\chapter{\sanskrit गुणत्रयविभागयोग} 
\paragraph{\sanskrit श्रीभगवानुवाच}
\begin{quotation} 
परं भूयः प्रवक्ष्यामि ज्ञानानां ज्ञानमुत्तमम् ।  

यज्ज्ञात्वा मुनयः सर्वे परां सिद्धिमितो गताः   ॥ १४.१ ॥  मूल श्लोक
\end{quotation}

\begin{quotation}

परं भूयः प्रवक्ष्यामि, ज्ञानानां ज्ञानम उत्तमम् ।  

यज् ज्ञात्वा मुनयः सर्वे, परां सिद्धिम इतो गताः  ॥ १४.१ ॥  उच्चारण

\noindent\rule{16cm}{0.4pt} 
\end{quotation}


\begin{quotation}

इदं ज्ञानमुपाश्रित्य मम साधर्म्यमागताः  ।  

सर्गेऽपि नोपजायन्ते प्रलये न व्यथन्ति च   ॥ १४.२ ॥  मूल श्लोक
\end{quotation}

\begin{quotation}

इदं ज्ञानम उपा-श्रित्य, मम साधर्म्-यम आगताः  ।  

सर्गेऽ अपि न उप-जायन्ते, प्रलये न व्यथन्ति च  ॥ १४.२ ॥  उच्चारण

\noindent\rule{16cm}{0.4pt} 
\end{quotation}


\begin{quotation}

मम योनिर्महद्ब्रह्म तस्मिन्गर्भं दधाम्यहम्‌  ।  

सम्भवः सर्वभूतानां ततो भवति भारत  ॥ १४.३ ॥  मूल श्लोक
\end{quotation}

\begin{quotation}

मम योनिर् महद् ब्रह्म, तस्मिन् गर्भं दधाम्य अहम्‌  ।  

सम्भवः सर्व-भूतानां, ततो भवति भारत  ॥ १४.३ ॥  उच्चारण

\noindent\rule{16cm}{0.4pt} 
\end{quotation}


\begin{quotation}

सर्वयोनिषु कौन्तेय मूर्तयः सम्भवन्ति याः  ।  

तासां ब्रह्म महद्योनिरहं बीजप्रदः पिता  ॥ १४.४ ॥  मूल श्लोक
\end{quotation}

\begin{quotation}

सर्व-योनिषु कौन्तेय, मूर्तयः सम्-भवन्ति याः  ।  

तासां ब्रह्म महद् योनिर, अहं बीज-प्रदः पिता  ॥ १४.४ ॥  उच्चारण

\noindent\rule{16cm}{0.4pt} 
\end{quotation}


\begin{quotation}

सत्त्वं रजस्तम इति गुणाः प्रकृतिसम्भवाः  ।  

निबध्नन्ति महाबाहो देहे देहिनमव्ययम्‌  ॥ १४.५ ॥  मूल श्लोक
\end{quotation}

\begin{quotation}

सत्त्वं रजस् तम इति, गुणाः प्रकृति-सम्भवाः  ।  

निबध्-नन्ति महाबाहो, देहे देहिनम अ-व्ययम्‌  ॥ १४.५ ॥  उच्चारण

\noindent\rule{16cm}{0.4pt} 
\end{quotation}


\begin{quotation}
तत्र सत्त्वं निर्मलत्वात्प्रकाशकमनामयम्‌  ।  

सुखसङ्‍गेन बध्नाति ज्ञानसङ्‍गेन चानघ  ॥ १४.६ ॥  मूल श्लोक
\end{quotation}

\begin{quotation}

तत्र सत्त्वं निर्मल-त्वात्, प्रकाशकम अनामयम्‌  ।  

सुख-संगेन बध्-नाति, ज्ञान संगेन चानघ  ॥ १४.६ ॥  उच्चारण

\noindent\rule{16cm}{0.4pt} 
\end{quotation}


\begin{quotation}

रजो रागात्मकं विद्धि तृष्णासङ्‍गसमुद्भवम्‌  ।  

तन्निबध्नाति कौन्तेय कर्मसङ्‍गेन देहिनम्‌  ॥ १४.७ ॥  मूल श्लोक
\end{quotation}

\begin{quotation}

रजो रागात्मकं विद्धि, तृष्णा संग समुद्-भवम्‌  ।  

तन्नि बध्-नाति कौन्तेय, कर्म-संगेन देहिनम्‌  ॥ १४.७ ॥  उच्चारण

\noindent\rule{16cm}{0.4pt} 
\end{quotation}


\begin{quotation}

तमस्त्वज्ञानजं विद्धि मोहनं सर्वदेहिनाम्‌  ।  

प्रमादालस्यनिद्राभिस्तन्निबध्नाति भारत  ॥ १४.८ ॥  मूल श्लोक
\end{quotation}

\begin{quotation}

तमस् त्व ज्ञानजं विद्धि, मोहनं सर्व देहिनाम्‌  ।  

प्रमाद आलस्य निद्राभिस्, तन्नि बध्-नाति भारत  ॥ १४.८ ॥  उच्चारण

\noindent\rule{16cm}{0.4pt} 
\end{quotation}


\begin{quotation}

सत्त्वं सुखे सञ्जयति रजः कर्मणि भारत  ।  

ज्ञानमावृत्य तु तमः प्रमादे सञ्जयत्युत  ॥ १४.९ ॥  मूल श्लोक
\end{quotation}

\begin{quotation}

सत्त्वं सुखे सञ्जयति, रजः कर्मणि भारत  ।  

ज्ञानम आवृत्य तु तमः, प्रमादे सञ्जयत्य उत  ॥ १४.९ ॥  उच्चारण

\noindent\rule{16cm}{0.4pt} 
\end{quotation}


\begin{quotation}

रजस्तमश्चाभिभूय सत्त्वं भवति भारत  ।  

रजः सत्त्वं तमश्चैव तमः सत्त्वं रजस्तथा  ॥ १४.१० ॥  मूल श्लोक
\end{quotation}

\begin{quotation}

रजस् तमस् च भि-भूय, सत्त्वं भवति भारत  ।  

रजः सत्त्वं तमस् च एैव, तमः सत्त्वं रजस् तथा  ॥ १४.१० ॥  उच्चारण

\noindent\rule{16cm}{0.4pt} 
\end{quotation}


\begin{quotation}

सर्वद्वारेषु देहेऽस्मिन्प्रकाश उपजायते  ।  

ज्ञानं यदा तदा विद्याद्विवृद्धं सत्त्वमित्युत  ॥ १४.११ ॥  मूल श्लोक
\end{quotation}

\begin{quotation}

सर्व-द्वारेषु देहेऽ अस्मिन्, प्रकाश उप-जायते  ।  

ज्ञानं यदा तदा विद्याद्, वि-वृद्धं सत्त्वम इत्य उत  ॥ १४.११ ॥  उच्चारण

\noindent\rule{16cm}{0.4pt} 
\end{quotation}


\begin{quotation}
लोभः प्रवृत्तिरारम्भः कर्मणामशमः स्पृहा  ।  

रजस्येतानि जायन्ते विवृद्धे भरतर्षभ  ॥ १४.१२ ॥  मूल श्लोक
\end{quotation}

\begin{quotation}

लोभः प्रवृत्तिर आरम्भः, कर्मणाम अशमः स्पृहा  ।  

रजस्य एतानि जायन्ते, वि-वृद्धे भरतर्षभ  ॥ १४.१२ ॥  उच्चारण

\noindent\rule{16cm}{0.4pt} 
\end{quotation}


\begin{quotation}

अप्रकाशोऽप्रवृत्तिश्च प्रमादो मोह एव च  ।  

तमस्येतानि जायन्ते विवृद्धे कुरुनन्दन  ॥ १४.१३ ॥  मूल श्लोक
\end{quotation}

\begin{quotation}

अ-प्रकाशोऽ अ-प्रवृत्तिश् च, प्रमादो मोह एव च  ।  

तमस्य एतानि जायन्ते, वि-वृद्धे कुरु नन्दन  ॥ १४.१३ ॥  उच्चारण

\noindent\rule{16cm}{0.4pt} 
\end{quotation}


\begin{quotation}

यदा सत्त्वे प्रवृद्धे तु प्रलयं याति देहभृत्‌  ।  

तदोत्तमविदां लोकानमलान्प्रतिपद्यते  ॥ १४.१४ ॥  मूल श्लोक
\end{quotation}

\begin{quotation}

यदा सत्त्वे प्रवृद्धे तु, प्रलयं याति देह भृत्‌  ।  

तद-उत्तम-विदां लोकान, अमलान् प्रति-पद्यते  ॥ १४.१४ ॥  उच्चारण

\noindent\rule{16cm}{0.4pt} 
\end{quotation}


\begin{quotation}

रजसि प्रलयं गत्वा कर्मसङ्‍गिषु जायते  ।  

तथा प्रलीनस्तमसि मूढयोनिषु जायते  ॥ १४.१५ ॥  मूल श्लोक
\end{quotation}

\begin{quotation}

रजसि प्रलयं गत्वा, कर्म सङ्‍गिषु जायते  ।  

तथा प्रलीनस् तमसि, मूढ योनिषु जायते  ॥ १४.१५ ॥  उच्चारण

\noindent\rule{16cm}{0.4pt} 
\end{quotation}


\begin{quotation}

कर्मणः सुकृतस्याहुः सात्त्विकं निर्मलं फलम्‌  ।  

रजसस्तु फलं दुःखमज्ञानं तमसः फलम्‌  ॥ १४.१६ ॥  मूल श्लोक
\end{quotation}

\begin{quotation}

कर्मणः सु-कृतस्य आहुः, सात्त्विकं निर्मलं फलम्‌  ।  

रजस-अस्तु फलं दुःखम, अज्ञानं तमसः फलम्‌  ॥ १४.१६ ॥  उच्चारण

\noindent\rule{16cm}{0.4pt} 
\end{quotation}


\begin{quotation}

सत्त्वात्सञ्जायते ज्ञानं रजसो लोभ एव च  ।  

प्रमादमोहौ तमसो भवतोऽज्ञानमेव च  ॥ १४.१७ ॥  मूल श्लोक
\end{quotation}

\begin{quotation}

सत्त्वात् सञ्जायते ज्ञानं, रजसो लोभ एव च  ।  

प्रमाद मोहौ तमसो, भवतोऽ अज्ञानम एव च  ॥ १४.१७ ॥  उच्चारण

\noindent\rule{16cm}{0.4pt} 
\end{quotation}


\begin{quotation}
ऊर्ध्वं गच्छन्ति सत्त्वस्था मध्ये तिष्ठन्ति राजसाः  ।  

जघन्यगुणवृत्तिस्था अधो गच्छन्ति तामसाः  ॥ १४.१८ ॥  मूल श्लोक
\end{quotation}

\begin{quotation}

ऊर्ध्वं गच्छन्ति सत्त्व-स्था, मध्ये तिष्ठन्ति राजसाः  ।  

जघन्य-गुण-वृत्ति-स्था, अधो गच्छन्ति तामसाः  ॥ १४.१८ ॥  उच्चारण

\noindent\rule{16cm}{0.4pt} 
\end{quotation}


\begin{quotation}

नान्यं गुणेभ्यः कर्तारं यदा द्रष्टानुपश्यति  ।  

गुणेभ्यश्च परं वेत्ति मद्भावं सोऽधिगच्छति  ॥ १४.१९ ॥  मूल श्लोक
\end{quotation}

\begin{quotation}

नान्यं गुणेभ्यः कर्तारं, यदा द्रष्टा अनु-पश्यति  ।  

गुणेभ्यश् च परं वेत्ति, मद्-भावं सोऽ अधि-गच्छति  ॥ १४.१९ ॥  उच्चारण

\noindent\rule{16cm}{0.4pt} 
\end{quotation}


\begin{quotation}

गुणानेतानतीत्य त्रीन्देही देहसमुद्भवान्‌  ।  

जन्ममृत्युजरादुःखैर्विमुक्तोऽमृतमश्नुते  ॥ १४.२० ॥  मूल श्लोक
\end{quotation}

\begin{quotation}

गुणान एतान अतीत्य त्रीन्, देही देह समुद्-भवान्‌  ।  

जन्म मृत्यु जरा दुःखैर्, विमुक्तोऽ अमृतम अश्-नुते  ॥ १४.२० ॥  उच्चारण

\noindent\rule{16cm}{0.4pt} 
\end{quotation}

\paragraph{\sanskrit अर्जुन उवाच}

\begin{quotation}

कैर्लिंगैस्त्रीन्गुणानेतानतीतो भवति प्रभो  ।  

किमाचारः कथं चैतांस्त्रीन्गुणानतिवर्तते  ॥ १४.२१ ॥  मूल श्लोक
\end{quotation}

\begin{quotation}

कैर् लिङ्गैस त्रीन् गुणान एतान, अतीतो भवति प्रभो  ।  

किम आचारः कथं चैतांस्, त्रीन् गुणान अति-वर्तते  ॥ १४.२१ ॥  उच्चारण

\noindent\rule{16cm}{0.4pt} 
\end{quotation}



\paragraph{\sanskrit श्रीभगवानुवाच}
\begin{quotation} 
प्रकाशं च प्रवृत्तिं च मोहमेव च पाण्डव  ।  

न द्वेष्टि सम्प्रवृत्तानि न निवृत्तानि काङ्‍क्षति  ॥ १४.२२ ॥  मूल श्लोक
\end{quotation}

\begin{quotation}

प्रकाशं च प्रवृत्तिं च, मोहम एव च पाण्डव  ।  

न द्वेष्टि सम्-प्रवृत्-तानि, न निवृत्-तानि कांक्षति  ॥ १४.२२ ॥  उच्चारण

\noindent\rule{16cm}{0.4pt} 
\end{quotation}


\begin{quotation}

उदासीनवदासीनो गुणैर्यो न विचाल्यते  ।  

गुणा वर्तन्त इत्येव योऽवतिष्ठति नेङ्‍गते  ॥ १४.२३ ॥  मूल श्लोक
\end{quotation}

\begin{quotation}

उदासीन-वद आसीनो, गुणैर् या न विचाल्यते  ।  

गुणा वर्तन्त इत्य एव, योऽ अव-तिष्ठति नेङ्‍गते  ॥ १४.२३ ॥  उच्चारण

\noindent\rule{16cm}{0.4pt} 
\end{quotation}


\begin{quotation}

समदुःखसुखः स्वस्थः समलोष्टाश्मकाञ्चनः  ।  

तुल्यप्रियाप्रियो धीरस्तुल्यनिन्दात्मसंस्तुतिः  ॥ १४.२४ ॥  मूल श्लोक
\end{quotation}

\begin{quotation}

सम दुःख सुखः स्व स्थः, सम लोष्टा-अश्म काञ्चनः  ।  

तुल्य प्रिया-अप्रियो धीरस्, तुल्य निन्दात्म संस्तुतिः  ॥ १४.२४ ॥  उच्चारण

\noindent\rule{16cm}{0.4pt} 
\end{quotation}


\begin{quotation}

मानापमानयोस्तुल्यस्तुल्यो मित्रारिपक्षयोः  ।  

सर्वारम्भपरित्यागी गुणातीतः सा उच्यते  ॥ १४.२५ ॥  मूल श्लोक
\end{quotation}

\begin{quotation}

माना अपमानयो: तुल्य:, तुल्यो मित्रारि पक्षयोः  ।  

सर्वारम्भ परित्यागी, गुणातीतः सा उच्यते  ॥ १४.२५ ॥  उच्चारण

\noindent\rule{16cm}{0.4pt} 
\end{quotation}


\begin{quotation}

मां च योऽव्यभिचारेण भक्तियोगेन सेवते  ।  

स गुणान्समतीत्येतान्ब्रह्मभूयाय कल्पते  ॥ १४.२६ ॥  मूल श्लोक
\end{quotation}

\begin{quotation}

मां च योऽ अव्यभि-चारेण, भक्ति-योगेन सेवते  ।  

स गुणान् सम-तीत्य एतान्, ब्रह्म भूयाय कल्पते  ॥  १४.२६ ॥  उच्चारण

\noindent\rule{16cm}{0.4pt} 
\end{quotation}


\begin{quotation}

ब्रह्मणो हि प्रतिष्ठाहममृतस्याव्ययस्य च  ।  

शाश्वतस्य च धर्मस्य सुखस्यैकान्तिकस्य च  ॥ १४.२७ ॥  मूल श्लोक
\end{quotation}

\begin{quotation}

ब्रह्मणो हि प्रतिष्ठा अहम, अमृतस्या अ-व्ययस्य च  ।  

शाश्वतस्य च धर्मस्य, सुखस्य एैकान्ति-कस्य च  ॥ १४.२७ ॥  उच्चारण

\noindent\rule{16cm}{0.4pt} 
\end{quotation}
\begin{center} ***** \end{center}


\begin{quotation}

ॐ तत् सद इति श्री मद्-भगवद्-गीतास उपनिषत्सु ब्रह्म विद्यायां योगशास्त्रे श्री कृष्णार्जुन संवादे गुणत्रयविभागयोगो नाम चतुर्दशोऽ अध्यायः  ॥  १४  ॥ 

\end{quotation}