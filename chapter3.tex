\chapter{\sanskrit कर्मयोग}

\sanskrit

\paragraph{\sanskrit अर्जुन उवाच}
\begin{quotation}
	
ज्यायसी चेत्कर्मणस्ते मता बुद्धिर्जनार्दन  ।  

तत्किं कर्मणि घोरे मां नियोजयसि केशव  ॥ ३.१ ॥  मूल श्लोक
\end{quotation}

\begin{quotation}

ज्यायसी चेत कर्मणस्ते, मता बुद्धिर जनार्दन  ।  

तत् किम् कर्मणि घोरे, माम् नियो-जयसि केशव  ॥ ३.१ ॥  उच्चारण

\noindent\rule{16cm}{0.4pt} 
\end{quotation}


\begin{quotation}

व्यामिश्रेणेव वाक्येन बुद्धिं मोहयसीव मे  ।  

तदेकं वद निश्चित्य येन श्रेयोऽहमाप्नुयाम्‌  ॥ ३.२ ॥  मूल श्लोक
\end{quotation}

\begin{quotation}

व्यामि-श्रेण एव वाक्येन, बुद्धिं मोहयस एव मे  ।  

तदेकं वद निश्चित्य, येन श्रेयोऽ अहम आप्नुयाम्‌  ॥ ३.२ ॥  उच्चारण

\noindent\rule{16cm}{0.4pt} 
\end{quotation}


\paragraph{\sanskrit श्रीभगवानुवाच}

\begin{quotation}



लोकेऽस्मिन्द्विविधा निष्ठा पुरा प्रोक्ता मयानघ  ।  

ज्ञानयोगेन साङ्‍ख्यानां कर्मयोगेन योगिनाम्‌  ॥ ३.३ ॥  मूल श्लोक
\end{quotation}

\begin{quotation}

लोकेऽ अस्मिन् द्-विविधा निष्ठा, पुरा प्रोक्ता मयानघ  ।  

ज्ञान-योगेन साङ्‍ख्यानां, कर्म-योगेन योगिनाम्‌  ॥ ३.३ ॥  उच्चारण

\noindent\rule{16cm}{0.4pt} 
\end{quotation}


\begin{quotation}

न कर्मणामनारंभान्नैष्कर्म्यं पुरुषोऽश्नुते  ।  

न च सन्न्यसनादेव सिद्धिं समधिगच्छति  ॥ ३.४ ॥  मूल श्लोक
\end{quotation}

\begin{quotation}

न कर्मणाम अनारंभान्, नैष्कर्म्यं पुरुषोऽ अश्नुते ।  

न च सन्न्यस-नाद एव, सिद्धिं सम-अधिगच्छति  ॥ ३.४ ॥  उच्चारण

\noindent\rule{16cm}{0.4pt} 
\end{quotation}


\begin{quotation}

न हि कश्चित्क्षणमपि जातु तिष्ठत्यकर्मकृत्‌  ।  

कार्यते ह्यवशः कर्म सर्वः प्रकृतिजैर्गुणैः  ॥ ३.५ ॥  मूल श्लोक
\end{quotation}

\begin{quotation}
न हि कश्चित क्षणम अपि, जातु तिष्ठत्य अकर्म-कृत्‌  ।  

कार्यते ह्यवशः कर्म, सर्वः प्रकृति-जैर् गुणैः  ॥ ३.५ ॥  उच्चारण

\noindent\rule{16cm}{0.4pt} 
\end{quotation}


\begin{quotation}

कर्मेन्द्रियाणि संयम्य य आस्ते मनसा स्मरन्‌  ।  

इन्द्रियार्थान्विमूढात्मा मिथ्याचारः स उच्यते  ॥ ३.६ ॥  मूल श्लोक
\end{quotation}

\begin{quotation}

कर्म एन्द्रियाणि संयम्य, य आस्ते मनसा स्मरन्‌  ।  

इन्द्रियार्थान विमूढ-आत्मा, मिथ्याचारः स उच्यते  ॥ ३.६ ॥  उच्चारण

\noindent\rule{16cm}{0.4pt} 
\end{quotation}


\begin{quotation}

यस्त्विन्द्रियाणि मनसा नियम्यारभतेऽर्जुन  ।  

कर्मेन्द्रियैः कर्मयोगमसक्तः स विशिष्यते  ॥ ३.७ ॥  मूल श्लोक
\end{quotation}

\begin{quotation}

यस् त्व इन्द्रियाणि मनसा, नियम्या-अरभतेऽ अर्जुन  ।  

कर्म-इन्द्रियैः कर्म-योगम, असक्तः स विशिष्यते  ॥ ३.७ ॥  उच्चारण

\noindent\rule{16cm}{0.4pt} 
\end{quotation}


\begin{quotation}

नियतं कुरु कर्म त्वं कर्म ज्यायो ह्यकर्मणः ।  

शरीरयात्रापि च ते न प्रसिद्धयेदकर्मणः  ॥ ३.८ ॥  मूल श्लोक
\end{quotation}

\begin{quotation}

नियतं कुरु कर्म त्वं, कर्म ज्यायो ह्यकर्मणः ।  

शरीर-यात्रापि च ते, न प्रसिद्ध-येद अकर्मणः  ॥ ३.८ ॥  उच्चारण

\noindent\rule{16cm}{0.4pt} 
\end{quotation}


\begin{quotation}

यज्ञार्थात्कर्मणोऽन्यत्र लोकोऽयं कर्मबंधनः  ।  

तदर्थं कर्म कौन्तेय मुक्तसंगः समाचर  ॥ ३.९ ॥  मूल श्लोक
\end{quotation}

\begin{quotation}

यज्ञार्थात कर्मणोऽ अन्यत्र, लोकोऽ अयं कर्म-बंधनः  ।  

तदर्थं कर्म कौन्तेय, मुक्त-संगः समाचर  ॥ ३.९ ॥  उच्चारण

\noindent\rule{16cm}{0.4pt} 
\end{quotation}


\begin{quotation}

सहयज्ञाः प्रजाः सृष्टा पुरोवाचप्रजापतिः  ।  

अनेन प्रसविष्यध्वमेष वोऽस्त्विष्टकामधुक्‌  ॥ ३.१० ॥  मूल श्लोक
\end{quotation}

\begin{quotation}

सह-यज्ञाः प्रजाः सृष्ट्वा, पुरोवाच प्रजापतिः  ।  

अनेन प्रस-विष्य-ध्वम, एष वोऽ अस्त्व इष्ट कामधुक्‌  ॥ ३.१० ॥  उच्चारण

\noindent\rule{16cm}{0.4pt} 
\end{quotation}


\begin{quotation}

देवान्भावयतानेन ते देवा भावयन्तु वः  ।  

परस्परं भावयन्तः श्रेयः परमवाप्स्यथ  ॥ ३.११ ॥  मूल श्लोक
\end{quotation}

\begin{quotation}
देवान् भाव-यता-नेन, ते देवा भाव-यन्तु वः  ।  

परस्परं भाव-यन्तः, श्रेयः परम अवाप्स्यथ  ॥ ३.११ ॥  उच्चारण

\noindent\rule{16cm}{0.4pt} 
\end{quotation}


\begin{quotation}

इष्टान्भोगान्हि वो देवा दास्यन्ते यज्ञभाविताः  ।  

तैर्दत्तानप्रदायैभ्यो यो भुंक्ते स्तेन एव सः  ॥ ३.१२ ॥  मूल श्लोक
\end{quotation}

\begin{quotation}

इष्टान् भोगान्हि वो देवा, दास्यन्ते यज्ञ-भाविताः  ।  

तैर-दत्तान अप्रदा-यैभ्यो, यो भुंक्ते स्तेन एव सः  ॥ ३.१२ ॥  उच्चारण

\noindent\rule{16cm}{0.4pt} 
\end{quotation}


\begin{quotation}

यज्ञशिष्टाशिनः सन्तो मुच्यन्ते सर्वकिल्बिषैः  ।  

भुञ्जते ते त्वघं पापा ये पचन्त्यात्मकारणात्‌  ॥ ३.१३ ॥  मूल श्लोक
\end{quotation}

\begin{quotation}

यज्ञ-शिष्टा-शिनः सन्तो, मुच्यन्ते सर्व-किल्बिषैः  ।  

भुञ्जते ते त्वघं पापा, ये पचन्त्य आत्म-कारणात्‌  ॥ ३.१३ ॥  उच्चारण

\noindent\rule{16cm}{0.4pt} 
\end{quotation}


\begin{quotation}

अन्नाद्भवन्ति भूतानि पर्जन्यादन्नसम्भवः  ।  

यज्ञाद्भवति पर्जन्यो यज्ञः कर्मसमुद्भवः  ॥ ३.१४ ॥  मूल श्लोक
\end{quotation}

\begin{quotation}

अन्नाद्-भवन्ति भूतानि, पर्जन्याद-अन्न-सम्भवः  ।  

यज्ञाद् भवति पर्जन्यो, यज्ञः कर्म-समुद् भवः  ॥ ३.१४ ॥  उच्चारण

\noindent\rule{16cm}{0.4pt} 
\end{quotation}


\begin{quotation}

कर्म ब्रह्मोद्भवं विद्धि ब्रह्माक्षरसमुद्भवम्‌  ।  

तस्मात्सर्वगतं ब्रह्म नित्यं यज्ञे प्रतिष्ठितम्‌  ॥ ३.१५ ॥  मूल श्लोक
\end{quotation}

\begin{quotation}

कर्म ब्रह्मोद् भवं विद्धि, ब्रह्माक्षर समुद्-भवम्‌  ।  

तस्मात सर्व-गतं ब्रह्म, नित्यं यज्ञे प्रतिष्ठितम्‌  ॥ ३.१५ ॥  उच्चारण

\noindent\rule{16cm}{0.4pt} 
\end{quotation}


\begin{quotation}

एवं प्रवर्तितं चक्रं नानुवर्तयतीह यः  ।  

अघायुरिन्द्रियारामो मोघं पार्थ स जीवति  ॥ ३.१६ ॥  मूल श्लोक
\end{quotation}

\begin{quotation}

एवं प्रवर्तितं चक्रं, नानु-वर्त-यतीह यः  ।  

अघायुर-इन्द्रिया-रामो, मोघं पार्थ स जीवति  ॥ ३.१६ ॥  उच्चारण

\noindent\rule{16cm}{0.4pt} 
\end{quotation}


\begin{quotation}

यस्त्वात्मरतिरेव स्यादात्मतृप्तश्च मानवः  ।  

आत्मन्येव च सन्तुष्टस्तस्य कार्यं न विद्यते  ॥ ३.१७ ॥  मूल श्लोक
\end{quotation}

\begin{quotation}
यस् त्व आत्म-रतिर एव स्याद, आत्म-तृप्तश् च मानवः  ।  

आत्मन्य एव च सन्तुष्टस, तस्य कार्यं न विद्यते  ॥ ३.१७ ॥  उच्चारण

\noindent\rule{16cm}{0.4pt} 
\end{quotation}


\begin{quotation}

नैव तस्य कृतेनार्थो नाकृतेनेह कश्चन  ।  

न चास्य सर्वभूतेषु कश्चिदर्थव्यपाश्रयः  ॥ ३.१८ ॥  मूल श्लोक
\end{quotation}

\begin{quotation}

नैव तस्य कृतेनार्थो, न अकृतेन इह कश्चन  ।  

न चास्य सर्व-भूतेषु, कश्चिद अर्थ-व्यपाश्रयः  ॥ ३.१८ ॥  उच्चारण

\noindent\rule{16cm}{0.4pt} 
\end{quotation}


\begin{quotation}

तस्मादसक्तः सततं कार्यं कर्म समाचर  ।  

असक्तो ह्याचरन्कर्म परमाप्नोति पुरुषः  ॥ ३.१९ ॥  मूल श्लोक
\end{quotation}

\begin{quotation}

तस्माद असक्तः सततं, कार्यं कर्म समाचर  ।  

असक्तो ह्य आचरन् कर्म, परम-अपनोति पुरुषः  ॥ ३.१९ ॥  उच्चारण

\noindent\rule{16cm}{0.4pt} 
\end{quotation}


\begin{quotation}

कर्मणैव हि संसिद्धिमास्थिता जनकादयः  ।  
 
लोकसंग्रहमेवापि सम्पश्यन्कर्तुमर्हसि  ॥ ३.२० ॥  मूल श्लोक
\end{quotation}

\begin{quotation}

कर्मण एैव हि संसिद्धिम, आस्थिता जनकादयः  ।  

लोक-संग्रहम एवापि, सम्पश्यं कर्तुम अर्हसि  ॥ ३.२० ॥  उच्चारण

\noindent\rule{16cm}{0.4pt} 
\end{quotation}


\begin{quotation}

यद्यदाचरति श्रेष्ठस्तत्तदेवेतरो जनः  ।  

स यत्प्रमाणं कुरुते लोकस्तदनुवर्तते  ॥ ३.२१ ॥  मूल श्लोक
\end{quotation}

\begin{quotation}

यद् यद अचरति श्रेष्ठस, तत् तद् एव इतरो जनः  ।  

स यत् प्रमाणं कुरुते, लोकस् तद अनुवर्तते  ॥ ३.२१ ॥  उच्चारण

\noindent\rule{16cm}{0.4pt} 
\end{quotation}


\begin{quotation}

न मे पार्थास्ति कर्तव्यं त्रिषु लोकेषु किंचन  ।  

नानवाप्तमवाप्तव्यं वर्त एव च कर्मणि  ॥ ३.२२ ॥  मूल श्लोक
\end{quotation}

\begin{quotation}

न मे पार्थास्ति कर्तव्यं, त्रिषु लोकेषु किंचन  ।  

ना अनवाप्तम अवाप्तव्यं, वर्त एव च कर्मणि  ॥ ३.२२ ॥  उच्चारण

\noindent\rule{16cm}{0.4pt} 
\end{quotation}


\begin{quotation}

यदि ह्यहं न वर्तेयं जातु कर्मण्यतन्द्रितः  ।  

मम वर्त्मानुवर्तन्ते मनुष्याः पार्थ सर्वशः  ॥ ३.२३ ॥  मूल श्लोक
\end{quotation}

\begin{quotation}
यदि ह्य अहं न वर्तेयं, जातु कर्मण्य अतन्द्रितः  ।  

मम वर्त्मा-अनुवर्तन्ते, मनुष्याः पार्थ सर्वशः  ॥ ३.२३ ॥  उच्चारण

\noindent\rule{16cm}{0.4pt} 
\end{quotation}


\begin{quotation}

उत्सीदेयुरिमे लोका न कुर्यां कर्म चेदहम्‌  ।  

संकरस्य च कर्ता स्यामुपहन्यामिमाः प्रजाः  ॥ ३.२४ ॥  मूल श्लोक
\end{quotation}

\begin{quotation}

उत्सीद-एयुर इमे लोकाः, न कुर्यां कर्म चेदहम्‌  ।  

संकरस्य च कर्ता स्याम, उपहन्यां इमाः प्रजाः  ॥ ३.२४ ॥  उच्चारण

\noindent\rule{16cm}{0.4pt} 
\end{quotation}


\begin{quotation}

सक्ताः कर्मण्यविद्वांसो यथा कुर्वन्ति भारत  ।  

कुर्याद्विद्वांस्तथासक्तश्चिकीर्षुर्लोकसंग्रहम्‌  ॥ ३.२५ ॥  मूल श्लोक
\end{quotation}

\begin{quotation}

सक्ताः कर्मण्य अविद्वांसो, यथा कुर्वन्ति भारत  ।  

कुर्याद विद्वानः तथा-असक्तः, चिकीर्षुर लोक-संग्रहम्‌  ॥ ३.२५ ॥  उच्चारण

\noindent\rule{16cm}{0.4pt} 
\end{quotation}


\begin{quotation}

न बुद्धिभेदं जनयेदज्ञानां कर्मसङि्गनाम्‌  ।  

जोषयेत्सर्वकर्माणि विद्वान्युक्तः समाचरन्‌  ॥ ३.२६ ॥  मूल श्लोक
\end{quotation}

\begin{quotation}

न बुद्धि-भेदं जनयेद, अज्ञानां कर्म-सङि्गनाम्‌  ।  

जोषयेत सर्व कर्माणि, विद्वान युक्तः समाचरन्‌  ॥ ३.२६ ॥  उच्चारण

\noindent\rule{16cm}{0.4pt} 
\end{quotation}


\begin{quotation}

प्रकृतेः क्रियमाणानि गुणैः कर्माणि सर्वशः  ।  

अहंकारविमूढात्मा कर्ताहमिति मन्यते  ॥ ३.२७ ॥  मूल श्लोक
\end{quotation}

\begin{quotation}

प्रकृतेः क्रिय-माणानि, गुणैः कर्माणि सर्वशः  ।  

अहंकार विमूढ-आत्मा, कर्ता अहम इति मन्यते  ॥ ३.२७ ॥  उच्चारण

\noindent\rule{16cm}{0.4pt} 
\end{quotation}


\begin{quotation}

तत्त्ववित्तु महाबाहो गुणकर्मविभागयोः  ।  

गुणा गुणेषु वर्तन्त इति मत्वा न सज्जते  ॥ ३.२८ ॥  मूल श्लोक
\end{quotation}

\begin{quotation}

तत्त्व-वित् तु महाबाहो, गुणकर्म विभागयोः  ।  

गुणा गुणेषु वर्तन्त, इति मत्वा न सज्जते  ॥ ३.२८ ॥  उच्चारण

\noindent\rule{16cm}{0.4pt} 
\end{quotation}


\begin{quotation}

प्रकृतेर्गुणसम्मूढ़ाः सज्जन्ते गुणकर्मसु  ।  

तानकृत्स्नविदो मन्दान्कृत्स्नविन्न विचालयेत्‌  ॥ ३.२९ ॥  मूल श्लोक
\end{quotation}

\begin{quotation}
प्रकृतेर  गुण-सम्मूढ़ाः, सज्जन्ते गुण-कर्मसु  ।  

तान अकृत्सं-विदो मन्दां, कृत्सं-विन न विचालयेत्‌  ॥ ३.२९ ॥  उच्चारण

\noindent\rule{16cm}{0.4pt} 
\end{quotation}


\begin{quotation}

मयि सर्वाणि कर्माणि सन्नयस्याध्यात्मचेतसा  ।  

निराशीर्निर्ममो भूत्वा युध्यस्व विगतज्वरः  ॥ ३.३० ॥  मूल श्लोक
\end{quotation}

\begin{quotation}

मयि सर्वाणि कर्माणि, संन्यसय-आध्यात्म चेतसा  ।  

निराशीर निर्ममो भूत्वा, युध्यस्व विगत-ज्वरः  ॥ ३.३० ॥  उच्चारण

\noindent\rule{16cm}{0.4pt} 
\end{quotation}


\begin{quotation}

ये मे मतमिदं नित्यमनुतिष्ठन्ति मानवाः  ।  

श्रद्धावन्तोऽनसूयन्तो मुच्यन्ते तेऽपि कर्मभिः  ॥ ३.३१ ॥  मूल श्लोक
\end{quotation}

\begin{quotation}

ये मे मतम इदं नित्यम, अनु-तिष्ठन्ति मानवाः  ।  

श्रद्धावन्तोऽ अन-सूयन्तो, मुच्यन्ते तेऽ अपि कर्मभिः  ॥ ३.३१ ॥  उच्चारण

\noindent\rule{16cm}{0.4pt} 
\end{quotation}


\begin{quotation}

ये त्वेतदभ्यसूयन्तो नानुतिष्ठन्ति मे मतम्‌  ।  

सर्वज्ञानविमूढांस्तान्विद्धि नष्टानचेतसः  ॥ ३.३२ ॥  मूल श्लोक
\end{quotation}

\begin{quotation}

ये त्व एतद अभ्य-सूयन्तो, नानु-तिष्ठन्ति मे मतम्‌  ।  

सर्व ज्ञान विमूढांस तान, विद्धि नष्टान-चेतसः  ॥ ३.३२ ॥  उच्चारण

\noindent\rule{16cm}{0.4pt} 
\end{quotation}


\begin{quotation}

सदृशं चेष्टते स्वस्याः प्रकृतेर्ज्ञानवानपि  ।  

प्रकृतिं यान्ति भूतानि निग्रहः किं करिष्यति  ॥ ३.३३ ॥  मूल श्लोक
\end{quotation}

\begin{quotation}

सदृशं चेष्टते स्वस्याः, प्रकृतेर ज्ञानवान अपि  ।  

प्रकृतिं यान्ति भूतानि, निग्रहः किं करिष्यति  ॥ ३.३३ ॥  उच्चारण

\noindent\rule{16cm}{0.4pt} 
\end{quotation}


\begin{quotation}

इन्द्रियस्येन्द्रियस्यार्थे रागद्वेषौ व्यवस्थितौ  ।  

तयोर्न वशमागच्छेत्तौ ह्यस्य परिपन्थिनौ  ॥ ३.३४ ॥  मूल श्लोक
\end{quotation}

\begin{quotation}

इन्द्रियस्य इन्द्रियस्या अर्थे, रागद्वेषौ व्यवस्थितौ  ।  

तयोर् न वशम अगच्छेत्, तौ ह्य अस्य परि-पन्थिनौ  ॥ ३.३४ ॥  उच्चारण

\noindent\rule{16cm}{0.4pt} 
\end{quotation}


\begin{quotation}

श्रेयान्स्वधर्मो विगुणः परधर्मात्स्वनुष्ठितात्‌  ।  

स्वधर्मे निधनं श्रेयः परधर्मो भयावहः  ॥ ३.३५ ॥  मूल श्लोक
\end{quotation}

\begin{quotation}
श्रेयान् स्व-धर्मो विगुणः, पर-धर्मात स्व-अनुष्ठि-तात्‌  ।  

स्व-धर्मे निधनं श्रेयः, पर-धर्मो भयावहः  ॥ ३.३५ ॥  उच्चारण

\noindent\rule{16cm}{0.4pt} 
\end{quotation}

\paragraph{\sanskrit अर्जुन उवाच}

\begin{quotation}



अथ केन प्रयुक्तोऽयं पापं चरति पुरुषः  ।  

अनिच्छन्नपि वार्ष्णेय बलादिव नियोजितः  ॥ ३.३६ ॥  मूल श्लोक
\end{quotation}

\begin{quotation}

अथ केन प्रयुक्तोऽ अयं, पापं चरति पुरुषः  ।  

अनिच्छन्न अपि वार्ष्णेय, बलादिव नियोजितः  ॥ ३.३६ ॥  उच्चारण

\noindent\rule{16cm}{0.4pt} 
\end{quotation}

\paragraph{\sanskrit श्रीभगवानुवाच}

\begin{quotation}



काम एष क्रोध एष रजोगुणसमुद्भवः  ।  

महाशनो महापाप्मा विद्धयेनमिह वैरिणम्‌  ॥ ३.३७ ॥  मूल श्लोक
\end{quotation}

\begin{quotation}

काम एष क्रोध एष,रजोगुण समुद्भवः  ।  

महाशनो महा-पाप्मा, विद्धये इनम् इह वैरिणम्‌  ॥ ३.३७ ॥  उच्चारण

\noindent\rule{16cm}{0.4pt} 
\end{quotation}


\begin{quotation}

धूमेनाव्रियते वह्निर्यथादर्शो मलेन च ।  

यथोल्बेनावृतो गर्भस्तथा तेनेदमावृतम्‌  ॥ ३.३८ ॥  मूल श्लोक
\end{quotation}

\begin{quotation}

धूमेना-अव्रियते वहनिर, यथा-दर्शो मलेन च ।  

यथोल्बेन्-आवृतो गर्भस, तथा तेनेदम् आवृतम्‌  ॥ ३.३८ ॥  उच्चारण

\noindent\rule{16cm}{0.4pt} 
\end{quotation}


\begin{quotation}

आवृतं ज्ञानमेतेन ज्ञानिनो नित्यवैरिणा  ।  

कामरूपेण कौन्तेय दुष्पूरेणानलेन च  ॥ ३.३९ ॥  मूल श्लोक
\end{quotation}

\begin{quotation}

आवृतं ज्ञानम एतेन, ज्ञानिनो नित्य वैरिणा  ।  

काम रूपेण कौन्तेय, दुष्पूरेणा-अनलेन च  ॥ ३.३९ ॥  उच्चारण

\noindent\rule{16cm}{0.4pt} 
\end{quotation}


\begin{quotation}


इन्द्रियाणि मनो बुद्धिरस्याधिष्ठानमुच्यते  ।  

एतैर्विमोहयत्येष ज्ञानमावृत्य देहिनम्‌  ॥ ३.४० ॥  मूल श्लोक
\end{quotation}

\begin{quotation}

इन्द्रियाणि मनो बुद्धिर, अस्य अधिष्ठानं उच्यते  ।  

एतै र्विमोह-यत्य एष, ज्ञानं आवृत्य देहिनम्‌  ॥ ३.४० ॥  उच्चारण

\noindent\rule{16cm}{0.4pt} 
\end{quotation}


\begin{quotation}

तस्मात्त्वमिन्द्रियाण्यादौ नियम्य भरतर्षभ  ।  

पाप्मानं प्रजहि ह्येनं ज्ञानविज्ञाननाशनम्‌  ॥ ३.४१ ॥  मूल श्लोक
\end{quotation}

\begin{quotation}

तस्मात त्वम इन्द्रियाण्य आदौ, नियम्य भरतर्षभ  ।  

पाप्मानं प्रजहि ह्य एनं, ज्ञान विज्ञान नाशनम्‌  ॥ ३.४१ ॥  उच्चारण

\noindent\rule{16cm}{0.4pt} 
\end{quotation}


\begin{quotation}

इन्द्रियाणि पराण्याहुरिन्द्रियेभ्यः परं मनः  ।  

मनसस्तु परा बुद्धिर्यो बुद्धेः परतस्तु सः  ॥ ३.४२ ॥  मूल श्लोक
\end{quotation}

\begin{quotation}

इन्द्रियाणि पराण्याहुर, इन्द्रिय-एभ्यः परं मनः  ।  

मन-सस्तु परा बुद्धिर, यो बुद्धेः पर-तस्तु सः  ॥ ३.४२ ॥  उच्चारण

\noindent\rule{16cm}{0.4pt} 
\end{quotation}


\begin{quotation}

एवं बुद्धेः परं बुद्धवा संस्तभ्यात्मानमात्मना  ।  

जहि शत्रुं महाबाहो कामरूपं दुरासदम्‌  ॥ ३.४३ ॥  मूल श्लोक
\end{quotation}

\begin{quotation}

एवं बुद्धेः परं बुद्धवा, संस्तभ्य-आत्मानम आत्मना  ।  

जहि शत्रुं महाबाहो, कामरूपं दुरासदम्‌  ॥ ३.४३ ॥  उच्चारण

\noindent\rule{16cm}{0.4pt} 
\end{quotation}





\begin{center} ***** \end{center}
\begin{quotation}
ॐ तत् सद इति श्री मद्-भगवद्-गीतास उपनिषत्सु ब्रह्म विद्यायां योगशास्त्रे श्री कृष्णार्जुन संवादे कर्मयोगो नाम तृतीयोऽ अध्यायः  ॥  ३  ॥ 
\end{quotation}

