\chapter{\sanskrit अक्षर ब्रह्मयोग} 
\paragraph{\sanskrit अर्जुन उवाच}
\begin{quotation} 
किं तद्ब्रह्म किमध्यात्मं किं पुरुषोत्तम  ।  

अधिभूतं च किं प्रोक्तमधिदैवं किमुच्यते  ॥ ८.१ ॥  मूल श्लोक
\end{quotation}

\begin{quotation}

किं तद् ब्रह्म किम अध्यात्मम, किं कर्म पुरुषोत्तम  ।  

अधिभूतं च किं प्रोक्तम, अधिदैवं किम उच्यते  ॥ ८.१ ॥  उच्चारण

\noindent\rule{16cm}{0.4pt} 
\end{quotation}


\begin{quotation} 

अधियज्ञः कथं कोऽत्र देहेऽस्मिन्मधुसूदन  ।  

प्रयाणकाले च कथं ज्ञेयोऽसि नियतात्मभिः  ॥ ८.२ ॥  मूल श्लोक
\end{quotation}

\begin{quotation}

अधियज्ञः कथं कोऽ अत्र, देहेऽ अस्मिन मधुसूदन  ।  

प्रयाण काले च कथं, ज्ञेयोऽ असि नियत आत्म-भिः  ॥ ८.२ ॥  उच्चारण

\noindent\rule{16cm}{0.4pt} 
\end{quotation}



\paragraph{\sanskrit श्रीभगवानुवाच}
\begin{quotation} 

अक्षरं ब्रह्म परमं स्वभावोऽध्यात्ममुच्यते  ।  

भूतभावोद्भवकरो विसर्गः कर्मसंज्ञितः  ॥ ८.३ ॥  मूल श्लोक
\end{quotation}

\begin{quotation}

अक्षरं ब्रह्म परमं, स्वभावोऽ अध्यात्मम उच्यते  ।  

भूत-भाव उद्भव-करो, विसर्गः कर्म संज्ञितः  ॥ ८.३ ॥  उच्चारण

\noindent\rule{16cm}{0.4pt} 
\end{quotation}


\begin{quotation} 

अधिभूतं क्षरो भावः पुरुषश्चाधिदैवतम्‌  ।  

अधियज्ञोऽहमेवात्र देहे देहभृतां वर  ॥ ८.४ ॥  मूल श्लोक
\end{quotation}

\begin{quotation}

अधिभूतं क्षरो भावः, पुरुषश् च अधि-दैवतम्‌  ।  

अधियज्ञोऽ अहम एवात्र, देहे देह-भृतां वर  ॥ ८.४ ॥  उच्चारण

\noindent\rule{16cm}{0.4pt} 
\end{quotation}


\begin{quotation} 

अंतकाले च मामेव स्मरन्मुक्त्वा कलेवरम्‌  ।  

यः प्रयाति स मद्भावं याति नास्त्यत्र संशयः  ॥ ८.५ ॥  मूल श्लोक
\end{quotation}

\begin{quotation}
अंत-काले च माम एव, स्मरन मुक्त्वा कलेवरम्‌  ।  

यः प्रयाति स मद्भावं, याति नास्-त्य अत्र संशयः  ॥ ८.५ ॥  उच्चारण

\noindent\rule{16cm}{0.4pt} 
\end{quotation}


\begin{quotation} 

यं यं वापि स्मरन्भावं त्यजत्यन्ते कलेवरम्‌  ।  

तं तमेवैति कौन्तेय सदा तद्भावभावितः  ॥ ८.६ ॥  मूल श्लोक
\end{quotation}

\begin{quotation}

यं यं वापि स्मरन भावं, त्यजत्य अन्ते कलेवरम्‌  ।  

तं तम एवैति कौन्तेय, सदा तद्भाव भावितः  ॥ ८.६ ॥  उच्चारण

\noindent\rule{16cm}{0.4pt} 
\end{quotation}


\begin{quotation} 

तस्मात्सर्वेषु कालेषु मामनुस्मर युद्ध च  ।  

मय्यर्पितमनोबुद्धिर्मामेवैष्यस्यसंशयम्‌  ॥ ८.७ ॥  मूल श्लोक
\end{quotation}

\begin{quotation}

तस्मात सर्वेषु कालेषु, माम अनुस्मर युद्ध च  ।  

मय्य अर्पित मनो बुद्धिर, माम एवे एष्-यस्य असंशयम्‌  ॥ ८.७ ॥  उच्चारण

\noindent\rule{16cm}{0.4pt} 
\end{quotation}


\begin{quotation} 

अभ्यासयोगयुक्तेन चेतसा नान्यगामिना  ।  

परमं पुरुषं दिव्यं याति पार्थानुचिन्तयन्‌  ॥ ८.८ ॥  मूल श्लोक
\end{quotation}

\begin{quotation}

अभ्यास-योग युक्तेन, चेतसा नान्य-गामिना  ।  

परमं पुरुषं दिव्यं, याति पार्था-अनु-चिन्तयन्‌  ॥ ८.८ ॥  उच्चारण

\noindent\rule{16cm}{0.4pt} 
\end{quotation}


\begin{quotation} 

कविं पुराणमनुशासितार-मणोरणीयांसमनुस्मरेद्यः  ।  

सर्वस्य धातारमचिन्त्यरूप-मादित्यवर्णं तमसः परस्तात्‌  ॥ ८.९ ॥  मूल श्लोक
\end{quotation}

\begin{quotation}

कविं पुराणम अनु-शासितारम्, 
अणोर अणीयां-सम अनु-स्मरेद यः  ।  

सर्वस्य धातारम अचिन्त्य रूपम, 
आदित्य वर्णं तमसः परस्तात्‌  ॥ ८.९ ॥  उच्चारण

\noindent\rule{16cm}{0.4pt} 
\end{quotation}


\begin{quotation} 

प्रयाण काले मनसाचलेन 
भक्त्या युक्तो योगबलेन चैव  ।  

भ्रुवोर्मध्ये प्राणमावेश्य सम्यक्‌ 
स तं परं पुरुषमुपैति दिव्यम्‌  ॥ ८.१० ॥  मूल श्लोक
\end{quotation}

\begin{quotation}


प्रयाण काले मनसा चलेन, 
भक्त्या युक्तो योगबलेन चैव  ।  

भ्रुवोर मध्ये प्राणम आवेश्य सम्यक्‌, 
स तं परं पुरुषम उपैति दिव्यम्‌  ॥ ८.१० ॥  उच्चारण

\noindent\rule{16cm}{0.4pt} 
\end{quotation}


\begin{quotation} 

यदक्षरं वेदविदो वदन्ति विशन्ति यद्यतयो वीतरागाः  ।  

यदिच्छन्तो ब्रह्मचर्यं चरन्ति तत्ते पदं संग्रहेण प्रवक्ष्ये  ॥ ८.११ ॥  मूल श्लोक
\end{quotation}

\begin{quotation}

यद अक्षरं वेद विदो वदन्ति, 
विशन्ति यद-तयो वीतरागाः  ।  

यदिच्-छन्तो ब्रह्मचर्यम् चरन्ति, 
तत्ते पदं संग्रहेण प्रवक्ष्ये  ॥ ८.११ ॥  उच्चारण

\noindent\rule{16cm}{0.4pt} 
\end{quotation}


\begin{quotation} 

सर्वद्वाराणि संयम्य मनो हृदि निरुध्य च  ।  

मूर्ध्न्याधायात्मनः प्राणमास्थितो योगधारणाम्‌  ॥ ८.१२ ॥  मूल श्लोक
\end{quotation}

\begin{quotation}

सर्व-द्वाराणि संयम य, मनो हृदि निरुध्य च  ।  

मूर्धन्य आधाय आत्मनः प्राणम, आस्थितो योग-धारणाम्‌  ॥ ८.१२ ॥  उच्चारण

\noindent\rule{16cm}{0.4pt} 
\end{quotation}


\begin{quotation} 

ओमित्येकाक्षरं ब्रह्म व्याहरन्मामनुस्मरन्‌  ।  

यः प्रयाति त्यजन्देहं स याति परमां गतिम्‌  ॥ ८.१३ ॥  मूल श्लोक
\end{quotation}

\begin{quotation}

ओम इति एक अक्षरं ब्रह्म, व्याहरन माम अनु-स्मरन्‌  ।  

यः प्रयाति त्यजन देहं, स याति परमां गतिम्‌  ॥ ८.१३ ॥  उच्चारण

\noindent\rule{16cm}{0.4pt} 
\end{quotation}


\begin{quotation} 

अनन्यचेताः सततं यो मां स्मरति नित्यशः  ।  

तस्याहं सुलभः पार्थ नित्ययुक्तस्य योगिनीः  ॥ ८.१४ ॥  मूल श्लोक
\end{quotation}

\begin{quotation}

अनन्य चेताः सततं, यो मां स्मरति नित्य शः  ।  

तस्याहं सुलभः पार्थ, नित्य युक्तस्य योगिनः  ॥ ८.१४ ॥  उच्चारण

\noindent\rule{16cm}{0.4pt} 
\end{quotation}


\begin{quotation} 

मामुपेत्य पुनर्जन्म दुःखालयमशाश्वतम्‌  ।  

नाप्नुवन्ति महात्मानः संसिद्धिं परमां गताः  ॥ ८.१५ ॥  मूल श्लोक
\end{quotation}

\begin{quotation}


माम उपेत्य पुनर्जन्म, दुःखालयम शाश्वतम्‌  ।  

न-अप्-नुवन्ति महात्मानः, संसिद्धिं परमां गताः  ॥ ८.१५ ॥  उच्चारण

\noindent\rule{16cm}{0.4pt} 
\end{quotation}


\begin{quotation} 

आब्रह्मभुवनाल्लोकाः पुनरावर्तिनोऽर्जुन  ।  

मामुपेत्य तु कौन्तेय पुनर्जन्म न विद्यते  ॥ ८.१६ ॥  मूल श्लोक
\end{quotation}

\begin{quotation}

अ-ब्रह्म-भुवनाल-लोकाः, पुनरावर्ति-नोऽ अर्जुन  ।  

माम उपेत्य तु कौन्तेय, पुनर्जन्म न विद्यते  ॥ ८.१६ ॥  उच्चारण

\noindent\rule{16cm}{0.4pt} 
\end{quotation}


\begin{quotation} 

सहस्रयुगपर्यन्तमहर्यद्ब्रह्मणो विदुः  ।  

रात्रिं युगसहस्रान्तां तेऽहोरात्रविदो जनाः  ॥ ८.१७ ॥  मूल श्लोक
\end{quotation}

\begin{quotation}

सहस्र युग पर्यन्तम, अहर-यद ब्रह्मणो विदुः  ।  

रात्रिं युग सहस्र-आन्तां, तेऽ अहो-रात्र-विदो जनाः  ॥ ८.१७ ॥  उच्चारण

\noindent\rule{16cm}{0.4pt} 
\end{quotation}


\begin{quotation} 

अव्यक्ताद्व्यक्तयः सर्वाः प्रभवन्त्यहरागमे  ।  

रात्र्यागमे प्रलीयन्ते तत्रैवाव्यक्तसंज्ञके  ॥ ८.१८ ॥  मूल श्लोक
\end{quotation}

\begin{quotation}

अव्यक्ताद व्यक्त-यः सर्वाः, प्रभवन्त्य अहर-अगमे  ।  

रात्र्य अगमे प्रलीयन्ते, तत्रैवा-व्यक्त संज्ञके  ॥ ८.१८ ॥  उच्चारण

\noindent\rule{16cm}{0.4pt} 
\end{quotation}


\begin{quotation} 

भूतग्रामः स एवायं भूत्वा भूत्वा प्रलीयते  ।  

रात्र्यागमेऽवशः पार्थ प्रभवत्यहरागमे  ॥ ८.१९ ॥  मूल श्लोक
\end{quotation}

\begin{quotation}

भूत-ग्रामः स एवायं, भूत्वा भूत्वा प्रलीयते  ।  

रात्र्य अगमेऽ अवशः पार्थ, प्रभवत्य अहर-अगमे  ॥ ८.१९ ॥  उच्चारण

\noindent\rule{16cm}{0.4pt} 
\end{quotation}


\begin{quotation} 

परस्तस्मात्तु भावोऽन्योऽव्यक्तोऽव्यक्तात्सनातनः  ।  

यः स सर्वेषु भूतेषु नश्यत्सु न विनश्यति  ॥ ८.२० ॥  मूल श्लोक
\end{quotation}

\begin{quotation}

परस्-तस्मात तु भावोऽ अन्योऽ, अव्यक्तोऽ अव्यक्तात सनातनः  ।  

यः स सर्वेषु भूतेषु, नश्यत-सु न विनश्यति  ॥ ८.२० ॥  उच्चारण

\noindent\rule{16cm}{0.4pt} 
\end{quotation}


\begin{quotation} 

अव्यक्तोऽक्षर इत्युक्तस्तमाहुः परमां गतिम्‌  ।  

यं प्राप्य न निवर्तन्ते तद्धाम परमं मम  ॥ ८.२१ ॥  मूल श्लोक
\end{quotation}

\begin{quotation}
अव्यक्तोऽ अक्षर इति युक्तस, तम आहुः परमां गतिम्‌  ।  

यं प्राप्य न नि-वर्तन्ते, तद धाम परमं मम  ॥ ८.२१ ॥  उच्चारण

\noindent\rule{16cm}{0.4pt} 
\end{quotation}


\begin{quotation} 

पुरुषः स परः पार्थ भक्त्या लभ्यस्त्वनन्यया  ।  

यस्यान्तः स्थानि भूतानि येन सर्वमिदं ततम्‌  ॥ ८.२२ ॥  मूल श्लोक
\end{quotation}

\begin{quotation}

पुरुषः स परः पार्थ, भक्त्या लभ्यस त्व अनन्यया  ।  

यस्यान्तः स्थानि भूतानि, येन सर्वम इदं ततम्‌  ॥ ८.२२ ॥  उच्चारण

\noindent\rule{16cm}{0.4pt} 
\end{quotation}


\begin{quotation} 

यत्र काले त्वनावत्तिमावृत्तिं चैव योगिनः  ।  

प्रयाता यान्ति तं कालं वक्ष्यामि भरतर्षभ  ॥ ८.२३ ॥  मूल श्लोक
\end{quotation}

\begin{quotation}

यत्र काले त्व अनावृत्तिम, आवृत्तिं चैव योगिनः  ।  

प्रयाता यान्ति तं कालं, वक्ष्यामि भरतर्षभ  ॥ ८.२३ ॥  उच्चारण

\noindent\rule{16cm}{0.4pt} 
\end{quotation}


\begin{quotation} 

अग्निर्ज्योतिरहः शुक्लः षण्मासा उत्तरायणम्‌  ।  

तत्र प्रयाता गच्छन्ति ब्रह्म ब्रह्मविदो जनाः  ॥ ८.२४ ॥  मूल श्लोक
\end{quotation}

\begin{quotation}

अग्निर ज्योतिर अहः शुक्लः, षण्मासा उत्तरा-यणम्‌  ।  

तत्र प्रयाता गच्छन्ति, ब्रह्म ब्रह्म-विदो जनाः  ॥ ८.२४ ॥  उच्चारण

\noindent\rule{16cm}{0.4pt} 
\end{quotation}


\begin{quotation} 

धूमो रात्रिस्तथा कृष्ण षण्मासा दक्षिणायनम्‌  ।  

तत्र चान्द्रमसं ज्योतिर्योगी प्राप्य निवर्तते  ॥ ८.२५ ॥  मूल श्लोक
\end{quotation}

\begin{quotation}

धूमो रात्रिस तथा कृष्ण, षण्मासा दक्षिणा-यनम्‌  ।  

तत्र चान्द्रमसं ज्योतिर, योगी प्राप्य नि-वर्तते  ॥ ८.२५ ॥  उच्चारण

\noindent\rule{16cm}{0.4pt} 
\end{quotation}


\begin{quotation} 

शुक्ल कृष्णे गती ह्येते जगतः शाश्वते मते  ।  

एकया यात्यनावृत्ति मन्ययावर्तते पुनः  ॥ ८.२६ ॥  मूल श्लोक
\end{quotation}

\begin{quotation}

शुक्ल कृष्णे गती ह्येते, जगतः शाश्वते मते  ।  

एकया यात्य अनावृत्तिम, अन्यय-आवर्तते पुनः  ॥ ८.२६ ॥  उच्चारण

\noindent\rule{16cm}{0.4pt} 
\end{quotation}


\begin{quotation} 

नैते सृती पार्थ जानन्योगी मुह्यति कश्चन  ।  

तस्मात्सर्वेषु कालेषु योगयुक्तो भवार्जुन  ॥ ८.२७ ॥  मूल श्लोक
\end{quotation}

\begin{quotation}
नैते सृती पार्थ जानन, योगी मुह्यति कश्चन  ।  

तस्मात सर्वेषु कालेषु, योग युक्तो भवार्जुन  ॥ ८.२७ ॥  उच्चारण

\noindent\rule{16cm}{0.4pt} 
\end{quotation}


\begin{quotation} 

वेदेषु यज्ञेषु तपःसु चैव दानेषु यत्पुण्यफलं प्रदिष्टम्‌  ।  

अत्येत तत्सर्वमिदं विदित्वा योगी परं स्थानमुपैति चाद्यम्‌  ॥ ८.२८ ॥  मूल श्लोक
\end{quotation}

\begin{quotation}

वेदेषु यज्ञेषु तपःसु चैव, 
दानेषु यत-पुण्य-फलं प्रदिष्टम्‌  ।  

अत्येती तत सर्वम इदं विदित्वा, 
योगी परं स्थानम उपैति चाद्यम्‌  ॥ ८.२८ ॥  उच्चारण

\noindent\rule{16cm}{0.4pt} 
\end{quotation}

\begin{center} ***** \end{center}

\begin{quotation} 





ॐ तत् सद इति श्री मद्-भगवद्-गीतास उपनिषत्सु ब्रह्म विद्यायां योगशास्त्रे श्री कृष्णार्जुन संवादे अक्षर ब्रह्मयोगो नामाष्टमोऽ अध्यायः  ॥  ८ ॥ 

\end{quotation}



