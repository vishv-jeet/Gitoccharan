\chapter{\sanskrit ज्ञानविज्ञानयोग}
\paragraph{\sanskrit श्रीभगवानुवाच}

\begin{quotation}
मय्यासक्तमनाः पार्थ योगं युञ्जन्मदाश्रयः  ।  

असंशयं समग्रं मां यथा ज्ञास्यसि तच्छृणु  ॥ ७.१ ॥  मूल श्लोक
\end{quotation}

\begin{quotation}
मय्य आसक्त मनाः पार्थ, योगं युञ्जन मद-आश्रयः  ।  

असंशयं समग्रं माम, यथा ज्ञास्यसि तच्-छ्रुणु  ॥ ७.१ ॥  उच्चारण

\noindent\rule{16cm}{0.4pt} 
\end{quotation}


\begin{quotation} 
ज्ञानं तेऽहं सविज्ञानमिदं वक्ष्याम्यशेषतः  ।  

यज्ज्ञात्वा नेह भूयोऽन्यज्ज्ञातव्यमवशिष्यते  ॥ ७.२ ॥  मूल श्लोक
\end{quotation}

\begin{quotation}
ज्ञानं तेऽ अहं सविज्ञानम्, इदं वक्ष्याम्य अशेषतः  ।  

यज ज्ञात्वा नेह भूयोऽ अन्यज, ज्ञातव्यम् अव-शिष्यते  ॥ ७.२ ॥  उच्चारण

\noindent\rule{16cm}{0.4pt} 
\end{quotation}


\begin{quotation} 
मनुष्याणां सहस्रेषु कश्चिद्यतति सिद्धये  ।  

यततामपि सिद्धानां कश्चिन्मां वेत्ति तत्वतः  ॥ ७.३ ॥  मूल श्लोक
\end{quotation}

\begin{quotation}
मनुष्याणां सहस्र-एषु, कश्चिद् यतति सिद्धये  ।  

यतताम् अपि सिद्धानां, कश्चिन माम् वेत्ति तत्वतः  ॥ ७.३ ॥  उच्चारण

\noindent\rule{16cm}{0.4pt} 
\end{quotation}


\begin{quotation} 
भूमिरापोऽनलो वायुः खं मनो बुद्धिरेव च  ।  

अहङ्‍कार इतीयं मे भिन्ना प्रकृतिरष्टधा  ॥ ७.४ ॥  मूल श्लोक
\end{quotation}

\begin{quotation}
भूमिर आपोऽ अनलो वायुः, खं मनो बुद्धिर एव च  ।  

अहन्कार इतीयं मे, भिन्ना प्रकृतिर अष्टधा  ॥ ७.४ ॥  उच्चारण

\noindent\rule{16cm}{0.4pt} 
\end{quotation}


\begin{quotation} 
अपरेयमितस्त्वन्यां प्रकृतिं विद्धि मे पराम्‌  ।  

जीवभूतां महाबाहो ययेदं धार्यते जगत्‌  ॥ ७.५ ॥  मूल श्लोक
\end{quotation}

\begin{quotation}
अपरेयम् इतस तव अन्यां, प्रकृतिं विद्धि मे पराम्‌  ।  

जीव भूतां महाबाहो, ययेदं धार्यते जगत्‌  ॥ ७.५ ॥  उच्चारण

\noindent\rule{16cm}{0.4pt} 
\end{quotation}


\begin{quotation} 

एतद्योनीनि भूतानि सर्वाणीत्युपधारय  ।  

अहं कृत्स्नस्य जगतः प्रभवः प्रलयस्तथा  ॥ ७.६ ॥  मूल श्लोक
\end{quotation}

\begin{quotation}
एतद-योनीनि भूतानि, सर्वाणीत्य उपधारय  ।  

अहं कृत्स्न-अस्य जगतः, प्रभवः प्रलयस तथा  ॥ ७.६ ॥  उच्चारण

\noindent\rule{16cm}{0.4pt} 
\end{quotation}


\begin{quotation} 
मत्तः परतरं नान्यत्किञ्चिदस्ति धनञ्जय  ।  

मयि सर्वमिदं प्रोतं सूत्रे मणिगणा इव  ॥ ७.७ ॥  मूल श्लोक
\end{quotation}

\begin{quotation}
मत्तः पर-तरं नान्यत, किञ्चिद अस्ति धनञ्जय  ।  

मयि सर्वम इदं प्रोतं, सूत्रे मणि-गणा इव  ॥ ७.७ ॥  उच्चारण

\noindent\rule{16cm}{0.4pt} 
\end{quotation}


\begin{quotation} 
रसोऽहमप्सु कौन्तेय प्रभास्मि शशिसूर्ययोः  ।  

प्रणवः सर्ववेदेषु शब्दः खे पौरुषं नृषु  ॥ ७.८ ॥  मूल श्लोक
\end{quotation}

\begin{quotation}
रसोऽ अहं अप्सु कौन्तेय, प्रभा अस्मि शशि सूर्ययोः  ।  

प्रणवः सर्व-वेदेषु, शब्दः खे पौरुषं नृषु  ॥ ७.८ ॥  उच्चारण

\noindent\rule{16cm}{0.4pt} 
\end{quotation}


\begin{quotation} 
पुण्यो गन्धः पृथिव्यां च तेजश्चास्मि विभावसौ  ।  

जीवनं सर्वभूतेषु तपश्चास्मि तपस्विषु  ॥ ७.९ ॥  मूल श्लोक
\end{quotation}

\begin{quotation}
पुण्यो गन्धः पृथिव्यां च, तेजस च अस्मि विभावसौ  ।  

जीवनं सर्व-भूतेषु, तपस च अस्मि तपस्विषु  ॥ ७.९ ॥  उच्चारण

\noindent\rule{16cm}{0.4pt} 
\end{quotation}


\begin{quotation} 
बीजं मां सर्वभूतानां विद्धि पार्थ सनातनम्‌  ।  

बुद्धिर्बुद्धिमतामस्मि तेजस्तेजस्विनामहम्‌  ॥ ७.१० ॥  मूल श्लोक
\end{quotation}

\begin{quotation}
बीजं मां सर्व-भूतानां, विद्धि पार्थ सनातनम्‌  ।  

बुद्धिर बुद्धिमताम् अस्मि, तेजस तेजस्विनाम् अहम्‌  ॥ ७.१० ॥  उच्चारण

\noindent\rule{16cm}{0.4pt} 
\end{quotation}


\begin{quotation} 
बलं बलवतां चाहं कामरागविवर्जितम्‌  ।  

धर्माविरुद्धो भूतेषु कामोऽस्मि भरतर्षभ  ॥ ७.११ ॥  मूल श्लोक
\end{quotation}

\begin{quotation}
बलं बल-वतां चाहं, काम राग वि-वर्जितम्‌  ।  

धर्मा विरुद्धो भूतेषु, कामोऽ अस्मि भरतर्षभ  ॥ ७.११ ॥  उच्चारण

\noindent\rule{16cm}{0.4pt} 
\end{quotation}


\begin{quotation} 
ये चैव सात्त्विका भावा राजसास्तामसाश्चये  ।  

मत्त एवेति तान्विद्धि न त्वहं तेषु ते मयि  ॥ ७.१२ ॥  मूल श्लोक
\end{quotation}

\begin{quotation}
ये चैव सात्त्विका भावा, राजसास तामसास चये  ।  

मत्त एवेति तान विद्धि, न त्व अहं तेषु ते मयि  ॥ ७.१२ ॥  उच्चारण

\noindent\rule{16cm}{0.4pt} 
\end{quotation}


\begin{quotation} 
त्रिभिर्गुणमयैर्भावैरेभिः सर्वमिदं जगत्‌  ।  

मोहितं नाभिजानाति मामेभ्यः परमव्ययम्‌  ॥ ७.१३ ॥  मूल श्लोक
\end{quotation}

\begin{quotation}
त्रिभिर गुण-मयैर भावैर, एभिः सर्वम इदं जगत्‌  ।  

मोहितं नाभि जानाति, मामेभ्यः परम अव्ययम्‌  ॥ ७.१३ ॥  उच्चारण

\noindent\rule{16cm}{0.4pt} 
\end{quotation}


\begin{quotation} 
दैवी ह्येषा गुणमयी मम माया दुरत्यया  ।  

मामेव ये प्रपद्यन्ते मायामेतां तरन्ति ते  ॥ ७.१४ ॥  मूल श्लोक
\end{quotation}

\begin{quotation}
दैवी ह्य एषा गुण-मयी, मम माया दुरत-यया  ।  

मामेव ये प्रपद्-यन्ते, मायाम् एतां तरन्ति ते  ॥ ७.१४ ॥  उच्चारण

\noindent\rule{16cm}{0.4pt} 
\end{quotation}


\begin{quotation} 
न मां दुष्कृतिनो मूढाः प्रपद्यन्ते नराधमाः  ।  

माययापहृतज्ञाना आसुरं भावमाश्रिताः  ॥ ७.१५ ॥  मूल श्लोक
\end{quotation}

\begin{quotation}
न मां दुष्कृतिनो मूढाः, प्रपद्-यन्ते नरा-धमाः  ।  

मायय अपहृत ज्ञाना, आसुरं भावम आश्रिताः  ॥ ७.१५ ॥  उच्चारण

\noindent\rule{16cm}{0.4pt} 
\end{quotation}


\begin{quotation} 
चतुर्विधा भजन्ते मां जनाः सुकृतिनोऽर्जुन  ।  

आर्तो जिज्ञासुरर्थार्थी ज्ञानी च भरतर्षभ  ॥ ७.१६ ॥  मूल श्लोक
\end{quotation}

\begin{quotation}
चतुर-विधा भजन्ते मां, जनाः सुकृतिनोऽ अर्जुन  ।  

आर्तो जिज्ञासुर अर्थार्थी, ज्ञानी च भरतर्षभ  ॥ ७.१६ ॥  उच्चारण

\noindent\rule{16cm}{0.4pt} 
\end{quotation}


\begin{quotation} 
तेषां ज्ञानी नित्ययुक्त एकभक्तिर्विशिष्यते  ।  

प्रियो हि ज्ञानिनोऽत्यर्थमहं स च मम प्रियः  ॥ ७.१७ ॥  मूल श्लोक
\end{quotation}

\begin{quotation}
तेषां ज्ञानी नित्य युक्त, एक भक्तिर विशिष्यते  ।  

प्रियो हि ज्ञानिनोऽ अत्य-अर्थम, अहं स च मम प्रियः  ॥ ७.१७ ॥  उच्चारण

\noindent\rule{16cm}{0.4pt} 
\end{quotation}


\begin{quotation} 
उदाराः सर्व एवैते ज्ञानी त्वात्मैव मे मतम्‌  ।  

आस्थितः स हि युक्तात्मा मामेवानुत्तमां गतिम्‌  ॥ ७.१८ ॥  मूल श्लोक
\end{quotation}

\begin{quotation}
उदाराः सर्व एवैते, ज्ञानी त्व आत्मैव मे मतम्‌  ।  

आस्थितः स हि युक्तात्मा, मामेवान उत्तमां गतिम्‌  ॥ ७.१८ ॥  उच्चारण

\noindent\rule{16cm}{0.4pt} 
\end{quotation}


\begin{quotation} 
बहूनां जन्मनामन्ते ज्ञानवान्मां प्रपद्यते  ।  

वासुदेवः सर्वमिति स महात्मा सुदुर्लभः  ॥ ७.१९ ॥  मूल श्लोक
\end{quotation}

\begin{quotation}
बहूनां जन्मनाम् अन्ते, ज्ञानवान माम् प्रपद्यते  ।  

वासुदेवः सर्वम इति, स महात्मा सुदुर्लभः  ॥ ७.१९ ॥  उच्चारण

\noindent\rule{16cm}{0.4pt} 
\end{quotation}


\begin{quotation} 
कामैस्तैस्तैर्हृतज्ञानाः प्रपद्यन्तेऽन्यदेवताः  ।  

तं तं नियममास्थाय प्रकृत्या नियताः स्वया  ॥ ७.२० ॥  मूल श्लोक
\end{quotation}

\begin{quotation}
कामैस तैस-तैर ह्रत-ज्ञानाः, प्रपद्-यन्तेऽ अन्य देवताः  ।  

तं तं नियमं आस्थाय, प्रकृत्या नियताः स्वया  ॥ ७.२० ॥  उच्चारण

\noindent\rule{16cm}{0.4pt} 
\end{quotation}


\begin{quotation} 
यो यो यां यां तनुं भक्तः श्रद्धयार्चितुमिच्छति  ।  

तस्य तस्याचलां श्रद्धां तामेव विदधाम्यहम्‌  ॥ ७.२१ ॥  मूल श्लोक
\end{quotation}

\begin{quotation}
यो यो यां यां तनुं भक्तः, श्रद्धय अर्चितुम ईच्छति  ।  

तस्य तस्या-चलां श्रद्धां, तामेव विदधाम्य अहम्‌  ॥ ७.२१ ॥  उच्चारण

\noindent\rule{16cm}{0.4pt} 
\end{quotation}


\begin{quotation} 
स तया श्रद्धया युक्तस्तस्याराधनमीहते  ।  

लभते च ततः कामान्मयैव विहितान्हि तान्‌  ॥ ७.२२ ॥  मूल श्लोक
\end{quotation}

\begin{quotation}
स तया श्रद्धया युक्तः, तस्य आराधनं इहते  ।  

लभते च ततः कामान, मयैव विहितान हि तान्‌  ॥ ७.२२ ॥  उच्चारण

\noindent\rule{16cm}{0.4pt} 
\end{quotation}


\begin{quotation} 
अन्तवत्तु फलं तेषां तद्भवत्यल्पमेधसाम्‌  ।  

देवान्देवयजो यान्ति मद्भक्ता यान्ति मामपि  ॥ ७.२३ ॥  मूल श्लोक
\end{quotation}

\begin{quotation}
अन्तवत् तु फलं तेषां, तद भवत्य अल्प-मेधसाम्‌  ।  

देवान देव-यजो यान्ति, मद्भक्ता यान्ति माम अपि  ॥ ७.२३ ॥  उच्चारण

\noindent\rule{16cm}{0.4pt} 
\end{quotation}


\begin{quotation} 
अव्यक्तं व्यक्तिमापन्नं मन्यन्ते मामबुद्धयः  ।  

परं भावमजानन्तो ममाव्ययमनुत्तमम्‌  ॥ ७.२४ ॥  मूल श्लोक
\end{quotation}

\begin{quotation}
अव्यक्तं व्यक्तिम आपन्नं, मन्यन्ते माम बुद्धयः  ।  

परं भावम जानन्तो, मम-अव्ययम् अन-उत्तमम्‌  ॥ ७.२४ ॥  उच्चारण

\noindent\rule{16cm}{0.4pt} 
\end{quotation}


\begin{quotation} 
नाहं प्रकाशः सर्वस्य योगमायासमावृतः  ।  

मूढोऽयं नाभिजानाति लोको मामजमव्ययम्‌  ॥ ७.२५ ॥  मूल श्लोक
\end{quotation}

\begin{quotation}
नाहं प्रकाशः सर्वस्य, योग माया समावृतः  ।  

मूढोऽ अयं नाभि-जानाति, लोको माम अजं अव्ययम्‌  ॥ ७.२५ ॥  उच्चारण

\noindent\rule{16cm}{0.4pt} 
\end{quotation}


\begin{quotation} 
वेदाहं समतीतानि वर्तमानानि चार्जुन  ।  

भविष्याणि च भूतानि मां तु वेद न कश्चन  ॥ ७.२६ ॥  मूल श्लोक
\end{quotation}

\begin{quotation}
वेदाहं समती-तानि, वर्तमान-आनि चार्जुन  ।  

भविष्याणि च भूतानि, मां तु वेद न कश्चन  ॥ ७.२६ ॥  उच्चारण

\noindent\rule{16cm}{0.4pt} 
\end{quotation}


\begin{quotation} 
इच्छाद्वेषसमुत्थेन द्वन्द्वमोहेन भारत  ।  

सर्वभूतानि सम्मोहं सर्गे यान्ति परन्तप  ॥ ७.२७ ॥  मूल श्लोक
\end{quotation}

\begin{quotation}
इच्छा द्वेष समुत्थेन, द्वन्द्व मोहेन भारत  ।  

सर्व-भूतानि सम्मोहं, सर्गे यान्ति परन्तप  ॥ ७.२७ ॥  उच्चारण

\noindent\rule{16cm}{0.4pt} 
\end{quotation}


\begin{quotation} 
येषां त्वन्तगतं पापं जनानां पुण्यकर्मणाम्‌  ।  

ते द्वन्द्वमोहनिर्मुक्ता भजन्ते मां दृढव्रताः  ॥ ७.२८ ॥  मूल श्लोक
\end{quotation}

\begin{quotation}
येषां त्व अन्त-गतं पापं, जनानां पुण्य कर्मणाम्‌  ।  

ते द्वन्द्व मोह-निर्मुक्ता, भजन्ते मां दृढ-व्रताः  ॥ ७.२८ ॥  उच्चारण

\noindent\rule{16cm}{0.4pt} 
\end{quotation}


\begin{quotation} 
जरामरणमोक्षाय मामाश्रित्य यतन्ति ये  ।  

ते ब्रह्म तद्विदुः कृत्स्नमध्यात्मं कर्म चाखिलम्‌  ॥ ७.२९ ॥  मूल श्लोक
\end{quotation}

\begin{quotation}
जरा मरण मोक्षाय, माम आश्रित्य यतन्ति ये  ।  

ते ब्रह्म तद विदुः कृत्स्-नम, अध्यात्मं कर्म च अखिलम्‌  ॥ ७.२९ ॥  उच्चारण

\noindent\rule{16cm}{0.4pt} 
\end{quotation}


\begin{quotation} 
साधिभूताधिदैवं मां साधियज्ञं च ये विदुः  ।  

प्रयाणकालेऽपि च मां ते विदुर्युक्तचेतसः  ॥ ७.३० ॥  मूल श्लोक
\end{quotation}

\begin{quotation}
साधि भूताधि दैवं मां, साधि यज्ञं च ये विदुः  ।  

प्रयाण कालेऽ अपि च मां, ते विदुर युक्त चेतसः  ॥ ७.३० ॥  उच्चारण

\noindent\rule{16cm}{0.4pt} 
\end{quotation}

\begin{center} ***** \end{center}

\begin{quotation}  

ॐ तत् सद इति श्री मद्-भगवद्-गीतास उपनिषत्सु ब्रह्म विद्यायां योगशास्त्रे श्री कृष्णार्जुन संवादे ज्ञानविज्ञानयोगो नाम सप्तमोऽ अध्यायः  ॥  ७ ॥ 

\end{quotation}
