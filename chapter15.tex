\chapter{\sanskrit पुरुषोत्तमयोग} 
\paragraph{\sanskrit श्रीभगवानुवाच}
\begin{quotation} 
ऊर्ध्वमूलमधः शाखमश्वत्थं प्राहुरव्ययम्‌  ।  

छन्दांसि यस्य पर्णानि यस्तं वेद स वेदवित्‌  ॥ १५.१ ॥  मूल श्लोक
\end{quotation}

\begin{quotation}

ऊर्ध्व मूलम अधः शाखम, अश्वत्थं प्राहुर अ-व्ययम्‌  ।  

छन्दांसि यस्य पर्णानि, यस्तं वेद स वेदवित्‌  ॥ १५.१ ॥  उच्चारण

\noindent\rule{16cm}{0.4pt} 
\end{quotation}


\begin{quotation}

अधश्चोर्ध्वं प्रसृतास्तस्य शाखा गुणप्रवृद्धा विषयप्रवालाः  ।  

अधश्च मूलान्यनुसन्ततानि कर्मानुबन्धीनि मनुष्यलोके  ॥ १५.२ ॥  मूल श्लोक
\end{quotation}

\begin{quotation}

अधश् च ऊर्ध्वं प्र-सृतास तस्य शाखा 
गुण प्रवृद्धा विषय प्रवालाः  ।  

अधश् च मूलान्य अनु-सन्त-तानि 
कर्मानु बन्धीनि मनुष्य लोके  ॥ १५.२ ॥  उच्चारण

\noindent\rule{16cm}{0.4pt} 
\end{quotation}


\begin{quotation}

न रूपमस्येह तथोपलभ्यते नान्तो न चादिर्न च सम्प्रतिष्ठा  ।  

अश्वत्थमेनं सुविरूढमूल मसङ्‍गशस्त्रेण दृढेन छित्त्वा  ॥ १५.३ ॥  मूल श्लोक
\end{quotation}

\begin{quotation}

न रूपम अस्येह तथो-उप-लभ्यते 
नान्तो न चादिर न च सम्प्रतिष्ठा  ।  

अश्वत्थम इनं सु-विरूढ-मूलम
असङ्‍ग शस्त्रेण दृढेन छित्त्वा  ॥ १५.३ ॥  उच्चारण

\noindent\rule{16cm}{0.4pt} 
\end{quotation}


\begin{quotation}

ततः पदं तत्परिमार्गितव्यं यस्मिन्गता न निवर्तन्ति भूयः  ।  

तमेव चाद्यं पुरुषं प्रपद्ये यतः प्रवृत्तिः प्रसृता पुराणी  ॥ १५.४ ॥  मूल श्लोक
\end{quotation}

\begin{quotation}

ततः पदं तत परि-मार्गि-तव्यं,
यस्मिन् गता न नि-वर्तन्ति भूयः  ।  

तमेव चाद्यं पुरुषं प्रपद्ये यतः 
प्रवृत्तिः प्र-सृता पुराणी  ॥ १५.४ ॥  उच्चारण

\noindent\rule{16cm}{0.4pt} 
\end{quotation}


\begin{quotation}
निर्मानमोहा जितसङ्गदोषाअध्यात्मनित्या विनिवृत्तकामाः  ।  

द्वन्द्वैर्विमुक्ताः सुखदुःखसञ्ज्ञैर्गच्छन्त्यमूढाः पदमव्ययं तत्‌  ॥ १५.५ ॥  मूल श्लोक
\end{quotation}

\begin{quotation}

निर्मान मोहा जित-सङ्ग-दोषा
अध्यात्म-नित्या वि-निवृत्त-कामाः  ।  

द्वन्द्वैर विमुक्ताः सुख दुःख सञ्ज्ञैर 
गच्छन्त्य मूढाः पदम अ-व्ययं तत्‌  ॥ १५.५ ॥  उच्चारण

\noindent\rule{16cm}{0.4pt} 
\end{quotation}


\begin{quotation}

न तद्भासयते सूर्यो न शशाङ्को न पावकः  ।  

यद्गत्वा न निवर्तन्ते तद्धाम परमं मम  ॥ १५.६ ॥  मूल श्लोक
\end{quotation}

\begin{quotation}

न तद भासयते सूर्यो, न शशाङ्को न पावकः  ।  

यद् गत्वा न नि-वर्तन्ते, तद धाम परमं मम  ॥ १५.६ ॥  उच्चारण

\noindent\rule{16cm}{0.4pt} 
\end{quotation}


\begin{quotation}

ममैवांशो जीवलोके जीवभूतः सनातनः  ।  

मनः षष्ठानीन्द्रियाणि प्रकृतिस्थानि कर्षति  ॥ १५.७ ॥  मूल श्लोक
\end{quotation}

\begin{quotation}

मम एैवांशो जीव लोके, जीव भूतः सनातनः  ।  

मनः षष्ठान-इन्द्रियाणि, प्रकृति स्थानि कर्षति  ॥ १५.७ ॥  उच्चारण

\noindent\rule{16cm}{0.4pt} 
\end{quotation}


\begin{quotation}

शरीरं यदवाप्नोति यच्चाप्युत्क्रामतीश्वरः  ।  

गृहीत्वैतानि संयाति वायुर्गन्धानिवाशयात्‌  ॥ १५.८ ॥  मूल श्लोक
\end{quotation}

\begin{quotation}

शरीरं यद अवाप्-नोति, यच चप्य उत-क्रामति ईश्वरः  ।  

गृहीत्वै तानि संयाति, वायुर गन्धान इव आशयात्‌  ॥ १५.८ ॥  उच्चारण

\noindent\rule{16cm}{0.4pt} 
\end{quotation}


\begin{quotation}

श्रोत्रं चक्षुः स्पर्शनं च रसनं घ्राणमेव च  ।  

अधिष्ठाय मनश्चायं विषयानुपसेवते ॥ १५.९ ॥  मूल श्लोक
\end{quotation}

\begin{quotation}

श्रोत्रं चक्षुः स्पर्शनं च, रसनं घ्राणम एव च  ।  

अधिष्ठाय मनश् चायं, विषयान उप-सेवते  ॥ १५.९ ॥  उच्चारण

\noindent\rule{16cm}{0.4pt} 
\end{quotation}


\begin{quotation}

उत्क्रामन्तं स्थितं वापि भुञ्जानं वा गुणान्वितम्‌  ।  

विमूढा नानुपश्यन्ति पश्यन्ति ज्ञानचक्षुषः  ॥ १५.१० ॥  मूल श्लोक
\end{quotation}

\begin{quotation}

उत्क्राम अन्तं स्थितं वापि, भुञ्जानं वा गुणा-अन्वितम्‌   ।  

विमूढा नानु पश्यन्ति ,पश्यन्ति ज्ञान चक्षुषः  ॥ १५.१० ॥  उच्चारण

\noindent\rule{16cm}{0.4pt} 
\end{quotation}


\begin{quotation}

यतन्तो योगिनश्चैनं पश्यन्त्यात्मन्यवस्थितम्‌  ।  

यतन्तोऽप्यकृतात्मानो नैनं पश्यन्त्यचेतसः  ॥ १५.११ ॥  मूल श्लोक
\end{quotation}

\begin{quotation}

यतन्तो योगिनश् चैनं, पश्यन्त्य आत्मन्य अवस्थितम्‌  ।  

यतन्तोऽ अप्य अकृता-आत्मानो, नैनं पश्यन्त्य चेतसः  ॥ १५.११ ॥  उच्चारण

\noindent\rule{16cm}{0.4pt} 
\end{quotation}


\begin{quotation}

यदादित्यगतं तेजो जगद्भासयतेऽखिलम्‌  ।  

यच्चन्द्रमसि यच्चाग्नौ तत्तेजो विद्धि मामकम्‌  ॥ १५.१२ ॥  मूल श्लोक
\end{quotation}

\begin{quotation}

यद आदित्य गतं तेजो, जगद  भासयतेऽ खिलम्‌  ।  

यच् चन्द्रमसि यच् चाग्नौ, तत्तेजो विद्धि माम-कम्‌  ॥ १५.१२ ॥  उच्चारण

\noindent\rule{16cm}{0.4pt} 
\end{quotation}


\begin{quotation}

गामाविश्य च भूतानि धारयाम्यहमोजसा  ।  

पुष्णामि चौषधीः सर्वाः सोमो भूत्वा रसात्मकः  ॥ १५.१३ ॥  मूल श्लोक
\end{quotation}

\begin{quotation}

गामा अविश्य च भूतानि, धारयाम्य अहम ओजसा  ।  

पुष्णामि चौषधीः सर्वाः, सोमो भूत्वा रसात्मकः  ॥ १५.१३ ॥  उच्चारण

\noindent\rule{16cm}{0.4pt} 
\end{quotation}


\begin{quotation}

अहं वैश्वानरो भूत्वा प्राणिनां देहमाश्रितः  ।  

प्राणापानसमायुक्तः पचाम्यन्नं चतुर्विधम्‌  ॥ १५.१४ ॥  मूल श्लोक
\end{quotation}

\begin{quotation}

अहं वैश्वानरो भूत्वा, प्राणिनां देहम आश्रितः  ।  

प्राणा पान समायुक्तः, पचाम्य अन्नम चतुर्विधम्‌  ॥ १५.१४ ॥  उच्चारण

\noindent\rule{16cm}{0.4pt} 
\end{quotation}


\begin{quotation}

सर्वस्य चाहं हृदि सन्निविष्टोमत्तः स्मृतिर्ज्ञानमपोहनं च  ।  

वेदैश्च सर्वैरहमेव वेद्योवेदान्तकृद्वेदविदेव चाहम्‌  ॥ १५.१५ ॥  मूल श्लोक
\end{quotation}

\begin{quotation}

सर्वस्य चाहं हृदि सन्नि विष्टो
मत्तः स्मृतिर ज्ञानम अपोहनं च  ।  

वेदैश् च सर्वैर अहम एव वेद्यो 
वेदान्त-कृद वेद-विद एव चाहम्‌  ॥ १५.१५ ॥  उच्चारण

\noindent\rule{16cm}{0.4pt} 
\end{quotation}


\begin{quotation}

द्वाविमौ पुरुषौ लोके क्षरश्चाक्षर एव च  ।  

क्षरः सर्वाणि भूतानि कूटस्थोऽक्षर उच्यते  ॥ १५.१६ ॥  मूल श्लोक
\end{quotation}

\begin{quotation}

द्वाविमौ पुरुषौ लोके, क्षरश् चाक्षर एव च  ।  

क्षरः सर्वाणि भूतानि, कूटस्थोऽ अक्षर उच्यते  ॥ १५.१६ ॥  उच्चारण

\noindent\rule{16cm}{0.4pt} 
\end{quotation}


\begin{quotation}

उत्तमः पुरुषस्त्वन्यः परमात्मेत्युदाहृतः  ।  

यो लोकत्रयमाविश्य बिभर्त्यव्यय ईश्वरः  ॥ १५.१७ ॥  मूल श्लोक
\end{quotation}

\begin{quotation}

उत्तमः पुरुषः त्व अन्यः, परमात्म एत्य उदाहृतः  ।  

यो लोकत्रयम आविश्य, बिभर्-त्य अव्यय ईश्वरः  ॥ १५.१७ ॥  उच्चारण

\noindent\rule{16cm}{0.4pt} 
\end{quotation}


\begin{quotation}

यस्मात्क्षरमतीतोऽहमक्षरादपि चोत्तमः  ।  

अतोऽस्मि लोके वेदे च प्रथितः पुरुषोत्तमः  ॥ १५.१८ ॥  मूल श्लोक
\end{quotation}

\begin{quotation}

यस्मात क्षरम अतीतोऽ अहम, अक्षराद अपि चोत्तमः  ।  

अतोऽ अस्मि लोके वेदे च, प्रथितः पुरुषोत्तमः  ॥ १५.१८ ॥  उच्चारण

\noindent\rule{16cm}{0.4pt} 
\end{quotation}


\begin{quotation}

यो मामेवमसम्मूढो जानाति पुरुषोत्तमम्‌  ।  

स सर्वविद्भजति मां सर्वभावेन भारत  ॥ १५.१९ ॥  मूल श्लोक
\end{quotation}

\begin{quotation}

यो माम एवम सम्मूढो, जानाति पुरुषोत्तमम्‌  ।  

स सर्व-विद् भजति मां, सर्व भावेन भारत  ॥ १५.१९ ॥  उच्चारण

\noindent\rule{16cm}{0.4pt} 
\end{quotation}


\begin{quotation}

इति गुह्यतमं शास्त्रमिदमुक्तं मयानघ  ।  

एतद्‍बुद्ध्वा बुद्धिमान्स्यात्कृतकृत्यश्च भारत  ॥ १५.२० ॥  मूल श्लोक
\end{quotation}

\begin{quotation}

इति गुह्यत-मम् शास्त्रम, इदम उक्तं मया-नघ  ।  

एतद्‍ बुद्ध्वा बुद्धिमान स्यात, कृत-कृत्यश् च भारत  ॥ १५.२० ॥  उच्चारण

\noindent\rule{16cm}{0.4pt} 
\end{quotation}
\begin{center} ***** \end{center}


\begin{quotation}


ॐ तत् सद इति श्री मद्-भगवद्-गीतास उपनिषत्सु ब्रह्म विद्यायां योगशास्त्रे श्री कृष्णार्जुन संवादे  पुरुषोत्तमयोगो नाम पञ्चदशोऽ अध्यायः  ॥  १५  ॥ 
\end{quotation} 