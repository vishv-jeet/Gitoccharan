\chapter{\sanskrit क्षेत्रक्षेत्रज्ञविभागयोग} 
\paragraph{\sanskrit अर्जुन उवाच}
\begin{quotation} 
प्रकृतिं पुरुषं चैव क्षेत्रं क्षेत्रज्ञमेव च  ।  

एतद्वेदितुमिच्छामि ज्ञानं ज्ञेयं च केशव  ॥ १३.१ ॥  मूल श्लोक
\end{quotation}

\begin{quotation}

प्रकृतिं पुरुषं चैव, क्षेत्रं क्षेत्रज्ञम एव च  ।  

एतद् वेदितुम इच्छामि, ज्ञानं ज्ञेयं च केशव  ॥ १३.१ ॥  उच्चारण

\noindent\rule{16cm}{0.4pt} 
\end{quotation}




\paragraph{\sanskrit श्रीभगवानुवाच}
\begin{quotation} 
इदं शरीरं कौन्तेय क्षेत्रमित्यभिधीयते ।  

एतद्यो वेत्ति तं प्राहुः क्षेत्रज्ञ इति तद्विदः ॥ १३.२ ॥  मूल श्लोक
\end{quotation}

\begin{quotation}

इदं शरीरं कौन्तेय, क्षेत्रम इत्य अभि-धीयते ।  

एतद् यो वेत्ति तं प्राहुः, क्षेत्रज्ञ इति तद्-विदः ॥ १३.२ ॥  उच्चारण

\noindent\rule{16cm}{0.4pt} 
\end{quotation}


\begin{quotation}

क्षेत्रज्ञं चापि मां विद्धि सर्वक्षेत्रेषु भारत ।  

क्षेत्रक्षेत्रज्ञयोर्ज्ञानं यत्तज्ज्ञानं मतं मम  ॥ १३.३ ॥  मूल श्लोक
\end{quotation}

\begin{quotation}

क्षेत्रज्ञं चापि मां विद्धि, सर्व क्षेत्रेषु भारत ।  

क्षेत्र क्षेत्रज्ञ-योर ज्ञानं, यत् तज ज्ञानं मतं मम  ॥ १३.३ ॥  उच्चारण

\noindent\rule{16cm}{0.4pt} 
\end{quotation}


\begin{quotation}

तत्क्षेत्रं यच्च यादृक्च यद्विकारि यतश्च यत्‌ ।  

स च यो यत्प्रभावश्च तत्समासेन मे श्रृणु  ॥ १३.४ ॥  मूल श्लोक
\end{quotation}

\begin{quotation}

तत क्षेत्रं यच् च यादृक् च, यद्-विकारि यतश् च यत्‌ ।  

स च यो यत प्रभावश् च, तत समासेन मे श्रृणु  ॥ १३.४ ॥  उच्चारण

\noindent\rule{16cm}{0.4pt} 
\end{quotation}


\begin{quotation}

ऋषिभिर्बहुधा गीतं छन्दोभिर्विविधैः पृथक्‌  ।  

ब्रह्मसूत्रपदैश्चैव हेतुमद्भिर्विनिश्चितैः  ॥ १३.५ ॥  मूल श्लोक
\end{quotation}

\begin{quotation}
ऋषिभिर बहुधा गीतं, छन्दोभिर विविधैः पृथक्‌  ।  

ब्रह्म सूत्र पदैश् चैव, हेतु-मद्-भिर वि-निश्चितैः  ॥ १३.५ ॥  उच्चारण

\noindent\rule{16cm}{0.4pt} 
\end{quotation}


\begin{quotation}

महाभूतान्यहङ्‍कारो बुद्धिरव्यक्तमेव च  ।  

इन्द्रियाणि दशैकं च पञ्च चेन्द्रियगोचराः  ॥ १३.६ ॥  मूल श्लोक
\end{quotation}

\begin{quotation}

महा भूतान्य अहङ्‍कारो, बुद्धिर अव्यक्तम एव च  ।  

इन्द्रियाणि दशैकं च, पञ्च चेन्द्रिय-गोचराः  ॥ १३.६ ॥  उच्चारण

\noindent\rule{16cm}{0.4pt} 
\end{quotation}


\begin{quotation}

इच्छा द्वेषः सुखं दुःखं सङ्‍घातश्चेतना धृतिः  ।  

एतत्क्षेत्रं समासेन सविकारमुदाहृतम्‌  ॥ १३.७ ॥  मूल श्लोक
\end{quotation}

\begin{quotation}

इच्छा द्वेषः सुखं दुःखं, सङ्‍घातश् चेतना धृतिः  ।  

एतत क्षेत्रं समासेन, सविकारम उदाहृतम्‌  ॥ १३.७ ॥  उच्चारण

\noindent\rule{16cm}{0.4pt} 
\end{quotation}


\begin{quotation}

अमानित्वमदम्भित्वमहिंसा क्षान्तिरार्जवम्‌  ।  

आचार्योपासनं शौचं स्थैर्यमात्मविनिग्रहः  ॥ १३.८ ॥  मूल श्लोक
\end{quotation}

\begin{quotation}

अमानि-त्वम अदम्भि-त्वम, अहिंसा क्षान्तिर आर्जवम्‌  ।  

आचार्यो-पासनं शौचं, स्थैर्यम आत्म-विनिग्रहः  ॥ १३.८ ॥  उच्चारण

\noindent\rule{16cm}{0.4pt} 
\end{quotation}


\begin{quotation}

इन्द्रियार्थेषु वैराग्यमनहङ्‍कार एव च  ।  

जन्ममृत्युजराव्याधिदुःखदोषानुदर्शनम्‌  ॥ १३.९ ॥  मूल श्लोक
\end{quotation}

\begin{quotation}

इन्द्रिय-आर्थेषु वैराग्यम, अन-अहंकार एव च  ।  

जन्म मृत्यु जरा व्याधि, दुःख दोषा अनु-दर्शनम्‌  ॥ १३.९ ॥  उच्चारण

\noindent\rule{16cm}{0.4pt} 
\end{quotation}


\begin{quotation}

असक्तिरनभिष्वङ्‍ग: पुत्रदारगृहादिषु  ।  

नित्यं च समचित्तत्वमिष्टानिष्टोपपत्तिषु  ॥ १३.१० ॥  मूल श्लोक
\end{quotation}

\begin{quotation}

असक्तिर अन-भिष्वंग:, पुत्र-दार गृहादिषु  ।  

नित्यं च समचित्-तत्वम,  इष्टा-अनिष्टो उप-पत्तिषु  ॥ १३.१० ॥  उच्चारण

\noindent\rule{16cm}{0.4pt} 
\end{quotation}


\begin{quotation}

मयि चानन्ययोगेन भक्तिरव्यभिचारिणी  ।  

विविक्तदेशसेवित्वमरतिर्जनसंसदि  ॥ १३.११ ॥  मूल श्लोक
\end{quotation}

\begin{quotation}
मयि चानन्य-योगेन, भक्तिर अव्यभि-चारिणी  ।  

विविक्त देश सेवित्वम, अरतिर जन-संसदि  ॥ १३.११ ॥  उच्चारण

\noindent\rule{16cm}{0.4pt} 
\end{quotation}


\begin{quotation}

अध्यात्मज्ञाननित्यत्वं तत्वज्ञानार्थदर्शनम्‌  ।  

एतज्ज्ञानमिति प्रोक्तमज्ञानं यदतोऽन्यथा  ॥ १३.१२ ॥  मूल श्लोक
\end{quotation}

\begin{quotation}

अध्यात्म ज्ञान नित्यत्वं, तत्व ज्ञानार्थ दर्शनम्‌  ।  

एतज ज्ञानम इति प्रोक्तम, अज्ञानं यद अतोऽ अन्यथा  ॥ १३.१२ ॥  उच्चारण

\noindent\rule{16cm}{0.4pt} 
\end{quotation}


\begin{quotation}

ज्ञेयं यत्तत्प्रवक्ष्यामि यज्ज्ञात्वाऽमृतमश्नुते ।  

अनादिमत्परं ब्रह्म न सत्तन्नासदुच्यते  ॥ १३.१३ ॥  मूल श्लोक
\end{quotation}

\begin{quotation}

ज्ञेयं यत् तत् प्रवक्ष्यामि, यज ज्ञात्वा-अमृतम अश्-नुते  ।  

अनादि मत्परं ब्रह्म, न सत् तन् नासद उच्यते  ॥ १३.१३ ॥  उच्चारण

\noindent\rule{16cm}{0.4pt} 
\end{quotation}


\begin{quotation}

सर्वतः पाणिपादं तत्सर्वतोऽक्षिशिरोमुखम्‌  ।  

सर्वतः श्रुतिमल्लोके सर्वमावृत्य तिष्ठति  ॥ १३.१४ ॥  मूल श्लोक
\end{quotation}

\begin{quotation}

सर्वतः पाणि पादं तत, सर्वतोऽ-अक्षिशिरो-मुखम्‌  ।  

सर्वतः श्रुतिमल लोके, सर्वम आवृत्य तिष्ठति  ॥ १३.१४ ॥  उच्चारण

\noindent\rule{16cm}{0.4pt} 
\end{quotation}


\begin{quotation}

सर्वेन्द्रियगुणाभासं सर्वेन्द्रियविवर्जितम्‌  ।  

असक्तं सर्वभृच्चैव निर्गुणं गुणभोक्तृ च  ॥ १३.१५ ॥  मूल श्लोक
\end{quotation}

\begin{quotation}

सर्व-एन्द्रिय गुणा-भासं, सर्व-एन्द्रिय वि-वर्जितम्‌  ।  

असक्तं सर्व-भृच चैव, निर्गुणं गुण भोक्तृ च  ॥ १३.१५ ॥  उच्चारण

\noindent\rule{16cm}{0.4pt} 
\end{quotation}


\begin{quotation}

बहिरन्तश्च भूतानामचरं चरमेव च  ।  

सूक्ष्मत्वात्तदविज्ञयं दूरस्थं चान्तिके च तत्‌  ॥ १३.१६ ॥  मूल श्लोक
\end{quotation}

\begin{quotation}

बहिर अन्तश् च भूतानाम, अचरं चरमेव च  ।  

सूक्ष्म-त्वात तद अविज्ञयं, दूरस्थं चान्तिके च तत्‌  ॥ १३.१६ ॥  उच्चारण

\noindent\rule{16cm}{0.4pt} 
\end{quotation}


\begin{quotation}

अविभक्तं च भूतेषु विभक्तमिव च स्थितम्‌  ।  

भूतभर्तृ च तज्ज्ञेयं ग्रसिष्णु प्रभविष्णु च  ॥ १३.१७ ॥  मूल श्लोक
\end{quotation}

\begin{quotation}
अ-विभक्तं च भूतेषु, विभक्तम एव च स्थितम्‌  ।  

भूत-भर्त-ऋ च तज ज्ञेयं, ग्रसिष्णु प्रभ-विष्णु च  ॥ १३.१७ ॥  उच्चारण

\noindent\rule{16cm}{0.4pt} 
\end{quotation}


\begin{quotation}

ज्योतिषामपि तज्ज्योतिस्तमसः परमुच्यते  ।  

ज्ञानं ज्ञेयं ज्ञानगम्यं हृदि सर्वस्य विष्ठितम्‌  ॥ १३.१८ ॥  मूल श्लोक
\end{quotation}

\begin{quotation}

ज्योतिषाम अपि तज ज्योतिस,  तमसः परम उच्यते  ।  

ज्ञानं ज्ञेयं ज्ञान गम्यं, हृदि सर्वस्य विष्ठि-तम्‌  ॥ १३.१८ ॥  उच्चारण

\noindent\rule{16cm}{0.4pt} 
\end{quotation}


\begin{quotation}

इति क्षेत्रं तथा ज्ञानं ज्ञेयं चोक्तं समासतः  ।  

मद्भक्त एतद्विज्ञाय मद्भावायोपपद्यते  ॥ १३.१९ ॥  मूल श्लोक
\end{quotation}

\begin{quotation}

इति क्षेत्रं तथा ज्ञानं, ज्ञेयं चोक्तं समा-सतः  ।  

मद्-भक्त एतद् विज्ञाय, मद्-भावायो उपपद्यते  ॥ १३.१९ ॥  उच्चारण

\noindent\rule{16cm}{0.4pt} 
\end{quotation}


\begin{quotation}

प्रकृतिं पुरुषं चैव विद्ध्‌यनादी उभावपि  ।  

विकारांश्च गुणांश्चैव विद्धि प्रकृतिसम्भवान्‌  ॥ १३.२० ॥  मूल श्लोक
\end{quotation}

\begin{quotation}

प्रकृतिं पुरुषं चैव, विद्ध्‌य अनादी उभाव अपि  ।  

विकारांश् च गुणांश् चैव, विद्धि प्रकृति-सम्भवान्‌  ॥ १३.२० ॥  उच्चारण

\noindent\rule{16cm}{0.4pt} 
\end{quotation}


\begin{quotation}

कार्यकरणकर्तृत्वे हेतुः प्रकृतिरुच्यते  ।  

पुरुषः सुखदुःखानां भोक्तृत्वे हेतुरुच्यते  ॥ १३.२१ ॥  मूल श्लोक
\end{quotation}

\begin{quotation}

कार्य करण कर्-तृत्वे, हेतुः प्रकृतिर उच्यते  ।  

पुरुषः सुख दुःखानां, भोक-तृत्वे हेतुर उच्यते  ॥ १३.२१ ॥  उच्चारण

\noindent\rule{16cm}{0.4pt} 
\end{quotation}


\begin{quotation}

पुरुषः प्रकृतिस्थो हि भुङ्‍क्ते प्रकृतिजान्गुणान्‌  ।  

कारणं गुणसंगोऽस्य सदसद्योनिजन्मसु  ॥ १३.२२ ॥  मूल श्लोक
\end{quotation}

\begin{quotation}

पुरुषः प्रकृति-स्थो हि, भुंक्ते प्रकृति-जान-गुणान्‌  ।  

कारणं गुण संगोऽ अस्य, सद असद् योनि जन्मसु  ॥ १३.२२ ॥  उच्चारण

\noindent\rule{16cm}{0.4pt} 
\end{quotation}


\begin{quotation}

उपद्रष्टानुमन्ता च भर्ता भोक्ता महेश्वरः  ।  

परमात्मेति चाप्युक्तो देहेऽस्मिन्पुरुषः परः  ॥ १३.२३ ॥  मूल श्लोक
\end{quotation}

\begin{quotation}
उप-द्रष्टा-अनुमन्ता च, भर्ता भोक्ता महेश्वरः  ।  

परमात्मेति चाप्य उक्तो, देहेऽ अस्मिन पुरुषः परः  ॥ १३.२३ ॥  उच्चारण

\noindent\rule{16cm}{0.4pt} 
\end{quotation}


\begin{quotation}

य एवं वेत्ति पुरुषं प्रकृतिं च गुणैः सह  ।  

सर्वथा वर्तमानोऽपि न स भूयोऽभिजायते  ॥ १३.२४ ॥  मूल श्लोक
\end{quotation}

\begin{quotation}

य एवं वेत्ति पुरुषं, प्रकृतिं च गुणैः सह  ।  

सर्वथा वर्तमानोऽ अपि, न स भूयोऽ अभिजायते  ॥ १३.२४ ॥  उच्चारण

\noindent\rule{16cm}{0.4pt} 
\end{quotation}


\begin{quotation}

ध्यानेनात्मनि पश्यन्ति केचिदात्मानमात्मना  ।  

अन्ये साङ्‍ख्येन योगेन कर्मयोगेन चापरे  ॥ १३.२५ ॥  मूल श्लोक
\end{quotation}

\begin{quotation}

ध्यानेन आत्मनि पश्यन्ति, केचिद आत्मानम आत्मना  ।  

अन्ये साङ्‍ख्येन योगेन, कर्म-योगेन चापरे  ॥ १३.२५ ॥  उच्चारण

\noindent\rule{16cm}{0.4pt} 
\end{quotation}


\begin{quotation}

अन्ये त्वेवमजानन्तः श्रुत्वान्येभ्य उपासते  ।  

तेऽपि चातितरन्त्येव मृत्युं श्रुतिपरायणाः  ॥ १३.२६ ॥  मूल श्लोक
\end{quotation}

\begin{quotation}

अन्ये त्वेवम अजानन्तः, श्रुत्वा अन्येभ्य उपासते  ।  

तेऽ अपि च आति-तरन्त्य एव, मृत्युं श्रुति-परायणाः  ॥ १३.२६ ॥  उच्चारण

\noindent\rule{16cm}{0.4pt} 
\end{quotation}


\begin{quotation}

यावत्सञ्जायते किञ्चित्सत्त्वं स्थावरजङ्गमम्‌  ।  

क्षेत्रक्षेत्रज्ञसंयोगात्तद्विद्धि भरतर्षभ  ॥ १३.२७ ॥  मूल श्लोक
\end{quotation}

\begin{quotation}

यावत सञ्जायते किञ्चित्, सत्त्वं स्थावर जङ्गमम्‌  ।  

क्षेत्र क्षेत्रज्ञ संयोगात्, तद् विद्धि भरतर्षभ  ॥ १३.२७ ॥  उच्चारण

\noindent\rule{16cm}{0.4pt} 
\end{quotation}


\begin{quotation}

समं सर्वेषु भूतेषु तिष्ठन्तं परमेश्वरम्‌  ।  

विनश्यत्स्वविनश्यन्तं यः पश्यति स पश्यति  ॥ १३.२८ ॥  मूल श्लोक
\end{quotation}

\begin{quotation}
समं सर्वेषु भूतेषु, तिष्ठन्तं परमेश्वरम्‌  ।  

विनश्यत-स्व विनश्यन्तं, यः पश्यति स पश्यति  ॥ १३.२८ ॥  उच्चारण

\noindent\rule{16cm}{0.4pt} 
\end{quotation}


\begin{quotation}

समं पश्यन्हि सर्वत्र समवस्थितमीश्वरम्‌  ।  

न हिनस्त्यात्मनात्मानं ततो याति परां गतिम्‌  ॥ १३.२९ ॥  मूल श्लोक
\end{quotation}

\begin{quotation}

समं पश्यन् ही सर्वत्र, समव-स्थितम ईश्वरम्‌  ।  

न हिनस्त्-य आत्मना आत्मानं, ततो याति परां गतिम्‌  ॥ १३.२९ ॥  उच्चारण

\noindent\rule{16cm}{0.4pt} 
\end{quotation}


\begin{quotation}

प्रकृत्यैव च कर्माणि क्रियमाणानि सर्वशः  ।  

यः पश्यति तथात्मानमकर्तारं स पश्यति  ॥ १३.३० ॥  मूल श्लोक
\end{quotation}

\begin{quotation}

प्रकृत्य एैव च कर्माणि, क्रियम-आणानि सर्वशः  ।  

यः पश्यति तथात्मानम, अकर्तारं स पश्यति  ॥ १३.३० ॥  उच्चारण

\noindent\rule{16cm}{0.4pt} 
\end{quotation}


\begin{quotation}

यदा भूतपृथग्भावमेकस्थमनुपश्यति  ।  

तत एव च विस्तारं ब्रह्म सम्पद्यते तदा  ॥ १३.३१ ॥  मूल श्लोक
\end{quotation}

\begin{quotation}

यदा भूत पृथग् भावम, एक-स्थम अनु-पश्यति  ।  

तत एव च विस्तारं, ब्रह्म सम्पद्-यते तदा  ॥ १३.३१ ॥  उच्चारण

\noindent\rule{16cm}{0.4pt} 
\end{quotation}


\begin{quotation}

अनादित्वान्निर्गुणत्वात्परमात्मायमव्ययः  ।  

शरीरस्थोऽपि कौन्तेय न करोति न लिप्यते  ॥ १३.३२ ॥  मूल श्लोक
\end{quotation}

\begin{quotation}

अनादित्-वान निर्गुणत्-वात्,  परमात्मायम अव्ययः  ।  

शरीरस्थोऽ अपि कौन्तेय, न करोति न लिप्यते  ॥ १३.३२ ॥  उच्चारण

\noindent\rule{16cm}{0.4pt} 
\end{quotation}


\begin{quotation}

यथा सर्वगतं सौक्ष्म्यादाकाशं नोपलिप्यते  ।  

सर्वत्रावस्थितो देहे तथात्मा नोपलिप्यते  ॥ १३.३३ ॥  मूल श्लोक
\end{quotation}

\begin{quotation}

यथा सर्व गतं सौक्ष्म्याद, आकाशं ना उप-लिप्यते  ।  

सर्वत्र आवस्थितो देहे, तथात्मा ना उप-लिप्यते  ॥ १३.३३ ॥  उच्चारण

\noindent\rule{16cm}{0.4pt} 
\end{quotation}


\begin{quotation}

यथा प्रकाशयत्येकः कृत्स्नं लोकमिमं रविः  ।  

क्षेत्रं क्षेत्री तथा कृत्स्नं प्रकाशयति भारत  ॥ १३.३४ ॥  मूल श्लोक
\end{quotation}

\begin{quotation}
यथा प्रकाशयत्य एकः, कृत्स्-नम् लोकम इमं रविः ।  

क्षेत्रं क्षेत्री तथा कृत्स्-नम्, प्रकाशयति भारत  ॥ १३.३४ ॥  उच्चारण

\noindent\rule{16cm}{0.4pt} 
\end{quotation}


\begin{quotation}

क्षेत्रक्षेत्रज्ञयोरेवमन्तरं ज्ञानचक्षुषा  ।  

भूतप्रकृतिमोक्षं च ये विदुर्यान्ति ते परम्‌  ॥ १३.३५ ॥  मूल श्लोक
\end{quotation}

\begin{quotation}

क्षेत्र क्षेत्र-ज्ञयोर एवम, अन्तरं ज्ञान चक्षुषा  ।  

भूत प्रकृति मोक्षं च, ये विदुर् यान्ति ते परम्‌  ॥ १३.३५ ॥  उच्चारण

\noindent\rule{16cm}{0.4pt} 
\end{quotation}




\begin{center} ***** \end{center}

\begin{quotation}

ॐ तत् सद इति श्री मद्-भगवद्-गीतास उपनिषत्सु ब्रह्म विद्यायां योगशास्त्रे श्री कृष्णार्जुन संवादे  क्षेत्रक्षेत्रज्ञविभागयोगो नाम त्रयोदशोऽ अध्यायः  ॥  १३  ॥ 

\end{quotation} 
