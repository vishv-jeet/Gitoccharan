\chapter{\sanskrit राजविद्या राजगुह्य योग}
\paragraph{\sanskrit श्रीभगवानुवाच}
\begin{quotation} 

इदं तु ते गुह्यतमं प्रवक्ष्याम्यनसूयवे  ।  

ज्ञानं विज्ञानसहितं यज्ज्ञात्वा मोक्ष्यसेऽशुभात्‌  ॥ ९.१ ॥  मूल श्लोक
\end{quotation}

\begin{quotation}

इदं तु ते गुह्य-तमं, प्रवक्ष्-याम्य अनसूयवे  ।  

ज्ञानं विज्ञान सहितं, यज्-ज्ञात्वा मोक्ष्यसेऽ अशुभात्‌  ॥ ९.१ ॥  उच्चारण

\noindent\rule{16cm}{0.4pt} 
\end{quotation}


\begin{quotation} 

राजविद्या राजगुह्यं पवित्रमिदमुत्तमम्‌  ।  

प्रत्यक्षावगमं धर्म्यं सुसुखं कर्तुमव्ययम्‌  ॥ ९.२ ॥  मूल श्लोक
\end{quotation}

\begin{quotation}

राजविद्या राज-गुह्यं, पवित्रम इदम उत्तमम्‌  ।  

प्रत्यक्ष आवगमं धर्म्यं, सु-सुखं कर्तुम अ-व्ययम्‌  ॥ ९.२ ॥  उच्चारण

\noindent\rule{16cm}{0.4pt} 
\end{quotation}


\begin{quotation} 

अश्रद्दधानाः पुरुषा धर्मस्यास्य परन्तप  ।  

अप्राप्य मां निवर्तन्ते मृत्युसंसारवर्त्मनि  ॥ ९.३ ॥  मूल श्लोक
\end{quotation}

\begin{quotation}

अ-श्रद्दधा-नाः पुरुषा, धर्मस्य-अस्य परन्तप  ।  

अप्राप्य मां नि-वर्तन्ते, मृत्यु संसार वर्त्मनि  ॥ ९.३ ॥  उच्चारण

\noindent\rule{16cm}{0.4pt} 
\end{quotation}


\begin{quotation} 

मया ततमिदं सर्वं जगदव्यक्तमूर्तिना  ।  

मत्स्थानि सर्वभूतानि न चाहं तेषवस्थितः  ॥ ९.४ ॥  मूल श्लोक
\end{quotation}

\begin{quotation}

मया ततम इदं सर्वं, जगद व्यक्त मूर्तिना  ।  

मत-स्थानि सर्व-भूतानि, न चाहं तेषव स्थितः  ॥ ९.४ ॥  उच्चारण

\noindent\rule{16cm}{0.4pt} 
\end{quotation}


\begin{quotation} 

न च मत्स्थानि भूतानि पश्य मे योगमैश्वरम्‌  ।  

भूतभृन्न च भूतस्थो ममात्मा भूतभावनः  ॥ ९.५ ॥  मूल श्लोक
\end{quotation}

\begin{quotation}

न च मत-स्थानि भूतानि, पश्य मे योगम एैश्वरम्‌  ।  

भूत भृन्न च भूतस्थो, मम-आत्मा भूत-भावनः  ॥ ९.५ ॥  उच्चारण

\noindent\rule{16cm}{0.4pt} 
\end{quotation}


\begin{quotation} 
यथाकाशस्थितो नित्यं वायुः सर्वत्रगो महान्‌  ।  

तथा सर्वाणि भूतानि मत्स्थानीत्युपधारय  ॥ ९.६ ॥  मूल श्लोक
\end{quotation}

\begin{quotation}

यथाऽऽ-आकाश स्थितो नित्यं, वायुः सर्वत्र-गो महान्‌  ।  

तथा सर्वाणि भूतानि, मत-स्थानीत्य उपधारय  ॥ ९.६ ॥  उच्चारण

\noindent\rule{16cm}{0.4pt} 
\end{quotation}


\begin{quotation} 

सर्वभूतानि कौन्तेय प्रकृतिं यान्ति मामिकाम्‌  ।  

कल्पक्षये पुनस्तानि कल्पादौ विसृजाम्यहम्‌  ॥ ९.७ ॥  मूल श्लोक
\end{quotation}

\begin{quotation}

सर्व-भूतानि कौन्तेय, प्रकृतिं यान्ति मामिकाम्‌  ।  

कल्पक्षये पुनस्तानि, कल्पादौ वि-सृजाम्य अहम्‌  ॥ ९.७ ॥  उच्चारण

\noindent\rule{16cm}{0.4pt} 
\end{quotation}


\begin{quotation} 

प्रकृतिं स्वामवष्टभ्य विसृजामि पुनः पुनः  ।  

भूतग्राममिमं कृत्स्नमवशं प्रकृतेर्वशात्‌  ॥ ९.८ ॥  मूल श्लोक
\end{quotation}

\begin{quotation}

प्रकृतिं स्वाम अवष्टभ्य, वि-सृजामि पुनः पुनः  ।  

भूत-ग्रामम इमं कृस्त-नम, अवशं प्रकृतेर वशात्‌  ॥ ९.८ ॥  उच्चारण

\noindent\rule{16cm}{0.4pt} 
\end{quotation}


\begin{quotation} 

न च मां तानि कर्माणि निबध्नन्ति धनञ्जय ।  

उदासीनवदासीनमसक्तं तेषु कर्मसु  ॥ ९.९ ॥  मूल श्लोक
\end{quotation}

\begin{quotation}

न च मां तानि कर्माणि, निबध्-नन्ति धनञ्जय ।  

उदासीन-वद आसीनम, असक्तं तेषु कर्मसु  ॥ ९.९ ॥  उच्चारण

\noindent\rule{16cm}{0.4pt} 
\end{quotation}


\begin{quotation} 

मयाध्यक्षेण प्रकृतिः सूयते सचराचरं  ।  

हेतुनानेन कौन्तेय जगद्विपरिवर्तते  ॥ ९.१० ॥  मूल श्लोक
\end{quotation}

\begin{quotation}

मया अध्यक्षेण प्रकृतिः, सूयते सचरा चरं  ।  

हेतु-नानेन कौन्तेय, जगद वि-परिवर्तते  ॥ ९.१० ॥  उच्चारण

\noindent\rule{16cm}{0.4pt} 
\end{quotation}


\begin{quotation} 

अवजानन्ति मां मूढा मानुषीं तनुमाश्रितम्‌ ।  

परं भावमजानन्तो मम भूतमहेश्वरम्‌  ॥ ९.११ ॥  मूल श्लोक
\end{quotation}

\begin{quotation}

अव-जानन्ति मां मूढा, मानुषीं तनुम आश्रितम्‌ ।  

परं भावम जानन्तो, मम भूत-महेश्वरम्‌  ॥ ९.११ ॥  उच्चारण

\noindent\rule{16cm}{0.4pt} 
\end{quotation}


\begin{quotation} 
मोघाशा मोघकर्माणो मोघज्ञाना विचेतसः  ।  

राक्षसीमासुरीं चैव प्रकृतिं मोहिनीं श्रिताः  ॥ ९.१२ ॥  मूल श्लोक
\end{quotation}

\begin{quotation}

मोघाशा मोघ कर्माणो, मोघ ज्ञाना वि-चेतसः  ।  

राक्षसीम आसुरीं चैव, प्रकृतिं मोहिनीं श्रिताः  ॥ ९.१२ ॥  उच्चारण

\noindent\rule{16cm}{0.4pt} 
\end{quotation}


\begin{quotation} 

महात्मानस्तु मां पार्थ दैवीं प्रकृतिमाश्रिताः  ।  

भजन्त्यनन्यमनसो ज्ञात्वा भूतादिमव्यम्‌  ॥ ९.१३ ॥  मूल श्लोक
\end{quotation}

\begin{quotation}

महात्मानस् तु मां पार्थ, दैवीं प्रकृतिम आश्रिताः  ।  

भजन्त्य अनन्य-मनसो, ज्ञात्वा भूतादिम अ-व्ययम्  ॥ ९.१३ ॥  उच्चारण

\noindent\rule{16cm}{0.4pt} 
\end{quotation}


\begin{quotation} 

सततं कीर्तयन्तो मां यतन्तश्च दृढ़व्रताः  ।  

नमस्यन्तश्च मां भक्त्या नित्ययुक्ता उपासते  ॥ ९.१४ ॥  मूल श्लोक
\end{quotation}

\begin{quotation}

सततं कीर्त-यन्तो मां, यतन्तश् च दृढ़-व्रताः  ।  

नमस-यन्तश् च मां भक्त्या, नित्य-युक्ता उपासते  ॥ ९.१४ ॥  उच्चारण

\noindent\rule{16cm}{0.4pt} 
\end{quotation}


\begin{quotation} 

ज्ञानयज्ञेन चाप्यन्ये यजन्तो मामुपासते ।  

एकत्वेन पृथक्त्वेन बहुधा विश्वतोमुखम्  ॥ ९.१५ ॥  मूल श्लोक
\end{quotation}

\begin{quotation}

ज्ञान यज्ञेन चाप्य अन्ये, यजन्तो मामु-पासते ।  

एक-त्वेन पृथक-त्वेन, बहुधा विश्वतो मुखम्  ॥ ९.१५ ॥  उच्चारण

\noindent\rule{16cm}{0.4pt} 
\end{quotation}


\begin{quotation} 

अहं क्रतुरहं यज्ञः स्वधाहमहमौषधम्‌  ।  

मंत्रोऽहमहमेवाज्यमहमग्निरहं हुतम्‌  ॥ ९.१६ ॥  मूल श्लोक
\end{quotation}

\begin{quotation}

अहं क्रतुर अहं यज्ञः, स्वधा अहम, अहम औषधम्‌  ।  

मंत्रोऽ अहम अहम एव आज्यम, अहम अग्निर अहं हुतम्‌  ॥ ९.१६ ॥  उच्चारण

\noindent\rule{16cm}{0.4pt} 
\end{quotation}


\begin{quotation} 

पिताहमस्य जगतो माता धाता पितामहः  ।  

वेद्यं पवित्रमोङ्कार ऋक्साम यजुरेव च  ॥ ९.१७ ॥   मूल श्लोक
\end{quotation}

\begin{quotation}

पिता अहम अस्य जगतो, माता धाता पितामहः  ।  

वेद्यं पवित्रम ओङ्कार, ऋक्साम यजुरेव च  ॥ ९.१७ ॥  उच्चारण

\noindent\rule{16cm}{0.4pt} 
\end{quotation}


\begin{quotation} 
गतिर्भर्ता प्रभुः साक्षी निवासः शरणं सुहृत्‌  ।  

प्रभवः प्रलयः स्थानं निधानं बीजमव्ययम्‌  ॥ ९.१८ ॥  मूल श्लोक
\end{quotation}

\begin{quotation}

गतिर भर्ता प्रभुः साक्षी, निवासः शरणं सुहृत्‌  ।  

प्रभवः प्रलयः स्थानं, निधानं बीजम अ-व्ययम्‌  ॥ ९.१८ ॥  उच्चारण

\noindent\rule{16cm}{0.4pt} 
\end{quotation}


\begin{quotation} 

तपाम्यहमहं वर्षं निगृह्‌णाम्युत्सृजामि च  ।  

अमृतं चैव मृत्युश्च सदसच्चाहमर्जुन  ॥ ९.१९ ॥  मूल श्लोक
\end{quotation}

\begin{quotation}

तपाम्य अहम अहं वर्षं, निगृह्‌ण-आम्य उत-सृजामि च  ।  

अमृतं चैव मृत्युश् च, सद असच् च अहम अर्जुन  ॥ ९.१९ ॥  उच्चारण

\noindent\rule{16cm}{0.4pt} 
\end{quotation}


\begin{quotation} 

त्रैविद्या मां सोमपाः पूतपापायज्ञैरिष्ट्‍वा स्वर्गतिं प्रार्थयन्ते ।  

ते पुण्यमासाद्य सुरेन्द्रलोकमश्नन्ति दिव्यान्दिवि देवभोगान्‌  ॥ ९.२० ॥  मूल श्लोक
\end{quotation}

\begin{quotation}

त्रैविद्या मां सोमपाः पूत-पापा ।  

यज्ञैर इष्ट्‍वा स्वर्गतिं प्रार्थयन्ते ।  

ते पुण्यम आसाद्य सुरेन्द्र लोकम ।  

अशनन्ति दिव्यान दिवि देव-भोगान्‌  ॥ ९.२० ॥  उच्चारण

\noindent\rule{16cm}{0.4pt} 
\end{quotation}


\begin{quotation} 

ते तं भुक्त्वा स्वर्गलोकं विशालंक्षीणे पुण्य मर्त्यलोकं विशन्ति ।  

एवं त्रयीधर्ममनुप्रपन्ना गतागतं कामकामा लभन्ते  ॥ ९.२१ ॥  मूल श्लोक
\end{quotation}

\begin{quotation}

ते तं भुक्त्वा स्वर्ग लोकं विशालं
क्षीणे पुण्य मर्त्य-लोकं विशन्ति ।  

एवं त्रयी धर्मम अनु-प्रपन्ना 
गता गतं काम कामा लभन्ते  ॥ ९.२१ ॥  उच्चारण

\noindent\rule{16cm}{0.4pt} 
\end{quotation}


\begin{quotation} 

अनन्याश्चिन्तयन्तो मां ये जनाः पर्युपासते  ।  

तेषां नित्याभियुक्तानां योगक्षेमं वहाम्यहम्‌  ॥ ९.२२ ॥  मूल श्लोक
\end{quotation}

\begin{quotation}

अ-नन्यश चिंत-यन्तो मां, ये जनाः पर्युपासते  ।  

तेषां नित्याभि-युक्तानां, योग-क्षेमं वहाम्य अहम्‌  ॥ ९.२२ ॥  उच्चारण

\noindent\rule{16cm}{0.4pt} 
\end{quotation}


\begin{quotation} 


येऽप्यन्यदेवता भक्ता यजन्ते श्रद्धयान्विताः  ।  

तेऽपि मामेव कौन्तेय यजन्त्यविधिपूर्वकम्‌  ॥ ९.२३ ॥  मूल श्लोक
\end{quotation}

\begin{quotation}

येऽ अप्य अन्य देवता भक्ता, यजन्ते श्रद्धय आन्विताः  ।  

तेऽ अपि मामेव कौन्तेय, यजन्त्य अविधि-पूर्वकम्‌  ॥ ९.२३ ॥  उच्चारण

\noindent\rule{16cm}{0.4pt} 
\end{quotation}


\begin{quotation} 

अहं हि सर्वयज्ञानां भोक्ता च प्रभुरेव च  ।  

न तु मामभिजानन्ति तत्त्वेनातश्च्यवन्ति ते  ॥ ९.२४ ॥  मूल श्लोक
\end{quotation}

\begin{quotation}

अहं हि सर्व-यज्ञानां, भोक्ता च प्रभुरेव च  ।  

न तु मामभि जानन्ति, तत्त्वे-नातश च्यवन्ति ते  ॥ ९.२४ ॥  उच्चारण

\noindent\rule{16cm}{0.4pt} 
\end{quotation}


\begin{quotation} 

यान्ति देवव्रता देवान्पितृन्यान्ति पितृव्रताः  ।  

भूतानि यान्ति भूतेज्या यान्ति मद्याजिनोऽपि माम्‌  ॥ ९.२५ ॥  मूल श्लोक
\end{quotation}

\begin{quotation}

यान्ति देव व्रता देवान, पितृन यान्ति पितृ व्रताः  ।  

भूतानि यान्ति भूतेज्या, यान्ति मद याजिनोऽ अपि माम्‌  ॥ ९.२५ ॥  उच्चारण

\noindent\rule{16cm}{0.4pt} 
\end{quotation}


\begin{quotation} 

पत्रं पुष्पं फलं तोयं यो मे भक्त्या प्रयच्छति  ।  

तदहं भक्त्युपहृतमश्नामि प्रयतात्मनः  ॥ ९.२६ ॥  मूल श्लोक
\end{quotation}

\begin{quotation}

पत्रं पुष्पं फलं तोयं, यो मे भक्त्या प्रयच्छति  ।  

तद अहं भक्त्य उपहृतम, अशनामि प्रयत-आत्मनः  ॥ ९.२६ ॥  उच्चारण

\noindent\rule{16cm}{0.4pt} 
\end{quotation}


\begin{quotation} 

यत्करोषि यदश्नासि यज्जुहोषि ददासि यत्‌  ।  

यत्तपस्यसि कौन्तेय तत्कुरुष्व मदर्पणम्‌  ॥ ९.२७ ॥  मूल श्लोक
\end{quotation}

\begin{quotation}

यत करोषि यद अश्-नासि, यज-जुहोषि ददासि यत्‌  ।  

यत्त पस्यसि कौन्तेय, तत कुरुष्व मद-अर्पणम्‌  ॥ ९.२७ ॥  उच्चारण

\noindent\rule{16cm}{0.4pt} 
\end{quotation}


\begin{quotation} 

शुभाशुभफलैरेवं मोक्ष्य से कर्मबंधनैः  ।  

सन्न्यासयोगमुक्तात्मा विमुक्तो मामुपैष्यसि  ॥ ९.२८ ॥  मूल श्लोक
\end{quotation}

\begin{quotation}

शुभा शुभ फलैर एवं, मोक्ष्य से कर्म बंधनैः  ।  

सन्न्यास योग युक्तात्मा, विमुक्तो माम उपैष्यसि  ॥ ९.२८ ॥  उच्चारण

\noindent\rule{16cm}{0.4pt} 
\end{quotation}


\begin{quotation} 
समोऽहं सर्वभूतेषु न मे द्वेष्योऽस्ति न प्रियः  ।  

ये भजन्ति तु मां भक्त्या मयि ते तेषु चाप्यहम्‌  ॥ ९.२९ ॥  मूल श्लोक
\end{quotation}

\begin{quotation}

समोऽ- अहं सर्व भूतेषु, न मे द्वेष्योऽ अस्ति न प्रियः  ।  

ये भजन्ति तु मां भक्त्या, मयि ते तेषु चाप्य अहम्‌  ॥ ९.२९ ॥  उच्चारण

\noindent\rule{16cm}{0.4pt} 
\end{quotation}


\begin{quotation} 

अपि चेत्सुदुराचारो भजते मामनन्यभाक्‌  ।  

साधुरेव स मन्तव्यः सम्यग्व्यवसितो हि सः  ॥ ९.३० ॥  मूल श्लोक
\end{quotation}

\begin{quotation}

अपि चेत सु-दुराचारो, भजते माम अनन्य-भाक्‌  ।  

साधुर एव स मन्तव्यः, सम्यग व्यवसितो हि सः  ॥ ९.३० ॥  उच्चारण

\noindent\rule{16cm}{0.4pt} 
\end{quotation}


\begin{quotation} 

क्षिप्रं भवति धर्मात्मा शश्वच्छान्तिं निगच्छति  ।  

कौन्तेय प्रतिजानीहि न मे भक्तः प्रणश्यति  ॥ ९.३१ ॥  मूल श्लोक
\end{quotation}

\begin{quotation}

क्षिप्रं भवति धर्मात्मा, शश्वच छान्तिं नि-गच्छति  ।  

कौन्तेय प्रति-जानीहि, न मे भक्तः प्रणश्यति  ॥ ९.३१ ॥  उच्चारण

\noindent\rule{16cm}{0.4pt} 
\end{quotation}


\begin{quotation} 

मां हि पार्थ व्यपाश्रित्य येऽपि स्यु पापयोनयः  ।  

स्त्रियो वैश्यास्तथा शूद्रास्तेऽपि यान्ति परां गतिम्‌  ॥ ९.३२ ॥  मूल श्लोक
\end{quotation}

\begin{quotation}

मां हि पार्थ व्यप-आश्रित्य, येऽ अपिस्यु पापयोनयः  ।  

स्त्रियो वैश्यास तथा शूद्रा, स्तेऽ अपि यान्ति परां गतिम्‌  ॥ ९.३२ ॥  उच्चारण

\noindent\rule{16cm}{0.4pt} 
\end{quotation}


\begin{quotation} 

किं पुनर्ब्राह्मणाः पुण्या भक्ता राजर्षयस्तथा  ।  

अनित्यमसुखं लोकमिमं प्राप्य भजस्व माम्‌  ॥ ९.३३ ॥  मूल श्लोक
\end{quotation}

\begin{quotation}

किं पुनर ब्राह्मणाः पुण्या, भक्ता राज-ऋषियस तथा  ।  

अनित्यम असुखं लोकम, इमं प्राप्य भजस्व माम्‌  ॥ ९.३३ ॥  उच्चारण

\noindent\rule{16cm}{0.4pt} 
\end{quotation}


\begin{quotation} 

मन्मना भव मद्भक्तो मद्याजी मां नमस्कुरु  ।  

मामेवैष्यसि युक्त्वैवमात्मानं मत्परायण:  ॥ ९.३४ ॥  मूल श्लोक
\end{quotation}

\begin{quotation}

मन्मना भव मद्भक्तो, मद्याजी मां नमस्कुरु  ।  

माम एवैष्यसि युक्त्व-एैवम, आत्मानं मत-परायण:  ॥ ९.३४ ॥  उच्चारण

\noindent\rule{16cm}{0.4pt} 
\end{quotation}

\begin{center} ***** \end{center}
\begin{quotation} 



ॐ तत् सद इति श्री मद्-भगवद्-गीतास उपनिषत्सु ब्रह्म विद्यायां योगशास्त्रे श्री कृष्णार्जुन संवादे राजविद्या राजगुह्य योगो नाम नवमोऽ अध्यायः  ॥  ९  ॥ 
\end{quotation} 
