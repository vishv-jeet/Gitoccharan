\chapter{\sanskrit विभूतियोग} 

\paragraph{\sanskrit श्रीभगवानुवाच}
\begin{quotation} 

भूय एव महाबाहो श्रृणु मे परमं वचः  ।  

यत्तेऽहं प्रीयमाणाय वक्ष्यामि हितकाम्यया  ॥ १०.१ ॥  मूल श्लोक
\end{quotation}

\begin{quotation}

भूय एव महाबाहो, श्रृणु मे परमं वचः  ।  

यत् तेऽ अहं प्रीय-माणाय, वक्ष्यामि हित-काम्यया  ॥ १०.१ ॥  उच्चारण

\noindent\rule{16cm}{0.4pt} 
\end{quotation}


\begin{quotation} 

न मे विदुः सुरगणाः प्रभवं न महर्षयः  ।  

अहमादिर्हि देवानां महर्षीणां च सर्वशः  ॥ १०.२ ॥  मूल श्लोक
\end{quotation}

\begin{quotation}

न मे विदुः सुर-गणाः, प्रभवं न महर्षयः  ।  

अहम आदिर हि देवानां, महर्षीणां च सर्वशः  ॥ १०.२ ॥  उच्चारण

\noindent\rule{16cm}{0.4pt} 
\end{quotation}


\begin{quotation} 

यो मामजमनादिं च वेत्ति लोकमहेश्वरम्‌  ।  

असम्मूढः स मर्त्येषु सर्वपापैः प्रमुच्यते  ॥ १०.३ ॥  मूल श्लोक
\end{quotation}

\begin{quotation}

यो माम अजम अनादिं च, वेत्ति लोक-महेश्वरम्‌  ।  

असम्मूढः स मर्त्येषु, सर्व-पापैः प्रमुच्यते  ॥ १०.३ ॥  उच्चारण

\noindent\rule{16cm}{0.4pt} 
\end{quotation}


\begin{quotation} 

बुद्धिर्ज्ञानमसम्मोहः क्षमा सत्यं दमः शमः  ।  

सुखं दुःखं भवोऽभावो भयं चाभयमेव च  ॥ १०.४ ॥  मूल श्लोक
\end{quotation}

\begin{quotation}

बुद्धिर ज्ञानम सम्मोहः, क्षमा सत्यं दमः शमः  ।  

सुखं दुःखं भवोऽ अभावो, भयं च अभयम एव च  ॥ १०.४ ॥  उच्चारण

\noindent\rule{16cm}{0.4pt} 
\end{quotation}


\begin{quotation} 

अहिंसा समता तुष्टिस्तपो दानं यशोऽयशः  ।  

भवन्ति भावा भूतानां मत्त एव पृथग्विधाः  ॥ १०.५ ॥  मूल श्लोक
\end{quotation}

\begin{quotation}

अहिंसा समता तुष्टिः, तपो दानं यशोऽ अयशः  ।  

भवन्ति भावा भूतानां, मत्त एव पृथग़-विधाः  ॥ १०.५ ॥  उच्चारण

\noindent\rule{16cm}{0.4pt} 
\end{quotation}


\begin{quotation} 

महर्षयः सप्त पूर्वे चत्वारो मनवस्तथा  ।  

मद्भावा मानसा जाता येषां लोक इमाः प्रजाः  ॥ १०.६ ॥  मूल श्लोक
\end{quotation}

\begin{quotation}

महर्षयः सप्त पूर्वे, चत्वारो मनवस् तथा  ।  

मद्भावा मानसा जाता, येषां लोक इमाः प्रजाः  ॥ १०.६ ॥  उच्चारण

\noindent\rule{16cm}{0.4pt} 
\end{quotation}


\begin{quotation} 

एतां विभूतिं योगं च मम यो वेत्ति तत्त्वतः ।  

सोऽविकम्पेन योगेन युज्यते नात्र संशयः  ॥ १०.७ ॥  मूल श्लोक
\end{quotation}

\begin{quotation}

एतां विभूतिं योगं च, मम यो वेत्ति तत्त्वतः  ।  

सोऽ अ-विकम्पेन योगेन, युज्यते नात्र संशयः  ॥ १०.७ ॥  उच्चारण

\noindent\rule{16cm}{0.4pt} 
\end{quotation}


\begin{quotation} 

अहं सर्वस्य प्रभवो मत्तः सर्वं प्रवर्तते  ।  

इति मत्वा भजन्ते मां बुधा भावसमन्विताः  ॥ १०.८ ॥  मूल श्लोक
\end{quotation}

\begin{quotation}

अहं सर्वस्य प्रभवो, मत्तः सर्वं प्रवर्तते  ।  

इति मत्वा भजन्ते मां, बुधा भाव-समन्विताः  ॥ १०.८ ॥  उच्चारण

\noindent\rule{16cm}{0.4pt} 
\end{quotation}


\begin{quotation} 

मच्चित्ता मद्गतप्राणा बोधयन्तः परस्परम्‌  ।  

कथयन्तश्च मां नित्यं तुष्यन्ति च रमन्ति च  ॥ १०.९ ॥  मूल श्लोक
\end{quotation}

\begin{quotation}

मच्चित्ता मद्-गत-प्राणा, बोध-यन्तः परस्परम्‌  ।  

कथ-यन्तश् च मां नित्यं, तुष्यन्ति च रमन्ति च  ॥ १०.९ ॥  उच्चारण

\noindent\rule{16cm}{0.4pt} 
\end{quotation}


\begin{quotation} 

तेषां सततयुक्तानां भजतां प्रीतिपूर्वकम्‌  ।  

ददामि बुद्धियोगं तं येन मामुपयान्ति ते  ॥ १०.१० ॥  मूल श्लोक
\end{quotation}

\begin{quotation}

तेषां सतत युक्तानां, भजतां प्रीति-पूर्वकम्‌  ।  

ददामि बुद्धि-योगं तं, येन माम उपयान्ति ते  ॥ १०.१० ॥  उच्चारण

\noindent\rule{16cm}{0.4pt} 
\end{quotation}


\begin{quotation} 

तेषामेवानुकम्पार्थमहमज्ञानजं तमः ।  

नाशयाम्यात्मभावस्थो ज्ञानदीपेन भास्वता  ॥ १०.११ ॥  मूल श्लोक
\end{quotation}

\begin{quotation}

तेषाम एव अनुकम्पा-अर्थम, अहम ज्ञानजं तमः ।  

नाशयाम्य आत्म भावस्थो, ज्ञान-दीपेन भास्वता  ॥ १०.११ ॥  उच्चारण

\noindent\rule{16cm}{0.4pt} 
\end{quotation}

\paragraph{\sanskrit अर्जुन उवाच}

\begin{quotation} 

परं ब्रह्म परं धाम पवित्रं परमं भवान्‌  ।  

पुरुषं शाश्वतं दिव्यमादिदेवमजं विभुम्‌  ॥ १०.१२ ॥  मूल श्लोक
\end{quotation}

\begin{quotation}

परं ब्रह्म परं धाम, पवित्रं परमं भवान्‌  ।  

पुरुषं शाश्वतं दिव्यम, आदि-देवम अजं विभुम्‌  ॥ १०.१२ ॥  उच्चारण

\noindent\rule{16cm}{0.4pt} 
\end{quotation}


\begin{quotation} 

आहुस्त्वामृषयः सर्वे देवर्षिर्नारदस्तथा  ।  

असितो देवलो व्यासः स्वयं चैव ब्रवीषि मे  ॥ १०.१३ ॥  मूल श्लोक
\end{quotation}

\begin{quotation}

आहुस-त्वाम ऋषयः सर्वे, देवर्षिर नारदस् तथा  ।  

असितो देवलो व्यासः, स्वयं चैव ब्रवीषि मे  ॥ १०.१३ ॥  उच्चारण

\noindent\rule{16cm}{0.4pt} 
\end{quotation}


\begin{quotation} 

सर्वमेतदृतं मन्ये यन्मां वदसि केशव ।  

न हि ते भगवन् व्यक्तिं विदुर्देवा न दानवाः ॥ १०.१४ ॥  मूल श्लोक
\end{quotation}

\begin{quotation}

सर्वम एतद् ऋतं मन्ये, यन्मां वदसि केशव  ।  

न हि ते भगवन व्यक्तिं, विदुर-देवा न दानवाः  ॥ १०.१४ ॥  उच्चारण

\noindent\rule{16cm}{0.4pt} 
\end{quotation}


\begin{quotation} 

स्वयमेवात्मनात्मानं वेत्थ त्वं पुरुषोत्तम  ।  

भूतभावन भूतेश देवदेव जगत्पते  ॥ १०.१५ ॥  मूल श्लोक
\end{quotation}

\begin{quotation}

स्वयम एव-आत्मन-आत्मानं, वेत्थ त्वं पुरुषोत्तम  ।  

भूत-भावन भूतेश, देव देव जगत्पते  ॥ १०.१५ ॥  उच्चारण

\noindent\rule{16cm}{0.4pt} 
\end{quotation}


\begin{quotation} 

वक्तुमर्हस्यशेषेण दिव्या ह्यात्मविभूतयः  ।  

याभिर्विभूतिभिर्लोकानिमांस्त्वं व्याप्य तिष्ठसि  ॥ १०.१६ ॥  मूल श्लोक
\end{quotation}

\begin{quotation}

वक्तुम अर्हस्य अशेषेण, दिव्या ह्य-आत्म विभूतयः  ।  

याभिर विभूति-भिर लोकान, इमांस-त्वं व्याप्य तिष्ठसि  ॥ १०.१६ ॥  उच्चारण

\noindent\rule{16cm}{0.4pt} 
\end{quotation}


\begin{quotation} 
कथं विद्यामहं योगिंस्त्वां सदा परिचिन्तयन्‌  ।  

केषु केषु च भावेषु चिन्त्योऽसि भगवन्मया  ॥ १०.१७ ॥  मूल श्लोक
\end{quotation}

\begin{quotation}

कथं विद्याम अहं योगिम्स, त्वां सदा परि-चिन्तयन्‌  ।  

केषु केषु च भावेषु, चिन्त्योऽ असि भगवन-मया  ॥ १०.१७ ॥  उच्चारण

\noindent\rule{16cm}{0.4pt} 
\end{quotation}


\begin{quotation} 

विस्तरेणात्मनो योगं विभूतिं च जनार्दन  ।  

भूयः कथय तृप्तिर्हि श्रृण्वतो नास्ति मेऽमृतम्‌  ॥ १०.१८ ॥  मूल श्लोक
\end{quotation}

\begin{quotation}

विस्तरेण आत्मनो योगं, विभूतिं च जनार्दन  ।  

भूयः कथय तृप्तिर-हि, श्रृण्वतो नास्ति मेऽ अमृतम्‌  ॥ १०.१८ ॥  उच्चारण

\noindent\rule{16cm}{0.4pt} 
\end{quotation}

\paragraph{\sanskrit श्रीभगवानुवाच}

\begin{quotation} 


हन्त ते कथयिष्यामि दिव्या ह्यात्मविभूतयः  ।  

प्राधान्यतः कुरुश्रेष्ठ नास्त्यन्तो विस्तरस्य मे  ॥ १०.१९ ॥  मूल श्लोक
\end{quotation}

\begin{quotation}

हन्त ते कथय-इष्यामि , दिव्या ह्य आत्म विभूतयः  ।  

प्राधान्य-तः कुरुश्रेष्ठ, नास्त्य अन्तो विस्तरस्य मे  ॥ १०.१९ ॥  उच्चारण

\noindent\rule{16cm}{0.4pt} 
\end{quotation}


\begin{quotation} 

अहमात्मा गुडाकेश सर्वभूताशयस्थितः  ।  

अहमादिश्च मध्यं च भूतानामन्त एव च  ॥ १०.२० ॥  मूल श्लोक
\end{quotation}

\begin{quotation}

अहम आत्मा गुडाकेश, सर्व भूताशय स्थितः  ।  

अहम आदिश् च मध्यं च, भूतानाम अन्त एव च  ॥ १०.२० ॥  उच्चारण

\noindent\rule{16cm}{0.4pt} 
\end{quotation}


\begin{quotation} 

आदित्यानामहं विष्णुर्ज्योतिषां रविरंशुमान्‌  ।  

मरीचिर्मरुतामस्मि नक्षत्राणामहं शशी  ॥ १०.२१ ॥  मूल श्लोक
\end{quotation}

\begin{quotation}

आदित्या-नाम अहं विष्णुर, ज्योतिषां रविर अंशुमान्‌  ।  

मरीचिर मरुताम अस्मि, नक्षत्राणाम् अहं शशी  ॥ १०.२१ ॥  उच्चारण

\noindent\rule{16cm}{0.4pt} 
\end{quotation}


\begin{quotation} 


वेदानां सामवेदोऽस्मि देवानामस्मि वासवः ।  

इन्द्रियाणां मनश्चास्मि भूतानामस्मि चेतना  ॥ १०.२२ ॥  मूल श्लोक
\end{quotation}

\begin{quotation}

वेदानां सामवेदोऽ अस्मि, देवानाम अस्मि वासवः  ।  

इंद्रियाणां मनश् च अस्मि, भूतानाम अस्मि चेतना  ॥ १०.२२ ॥  उच्चारण

\noindent\rule{16cm}{0.4pt} 
\end{quotation}


\begin{quotation} 

रुद्राणां शङ्करश्चास्मि वित्तेशो यक्षरक्षसाम् ।  

वसूनां पावकश्चास्मि मेरुः शिखरिणामहम्  ॥ १०.२३ ॥  मूल श्लोक
\end{quotation}

\begin{quotation}

रुद्राणां शङ्करश्  च अस्मि, वित्तेशो यक्ष-रक्षसाम्‌  ।  

वसूनां पावकश् च अस्मि, मेरुः शिखरि-णाम अहम्‌  ॥ १०.२३ ॥  उच्चारण

\noindent\rule{16cm}{0.4pt} 
\end{quotation}


\begin{quotation} 

पुरोधसां च मुख्यं मां विद्धि पार्थ बृहस्पतिम्‌  ।  

सेनानीनामहं स्कन्दः सरसामस्मि सागरः  ॥ १०.२४ ॥  मूल श्लोक
\end{quotation}

\begin{quotation}

पुरोधसां च मुख्यं मां, विद्धि पार्थ बृहस्पतिम्‌  ।  

सेनानी-नाम अहं स्कन्दः, सरसाम अस्मि सागरः  ॥ १०.२४ ॥  उच्चारण

\noindent\rule{16cm}{0.4pt} 
\end{quotation}


\begin{quotation} 

महर्षीणां भृगुरहं गिरामस्म्येकमक्षरम्‌  ।  

यज्ञानां जपयज्ञोऽस्मि स्थावराणां हिमालयः  ॥ १०.२५ ॥  मूल श्लोक
\end{quotation}

\begin{quotation}

महर्षीणां भृगुर अहं, गिराम अस्म्य एकम अक्षरम्‌  ।  

यज्ञानां जप-यज्ञोऽ अस्मि, स्थावरा-णाम् हिमालयः  ॥ १०.२५ ॥  उच्चारण

\noindent\rule{16cm}{0.4pt} 
\end{quotation}


\begin{quotation} 

अश्वत्थः सर्ववृक्षाणां देवर्षीणां च नारदः ।  

गन्धर्वाणां चित्ररथः सिद्धानां कपिलो मुनिः  ॥ १०.२६ ॥  मूल श्लोक
\end{quotation}

\begin{quotation}

अश्वत्थः सर्व-वृक्षाणां, देवर्षीणां च नारदः  ।  

गन्धर्वाणां चित्ररथः, सिद्धानां कपिलो मुनिः  ॥ १०.२६ ॥  उच्चारण

\noindent\rule{16cm}{0.4pt} 
\end{quotation}


\begin{quotation} 

उच्चैःश्रवसमश्वानां विद्धि माममृतोद्धवम्‌  ।  

एरावतं गजेन्द्राणां नराणां च नराधिपम्‌  ॥ १०.२७ ॥  मूल श्लोक
\end{quotation}

\begin{quotation}

उच्चैः-श्रवसम अश्वानां, विद्धि माम अमृतो-उद्भवम्  ।  

एरावतं गजेन्द्राणां, नराणां च नराधिपम्‌  ॥ १०.२७ ॥  उच्चारण

\noindent\rule{16cm}{0.4pt} 
\end{quotation}


\begin{quotation} 
आयुधानामहं वज्रं धेनूनामस्मि कामधुक्‌  ।  

प्रजनश्चास्मि कन्दर्पः सर्पाणामस्मि वासुकिः  ॥ १०.२८ ॥  मूल श्लोक
\end{quotation}

\begin{quotation}

आयुधानाम अहं वज्रं, धेनूनाम अस्मि कामधुक्‌  ।  

प्रजनश् च अस्मि कन्दर्पः, सर्पाणाम अस्मि वासुकिः  ॥ १०.२८ ॥  उच्चारण

\noindent\rule{16cm}{0.4pt} 
\end{quotation}


\begin{quotation} 
अनन्तश्चास्मि नागानां वरुणो यादसामहम् ।  

पितृ़णामर्यमा चास्मि यमः संयमतामहम्  ॥ १०.२९ ॥  मूल श्लोक
\end{quotation}

\begin{quotation}

अनन्तश् च अस्मि नागानां, वरुणो यादसाम अहम्‌  ।  

पितृ़णाम आर्यमा चास्मि, यमः संयम-ताम अहम्‌  ॥ १०.२९ ॥  उच्चारण

\noindent\rule{16cm}{0.4pt} 
\end{quotation}


\begin{quotation} 

प्रह्लादश्चास्मि दैत्यानां कालः कलयतामहम् ।  

मृगाणां च मृगेन्द्रोऽहं वैनतेयश्च पक्षिणाम्  ॥ १०.३० ॥  मूल श्लोक
\end{quotation}

\begin{quotation}

प्रह्लादश् च अस्मि दैत्यानां, कालः कलय-ताम अहम्‌  ।  

मृगाणां च मृगेन्द्रोऽ अहं, वैनते-यश् च पक्षिणाम्‌  ॥ १०.३० ॥  उच्चारण

\noindent\rule{16cm}{0.4pt} 
\end{quotation}


\begin{quotation} 

पवनः पवतामस्मि रामः शस्त्रभृतामहम्‌  ।  

झषाणां मकरश्चास्मि स्रोतसामस्मि जाह्नवी  ॥ १०.३१ ॥  मूल श्लोक
\end{quotation}

\begin{quotation}

पवनः पवताम अस्मि, रामः शस्त्र-भृताम अहम्‌  ।  

झषाणां मकरश् च अस्मि, स्रोत-साम अस्मि जाह्नवी  ॥ १०.३१ ॥  उच्चारण

\noindent\rule{16cm}{0.4pt} 
\end{quotation}


\begin{quotation} 

सर्गाणामादिरन्तश्च मध्यं चैवाहमर्जुन  ।  

अध्यात्मविद्या विद्यानां वादः प्रवदतामहम्‌  ॥ १०.३२ ॥  मूल श्लोक
\end{quotation}

\begin{quotation}

सर्गाणाम आदिर अन्तश् च, मध्यं चैवाहम अर्जुन  ।  

अध्यात्म विद्या विद्यानां, वादः प्रवद-ताम अहम्‌  ॥ १०.३२ ॥  उच्चारण

\noindent\rule{16cm}{0.4pt} 
\end{quotation}


\begin{quotation} 

अक्षराणामकारोऽस्मि द्वन्द्वः सामासिकस्य च ।  

अहमेवाक्षयः कालो धाताऽहं विश्वतोमुखः  ॥ १०.३३ ॥  मूल श्लोक
\end{quotation}

\begin{quotation}

अक्षराणाम अ-कारोऽ अस्मि, द्वंद्वः सामा-सिकस्य च  ।  

अहम एव अक्षयः कालो, धाताहं विश्वतो मुखः  ॥ १०.३३ ॥  उच्चारण

\noindent\rule{16cm}{0.4pt} 
\end{quotation}


\begin{quotation} 

मृत्युः सर्वहरश्चाहमुद्भवश्च भविष्यताम्‌  ।  

कीर्तिः श्रीर्वाक्च नारीणां स्मृतिर्मेधा धृतिः क्षमा  ॥ १०.३४ ॥  मूल श्लोक
\end{quotation}

\begin{quotation}

मृत्युः सर्वहरश् च अहम, उद्भवश् च भविष्यताम्‌  ।  

कीर्तिः श्रीर वाक् च नारीणां, स्मृतिर मेधा धृतिः क्षमा  ॥ १०.३४ ॥  उच्चारण

\noindent\rule{16cm}{0.4pt} 
\end{quotation}


\begin{quotation} 

बृहत्साम तथा साम्नां गायत्री छन्दसामहम्‌  ।  

मासानां मार्गशीर्षोऽहमृतूनां कुसुमाकरः ॥ १०.३५ ॥  मूल श्लोक
\end{quotation}

\begin{quotation}

बृहत्साम तथा साम्-नाम, गायत्री छन्द सामहम्‌  ।  

मासानां मार्गशीर्षोऽ अहम, ऋतूनां कुसुमाकरः ॥ १०.३५ ॥  उच्चारण

\noindent\rule{16cm}{0.4pt} 
\end{quotation}


\begin{quotation} 

द्यूतं छलयतामस्मि तेजस्तेजस्विनामहम् ।  

जयोऽ अस्मि व्यवसायोऽ, अस्मि सत्त्वं सत्त्ववतामहम्  ॥ १०.३६ ॥  मूल श्लोक
\end{quotation}

\begin{quotation}

द्यूतं छलयताम अस्मि, तेजस तेजस्वि-नाम अहम्‌  ।  

जयोऽ अस्मि व्यवसायोऽ अस्मि, सत्त्वं सत्त्व-वताम अहम्‌  ॥ १०.३६ ॥  उच्चारण

\noindent\rule{16cm}{0.4pt} 
\end{quotation}


\begin{quotation} 

वृष्णीनां वासुदेवोऽस्मि पाण्डवानां धनंजयः ।  

मुनीनामप्यहं व्यासः कवीनामुशना कविः  ॥ १०.३७ ॥  मूल श्लोक
\end{quotation}

\begin{quotation}

वृष्णी नाम वासुदेवोऽ अस्मि, पाण्डवा नाम धनञ्जयः  ।  

मुनीनाम अप्य अहं व्यासः, कवीनाम उशना कविः  ॥ १०.३७ ॥  उच्चारण

\noindent\rule{16cm}{0.4pt} 
\end{quotation}


\begin{quotation} 

दण्डो दमयतामस्मि नीतिरस्मि जिगीषताम् ।  

मौनं चैवास्मि गुह्यानां ज्ञानं ज्ञानवतामहम्  ॥ १०.३८ ॥  मूल श्लोक
\end{quotation}

\begin{quotation}

दण्डो दमयताम अस्मि, नीतिर अस्मि जिगीषताम्‌  ।  

मौनं चैवास्मि गुह्यानां, ज्ञानं ज्ञानवताम अहम्‌  ॥ १०.३८ ॥  उच्चारण

\noindent\rule{16cm}{0.4pt} 
\end{quotation}


\begin{quotation} 

यच्चापि सर्वभूतानां बीजं तदहमर्जुन  ।  

न तदस्ति विना यत्स्यान्मया भूतं चराचरम्‌  ॥ १०.३९ ॥  मूल श्लोक
\end{quotation}

\begin{quotation}

यच्चापि सर्व भूतानां, बीजं तद अहम अर्जुन  ।  

न तदस्ति विना यत-स्यान, मया भूतं चराचरम्‌  ॥ १०.३९ ॥  उच्चारण

\noindent\rule{16cm}{0.4pt} 
\end{quotation}


\begin{quotation} 
नान्तोऽस्ति मम दिव्यानां विभूतीनां परन्तप  ।  

एष तूद्देशतः प्रोक्तो विभूतेर्विस्तरो मया  ॥ १०.४० ॥  मूल श्लोक
\end{quotation}

\begin{quotation}

नान्तोऽ अस्ति मम, दिव्यानां विभूतीनां परन्तप  ।  

एष तु उद्देशतः प्रोक्तो, विभूतेर विस्तरो मया  ॥ १०.४० ॥  उच्चारण

\noindent\rule{16cm}{0.4pt} 
\end{quotation}


\begin{quotation} 

यद्यद्विभूतिमत्सत्त्वं श्रीमदूर्जितमेव वा  ।  

तत्तदेवावगच्छ त्वं मम तेजोंऽशसम्भवम्‌  ॥ १०.४१ ॥  मूल श्लोक
\end{quotation}

\begin{quotation}

यद् यद् विभूति-मत सत्त्वं, श्रीमद उर्जितम एव वा  ।  

तत तद एवाव-गच्छ त्वं, मम तेजोऽ अंश सम्भवम्‌  ॥ १०.४१ ॥  उच्चारण

\noindent\rule{16cm}{0.4pt} 
\end{quotation}


\begin{quotation} 

अथवा बहुनैतेन किं ज्ञातेन तवार्जुन  ।  

विष्टभ्याहमिदं कृत्स्नमेकांशेन स्थितो जगत्‌  ॥ १०.४२ ॥  मूल श्लोक
\end{quotation}

\begin{quotation}

अथवा बहुन एैतेन, किं ज्ञातेन तवार्जुन  ।  

विष्ट-अभ्याहम इदं कृत्स्-नम, एकांशेन स्थितो जगत्‌  ॥ १०.४२ ॥  उच्चारण

\noindent\rule{16cm}{0.4pt} 
\end{quotation}

\begin{center} ***** \end{center}


\begin{quotation} 


ॐ तत् सद इति श्री मद्-भगवद्-गीतास उपनिषत्सु ब्रह्म विद्यायां योगशास्त्रे श्री कृष्णार्जुन संवादे विभूतियोगो नाम दशमोऽ अध्यायः  ॥  १०  ॥ 
\end{quotation} 