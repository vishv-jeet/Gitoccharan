\chapter{\sanskrit आत्मसंयमयोग}



\paragraph{\sanskrit श्रीभगवानुवाच}
\begin{quotation}
अनाश्रितः कर्मफलं कार्यं कर्म करोति यः  ।  

स सन्न्यासी च योगी च न निरग्निर्न चाक्रियः  ॥ ६.१ ॥  मूल श्लोक
\end{quotation}

\begin{quotation}

अनाश्रितः कर्मफलं, कार्यं कर्म करोति यः  ।  

स सन्न्यासी च योगी च, न निर-अग्निर न चाक्रियः  ॥ ६.१ ॥  उच्चारण

\noindent\rule{16cm}{0.4pt} 
\end{quotation}


\begin{quotation}  

यं सन्न्यासमिति प्राहुर्योगं तं विद्धि पाण्डव  ।  

न ह्यसन्न्यस्तसङ्‍कल्पो योगी भवति कश्चन  ॥ ६.२ ॥  मूल श्लोक
\end{quotation}

\begin{quotation}

यं सन्न्यासम् इति प्राहुर, योगं तं विद्धि पाण्डव  ।  

न ह्य असन्न्यस्त सङ्‍कल्पो, योगी भवति कश्चन  ॥ ६.२ ॥  उच्चारण

\noindent\rule{16cm}{0.4pt} 
\end{quotation}


\begin{quotation}  

आरुरुक्षोर्मुनेर्योगं कर्म कारणमुच्यते  ।  

योगारूढस्य तस्यैव शमः कारणमुच्यते  ॥ ६.३ ॥  मूल श्लोक
\end{quotation}

\begin{quotation}

आरु-रुक्षोर मुनेर योगं, कर्म कारणम उच्यते  ।  

योगा रूढस्य तस्यैव, शमः कारणम उच्यते  ॥ ६.३ ॥  उच्चारण

\noindent\rule{16cm}{0.4pt} 
\end{quotation}


\begin{quotation}  

यदा हि नेन्द्रियार्थेषु न कर्मस्वनुषज्जते  ।  

सर्वसङ्‍कल्पसन्न्यासी योगारूढ़स्तदोच्यते  ॥ ६.४ ॥  मूल श्लोक
\end{quotation}

\begin{quotation}

यदा हि न एन्द्रिय-अर्थेषु, न कर्मस्व अनु-षज्जते  ।  

सर्व सङ्‍कल्प सन्न्यासी, योगारूढ़ः तद उच्यते  ॥ ६.४ ॥  उच्चारण

\noindent\rule{16cm}{0.4pt} 
\end{quotation}


\begin{quotation}  

उद्धरेदात्मनाऽत्मानं नात्मानमवसादयेत्‌  ।  

आत्मैव ह्यात्मनो बन्धुरात्मैव रिपुरात्मनः  ॥ ६.५ ॥  मूल श्लोक
\end{quotation}

\begin{quotation}

उद्धरेद आत्मनाऽ आत्मानं, न आत्मानम अवसादयेत्‌  ।  

आत्मैव ह्य आत्मनो बन्धुर, आत्मैव रिपुर-आत्मनः  ॥ ६.५ ॥  उच्चारण

\noindent\rule{16cm}{0.4pt} 
\end{quotation}


\begin{quotation}  
बन्धुरात्मात्मनस्तस्य येनात्मैवात्मना जितः  ।  

अनात्मनस्तु शत्रुत्वे वर्तेतात्मैव शत्रुवत्‌  ॥ ६.६ ॥  मूल श्लोक
\end{quotation}

\begin{quotation}

बन्धुर आत्मा-आत्मनस् तस्य, येन आत्मैव आत्मना-जितः  ।  

अनात्-मनस्तु शत्रुत्वे, वर्तेत-आत्मैव शत्रुवत्‌  ॥ ६.६ ॥  उच्चारण

\noindent\rule{16cm}{0.4pt} 
\end{quotation}


\begin{quotation}  

जितात्मनः प्रशान्तस्य परमात्मा समाहितः  ।  

शीतोष्णसुखदुःखेषु तथा मानापमानयोः  ॥ ६.७ ॥  मूल श्लोक
\end{quotation}

\begin{quotation}

जितात्मनः प्रशान्तस्य, परमात्मा समाहितः  ।  

शीतोष्ण सुख-दुःखेषु, तथा मान-अपमानयोः  ॥ ६.७ ॥  उच्चारण

\noindent\rule{16cm}{0.4pt} 
\end{quotation}


\begin{quotation}  

ज्ञानविज्ञानतृप्तात्मा कूटस्थो विजितेन्द्रियः  ।  

युक्त इत्युच्यते योगी समलोष्टाश्मकांचनः  ॥ ६.८ ॥  मूल श्लोक
\end{quotation}

\begin{quotation}

ज्ञान विज्ञान तृप्तात्मा, कूटस्थो विजित-एन्द्रियः  ।  

युक्त इती-उच्यते योगी, सम-लोष्टाश्म-कांचनः  ॥ ६.८ ॥  उच्चारण

\noindent\rule{16cm}{0.4pt} 
\end{quotation}


\begin{quotation}  

सुहृन्मित्रार्युदासीनमध्यस्थद्वेष्यबन्धुषु  ।  

साधुष्वपि च पापेषु समबुद्धिर्विशिष्यते  ॥ ६.९ ॥  मूल श्लोक
\end{quotation}

\begin{quotation}

सुह-ऋन् मित्रय-रि उदासीन, मध्यस्थ द्वेष्य बन्धुषु  ।  

साधुष्व अपि च पापेषु, सम-बुद्धिर विशिष्यते  ॥ ६.९ ॥  उच्चारण

\noindent\rule{16cm}{0.4pt} 
\end{quotation}


\begin{quotation}  

योगी युञ्जीत सततमात्मानं रहसि स्थितः  ।  

एकाकी यतचित्तात्मा निराशीरपरिग्रहः  ॥ ६.१० ॥  मूल श्लोक
\end{quotation}

\begin{quotation}

योगी युञ्जीत सततम्, आत्मानं रहसि स्थितः  ।  

एकाकी यत-चित्तात्मा, निराशीर परिग्रहः  ॥ ६.१० ॥  उच्चारण

\noindent\rule{16cm}{0.4pt} 
\end{quotation}


\begin{quotation}  

शुचौ देशे प्रतिष्ठाप्य स्थिरमासनमात्मनः  ।  

नात्युच्छ्रितं नातिनीचं चैलाजिनकुशोत्तरम्‌  ॥ ६.११ ॥  मूल श्लोक
\end{quotation}

\begin{quotation}

शुचौ देशे प्रतिष्ठाप्य, स्थिरम् आसनम् आत्मनः  ।  

नात्य उच्छ्रितं नाति-नीचं, चैलाजिन कुशोत्तरम्‌  ॥ ६.११ ॥  उच्चारण

\noindent\rule{16cm}{0.4pt} 
\end{quotation}


\begin{quotation}  
तत्रैकाग्रं मनः कृत्वा यतचित्तेन्द्रियक्रियः  ।  

उपविश्यासने युञ्ज्याद्योगमात्मविशुद्धये  ॥ ६.१२ ॥  मूल श्लोक
\end{quotation}

\begin{quotation}

तत्र-एैकाग्रं मनः कृत्वा, यत-चित्तेन्द्रिय-क्रियः  ।  

उप-विश्य-आसने युञ्ज्याद, योगम् आत्म विशुद्धये  ॥ ६.१२ ॥  उच्चारण

\noindent\rule{16cm}{0.4pt} 
\end{quotation}


\begin{quotation}  

समं कायशिरोग्रीवं धारयन्नचलं स्थिरः  ।  

सम्प्रेक्ष्य नासिकाग्रं स्वं दिशश्चानवलोकयन्‌  ॥ ६.१३ ॥  मूल श्लोक
\end{quotation}

\begin{quotation}

समं काय-शिरो-ग्रीवं, धारयन्न-अचलं स्थिरः  ।  

सम्प्रेक्ष्य नासिका अग्रं स्वं, दिशश् च अन-अवलोकयन्‌  ॥ ६.१३ ॥  उच्चारण

\noindent\rule{16cm}{0.4pt} 
\end{quotation}


\begin{quotation}  

प्रशान्तात्मा विगतभीर्ब्रह्मचारिव्रते स्थितः  ।  

मनः संयम्य मच्चित्तो युक्त आसीत मत्परः  ॥ ६.१४ ॥  मूल श्लोक
\end{quotation}

\begin{quotation}

प्रशान्तात्मा विगत-भीर, ब्रह्मचारि व्रते स्थितः  ।  

मनः संयम्य मच् चित्तो, युक्त आसीत मत्परः  ॥ ६.१४ ॥  उच्चारण

\noindent\rule{16cm}{0.4pt} 
\end{quotation}


\begin{quotation}  

युञ्जन्नेवं सदात्मानं योगी नियतमानसः  ।  

शान्तिं निर्वाणपरमां मत्संस्थामधिगच्छति  ॥ ६.१५ ॥  मूल श्लोक
\end{quotation}

\begin{quotation}

युञ्जन्ं एवं सदात्मानं, योगी नियत मानसः  ।  

शान्तिं निर्वाण परमां, मत-संस्थाम अधिगच्छति  ॥ ६.१५ ॥  उच्चारण

\noindent\rule{16cm}{0.4pt} 
\end{quotation}


\begin{quotation}  

नात्यश्नतस्तु योगोऽस्ति न चैकान्तमनश्नतः  ।  

न चाति स्वप्नशीलस्य जाग्रतो नैव चार्जुन  ॥ ६.१६ ॥  मूल श्लोक
\end{quotation}

\begin{quotation}

नात्य अश्-नत: तु योगोऽ अस्ति, न चैकान्तं अनश्-नतः  ।  

न चाति स्वप्न शीलस्य, जाग्रतो नैव चार्जुन  ॥ ६.१६ ॥  उच्चारण

\noindent\rule{16cm}{0.4pt} 
\end{quotation}


\begin{quotation}  

युक्ताहारविहारस्य युक्तचेष्टस्य कर्मसु  ।  

युक्तस्वप्नावबोधस्य योगो भवति दुःखहा  ॥ ६.१७ ॥  मूल श्लोक
\end{quotation}

\begin{quotation}

युक्ताहार विहारस्य, युक्त चेष्टस्य कर्मसु  ।  

युक्त स्वप्-नाव-बोधस्य, योगो भवति दुःखहा  ॥ ६.१७ ॥  उच्चारण

\noindent\rule{16cm}{0.4pt} 
\end{quotation}


\begin{quotation}  
यदा विनियतं चित्तमात्मन्येवावतिष्ठते  ।  

निःस्पृहः सर्वकामेभ्यो युक्त इत्युच्यते तदा  ॥ ६.१८ ॥  मूल श्लोक
\end{quotation}

\begin{quotation}

यदा विनियतं चित्तम्, आत्मन्य एवाअवतिष्ठते  ।  

निःस्पृहः सर्व कामेभ्यो, युक्त इत्य-उच्यते तदा  ॥ ६.१८ ॥  उच्चारण

\noindent\rule{16cm}{0.4pt} 
\end{quotation}


\begin{quotation}  

यथा दीपो निवातस्थो नेंगते सोपमा स्मृता  ।  

योगिनो यतचित्तस्य युञ्जतो योगमात्मनः  ॥ ६.१९ ॥  मूल श्लोक
\end{quotation}

\begin{quotation}

यथा दीपो निवात-स्थो, नेंगते सोपमा स्मृता  ।  

योगिनो यत-चित्तस्य, युञ्जतो योगं आत्मनः  ॥ ६.१९ ॥  उच्चारण

\noindent\rule{16cm}{0.4pt} 
\end{quotation}


\begin{quotation}  

यत्रोपरमते चित्तं निरुद्धं योगसेवया  ।  

यत्र चैवात्मनात्मानं पश्यन्नात्मनि तुष्यति  ॥ ६.२० ॥  मूल श्लोक
\end{quotation}

\begin{quotation}

यत्रो परमते चित्तं, निरुद्धं योग सेवया  ।  

यत्र चैव-आत्मना-आत्मानं, पश्यन्ना आत्मनि तुष्यति  ॥ ६.२० ॥  उच्चारण

\noindent\rule{16cm}{0.4pt} 
\end{quotation}


\begin{quotation}  

सुखमात्यन्तिकं यत्तद्‍बुद्धिग्राह्यमतीन्द्रियम्‌  ।  

वेत्ति यत्र न चैवायं स्थितश्चलति तत्त्वतः  ॥ ६.२१ ॥  मूल श्लोक
\end{quotation}

\begin{quotation}

सुखं आत्यन्तिकं यत् तद्‍, बुद्धि-ग्राह्यम् अति-इन्द्रियम्‌  ।  

वेत्ति यत्र न चैवायं, स्थितश् चलति तत्त्व-तः  ॥ ६.२१ ॥  उच्चारण

\noindent\rule{16cm}{0.4pt} 
\end{quotation}


\begin{quotation}  

यं लब्ध्वा चापरं लाभं मन्यते नाधिकं ततः  ।  

यस्मिन्स्थितो न दुःखेन गुरुणापि विचाल्यते  ॥ ६.२२ ॥  मूल श्लोक
\end{quotation}

\begin{quotation}

यं लब्ध्वा चापरं लाभं, मन्यते नाधिकं ततः  ।  

यस्मिं स्थितो न दुःखेन, गुरुणा अपि विचाल्यते  ॥ ६.२२ ॥  उच्चारण

\noindent\rule{16cm}{0.4pt} 
\end{quotation}


\begin{quotation}  

तं विद्याद् दुःखसंयोगवियोगं योगसञ्ज्ञितम् ।  

स निश्चयेन योक्तव्यो योगोऽनिर्विण्णचेतसा  ॥ ६.२३ ॥  मूल श्लोक
\end{quotation}

\begin{quotation}

तं विद्याद् दुःख संयोग, वियोगं योग सञ्ज्ञितम् ।  

स निश्चयेन योक्-तव्यो, योगोऽ अनिर्-विण्ण चेतसा  ॥ ६.२३ ॥  उच्चारण

\noindent\rule{16cm}{0.4pt} 
\end{quotation}


\begin{quotation}  
सङ्‍कल्पप्रभवान्कामांस्त्यक्त्वा सर्वानशेषतः  ।  

मनसैवेन्द्रियग्रामं विनियम्य समन्ततः  ॥ ६.२४ ॥  मूल श्लोक
\end{quotation}

\begin{quotation}

सङ्‍कल्प प्रभवान् कामांस, त्यक्त्वा सर्वान अशेषतः  ।  

मनस एव इन्द्रिय ग्रामं, विनियम्य समन्ततः  ॥ ६.२४ ॥  उच्चारण

\noindent\rule{16cm}{0.4pt} 
\end{quotation}


\begin{quotation}  

शनैः शनैरुपरमेद्‍बुद्धया धृतिगृहीतया ।  

आत्मसंस्थं मनः कृत्वा न किंचिदपि चिन्तयेत्‌  ॥ ६.२५ ॥  मूल श्लोक
\end{quotation}

\begin{quotation}

शनैः शनैर उपरमेद्‍, बुद्धया धृति गृहीतया ।  

आत्म-संस्थम् मनः कृत्वा, न किंचिद अपि चिन्तयेत्‌  ॥ ६.२५ ॥  उच्चारण

\noindent\rule{16cm}{0.4pt} 
\end{quotation}


\begin{quotation}  

यतो यतो निश्चरति मनश्चञ्चलमस्थिरम्‌  ।  

ततस्ततो नियम्यैतदात्मन्येव वशं नयेत्‌  ॥ ६.२६ ॥  मूल श्लोक
\end{quotation}

\begin{quotation}

यतो यतो निश्चरति, मन चंचलम् अस्थिरम्‌  ।  

ततस् ततो नियम्य-ऐतद, आत्मन्य एव वशं नयेत्‌  ॥ ६.२६ ॥  उच्चारण

\noindent\rule{16cm}{0.4pt} 
\end{quotation}


\begin{quotation}  

प्रशान्तमनसं ह्येनं योगिनं सुखमुत्तमम्‌  ।  

उपैति शांतरजसं ब्रह्मभूतमकल्मषम्‌  ॥ ६.२७ ॥  मूल श्लोक
\end{quotation}

\begin{quotation}

प्रशान्त मनसं ह्य इनं, योगिनं सुखं उत्तमम्‌  ।  

उपैति शांत रजसं, ब्रह्म भूतम अकल्मषम्‌  ॥ ६.२७ ॥  उच्चारण

\noindent\rule{16cm}{0.4pt} 
\end{quotation}


\begin{quotation}  

युञ्जन्नेवं सदात्मानं योगी विगतकल्मषः  ।  

सुखेन ब्रह्मसंस्पर्शमत्यन्तं सुखमश्नुते  ॥ ६.२८ ॥  मूल श्लोक
\end{quotation}

\begin{quotation}

युञ्जन्न एवं सदात्मानं, योगी विगत कल्मषः  ।  

सुखेन ब्रह्म संस्पर्शम, अत्यन्तं सुखं अश्-नुते  ॥ ६.२८ ॥  उच्चारण

\noindent\rule{16cm}{0.4pt} 
\end{quotation}


\begin{quotation}  

सर्वभूतस्थमात्मानं सर्वभूतानि चात्मनि  ।  

ईक्षते योगयुक्तात्मा सर्वत्र समदर्शनः  ॥ ६.२९ ॥  मूल श्लोक
\end{quotation}

\begin{quotation}

सर्व-भूत-स्थं आत्मानं, सर्व-भूतानि चात्मनि  ।  

ईक्षते योग-युक्तात्मा, सर्वत्र सम दर्शनः  ॥ ६.२९ ॥  उच्चारण

\noindent\rule{16cm}{0.4pt} 
\end{quotation}


\begin{quotation}  
यो मां पश्यति सर्वत्र सर्वं च मयि पश्यति  ।  

तस्याहं न प्रणश्यामि स च मे न प्रणश्यति  ॥ ६.३० ॥  मूल श्लोक
\end{quotation}

\begin{quotation}

यो माम पश्यति सर्वत्र, सर्वं च मयि पश्यति  ।  

तस्याहं न प्रणश्यामि, स च मे न प्रणश्यति  ॥ ६.३० ॥  उच्चारण

\noindent\rule{16cm}{0.4pt} 
\end{quotation}


\begin{quotation}  

सर्वभूतस्थितं यो मां भजत्येकत्वमास्थितः  ।  

सर्वथा वर्तमानोऽपि स योगी मयि वर्तते  ॥ ६.३१ ॥  मूल श्लोक
\end{quotation}

\begin{quotation}

सर्व-भूत-स्थितं यो माम, भजत्य एकत्वम स्थितः  ।  

सर्वथा वर्तमानोऽ अपि स, योगी मयि वर्तते  ॥ ६.३१ ॥  उच्चारण

\noindent\rule{16cm}{0.4pt} 
\end{quotation}


\begin{quotation}  

आत्मौपम्येन सर्वत्र समं पश्यति योऽर्जुन  ।  

सुखं वा यदि वा दुःखं स योगी परमो मतः  ॥ ६.३२ ॥  मूल श्लोक
\end{quotation}

\begin{quotation}

आत्मौपम्-येन सर्वत्र, समं पश्यति योऽ अर्जुन  ।  

सुखं वा यदि वा दुःखं, स योगी परमो मतः  ॥ ६.३२ ॥  उच्चारण

\noindent\rule{16cm}{0.4pt} 
\end{quotation}


\paragraph{\sanskrit अर्जुन उवाच}
\begin{quotation}  



योऽयं योगस्त्वया प्रोक्तः साम्येन मधुसूदन  ।  

एतस्याहं न पश्यामि चञ्चलत्वात्स्थितिं स्थिराम्‌  ॥ ६.३३ ॥  मूल श्लोक
\end{quotation}

\begin{quotation}

योऽ अयं योगस त्वया प्रोक्तः, साम्येन मधु-सूदन  ।  

एतस्या अहं न पश्यामि, चञ्चल-त्वा स्थितिं स्थिराम्‌  ॥ ६.३३ ॥  उच्चारण

\noindent\rule{16cm}{0.4pt} 
\end{quotation}


\begin{quotation}  

चञ्चलं हि मनः कृष्ण प्रमाथि बलवद्दृढम्‌  ।  

तस्याहं निग्रहं मन्ये वायोरिव सुदुष्करम्‌  ॥ ६.३४ ॥  मूल श्लोक
\end{quotation}

\begin{quotation}

चञ्चलं हि मनः कृष्ण, प्रमाथि बलवद् दृढ़म   ।  

तस्याहं निग्रहं मन्ये, वायुर एव सु-दुष्करम्‌  ॥ ६.३४ ॥  उच्चारण

\noindent\rule{16cm}{0.4pt} 
\end{quotation}

\paragraph{\sanskrit श्रीभगवानुवाच}

\begin{quotation}  




असंशयं महाबाहो मनो दुर्निग्रहं चलम्‌  ।  

अभ्यासेन तु कौन्तेय वैराग्येण च गृह्यते  ॥ ६.३५ ॥  मूल श्लोक
\end{quotation}

\begin{quotation}

असंशयं महाबाहो, मनो दुर्-निग्रहं चलम्‌  ।  

अभ्यासेन तु कौन्तेय, वैराग्येण च गृह्यते  ॥ ६.३५ ॥  उच्चारण

\noindent\rule{16cm}{0.4pt} 
\end{quotation}


\begin{quotation}  

असंयतात्मना योगो दुष्प्राप इति मे मतिः  ।  

वश्यात्मना तु यतता शक्योऽवाप्तुमुपायतः  ॥ ६.३६ ॥  मूल श्लोक
\end{quotation}

\begin{quotation}

असम्यत आत्मना योगो, दुष्प्राप इति मे मतिः  ।  

वश्य आत्मना तु यतता, शक्योऽ अवाप्तुम् उपायतः  ॥ ६.३६ ॥  उच्चारण

\noindent\rule{16cm}{0.4pt} 
\end{quotation}


\paragraph{\sanskrit अर्जुन उवाच}
\begin{quotation}  



अयतिः श्रद्धयोपेतो योगाच्चलितमानसः  ।  

अप्राप्य योगसंसिद्धिं कां गतिं कृष्ण गच्छति  ॥ ६.३७ ॥  मूल श्लोक
\end{quotation}

\begin{quotation}

अयतिः श्रद्धय-उपेतो, योगाच् चलित मानसः  ।  

अप्राप्य योग संसिद्धिं, कां गतिं कृष्ण गच्छति  ॥ ६.३७ ॥  उच्चारण

\noindent\rule{16cm}{0.4pt} 
\end{quotation}


\begin{quotation}  

कच्चिन्नोभयविभ्रष्टश्छिन्नाभ्रमिव नश्यति  ।  

अप्रतिष्ठो महाबाहो विमूढो ब्रह्मणः पथि  ॥ ६.३८ ॥  मूल श्लोक
\end{quotation}

\begin{quotation}

कच्चिन् नौभय विभ्रष्टः, छिन्नाभ्रम इव नश्यति  ।  

अप्रतिष्ठो महाबाहो, विमूढो ब्रह्मणः पथि  ॥ ६.३८ ॥  उच्चारण

\noindent\rule{16cm}{0.4pt} 
\end{quotation}


\begin{quotation}  

एतन्मे संशयं कृष्ण छेत्तुमर्हस्यशेषतः  ।  

त्वदन्यः संशयस्यास्य छेत्ता न ह्युपपद्यते  ॥ ६.३९ ॥  मूल श्लोक
\end{quotation}

\begin{quotation}

एतन्मे संशयं कृष्ण, छेत्तुं अर्हस्य अशेषतः  ।  

त्वदन्यः संशयस्य-अस्य, छेत्ता न ह्य उपपद्यते  ॥ ६.३९ ॥  उच्चारण

\noindent\rule{16cm}{0.4pt} 
\end{quotation}


\paragraph{\sanskrit श्रीभगवानुवाच}
\begin{quotation}  


पार्थ नैवेह नामुत्र विनाशस्तस्य विद्यते  ।  

न हि कल्याणकृत्कश्चिद्दुर्गतिं तात गच्छति  ॥ ६.४० ॥  मूल श्लोक
\end{quotation}

\begin{quotation}

पार्थ नैवेह ना अमुत्र, विनाशस्त अस्य विद्यते  ।  

न हि कल्याण-कृत कश्चिद, दुर्गतिं तात गच्छति  ॥ ६.४० ॥  उच्चारण

\noindent\rule{16cm}{0.4pt} 
\end{quotation}


\begin{quotation}  

प्राप्य पुण्यकृतां लोकानुषित्वा शाश्वतीः समाः  ।  

शुचीनां श्रीमतां गेहे योगभ्रष्टोऽभिजायते  ॥ ६.४१ ॥  मूल श्लोक
\end{quotation}

\begin{quotation}

प्राप्य पुण्य कृतां लोकान, उषित्वा शाश्वतीः समाः  ।  

शुचीनां श्रीमतां गेहे, योगभ्रष्टोऽ अभिजायते  ॥ ६.४१ ॥  उच्चारण

\noindent\rule{16cm}{0.4pt} 
\end{quotation}


\begin{quotation}  

अथवा योगिनामेव कुले भवति धीमताम्‌  ।  

एतद्धि दुर्लभतरं लोके जन्म यदीदृशम्‌  ॥ ६.४२ ॥  मूल श्लोक
\end{quotation}

\begin{quotation}

अथवा योगिनाम एव, कुले भवति धीमताम्‌  ।  

एतद धी दुर्लभ-तरं, लोके जन्म यद इदृशम्‌  ॥ ६.४२ ॥  उच्चारण

\noindent\rule{16cm}{0.4pt} 
\end{quotation}


\begin{quotation}  

तत्र तं बुद्धिसंयोगं लभते पौर्वदेहिकम्‌  ।  

यतते च ततो भूयः संसिद्धौ कुरुनन्दन  ॥ ६.४३ ॥  मूल श्लोक
\end{quotation}

\begin{quotation}

तत्र तं बुद्धि संयोगं, लभते पौर्व देहिकम्‌  ।  

यतते च ततो भूयः, संसिद्धौ कुरु नन्दन  ॥ ६.४३ ॥  उच्चारण

\noindent\rule{16cm}{0.4pt} 
\end{quotation}


\begin{quotation}  

पूर्वाभ्यासेन तेनैव ह्रियते ह्यवशोऽपि सः  ।  

जिज्ञासुरपि योगस्य शब्दब्रह्मातिवर्तते  ॥ ६.४४ ॥  मूल श्लोक
\end{quotation}

\begin{quotation}

पूर्व-अभ्यासेन तेन एैव, ह्रियते ह्य अवशोऽ अपि सः  ।  

जिज्ञासुर अपि योगस्य, शब्द ब्रह्माति वर्तते  ॥ ६.४४ ॥  उच्चारण

\noindent\rule{16cm}{0.4pt} 
\end{quotation}


\begin{quotation}  

प्रयत्नाद्यतमानस्तु योगी संशुद्धकिल्बिषः  ।  

अनेकजन्मसंसिद्धस्ततो यात परां गतिम्‌  ॥ ६.४५ ॥  मूल श्लोक
\end{quotation}

\begin{quotation}
प्रयत्न-आद यत-मानस्तु, योगी संशुद्ध किल्बिषः  ।  

अनेक जन्म संसिद्धः, ततो यात परां गतिम्‌  ॥ ६.४५ ॥  उच्चारण

\noindent\rule{16cm}{0.4pt} 
\end{quotation}


\begin{quotation}  

तपस्विभ्योऽधिको योगी ज्ञानिभ्योऽपि मतोऽधिकः  ।  

कर्मिभ्यश्चाधिको योगी तस्माद्योगी भवार्जुन  ॥ ६.४६ ॥  मूल श्लोक
\end{quotation}

\begin{quotation}

तपस-विभ्योऽ अधिको योगी, ज्ञानिभ्योऽ अपि मतोऽ अधिकः  ।  

कर्मिभ्य: चाधिको योगी, तस्माद् योगी भवार्जुन  ॥ ६.४६ ॥  उच्चारण

\noindent\rule{16cm}{0.4pt} 
\end{quotation}


\begin{quotation}  

योगिनामपि सर्वेषां मद्गतेनान्तरात्मना  ।  

श्रद्धावान्भजते यो मां स मे युक्ततमो मतः  ॥ ६.४७ ॥  मूल श्लोक
\end{quotation}

\begin{quotation}

योगिनाम् अपि सर्वेषां, मद्-गतेन आन्तर-आत्मना  ।  

श्रद्धावान् भजते यो माम, स मे युक्त-तमो मतः  ॥ ६.४७ ॥  उच्चारण

\noindent\rule{16cm}{0.4pt} 
\end{quotation}

\begin{center} ***** \end{center}


\begin{quotation}  


ॐ तत् सद इति श्री मद्-भगवद्-गीतास उपनिषत्सु ब्रह्म विद्यायां योगशास्त्रे श्री कृष्णार्जुन संवादे आत्मसंयमयोगो नाम षष्ठोऽ अध्यायः  ॥  ६  ॥ 

\end{quotation}