\chapter{\sanskrit भक्तियोग} 
\paragraph{\sanskrit अर्जुन उवाच}
\begin{quotation} 
एवं सततयुक्ता ये भक्तास्त्वां पर्युपासते  ।  

ये चाप्यक्षरमव्यक्तं तेषां के योगवित्तमाः  ॥ १२.१ ॥  मूल श्लोक
\end{quotation}

\begin{quotation}

एवं सतत युक्ता ये, भक्तास त्वां पर्युपासते  ।  

ये चाप्य अक्षरम अव्यक्तं, तेषां के योग-वित्तमाः  ॥ १२.१ ॥  उच्चारण

\noindent\rule{16cm}{0.4pt} 
\end{quotation}



\paragraph{\sanskrit श्रीभगवानुवाच}
\begin{quotation} 
मय्यावेश्य मनो ये मां नित्ययुक्ता उपासते  ।  

श्रद्धया परयोपेतास्ते मे युक्ततमा मताः  ॥ १२.२ ॥  मूल श्लोक
\end{quotation}

\begin{quotation}

मय्य आवेश्य मनो ये मां, नित्य-युक्ता उपासते  ।  

श्रद्धया परय-उपेतास्, ते मे युक्त-तमा मताः  ॥ १२.२ ॥  उच्चारण

\noindent\rule{16cm}{0.4pt} 
\end{quotation}


\begin{quotation} 

ये त्वक्षरमनिर्देश्यमव्यक्तं पर्युपासते ।  

सर्वत्रगमचिन्त्यं च कूटस्थमचलं ध्रुवम्‌  ॥ १२.३ ॥  मूल श्लोक
\end{quotation}

\begin{quotation}

ये त्व अक्षरम निर्देश्यम, अव्यक्तं पर्युपासते ।  

सर्वत्र-गम अचिन्त्यं च, कूटस्थम अचलं ध्रुवम्‌  ॥ १२.३ ॥  उच्चारण

\noindent\rule{16cm}{0.4pt} 
\end{quotation}


\begin{quotation} 

सन्नियम्येन्द्रियग्रामं सर्वत्र समबुद्धयः  ।  

ते प्राप्नुवन्ति मामेव सर्वभूतहिते रताः  ॥ १२.४ ॥  मूल श्लोक
\end{quotation}

\begin{quotation}

सन्नियम्य इन्द्रिय ग्रामं, सर्वत्र सम-बुद्धयः  ।  

ते प्राप्-नुवन्ति मामेव, सर्वभूत हिते रताः  ॥ १२.४ ॥  उच्चारण

\noindent\rule{16cm}{0.4pt} 
\end{quotation}


\begin{quotation} 

क्लेशोऽधिकतरस्तेषामव्यक्तासक्तचेतसाम्‌  ।  

अव्यक्ता हि गतिर्दुःखं देहवद्भिरवाप्यते  ॥ १२.५ ॥  मूल श्लोक
\end{quotation}

\begin{quotation}

क्लेशोऽ अधिक-तरस तेषाम, अव्यक्ता सक्त चेतसाम्‌  ।  

अव्यक्ता हि गतिर दुःखं, देह-वदभिर अवाप्यते  ॥ १२.५ ॥  उच्चारण

\noindent\rule{16cm}{0.4pt} 
\end{quotation}


\begin{quotation} 

ये तु सर्वाणि कर्माणि मयि सन्नयस्य मत्पराः  ।  

अनन्येनैव योगेन मां ध्यायन्त उपासते  ॥ १२.६ ॥  मूल श्लोक
\end{quotation}

\begin{quotation}

ये तु सर्वाणि कर्माणि, मयि सन्नयस्य मत्पराः  ।  

अनन्येन एैव योगेन, मां ध्यायन्त उपासते  ॥ १२.६ ॥  उच्चारण

\noindent\rule{16cm}{0.4pt} 
\end{quotation}


\begin{quotation} 

तेषामहं समुद्धर्ता मृत्युसंसारसागरात्‌  ।  

भवामि नचिरात्पार्थ मय्यावेशितचेतसाम्‌  ॥ १२.७ ॥  मूल श्लोक
\end{quotation}

\begin{quotation}

तेषाम अहं समुद्धर्ता, मृत्यु संसार सागरात्‌  ।  

भवामि न चिरात पार्थ, मय्य आवेशित चेतसाम्‌  ॥ १२.७ ॥  उच्चारण

\noindent\rule{16cm}{0.4pt} 
\end{quotation}


\begin{quotation} 

मय्येव मन आधत्स्व मयि बुद्धिं निवेशय  ।  

निवसिष्यसि मय्येव अत ऊर्ध्वं न संशयः  ॥ १२.८ ॥  मूल श्लोक
\end{quotation}

\begin{quotation}

मय्येव मन आधत्-स्व, मयि बुद्धिं निवेशय  ।  

निवस-इष्यसि मय्येव, अत ऊर्ध्वं न संशयः  ॥ १२.८ ॥  उच्चारण

\noindent\rule{16cm}{0.4pt} 
\end{quotation}


\begin{quotation} 

अथ चित्तं समाधातुं न शक्रोषि मयि स्थिरम्‌  ।  

अभ्यासयोगेन ततो मामिच्छाप्तुं धनञ्जय  ॥ १२.९ ॥  मूल श्लोक
\end{quotation}

\begin{quotation}

अथ चित्तं समा-धातुं, न शक-नोषि मयि स्थिरम्‌  ।  

अभ्यास योगेन ततो, माम इच्छा-अप्तुं धनञ्जय  ॥ १२.९ ॥  उच्चारण

\noindent\rule{16cm}{0.4pt} 
\end{quotation}


\begin{quotation} 

अभ्यासेऽप्यसमर्थोऽसि मत्कर्मपरमो भव  ।  

मदर्थमपि कर्माणि कुर्वन्सिद्धिमवाप्स्यसि  ॥ १२.१० ॥  मूल श्लोक
\end{quotation}

\begin{quotation}

अभ्यासेऽ अप्य असमर्थोऽ असि, मत्कर्म परमो भव  ।  

मद्-अर्थम अपि कर्माणि, कुर्वन सिद्धिम अवाप्स्यसि  ॥ १२.१० ॥  उच्चारण

\noindent\rule{16cm}{0.4pt} 
\end{quotation}


\begin{quotation} 

अथैतदप्यशक्तोऽसि कर्तुं मद्योगमाश्रितः  ।  

सर्वकर्मफलत्यागं ततः कुरु यतात्मवान्‌  ॥ १२.११ ॥  मूल श्लोक
\end{quotation}

\begin{quotation}
अथ-एैतद-अप्य अशक्तोऽ असि, कर्तुं मद-योगम आश्रितः  ।  

सर्व कर्म फल त्यागं, ततः कुरु यत-आत्मवान्‌  ॥ १२.११ ॥  उच्चारण

\noindent\rule{16cm}{0.4pt} 
\end{quotation}


\begin{quotation} 

श्रेयो हि ज्ञानमभ्यासाज्ज्ञानाद्धयानं विशिष्यते  ।  

ध्यानात्कर्मफलत्यागस्त्यागाच्छान्तिरनन्तरम्‌  ॥ १२.१२ ॥  मूल श्लोक
\end{quotation}

\begin{quotation}

श्रेयो हि ज्ञानम अभ्यासाज, ज्ञानाद ध्यानं विशिष्यते  ।  

ध्यानात कर्मफल त्यागस,  त्यागाच छान्तिर अनन्तरम्‌  ॥ १२.१२ ॥  उच्चारण

\noindent\rule{16cm}{0.4pt} 
\end{quotation}




\paragraph{\sanskrit अर्जुन उवाच}
\begin{quotation} 
अद्वेष्टा सर्व-भूतानां मैत्रः करुण एव च  ।  

निर्ममो निरहङ्‍कारः समदुःखसुखः क्षमी  ॥ १२.१३ ॥  मूल श्लोक
\end{quotation}

\begin{quotation}

अद्वेष्टा सर्व-भूतानां, मैत्रः करुण एव च  ।  

निर्ममो निर-अहंकारः, सम दुःख सुखः क्षमी  ॥ १२.१३ ॥  उच्चारण

\noindent\rule{16cm}{0.4pt} 
\end{quotation}


\begin{quotation} 

संतुष्टः सततं योगी यतात्मा दृढ़निश्चयः ।  

मय्यर्पितमनोबुद्धिर्यो मद्भक्तः स मे प्रियः  ॥ १२.१४ ॥  मूल श्लोक
\end{quotation}

\begin{quotation}

संतुष्टः सततं योगी, यतात्मा दृढ़-निश्चयः ।  

मय्य अर्पित मनो बुद्धिर्यो, मद-भक्तः स मे प्रियः  ॥ १२.१४ ॥  उच्चारण

\noindent\rule{16cm}{0.4pt} 
\end{quotation}

\paragraph{\sanskrit श्रीभगवानुवाच}

\begin{quotation} 


यस्मान्नोद्विजते लोको लोकान्नोद्विजते च यः ।  

हर्षामर्षभयोद्वेगैर्मुक्तो यः स च मे प्रियः  ॥ १२.१५ ॥  मूल श्लोक
\end{quotation}

\begin{quotation}

यस्मान न उद्-विजते लोको, लोकान न उद्-विजते च यः ।  

हर्षामर्ष भयोद्-वेगैर, मुक्तो यः स च मे प्रियः  ॥ १२.१५ ॥  उच्चारण

\noindent\rule{16cm}{0.4pt} 
\end{quotation}


\begin{quotation} 


अनपेक्षः शुचिर्दक्ष उदासीनो गतव्यथः ।  

सर्वारम्भपरित्यागी यो मद-भक्तः स मे प्रियः  ॥ १२.१६ ॥  मूल श्लोक
\end{quotation}

\begin{quotation}

अनपेक्षः शुचिर दक्ष, उदासीनो गत व्यथः ।  

सर्वारम्भ परित्यागी, यो मद्भक्तः स मे प्रियः  ॥ १२.१६ ॥  उच्चारण

\noindent\rule{16cm}{0.4pt} 
\end{quotation}


\begin{quotation} 

यो न हृष्यति न द्वेष्टि न शोचति न काङ्‍क्षति ।  

शुभाशुभपरित्यागी भक्तिमान्यः स मे प्रियः ॥ १२.१७ ॥  मूल श्लोक
\end{quotation}

\begin{quotation}

यो न हृष्यति न द्वेष्टि, न शोचति न काङ्‍क्षति ।  

शुभा शुभ परित्यागी, भक्ति मान्यः स मे प्रियः  ॥ १२.१७ ॥  उच्चारण

\noindent\rule{16cm}{0.4pt} 
\end{quotation}


\begin{quotation} 

समः शत्रौ च मित्रे च तथा मानापमानयोः ।  

शीतोष्णसुखदुःखेषु समः सङ्‍गविवर्जितः  ॥ १२.१८ ॥  मूल श्लोक
\end{quotation}

\begin{quotation}

समः शत्रौ च मित्रे च, तथा मान-आपमानयोः ।  

शीतोष्ण सुख दुःखेषु, समः सङ्‍ग-विवर्जितः  ॥ १२.१८ ॥  उच्चारण

\noindent\rule{16cm}{0.4pt} 
\end{quotation}


\begin{quotation} 

तुल्यनिन्दास्तुतिर्मौनी सन्तुष्टो येन केनचित्‌ ।  

अनिकेतः स्थिरमतिर्भक्तिमान्मे प्रियो नरः  ॥ १२.१९ ॥  मूल श्लोक
\end{quotation}

\begin{quotation}

तुल्य निन्दा स्तुतिर मौनी, सन्तुष्टो येन केनचित्‌ ।  

अनिकेतः स्थिर-मतिर, भक्तिमान मे प्रियो नरः  ॥ १२.१९ ॥  उच्चारण

\noindent\rule{16cm}{0.4pt} 
\end{quotation}


\begin{quotation} 

ये तु धर्म्यामृतमिदं यथोक्तं पर्युपासते ।  

श्रद्धाना मत्परमा भक्तास्तेऽतीव मे प्रियाः  ॥ १२.२० ॥  मूल श्लोक
\end{quotation}

\begin{quotation}

ये तु धर्म्या-अमृतम इदं, यथ उक्तं पर्युपासते ।  

श्रद्धाना मत्परमा, भक्तास्तेऽ अतीव मे प्रियाः  ॥ १२.२० ॥  उच्चारण

\noindent\rule{16cm}{0.4pt} 
\end{quotation}

\begin{center} ***** \end{center}
\begin{quotation} 


ॐ तत् सद इति श्री मद्-भगवद्-गीतास उपनिषत्सु ब्रह्म विद्यायां योगशास्त्रे श्री कृष्णार्जुन संवादे भक्तियोगो नाम द्वादशोऽ अध्यायः  ॥  १२  ॥ 
\end{quotation} 