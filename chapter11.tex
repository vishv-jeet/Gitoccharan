\chapter{\sanskrit विश्वरूपदर्शनयोग}
\paragraph{\sanskrit अर्जुन उवाच}
\begin{quotation} 
मदनुग्रहाय परमं गुह्यमध्यात्मसञ्ज्ञितम्‌  ।  

यत्त्वयोक्तं वचस्तेन मोहोऽयं विगतो मम  ॥ ११.१ ॥  मूल श्लोक
\end{quotation}

\begin{quotation}

मद अनुग्रहाय परमं, गुह्यम अध्यात्म सञ्ज्ञितम्‌  ।  

यत्त्व योक्तं वचस्तेन, मोहोऽ अयं विगतो मम  ॥ ११.१ ॥  उच्चारण

\noindent\rule{16cm}{0.4pt} 
\end{quotation}


\begin{quotation} 

भवाप्ययौ हि भूतानां श्रुतौ विस्तरशो मया  ।  

त्वतः कमलपत्राक्ष माहात्म्यमपि चाव्ययम्‌  ॥ ११.२ ॥  मूल श्लोक
\end{quotation}

\begin{quotation}

भवा अप्ययौ हि भूतानां, श्रुतौ विस्तरशो मया  ।  

त्वतः कमल-पत्राक्ष, माहात्म्-यम अपि च अ-व्ययम्‌  ॥ ११.२ ॥  उच्चारण

\noindent\rule{16cm}{0.4pt} 
\end{quotation}


\begin{quotation} 

एवमेतद्यथात्थ त्वमात्मानं परमेश्वर  ।  

द्रष्टुमिच्छामि ते रूपमैश्वरं पुरुषोत्तम  ॥ ११.३ ॥  मूल श्लोक
\end{quotation}

\begin{quotation}

एवम एतद यथात्थ त्वम, आत्मानं परमेश्वर  ।  

द्रष्टुम इच्छामि ते रूपम, एैश्वरं पुरुषोत्तम  ॥ ११.३ ॥  उच्चारण

\noindent\rule{16cm}{0.4pt} 
\end{quotation}


\begin{quotation} 

मन्यसे यदि तच्छक्यं मया द्रष्टुमिति प्रभो  ।  

योगेश्वर ततो मे त्वं दर्शयात्मानमव्ययम्‌  ॥ ११.४ ॥  मूल श्लोक
\end{quotation}

\begin{quotation}

मन्यसे यदि तच छक्यं, मया द्रष्टुम इति प्रभो  ।  

योगेश्वर ततो मे त्वं, दर्शय-आत्मानम अ-व्ययम्‌  ॥ ११.४ ॥  उच्चारण

\noindent\rule{16cm}{0.4pt} 
\end{quotation}




\paragraph{\sanskrit श्रीभगवानुवाच}
\begin{quotation} 
पश्य मे पार्थ रूपाणि शतशोऽथ सहस्रशः  ।  

नानाविधानि दिव्यानि नानावर्णाकृतीनि च  ॥ ११.५ ॥  मूल श्लोक
\end{quotation}

\begin{quotation}
पश्य मे पार्थ रूपाणि, शतशोऽ अथ सहस्र-शः  ।  

नाना विधानि दिव्यानि, नाना वर्णा-कृतीनि च  ॥ ११.५ ॥  उच्चारण

\noindent\rule{16cm}{0.4pt} 
\end{quotation}


\begin{quotation} 

पश्यादित्यान्वसून्रुद्रानश्विनौ मरुतस्तथा  ।  

बहून्यदृष्टपूर्वाणि पश्याश्चर्याणि भारत  ॥ ११.६ ॥  मूल श्लोक
\end{quotation}

\begin{quotation}

पश्य आदित्यान वसून रूद्रान, अश्विनौ मरुतस् तथा  ।  

बहून्य अदृष्ट पूर्वाणि, पश्याश्-चर्याणि भारत  ॥ ११.६ ॥  उच्चारण

\noindent\rule{16cm}{0.4pt} 
\end{quotation}


\begin{quotation} 

इहैकस्थं जगत्कृत्स्नं पश्याद्य सचराचरम्‌  ।  

मम देहे गुडाकेश यच्चान्यद्द्रष्टमिच्छसि  ॥ ११.७ ॥  मूल श्लोक
\end{quotation}

\begin{quotation}

इहैक-स्थं जगत-कृत्-सन्म् , पश्याद्य सचरा-चरम्‌  ।  

मम देहे गुडाकेश, यच् चान्यद द्रष्टम इच्छसि  ॥ ११.७ ॥  उच्चारण

\noindent\rule{16cm}{0.4pt} 
\end{quotation}


\begin{quotation} 

न तु मां शक्यसे द्रष्टुमनेनैव स्वचक्षुषा ।  

दिव्यं ददामि ते चक्षुः पश्य मे योगमैश्वरम्  ॥ ११.८ ॥  मूल श्लोक
\end{quotation}

\begin{quotation}

न तु मां शक्यसे द्रष्टुम, अनेनैव स्व-चक्षुषा  ।  

दिव्यं ददामि ते चक्षुः, पश्य मे योगम एैश्वरम्‌  ॥ ११.८ ॥  उच्चारण

\noindent\rule{16cm}{0.4pt} 
\end{quotation}


\paragraph{\sanskrit सञ्जय उवाच}
\begin{quotation} 


एवमुक्त्वा ततो राजन्महायोगेश्वरो हरिः ।  

दर्शयामास पार्थाय परमं रूपमैश्वरम्  ॥ ११.९ ॥  मूल श्लोक
\end{quotation}

\begin{quotation}

एवम उक्त्वा ततो राजन, महा-योगेश्वरो हरिः  ।  

दर्शयाम आस पार्थाय, परमं रूपम एैश्वरम्  ॥ ११.९ ॥  उच्चारण

\noindent\rule{16cm}{0.4pt} 
\end{quotation}

%\noindent\rule{16cm}{0.4pt}
\begin{quotation} 

अनेकवक्त्रनयनमनेकाद्भुतदर्शनम्‌  ।  

अनेकदिव्याभरणं दिव्यानेकोद्यतायुधम्‌  ॥ ११.१० ॥  मूल श्लोक
\end{quotation}

\begin{quotation}

अनेक वक्त्र नयनम, अनेक अद्भुत दर्शनम्‌  ।  

अनेक दिव्य अभरणं, दिव्य अनेक उद्यत अयुधम्‌  ॥ ११.१० ॥  उच्चारण

\noindent\rule{16cm}{0.4pt} 
\end{quotation}

%\noindent\rule{16cm}{0.4pt}
\begin{quotation} 
दिव्यमाल्याम्बरधरं दिव्यगन्धानुलेपनम्‌  ।  

सर्वाश्चर्यमयं देवमनन्तं विश्वतोमुखम्‌  ॥ ११.११ ॥  मूल श्लोक
\end{quotation}

\begin{quotation}

दिव्य माल्य अम्बर धरं, दिव्य गन्ध अनु-लेपनम्‌  ।  

सर्व-आश्चर्य-मयं देवम, अनन्तं विश्वतो मुखम्‌  ॥ ११.११ ॥  उच्चारण

\noindent\rule{16cm}{0.4pt} 
\end{quotation}


\begin{quotation} 

दिवि सूर्यसहस्रस्य भवेद्युगपदुत्थिता  ।  

यदि भाः सदृशी सा स्याद्भासस्तस्य महात्मनः  ॥ ११.१२ ॥  मूल श्लोक
\end{quotation}

\begin{quotation}

दिवि सूर्य सहस्र-अस्य, भवेद् युगपद उत्थिता  ।  

यदि भाः सदृशी सा स्याद, भासस तस्य महात्मनः  ॥ ११.१२ ॥  उच्चारण

\noindent\rule{16cm}{0.4pt} 
\end{quotation}


\begin{quotation} 

तत्रैकस्थं जगत्कृत्स्नं प्रविभक्तमनेकधा  ।  

अपश्यद्देवदेवस्य शरीरे पाण्डवस्तदा  ॥ ११.१३ ॥  मूल श्लोक
\end{quotation}

\begin{quotation}

तत्र एैक-स्थं जगत-कृत्-सन्म्, प्रवि-भक्तम अनेकधा  ।  

अपश्यद् देव देवस्य, शरीरे पाण्डवस् तदा  ॥ ११.१३ ॥  उच्चारण

\noindent\rule{16cm}{0.4pt} 
\end{quotation}


\begin{quotation} 

ततः स विस्मयाविष्टो हृष्टरोमा धनञ्जयः  ।  

प्रणम्य शिरसा देवं कृताञ्जलिरभाषत  ॥ ११.१४ ॥  मूल श्लोक
\end{quotation}

\begin{quotation}

ततः स विस्मय-आविष्टो, हृष्ट-रोमा धनञ्जयः  ।  

प्रणम्य शिरसा देवं, कृताञ्जलिर भाषत  ॥ ११.१४ ॥  उच्चारण

\noindent\rule{16cm}{0.4pt} 
\end{quotation}




\paragraph{\sanskrit अर्जुन उवाच}
\begin{quotation} 
पश्यामि देवांस्तव देव देहे सर्वांस्तथा भूतविशेषसङ्‍घान्‌  ।  

ब्रह्माणमीशं कमलासनस्थमृषींश्च सर्वानुरगांश्च दिव्यान्‌  ॥ ११.१५ ॥  मूल श्लोक
\end{quotation}

\begin{quotation}

पश्यामि देवांस् तव देव देहे 

सर्वांस तथा भूत विशेष सङ्‍घान्‌  ।  

ब्रह्माणम इशं कमलासन स्थम 

ऋषींश् च सर्वान उरगांश् च दिव्यान्‌  ॥ ११.१५ ॥  उच्चारण

\noindent\rule{16cm}{0.4pt} 
\end{quotation}


\begin{quotation} 
अनेकबाहूदरवक्त्रनेत्रंपश्यामि त्वां सर्वतोऽनन्तरूपम्‌  ।  

नान्तं न मध्यं न पुनस्तवादिंपश्यामि विश्वेश्वर विश्वरूप  ॥ ११.१६ ॥  मूल श्लोक
\end{quotation}

\begin{quotation}

अनेक बाहु उदर वक्त्र नेत्रं

पश्यामि त्वां सर्वतोऽ अनन्त रूपम्‌  ।  

नान्तं न मध्यं न पुनस्तव आदिं

पश्यामि विश्वेश्वर विश्व-रूप  ॥ ११.१६ ॥  उच्चारण

\noindent\rule{16cm}{0.4pt} 
\end{quotation}


\begin{quotation} 

किरीटिनं गदिनं चक्रिणं च तेजोराशिं सर्वतो दीप्तिमन्तम्‌  ।  

पश्यामि त्वां दुर्निरीक्ष्यं समन्ताद्दीप्तानलार्कद्युतिमप्रमेयम्‌  ॥ ११.१७ ॥  मूल श्लोक
\end{quotation}

\begin{quotation}

किरीटिनं गदिनं चक्रिणं च

तेजो-राशिं सर्वतो दीप्ति-मन्तम्‌  ।  

पश्यामि त्वां दुर-निरीक्ष्यं समन्ताद 

दीप्त-आनला-अर्क द्युतिम अप्रमेयम्‌  ॥ ११.१७ ॥  उच्चारण

\noindent\rule{16cm}{0.4pt} 
\end{quotation}


\begin{quotation} 

त्वमक्षरं परमं वेदितव्यंत्वमस्य विश्वस्य परं निधानम्‌  ।  

त्वमव्ययः शाश्वतधर्मगोप्ता सनातनस्त्वं पुरुषो मतो मे  ॥ ११.१८ ॥  मूल श्लोक
\end{quotation}

\begin{quotation}

त्वम अक्षरं परमं वेदि-तव्यं 

त्वमस्य विश्वस्य परं निधानम्‌  ।  

त्वम अव्ययः शाश्वत धर्म-गोप्ता 

सनातनस त्वं पुरुषो मतो मे  ॥ ११.१८ ॥  उच्चारण

\noindent\rule{16cm}{0.4pt} 
\end{quotation}


\begin{quotation} 

अनादिमध्यान्तमनन्तवीर्यमनन्तबाहुं शशिसूर्यनेत्रम्‌  ।  

पश्यामि त्वां दीप्तहुताशवक्त्रंस्वतेजसा विश्वमिदं तपन्तम्‌  ॥ ११.१९ ॥  मूल श्लोक
\end{quotation}

\begin{quotation}

अनादि मध्य-आन्तम अनन्त वीर्यम 

अनन्त बाहुं शशि सूर्य नेत्रम्‌  ।  

पश्यामि त्वां दीप्त-हुताश-वक्त्रं 

स्व तेजसा विश्वम इदं तपन्तम्‌  ॥ ११.१९ ॥  उच्चारण

\noindent\rule{16cm}{0.4pt} 
\end{quotation}


\begin{quotation} 

द्यावापृथिव्योरिदमन्तरं हि व्याप्तं त्वयैकेन दिशश्च सर्वाः  ।  

दृष्ट्वाद्भुतं रूपमुग्रं तवेदंलोकत्रयं प्रव्यथितं महात्मन्‌  ॥ ११.२० ॥  मूल श्लोक
\end{quotation}

\begin{quotation}

द्याव आ-पृथिव्योर इदम अन्तरं हि

व्याप्तं त्वय-एैकेन दिशश् च सर्वाः  ।  

दृष्टाव अद्भुतं रूपम उग्रं तवेदं 

लोक त्रयं प्र-व्यथितं महात्मन्‌  ॥ ११.२० ॥  उच्चारण

\noindent\rule{16cm}{0.4pt} 
\end{quotation}


\begin{quotation} 

अमी हि त्वां सुरसङ्‍घा विशन्ति केचिद्भीताः प्राञ्जलयो गृणन्ति ।  

स्वस्तीत्युक्त्वा महर्षिसिद्धसङ्‍घा: स्तुवन्ति त्वां स्तुतिभिः पुष्कलाभिः  ॥ ११.२१ ॥  मूल श्लोक
\end{quotation}

\begin{quotation}

अमी हि त्वां सुरसङ्‍घा विशन्ति 

केचिद भीताः प्राञ्जलयो गृणन्ति ।  

स्वस्त-इत्य उक्त्वा महर्षि सिद्ध सङ्‍घा: 

स्तुवन्ति त्वां स्तुति-भिः पुष्कला-भिः  ॥ ११.२१ ॥  उच्चारण

\noindent\rule{16cm}{0.4pt} 
\end{quotation}


\begin{quotation} 

रुद्रादित्या वसवो ये च साध्याविश्वेऽश्विनौ मरुतश्चोष्मपाश्च  ।  

गंधर्वयक्षासुरसिद्धसङ्‍घावीक्षन्ते त्वां विस्मिताश्चैव सर्वे  ॥ ११.२२ ॥  मूल श्लोक
\end{quotation}

\begin{quotation}

रुद्र आदित्या वसवो ये च साध्या 

विश्वेऽ-अश्विनौ मरुतस च ओष्म-पाश् च ।  

गंधर्व यक्षा असुर सिद्ध सङ्‍घा 

वीक्षन्ते त्वां विस्मिताश चैव् सर्वे  ॥ ११.२२ ॥  उच्चारण

\noindent\rule{16cm}{0.4pt} 
\end{quotation}


\begin{quotation} 

रूपं महत्ते बहुवक्त्रनेत्रंमहाबाहो बहुबाहूरूपादम्‌  ।  

बहूदरं बहुदंष्ट्राकरालंदृष्टवा लोकाः प्रव्यथितास्तथाहम्‌  ॥ ११.२३ ॥  मूल श्लोक
\end{quotation}

\begin{quotation}

रूपं महत्ते बहु वक्त्र नेत्रं 

महाबाहो बहु-बाहूरू पादम्‌  ।  

बहु-उदरं बहु-दंष्ट्रा करालं 

दृष्टवा लोकाः प्र-व्यथितास् तथाहम्‌  ॥ ११.२३ ॥  उच्चारण

\noindent\rule{16cm}{0.4pt} 
\end{quotation}


\begin{quotation} 

नभःस्पृशं दीप्तमनेकवर्णंव्यात्ताननं दीप्तविशालनेत्रम्‌  ।  

दृष्टवा हि त्वां प्रव्यथितान्तरात्मा धृतिं न विन्दामि शमं च विष्णो  ॥ ११.२४ ॥  मूल श्लोक
\end{quotation}

\begin{quotation}




नभः स्पृशं दीप्तम अनेक वर्णं

व्यात्ता-ननं दीप्त विशाल नेत्रम्‌  ।  

दृष्टवा हि त्वां प्र-व्यथित अन्तर आत्मा

धृतिं न विन्दामि शमं च विष्णो  ॥ ११.२४ ॥  उच्चारण

\noindent\rule{16cm}{0.4pt} 
\end{quotation}


\begin{quotation} 

दंष्ट्राकरालानि च ते मुखानिदृष्टैव कालानलसन्निभानि  ।  

दिशो न जाने न लभे च शर्म प्रसीद देवेश जगन्निवास  ॥ ११.२५ ॥  मूल श्लोक
\end{quotation}

\begin{quotation}

दंष्ट्रा करालानि च ते मुखानि 

दृष्टैव कालानल सन्नि-भानि  ।  

दिशो न जाने न लभे च शर्म 

प्रसीद देवेश जगन्निवास  ॥ ११.२५ ॥  उच्चारण

\noindent\rule{16cm}{0.4pt} 
\end{quotation}


\begin{quotation} 

अमी च त्वां धृतराष्ट्रस्य पुत्राः सर्वे सहैवावनिपालसंघैः  ।  

भीष्मो द्रोणः सूतपुत्रस्तथासौ सहास्मदीयैरपि योधमुख्यैः  ॥ ११.२६ ॥  मूल श्लोक
\end{quotation}

\begin{quotation}

अमी च त्वां धृत-राष्ट्रस्य पुत्राः

सर्वे सह एैव अवनिपाल संघैः  ।  

भीष्मो द्रोणः सूत-पुत्रस तथासौ

सह अस्मदीयैर अपि योध मुख्यैः  ॥ ११.२६ ॥  उच्चारण

\noindent\rule{16cm}{0.4pt} 
\end{quotation}


\begin{quotation} 

वक्त्राणि ते त्वरमाणा विशन्ति दंष्ट्राकरालानि भयानकानि  ।  

केचिद्विलग्ना दशनान्तरेषु सन्दृश्यन्ते चूर्णितैरुत्तमाङ्‍गै  ॥ ११.२७ ॥  मूल श्लोक
\end{quotation}

\begin{quotation}

वक्त्राणि ते त्वर-माणा विशन्ति

दंष्ट्रा करालानि भयान-कानि  ।  

केचिद् विलग्ना दशनान्त-रेषु 

संदृश्यन्ते चूर्ण-इतैर उत्तमाङ्‍गै  ॥ ११.२७ ॥  उच्चारण

\noindent\rule{16cm}{0.4pt} 
\end{quotation}


\begin{quotation} 

यथा नदीनां बहवोऽम्बुवेगाः समुद्रमेवाभिमुखा द्रवन्ति  ।  

तथा तवामी नरलोकवीराविशन्ति वक्त्राण्यभिविज्वलन्ति  ॥ ११.२८ ॥  मूल श्लोक
\end{quotation}

\begin{quotation}




यथा नदीनां बहवोऽ अम्बु वेगाः

समुद्रम एव अभिमुखा द्रवन्ति  ।  

तथा तवामी नर लोक वीरा

विशन्ति वक्त्राण्य अभि-विज्वलन्ति  ॥ ११.२८ ॥  उच्चारण

\noindent\rule{16cm}{0.4pt} 
\end{quotation}


\begin{quotation} 

यथा प्रदीप्तं ज्वलनं पतंगाविशन्ति नाशाय समृद्धवेगाः  ।  

तथैव नाशाय विशन्ति लोकास्तवापि वक्त्राणि समृद्धवेगाः  ॥ ११.२९ ॥  मूल श्लोक
\end{quotation}

\begin{quotation}

यथा प्रदीप्तं ज्वलनं पतंगा

विशन्ति नाशाय समृद्ध वेगाः  ।  

तथैव नाशाय विशन्ति लोका 

स्तवापि वक्त्राणि समृद्ध वेगाः  ॥ ११.२९ ॥  उच्चारण

\noindent\rule{16cm}{0.4pt} 
\end{quotation}


\begin{quotation} 

लेलिह्यसे ग्रसमानः समन्ताल्लोकान्समग्रान्वदनैर्ज्वलद्भिः ।  

तेजोभिरापूर्य जगत्समग्रंभासस्तवोग्राः प्रतपन्ति विष्णो  ॥ ११.३० ॥  मूल श्लोक
\end{quotation}

\begin{quotation}

लेलिह्यसे ग्रसमानः समन्ताल

लोकान समग्रान वदनैर ज्वलद-भी:  ।  

तेजोभिर अपूर्य जगत समग्रं

भासस् तवोग्राः प्रतपन्ति विष्णो  ॥ ११.३० ॥  उच्चारण

\noindent\rule{16cm}{0.4pt} 
\end{quotation}


\begin{quotation} 

आख्याहि मे को भवानुग्ररूपोनमोऽस्तु ते देववर प्रसीद  ।  

विज्ञातुमिच्छामि भवन्तमाद्यंन हि प्रजानामि तव प्रवृत्तिम्‌  ॥ ११.३१ ॥  मूल श्लोक
\end{quotation}

\begin{quotation}

आख्याहि मे को भवान उग्र-रूपो 

नमोऽस्तु ते देव-वर प्रसीद  ।  

विज्ञातुम इच्छामि भवन्तम आद्यम

न हि प्रजानामि तव प्रवृत्तिम्‌  ॥ ११.३१ ॥  उच्चारण

\noindent\rule{16cm}{0.4pt} 
\end{quotation}



\paragraph{\sanskrit श्रीभगवानुवाच}
\begin{quotation} 
कालोऽस्मि लोकक्षयकृत्प्रवृद्धोलोकान्समाहर्तुमिह प्रवृत्तः  ।  

ऋतेऽपि त्वां न भविष्यन्ति सर्वे येऽवस्थिताः प्रत्यनीकेषु योधाः  ॥ ११.३२ ॥  मूल श्लोक
\end{quotation}

\begin{quotation}

कालोऽ अस्मि लोकक्षय कृत-प्रवृद्धो 

लोकान समाहर्तुम इह प्रवृत्तः  ।  

ऋतेऽ अपि त्वां न भविष्यन्ति सर्वे 

येऽ अवस्थिताः प्रत्य-अनीकेषु योधाः  ॥ ११.३२ ॥  उच्चारण

\noindent\rule{16cm}{0.4pt} 
\end{quotation}


\begin{quotation} 

तस्मात्त्वमुक्तिष्ठ यशो लभस्व जित्वा शत्रून्भुङ्‍क्ष्व राज्यं समृद्धम्‌  ।  

मयैवैते निहताः पूर्वमेव निमित्तमात्रं भव सव्यसाचिन्‌  ॥ ११.३३ ॥  मूल श्लोक
\end{quotation}

\begin{quotation}

तस्मात त्वम उक्तिष्ठ यशो लभस्व

जित्वा शत्रून भुन्क्ष्व राज्यं समृद्धम्‌  ।  

मय एैवै ते निहताः पूर्वम एव 

निमित्त मात्रं भव सव्य-साचिन्‌  ॥ ११.३३ ॥  उच्चारण

\noindent\rule{16cm}{0.4pt} 
\end{quotation}


\begin{quotation} 

द्रोणं च भीष्मं च जयद्रथं च कर्णं तथान्यानपि योधवीरान्‌  ।  

मया हतांस्त्वं जहि मा व्यथिष्ठायुध्यस्व जेतासि रणे सपत्नान्‌  ॥ ११.३४ ॥  मूल श्लोक
\end{quotation}

\begin{quotation}

द्रोणं च भीष्मं च जयद्रथं च 

कर्णं तथान्यान अपि योध-वीरान्‌  ।  

मया हतांस त्वं जहि मा व्यथिष्ठा 

युध्यस्व जेतासि रणे सपत्-नान्‌  ॥ ११.३४ ॥  उच्चारण

\noindent\rule{16cm}{0.4pt} 
\end{quotation}


\paragraph{\sanskrit सञ्जय उवाच}

\begin{quotation} 


एतच्छ्रुत्वा वचनं केशवस्य कृतांजलिर्वेपमानः किरीटी  ।  

नमस्कृत्वा भूय एवाह कृष्णंसगद्गदं भीतभीतः प्रणम्य  ॥ ११.३५ ॥  मूल श्लोक
\end{quotation}

\begin{quotation}

एतच् छ्रुत्वा वचनं केशवस्य 

कृतांजलिर वेपमानः किरीटी  ।  

नमस्-कृत्वा भूय एवाह कृष्णं

स-गद्-गदं  भीत-भीतः प्रणम्य  ॥ ११.३५ ॥  उच्चारण

\noindent\rule{16cm}{0.4pt} 
\end{quotation}






\paragraph{\sanskrit अर्जुन उवाच}
\begin{quotation} 
स्थाने हृषीकेश तव प्रकीर्त्या जगत्प्रहृष्यत्यनुरज्यते च  ।  

रक्षांसि भीतानि दिशो द्रवन्ति सर्वे नमस्यन्ति च सिद्धसङ्‍घा:  ॥ ११.३६ ॥  मूल श्लोक
\end{quotation}

\begin{quotation}

स्थाने हृषीकेश तव प्रकीर्त्या

जगत  प्रहृष्य-अत्य अनु-रज्यते च  ।  

रक्षांसि भीतानि दिशो द्रवन्ति 

सर्वे नमस्यन्ति च सिद्ध सङ्‍घा:  ॥ ११.३६ ॥  उच्चारण

\noindent\rule{16cm}{0.4pt} 
\end{quotation}


\begin{quotation} 

कस्माच्च ते न नमेरन्महात्मन्‌ गरीयसे ब्रह्मणोऽप्यादिकर्त्रे  ।  

अनन्त देवेश जगन्निवास त्वमक्षरं सदसत्तत्परं यत्‌  ॥ ११.३७ ॥  मूल श्लोक
\end{quotation}

\begin{quotation}

कस्माच्च ते न नमेरन महात्मन्‌ 

गरीयसे ब्रह्मणोऽ अप्य आदि कर्त्रे ।  

अनन्त देवेश जगन्निवास 

त्वम अक्षरं सद-असत तत्परं यत्‌  ॥ ११.३७ ॥  उच्चारण

\noindent\rule{16cm}{0.4pt} 
\end{quotation}


\begin{quotation} 

त्वमादिदेवः पुरुषः पुराणस्त्वमस्य विश्वस्य परं निधानम्‌  ।  

वेत्तासि वेद्यं च परं च धाम त्वया ततं विश्वमनन्तरूप  ॥ ११.३८ ॥  मूल श्लोक
\end{quotation}

\begin{quotation}

त्वम आदिदेवः पुरुषः पुराणस   

त्वम अस्य विश्वस्य परं निधानम्‌  ।  

वेत्तासि वेद्यं च परं च धाम 

त्वया ततं विश्वम अनन्त-रूप  ॥ ११.३८ ॥  उच्चारण

\noindent\rule{16cm}{0.4pt} 
\end{quotation}


\begin{quotation} 

वायुर्यमोऽग्निर्वरुणः शशाङ्‍क: प्रजापतिस्त्वं प्रपितामहश्च ।  

नमो नमस्तेऽस्तु सहस्रकृत्वः पुनश्च भूयोऽपि नमो नमस्ते  ॥ ११.३९ ॥  मूल श्लोक
\end{quotation}

\begin{quotation}

वायुर यमोऽ अग्निर वरुणः शशाङ्‍क: 

प्रजापतिस त्वं प्रपिता-महश् च  ।  

नमो नमस्तेऽ अस्तु सहस्र कृत्वः 

पुनश् च भूयोऽ अपि नमो नमस्ते  ॥ ११.३९ ॥  उच्चारण

\noindent\rule{16cm}{0.4pt} 
\end{quotation}


\begin{quotation} 

नमः पुरस्तादथ पृष्ठतस्ते नमोऽस्तु ते सर्वत एव सर्व  ।  
 
अनन्तवीर्यामितविक्रमस्त्वंसर्वं समाप्नोषि ततोऽसि सर्वः  ॥ ११.४० ॥  मूल श्लोक
\end{quotation}

\begin{quotation}

नमः पुरस्ताद अथ पृष्ठ-तस ते 

नमोऽस्तु ते सर्वत एव सर्व ।  
 
अनन्त वीर्या-अमित विक्रमस त्वं 

सर्वं समाप-नोषि ततोऽ असि सर्वः  ॥ ११.४० ॥  उच्चारण

\noindent\rule{16cm}{0.4pt} 
\end{quotation}


\begin{quotation} 

सखेति मत्वा प्रसभं यदुक्तं हे कृष्ण हे यादव हे सखेति ।  

अजानता महिमानं तवेदंमया प्रमादात्प्रणयेन वापि  ॥ ११.४१ ॥  मूल श्लोक
\end{quotation}

\begin{quotation}

सखेति मत्वा प्रसभम् यद उक्तं 

हे कृष्ण हे यादव हे सखेति ।  

अजानता महिमानं तवेदं 

मया प्रमादात  प्रणयेन वापि  ॥ ११.४१ ॥  उच्चारण

\noindent\rule{16cm}{0.4pt} 
\end{quotation}


\begin{quotation} 

यच्चावहासार्थमसत्कृतोऽसि विहारशय्यासनभोजनेषु  ।  

एकोऽथवाप्यच्युत तत्समक्षंतत्क्षामये त्वामहमप्रमेयम्‌  ॥ ११.४२ ॥  मूल श्लोक
\end{quotation}

\begin{quotation}

यच च अवहास-अर्थम असत-कृतोऽसि 

विहार शय्यासन भोजनेषु  ।  

एकोऽथ वप्य अच्युत तत समक्षं

तत क्षामये त्वाम-अहम अ-प्रमेयम्‌  ॥ ११.४२ ॥  उच्चारण

\noindent\rule{16cm}{0.4pt} 
\end{quotation}


\begin{quotation} 

पितासि लोकस्य चराचरस्य त्वमस्य पूज्यश्च गुरुर्गरीयान्‌ ।  

न त्वत्समोऽस्त्यभ्यधिकः कुतोऽन्योलोकत्रयेऽप्यप्रतिमप्रभाव  ॥ ११.४३ ॥  मूल श्लोक
\end{quotation}

\begin{quotation}

पितासि लोकस्य चरा-चरस्य 

त्वमस्य पूज्यश् च गुरुर गरीयान्‌  ।  

न त्वत-समोऽ अस्त्य अभ्य-धिकः कुतोऽन्यो
 
लोकत्रयेऽ अप्य अप्रतिम प्रभाव  ॥ ११.४३ ॥  उच्चारण

\noindent\rule{16cm}{0.4pt} 
\end{quotation}


\begin{quotation} 

तस्मात्प्रणम्य प्रणिधाय कायंप्रसादये त्वामहमीशमीड्यम्‌  ।  

पितेव पुत्रस्य सखेव सख्युः प्रियः प्रियायार्हसि देव सोढुम्‌  ॥ ११.४४ ॥  मूल श्लोक
\end{quotation}

\begin{quotation}

तस्मात प्रणम्य प्रणिधाय कायं

प्रसादये त्वाम अहम इशम् इड्-यम्‌ ।  

पितेव पुत्रस्य सखेव सख्युः 

प्रियः प्रियाया-अर्हसि देव सोढुम्‌  ॥ ११.४४ ॥  उच्चारण

\noindent\rule{16cm}{0.4pt} 
\end{quotation}


\begin{quotation} 

अदृष्टपूर्वं हृषितोऽस्मि दृष्ट्वा भयेन च प्रव्यथितं मनो मे  ।  

तदेव मे दर्शय देवरूपंप्रसीद देवेश जगन्निवास  ॥ ११.४५ ॥   मूल श्लोक
\end{quotation}

\begin{quotation}

अदृष्ट पूर्वं हृषितोऽ अस्मि दृष्ट्वा 

भयेन च प्र-व्यथितं मनो मे  ।  

तद एव मे दर्शय देव रूपं 

प्रसीद देवेश जगन्निवास  ॥ ४५ ॥  उच्चारण

\noindent\rule{16cm}{0.4pt} 
\end{quotation}


\begin{quotation} 

किरीटिनं गदिनं चक्रहस्तमिच्छामि त्वां द्रष्टुमहं तथैव  ।  

तेनैव रूपेण चतुर्भुजेनसहस्रबाहो भव विश्वमूर्ते  ॥ ११.४६ ॥  मूल श्लोक
\end{quotation}

\begin{quotation}

किरीटिनं गदिनं चक्र-हस्तम 

इच्छामि त्वां द्रष्टुम अहं तथैव  ।  

तेनैव रूपेण चतुर्भुजेन 

सहस्र बाहो भव विश्व मूर्ते ॥ ११.४६ ॥  उच्चारण

\noindent\rule{16cm}{0.4pt} 
\end{quotation}




\paragraph{\sanskrit श्रीभगवानुवाच}
\begin{quotation} 
मया प्रसन्नेन तवार्जुनेदंरूपं परं दर्शितमात्मयोगात्‌  ।  

तेजोमयं विश्वमनन्तमाद्यंयन्मे त्वदन्येन न दृष्टपूर्वम्‌  ॥ ११.४७ ॥  मूल श्लोक
\end{quotation}

\begin{quotation}

मया प्रसन्नेन तव अर्जुनेदं 

रूपं परं दर्शितम आत्म योगात्‌  ।  

तेजो मयं विश्वम अनन्तम आदयं 

यन मे त्वद अन्येन न दृष्ट पूर्वम्‌  ॥ ११.४७ ॥  उच्चारण

\noindent\rule{16cm}{0.4pt} 
\end{quotation}


\begin{quotation} 

न वेदयज्ञाध्ययनैर्न दानैर्न च क्रियाभिर्न तपोभिरुग्रैः  ।  

एवं रूपः शक्य अहं नृलोके द्रष्टुं त्वदन्येन कुरुप्रवीर  ॥ ११.४८ ॥  मूल श्लोक
\end{quotation}

\begin{quotation}
न वेद-यज्ञ अध्ययनैर न दानैर 

न च क्रियाभिर न तपोभिर उग्रैः  ।  

एवं रूपः शक्य अहं नृलोके,

द्रष्टुं त्वद अन्येन कुरु प्रवीर  ॥ ११.४८ ॥  उच्चारण

\noindent\rule{16cm}{0.4pt} 
\end{quotation}


\begin{quotation} 

मा ते व्यथा मा च विमूढभावोदृष्ट्वा रूपं घोरमीदृङ्‍ममेदम्‌  ।  

व्यतेपभीः प्रीतमनाः पुनस्त्वंतदेव मे रूपमिदं प्रपश्य  ॥ ११.४९ ॥  मूल श्लोक
\end{quotation}

\begin{quotation}

मा ते व्यथा मा च विमूढ भावो 

दृष्ट्वा रूपं घोरम इदृन मम इदम्‌  ।  

व्यतेप-भीः प्रीत-मनाः पुनस्त्वं

तदेव मे रूपम इदं प्रपश्य  ॥ ११.४९ ॥  उच्चारण

\noindent\rule{16cm}{0.4pt} 
\end{quotation}


\begin{quotation} 

संजय उवाच

इत्यर्जुनं वासुदेवस्तथोक्त्वा स्वकं रूपं दर्शयामास भूयः  ।  

आश्वासयामास च भीतमेनंभूत्वा पुनः सौम्यवपुर्महात्मा  ॥ ११.५० ॥  मूल श्लोक
\end{quotation}

\begin{quotation}

इत्य अर्जुनं वासुदेवस तथोक्त्वा 

स्वकं रूपं दर्शयामास भूयः  ।  

आश्वासयाम आस च भीत मेनं 

भूत्वा पुनः सौम्यव-पुर-महात्मा  ॥ ११.५० ॥  उच्चारण

\noindent\rule{16cm}{0.4pt} 
\end{quotation}



\paragraph{\sanskrit अर्जुन उवाच}
\begin{quotation} 
दृष्ट्वेदं मानुषं रूपं तव सौम्यं जनार्दन  ।  

इदानीमस्मि संवृत्तः सचेताः प्रकृतिं गतः  ॥ ११.५१ ॥  मूल श्लोक
\end{quotation}

\begin{quotation}

दृष्ट्व इदं मानुषं रूपं, तव सौम्यं जनार्दन  ।  

इदानीम अस्मि संवृत्तः, सचेताः प्रकृतिं गतः  ॥ ११.५१ ॥  उच्चारण

\noindent\rule{16cm}{0.4pt} 
\end{quotation}


\begin{quotation} 

सुदुर्दर्शमिदं रूपं दृष्टवानसि यन्मम  ।  

देवा अप्यस्य रूपस्य नित्यं दर्शनकाङ्‍क्षिणः  ॥ ११.५२ ॥  मूल श्लोक
\end{quotation}

\begin{quotation}
सु-दुर-दर्शम इदं रूपं, दृष्टवान असि यन म म ।  

देवा अप्य अस्य रूपस्य, नित्यं दर्शन-काङ्‍क्षिणः  ॥ ११.५२ ॥  उच्चारण

\noindent\rule{16cm}{0.4pt} 
\end{quotation}


\begin{quotation} 

नाहं वेदैर्न तपसा न दानेन न चेज्यया  ।  

शक्य एवं विधो द्रष्टुं दृष्ट्वानसि मां यथा  ॥ ११.५३ ॥  मूल श्लोक
\end{quotation}

\begin{quotation}

नाहं वेदैर न तपसा, न दानेन न चेज्यया  ।  

शक्य एवं विधो द्रष्टुं, दृष्ट्-वान असि मां यथा  ॥ ११.५३ ॥  उच्चारण

\noindent\rule{16cm}{0.4pt} 
\end{quotation}


\begin{quotation} 

भक्त्या त्वनन्यया शक्य अहमेवंविधोऽर्जुन  ।  

ज्ञातुं द्रष्टुं च तत्वेन प्रवेष्टुं च परन्तप  ॥ ११.५४ ॥  मूल श्लोक
\end{quotation}

\begin{quotation}

भक्त्या त्व अनन्यया शक्य, अहम एवं विधोऽ अर्जुन  ।  

ज्ञातुं द्रष्टुं च तत्वेन, प्रवेष्टुं च परन्तप  ॥ ११.५४ ॥  उच्चारण

\noindent\rule{16cm}{0.4pt} 
\end{quotation}


\begin{quotation} 

मत्कर्मकृन्मत्परमो मद्भक्तः सङ्‍गवर्जितः  ।  

निर्वैरः सर्वभूतेषु यः स मामेति पाण्डव  ॥ ११.५५ ॥  मूल श्लोक
\end{quotation}

\begin{quotation}

मत-कर्म-कृन  मत्-परमो, मद्-भक्तः सङ्‍ग वर्जितः  ।  

निर्वैरः सर्व-भूतेषु, यः स मामेति पाण्डव  ॥ ११.५५ ॥  उच्चारण

\noindent\rule{16cm}{0.4pt} 
\end{quotation}

\begin{center} ***** \end{center}
\begin{quotation} 



ॐ तत् सद इति श्री मद्-भगवद्-गीतास उपनिषत्सु ब्रह्म विद्यायां योगशास्त्रे श्री कृष्णार्जुन संवादे विश्व रूप दर्शन योगो नाम एैकादशोऽ अध्यायः  ॥  ११  ॥ 


\end{quotation} 