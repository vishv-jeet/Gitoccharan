\chapter{\sanskrit  श्रद्धात्रयविभागयोग} 
\paragraph{\sanskrit अर्जुन उवाच}
\begin{quotation} 
ये शास्त्रविधिमुत्सृज्य यजन्ते श्रद्धयान्विताः ।  

तेषां निष्ठा तु का कृष्ण सत्त्वमाहो रजस्तमः  ॥ १७.१ ॥  मूल श्लोक
\end{quotation}

\begin{quotation}

ये शास्त्र विधिम उत्सृज्य, यजन्ते श्रद्धया अन्विताः ।  

तेषां निष्ठा तु का कृष्ण, सत्त्वम आहो रजस् तमः  ॥ १७.१ ॥  उच्चारण

\noindent\rule{16cm}{0.4pt} 
\end{quotation}

\paragraph{\sanskrit श्रीभगवानुवाच}

\begin{quotation}


त्रिविधा भवति श्रद्धा देहिनां सा स्वभावजा ।  

सात्त्विकी राजसी चैव तामसी चेति तां श्रृणु  ॥ १७.२ ॥  मूल श्लोक
\end{quotation}

\begin{quotation}

त्रि-विधा भवति श्रद्धा, देहिनां सा स्वभाव-जा ।  

सात्त्विकी राजसी चैव, तामसी चेति तां श्रृणु  ॥ १७.२ ॥  उच्चारण

\noindent\rule{16cm}{0.4pt} 
\end{quotation}


\begin{quotation}

सत्त्वानुरूपा सर्वस्य श्रद्धा भवति भारत ।  

श्रद्धामयोऽयं पुरुषो यो यच्छ्रद्धः स एव सः  ॥ १७.३ ॥  मूल श्लोक
\end{quotation}

\begin{quotation}

सत्त्वा अनुरूपा सर्वस्य, श्रद्धा भवति भारत ।  

श्रद्धा मयोऽ अयं पुरुषो, यो यच् छ्रद्धः स एव सः  ॥ १७.३ ॥  उच्चारण

\noindent\rule{16cm}{0.4pt} 
\end{quotation}


\begin{quotation}

यजन्ते सात्त्विका देवान्यक्षरक्षांसि राजसाः ।  

प्रेतान्भूतगणांश्चान्ये जयन्ते तामसा जनाः  ॥ १७.४ ॥  मूल श्लोक
\end{quotation}

\begin{quotation}

यजन्ते सात्त्विका देवान्, यक्ष रक्षांसि राजसाः ।  

प्रेतान् भूत-गणांश् च अन्ये, जयन्ते तामसा जनाः  ॥ १७.४ ॥  उच्चारण

\noindent\rule{16cm}{0.4pt} 
\end{quotation}


\begin{quotation}

अशास्त्रविहितं घोरं तप्यन्ते ये तपो जनाः ।  

दम्भाहङ्‍कारसंयुक्ताः कामरागबलान्विताः  ॥ १७.५ ॥  मूल श्लोक
\end{quotation}

\begin{quotation}
अशास्त्र विहितं घोरं, तप्यन्ते ये तपो जनाः ।  

दम्भा अहङ्‍कार संयुक्ताः, काम राग बलान् अन्विताः  ॥ १७.५ ॥  उच्चारण

\noindent\rule{16cm}{0.4pt} 
\end{quotation}


\begin{quotation}

कर्शयन्तः शरीरस्थं भूतग्राममचेतसः ।  

मां चैवान्तःशरीरस्थं तान्विद्ध्‌यासुरनिश्चयान्‌  ॥ १७.६ ॥  मूल श्लोक
\end{quotation}

\begin{quotation}

कर्शयन्तः शरीर स्थं, भूत ग्रामम चेतसः ।  

मां चैवान्तः शरीर स्थं, तान् विद्ध्‌य आसुर निश्चयान्‌  ॥ १७.६ ॥  उच्चारण

\noindent\rule{16cm}{0.4pt} 
\end{quotation}


\begin{quotation}

आहारस्त्वपि सर्वस्य त्रिविधो भवति प्रियः ।  

यज्ञस्तपस्तथा दानं तेषां भेदमिमं श्रृणु  ॥ १७.७ ॥  मूल श्लोक
\end{quotation}

\begin{quotation}

आहारस् त्व अपि सर्वस्य, त्रि विधो भवति प्रियः ।  

यज्ञस् तपस् तथा दानं, तेषां भेदम इमं श्रृणु  ॥ १७.७ ॥  उच्चारण

\noindent\rule{16cm}{0.4pt} 
\end{quotation}


\begin{quotation}

आयुः सत्त्वबलारोग्यसुखप्रीतिविवर्धनाः ।  

रस्याः स्निग्धाः स्थिरा हृद्या आहाराः सात्त्विकप्रियाः  ॥ १७.८ ॥  मूल श्लोक
\end{quotation}

\begin{quotation}

आयुः सत्त्व बल आरोग्य, सुख प्रीति विवर्धनाः ।  

रस्याः स्निग्धाः स्थिरा ह्रदया, आहाराः सात्त्विक प्रियाः  ॥ १७.८ ॥  उच्चारण

\noindent\rule{16cm}{0.4pt} 
\end{quotation}


\begin{quotation}

कट्वम्ललवणात्युष्णतीक्ष्णरूक्षविदाहिनः ।  

आहारा राजसस्येष्टा दुःखशोकामयप्रदाः  ॥ १७.९ ॥  मूल श्लोक
\end{quotation}

\begin{quotation}

कटव् अम्ल लवणात्य उष्ण, तीक्ष्ण रूक्ष विदाहिनः ।  

आहारा राजस-अस्य एष्टा, दुःख शोकामय प्रदाः  ॥ १७.९ ॥  उच्चारण

\noindent\rule{16cm}{0.4pt} 
\end{quotation}


\begin{quotation}

यातयामं गतरसं पूति पर्युषितं च यत्‌ ।  

उच्छिष्टमपि चामेध्यं भोजनं तामसप्रियम्‌  ॥ १७.१० ॥  मूल श्लोक
\end{quotation}

\begin{quotation}

यात-यामं गत-रसं, पूति पर्यु-षितं च यत्‌ ।  

उच्छिष्टम अपि चामेध्यं, भोजनं तामस प्रियम्‌  ॥ १७.१० ॥  उच्चारण

\noindent\rule{16cm}{0.4pt} 
\end{quotation}


\begin{quotation}

अफलाकाङ्क्षिभिर्यज्ञो विधिदृष्टो य इज्यते ।  

यष्टव्यमेवेति मनः समाधाय स सात्त्विकः  ॥ १७.११ ॥  मूल श्लोक
\end{quotation}

\begin{quotation}
अफल आकान्क्षि-भिर् यज्ञो, विधि दृष्टो य इज्यते ।  

यष्ट-व्यम एवेति मनः, समाधाय स सात्त्विकः  ॥ १७.११ ॥  उच्चारण

\noindent\rule{16cm}{0.4pt} 
\end{quotation}


\begin{quotation}

अभिसन्धाय तु फलं दम्भार्थमपि चैव यत्‌ ।  

इज्यते भरतश्रेष्ठ तं यज्ञं विद्धि राजसम्‌  ॥ १७.१२ ॥  मूल श्लोक
\end{quotation}

\begin{quotation}

अभि-सन्धाय तु फलं, दम्भार्थम अपि चैव यत्‌ ।  

इज्यते भरत श्रेष्ठ तं, यज्ञं विद्धि राजसम्‌  ॥ १७.१२ ॥  उच्चारण

\noindent\rule{16cm}{0.4pt} 
\end{quotation}


\begin{quotation}

विधिहीनमसृष्टान्नं मन्त्रहीनमदक्षिणम्‌ ।  

श्रद्धाविरहितं यज्ञं तामसं परिचक्षते  ॥ १७.१३ ॥  मूल श्लोक
\end{quotation}

\begin{quotation}

विधि-हीनम असृष्टान्नं, मन्त्र हीनम अ-दक्षिणम्‌ ।  

श्रद्धा वि-रहितं यज्ञं, तामसं परि-चक्षते  ॥ १७.१३ ॥  उच्चारण

\noindent\rule{16cm}{0.4pt} 
\end{quotation}


\begin{quotation}

देवद्विजगुरुप्राज्ञपूजनं शौचमार्जवम्‌ ।  

ब्रह्मचर्यमहिंसा च शारीरं तप उच्यते  ॥ १७.१४ ॥  मूल श्लोक
\end{quotation}

\begin{quotation}

देव द्विज गुरु प्राज्ञ, पूजनं शौचम आर्जवम्‌ ।  

ब्रह्मचर्यम अहिंसा च, शारीरं तप उच्यते  ॥ १७.१४ ॥  उच्चारण

\noindent\rule{16cm}{0.4pt} 
\end{quotation}


\begin{quotation}

अनुद्वेगकरं वाक्यं सत्यं प्रियहितं च यत्‌ ।  

स्वाध्यायाभ्यसनं चैव वाङ्‍मयं तप उच्यते  ॥ १७.१५ ॥  मूल श्लोक
\end{quotation}

\begin{quotation}

अनुद्वेग करं वाक्यं, सत्यं प्रिय हितं च यत्‌ ।  

स्वाध्याया अभ्यसनं चैव, वाङ्‍मयं तप उच्यते  ॥ १७.१५ ॥  उच्चारण

\noindent\rule{16cm}{0.4pt} 
\end{quotation}


\begin{quotation}

मनः प्रसादः सौम्यत्वं मौनमात्मविनिग्रहः ।  

भावसंशुद्धिरित्येतत्तपो मानसमुच्यते  ॥ १७.१६ ॥  मूल श्लोक
\end{quotation}

\begin{quotation}

मनः प्रसादः सौम्य त्वं, मौनम आत्म विनिग्रहः ।  

भाव संशुद्धिर इत्य एतत्, तपो मानसम उच्यते  ॥ १७.१६ ॥  उच्चारण

\noindent\rule{16cm}{0.4pt} 
\end{quotation}


\begin{quotation}

श्रद्धया परया तप्तं तपस्तत्त्रिविधं नरैः ।  

अफलाकाङ्क्षिभिर्युक्तैः सात्त्विकं परिचक्षते  ॥ १७.१७ ॥  मूल श्लोक
\end{quotation}

\begin{quotation}
श्रद्धया परया तप्तं, तपस् तत् त्रिविधं नरैः ।  

अफल आकान्क्षि-भिर् युक्तैः, सात्त्विकं परि-चक्षते  ॥ १७.१७ ॥  उच्चारण

\noindent\rule{16cm}{0.4pt} 
\end{quotation}


\begin{quotation}

सत्कारमानपूजार्थं तपो दम्भेन चैव यत्‌ ।  

क्रियते तदिह प्रोक्तं राजसं चलमध्रुवम्‌  ॥ १७.१८ ॥  मूल श्लोक
\end{quotation}

\begin{quotation}

सत्कार-मान-पूजार्थं, तपो दम्भेन चैव यत्‌ ।  

क्रियते तदिह प्रोक्तं, राजसं चलम अध्रुवम्‌  ॥ १७.१८ ॥  उच्चारण

\noindent\rule{16cm}{0.4pt} 
\end{quotation}


\begin{quotation}

मूढग्राहेणात्मनो यत्पीडया क्रियते तपः ।  

परस्योत्सादनार्थं वा तत्तामसमुदाहृतम्‌  ॥ १७.१९ ॥  मूल श्लोक
\end{quotation}

\begin{quotation}

मूढ ग्राहेण आत्मनो यत्, पीडया क्रियते तपः ।  

परस्य उत्सादना अर्थं वा, तत् ताम सम उदा-हृतम्‌  ॥ १७.१९ ॥  उच्चारण

\noindent\rule{16cm}{0.4pt} 
\end{quotation}


\begin{quotation}

दातव्यमिति यद्दानं दीयतेऽनुपकारिणे ।  

देशे काले च पात्रे च तद्दानं सात्त्विकं स्मृतम्‌  ॥ १७.२० ॥  मूल श्लोक
\end{quotation}

\begin{quotation}

दा-तव्यम इति यद् दानं, दीयतेऽ अनुप-कारिणे ।  

देशे काले च पात्रे च, तद् दानं सात्त्विकं स्मृतम्‌  ॥ १७.२० ॥  उच्चारण

\noindent\rule{16cm}{0.4pt} 
\end{quotation}


\begin{quotation}

यत्तु प्रत्युपकारार्थं फलमुद्दिश्य वा पुनः ।  

दीयते च परिक्लिष्टं तद्दानं राजसं स्मृतम्‌  ॥ १७.२१ ॥  मूल श्लोक
\end{quotation}

\begin{quotation}

यत्तु प्रत्य उपकारा अर्थं, फलम उद्दिश्य वा पुनः ।  

दीयते च परि-क्लिष्टं, तद् दानं राजसं स्मृतम्‌  ॥ १७.२१ ॥  उच्चारण

\noindent\rule{16cm}{0.4pt} 
\end{quotation}


\begin{quotation}

अदेशकाले यद्दानमपात्रेभ्यश्च दीयते ।  

असत्कृतमवज्ञातं तत्तामसमुदाहृतम्‌  ॥ १७.२२ ॥  मूल श्लोक
\end{quotation}

\begin{quotation}

अदेश काले यद् दानम, अपात्रे-भ्यश् च दीयते ।  

असत् कृतम अव-ज्ञातं, तत् तामसम उदा-हृतम्‌  ॥ १७.२२ ॥  उच्चारण

\noindent\rule{16cm}{0.4pt} 
\end{quotation}


\begin{quotation}

ॐ तत्सदिति निर्देशो ब्रह्मणस्त्रिविधः स्मृतः ।  

ब्राह्मणास्तेन वेदाश्च यज्ञाश्च विहिताः पुरा  ॥ १७.२३ ॥  मूल श्लोक
\end{quotation}

\begin{quotation}
ॐ तत् सद इति निर्देशो, ब्रह्मणस् त्रि-विधः स्मृतः ।  

ब्राह्मणास् तेन वेदाश् च, यज्ञाश् च विहिताः पुरा  ॥ १७.२३ ॥  उच्चारण

\noindent\rule{16cm}{0.4pt} 
\end{quotation}


\begin{quotation}

तस्मादोमित्युदाहृत्य यज्ञदानतपः क्रियाः ।  

प्रवर्तन्ते विधानोक्तः सततं ब्रह्मवादिनाम्‌  ॥ १७.२४ ॥  मूल श्लोक
\end{quotation}

\begin{quotation}

तस्माद ओम इत्य उदा-हृत्य, यज्ञ दान तपः क्रियाः ।  

प्र-वर्तन्ते विधान उक्तः, सततं ब्रह्म-वादि-नाम्‌  ॥ १७.२४ ॥  उच्चारण

\noindent\rule{16cm}{0.4pt} 
\end{quotation}


\begin{quotation}

तदित्यनभिसन्दाय फलं यज्ञतपःक्रियाः ।  

दानक्रियाश्चविविधाः क्रियन्ते मोक्षकाङ्क्षिभिः  ॥ १७.२५ ॥  मूल श्लोक
\end{quotation}

\begin{quotation}

तद इत्य अनभि-सन्दाय, फलं यज्ञ तपः क्रियाः ।  

दान क्रियाश् च विविधाः, क्रियन्ते मोक्ष-कान्क्षिभि:  ॥ १७.२५ ॥  उच्चारण

\noindent\rule{16cm}{0.4pt} 
\end{quotation}


\begin{quotation}

सद्भावे साधुभावे च सदित्यतत्प्रयुज्यते ।  

प्रशस्ते कर्मणि तथा सच्छब्दः पार्थ युज्यते  ॥ १७.२६ ॥  मूल श्लोक
\end{quotation}

\begin{quotation}

सद् भावे साधु भावे च, सद इत्य तत् प्रयुज्यते ।  

प्रशस्ते कर्मणि तथा, सच् छब्दः पार्थ युज्यते  ॥ १७.२६ ॥  उच्चारण

\noindent\rule{16cm}{0.4pt} 
\end{quotation}


\begin{quotation}

यज्ञे तपसि दाने च स्थितिः सदिति चोच्यते ।  

कर्म चैव तदर्थीयं सदित्यवाभिधीयते  ॥ १७.२७ ॥  मूल श्लोक
\end{quotation}

\begin{quotation}

यज्ञे तपसि दाने च, स्थितिः सद इति च उच्यते ।  

कर्म चैव तद अर्थीयं, सद इत्य एवा अभि-धीयते  ॥ १७.२७ ॥  उच्चारण

\noindent\rule{16cm}{0.4pt} 
\end{quotation}


\begin{quotation}

अश्रद्धया हुतं दत्तं तपस्तप्तं कृतं च यत्‌ ।  

असदित्युच्यते पार्थ न च तत्प्रेत्य नो इह  ॥ १७.२८ ॥  मूल श्लोक
\end{quotation}

\begin{quotation}

अ-श्रद्धया हुतं दत्तं, तपस् तप्तं कृतं च यत्‌ ।  

असद इत्य उच्यते पार्थ, न च तत् प्रेत्य नो इह  ॥ १७.२८ ॥  उच्चारण

\noindent\rule{16cm}{0.4pt} 
\end{quotation}

\begin{center} ***** \end{center}

\begin{quotation}

ॐ तत् सद इति श्री मद्-भगवद्-गीतास उपनिषत्सु ब्रह्म विद्यायां योगशास्त्रे श्री कृष्णार्जुन संवादे  श्रद्धात्रयविभागयोगो नाम सप्तदशोऽ अध्यायः  ॥  १७  ॥ 

\end{quotation} 
