\chapter{\sanskrit{अर्जुनविषादयोग}}
\sanskrit

\paragraph{\sanskrit धृतराष्ट्र उवाच}

\begin{quotation}
	धर्मक्षेत्रे कुरुक्षेत्रे समवेता युयुत्सवः  । 
	
	मामकाः पाण्डवाश्चैव किमकुर्वत संजय  ॥ १.१ ॥  मूल श्लोक

\end{quotation}

\begin{quotation}
धर्म-क्षेत्रे कुरु-क्षेत्रे, समवेता युयुत्सवः  । 

मामकाः पाण्डवाश् चैव, किम अकुर्वत संजय   ॥ १.१ ॥   उच्चारण

\noindent\rule{16cm}{0.4pt} 
\end{quotation}

\paragraph{\sanskrit सञ्जय उवाच}

\begin{quotation}
दृष्टवा तु पाण्डवानीकं व्यूढं दुर्योधनस्तदा  । 

आचार्यमुपसंगम्य राजा वचनमब्रवीत्‌  ॥ १.२ ॥  मूल श्लोक
\end{quotation}

\begin{quotation}

दृष्ट्वा तु पाण्डवा-अनीकं, व्यूढं दुर्योधनस् तदा  । 

आचार्यम उप-संगम्य, राजा वचनम् अब्रवीत्  ॥ १.२ ॥  उच्चारण

\noindent\rule{16cm}{0.4pt} 
\end{quotation}


\begin{quotation} 

पश्यैतां पाण्डुपुत्राणामाचार्य महतीं चमूम्‌  । 

व्यूढां द्रुपदपुत्रेण तव शिष्येण धीमता  ॥ १.३ ॥  मूल श्लोक
\end{quotation}

\begin{quotation} 

पश्यैतां पाण्डु पुत्राणाम्, आचार्य महतीं चमूम  ।  

व्यूढां द्रुपद पुत्रेण, तव शिष्येण धीमता  ॥ १.३ ॥  उच्चारण

\noindent\rule{16cm}{0.4pt} 
\end{quotation}


\begin{quotation} 

अत्र शूरा महेष्वासा भीमार्जुनसमा युधि  ।  

युयुधानो विराटश्च द्रुपदश्च महारथः  ॥ १.४ ॥  मूल श्लोक
\end{quotation}

\begin{quotation}

अत्र शूरा महेष्वासा, भीमार्जुन-समा युधि  ।  

युयुधानो विराटश् च, द्रुपदश् च महारथः  ॥ १.४ ॥  उच्चारण

\noindent\rule{16cm}{0.4pt} 
\end{quotation}


\begin{quotation} 


धृष्टकेतुश्चेकितानः काशिराजश्च वीर्यवान्‌  ।  

पुरुजित्कुन्तिभोजश्च शैब्यश्च नरपुङवः  ॥ १.५ ॥  मूल श्लोक
\end{quotation}

\begin{quotation}

धृष्ट-केतुश चेकितानः, काशि-राजश् च वीर्यवान्  ।  

पुरुजित कुन्ति भोजश् च, शैब्यश्  च नर नरपुङवः  ॥ १.५ ॥  उच्चारण

\noindent\rule{16cm}{0.4pt} 
\end{quotation}


\begin{quotation} 

युधामन्युश्च विक्रान्त उत्तमौजाश्च वीर्यवान्‌  ।  

सौभद्रो द्रौपदेयाश्च सर्व एव महारथाः  ॥ १.६ ॥  मूल श्लोक
\end{quotation}

\begin{quotation}

युधामन्युश् च विक्रान्त, उत्तमौजाश् च वीर्यवान्  ।  

सौभद्रो द्रौपदेयाश् च, सर्व एव महारथाः  ॥ १.६ ॥  उच्चारण

\noindent\rule{16cm}{0.4pt} 
\end{quotation}


\begin{quotation} 

अस्माकं तु विशिष्टा ये तान्निबोध द्विजोत्तम  ।  

नायका मम सैन्यस्य सञ्ज्ञार्थं तान्ब्रवीमि ते  ॥ १.७ ॥  मूल श्लोक
\end{quotation}

\begin{quotation}

अस्माकं तु विशिष्टा ये, तान निबोध द्विजोत्तम  ।  

नायका मम सैन्यस्य, संज्ञार्थं तान ब्रवीमि ते  ॥ १.७ ॥  उच्चारण

\noindent\rule{16cm}{0.4pt} 
\end{quotation}


\begin{quotation} 

भवान्भीष्मश्च कर्णश्च कृपश्च समितिञ्जयः  ।  

अश्वत्थामा विकर्णश्च सौमदत्तिस्तथैव च  ॥ १.८ ॥  मूल श्लोक
\end{quotation}

\begin{quotation}

भवान् भीष्मश् च कर्णश् च, कृपश् च समितिञ् जयः ।  

अश्वत्थामा विकर्णश् च, सौमदत्तिस् तथैव च  ॥ १.८ ॥  उच्चारण

\noindent\rule{16cm}{0.4pt} 
\end{quotation}


\begin{quotation} 

अन्ये च बहवः शूरा मदर्थे त्यक्तजीविताः  ।  

नानाशस्त्रप्रहरणाः सर्वे युद्धविशारदाः  ॥ १.९ ॥  मूल श्लोक
\end{quotation}

\begin{quotation}

अन्ये च बहवः शूरा, मदर्थे त्यक्त जीविताः  ।  

नाना शस्त्र प्रहरणाः, सर्वे युद्ध विशारदाः  ॥ १.९ ॥  उच्चारण

\noindent\rule{16cm}{0.4pt} 
\end{quotation}


\begin{quotation} 

अपर्याप्तं तदस्माकं बलं भीष्माभिरक्षितम्‌  ।  

पर्याप्तं त्विदमेतेषां बलं भीमाभिरक्षितम्‌  ॥ १.१० ॥  मूल श्लोक
\end{quotation}

\begin{quotation}

अपर्याप्तं तद अस्माकं, बलं भीष्मा भिरक्षितम्  ।  

पर्याप्तं त्व इदम इतेषां, बलं भीमा भिरक्षितम्  ॥ १.१० ॥  उच्चारण

\noindent\rule{16cm}{0.4pt} 
\end{quotation}


\begin{quotation} 
अयनेषु च सर्वेषु यथाभागमवस्थिताः  ।  

भीष्ममेवाभिरक्षन्तु भवन्तः सर्व एव हि  ॥ १.११ ॥  मूल श्लोक
\end{quotation}

\begin{quotation}

अयनेषु च सर्वेषु, यथा भागम अवस्थिताः  ।  

भीष्मम् एवा भिरक्षन्तु, भवन्तः सर्व एव हि  ॥ १.११ ॥  उच्चारण

\noindent\rule{16cm}{0.4pt} 
\end{quotation}


\begin{quotation} 

तस्य सञ्जनयन्हर्षं कुरुवृद्धः पितामहः  ।  

सिंहनादं विनद्योच्चैः शङ्खं दध्मो प्रतापवान्‌   ॥ १.१२ ॥  मूल श्लोक
\end{quotation}

\begin{quotation}

तस्य संजनयन् हर्षं, कुरु वृद्धः पितामहः  ।  

सिंहनादं विनद्योच्चैः, शङ्खं दध्मौ प्रतापवान्  ॥ १.१२ ॥  उच्चारण

\noindent\rule{16cm}{0.4pt} 
\end{quotation}


\begin{quotation} 

ततः शंखाश्च भेर्यश्च पणवानकगोमुखाः  ।  

सहसैवाभ्यहन्यन्त स शब्दस्तुमुलोऽभवत्‌  ॥ १.१३ ॥  मूल श्लोक
\end{quotation}

\begin{quotation}

ततः शङ्खाश् च भेर्यश् च, पण-वानक गोमुखाः  ।  

सहसा एवा अभय-अन्यन्त, स शब्दस तूमुलोऽ अभवत्  ॥ १.१३ ॥  उच्चारण

\noindent\rule{16cm}{0.4pt} 
\end{quotation}


\begin{quotation} 

ततः श्वेतैर्हयैर्युक्ते महति स्यन्दने स्थितौ  ।  

माधवः पाण्डवश्चैव दिव्यौ शंखौ प्रदध्मतुः   ॥ १.१४ ॥  मूल श्लोक
\end{quotation}

\begin{quotation}

ततः श्वेतैर् हयैर् युक्ते, महति स्यन्दने स्थितौ  ।  

माधवः पाण्डवश् च एैव, दिव्यौ शङ्खौ प्रदध्-मतुः  ॥ १.१४ ॥  उच्चारण

\noindent\rule{16cm}{0.4pt} 
\end{quotation}


\begin{quotation} 

पाञ्चजन्यं हृषीकेशो देवदत्तं धनञ्जयः  ।  

पौण्ड्रं दध्मौ महाशंख भीमकर्मा वृकोदरः  ॥ १.१५ ॥  मूल श्लोक
\end{quotation}

\begin{quotation}

पाञ्चजन्यं हृषीकेशो, देवदत्तं धनंजयः  ।  

पौण्ड्रं दध्मौ महाशङ्खं, भीम कर्मा वृकोदरः  ॥ १.१५ ॥  उच्चारण

\noindent\rule{16cm}{0.4pt} 
\end{quotation}


\begin{quotation} 

अनन्तविजयं राजा कुन्तीपुत्रो युधिष्ठिरः  ।  

नकुलः सहदेवश्च सुघोषमणिपुष्पकौ ।  
  ॥ १.१६ ॥  मूल श्लोक
\end{quotation}

\begin{quotation}

अनन्त विजयं राजा, कुन्ती पुत्रो युधिष्ठिरः  ।  

नकुलः सहदेवश् च, सुघोष मणि पुष्पकौ  ॥ १.१६ ॥  उच्चारण

\noindent\rule{16cm}{0.4pt} 
\end{quotation}


\begin{quotation} 
काश्यश्च परमेष्वासः शिखण्डी च महारथः  ।  

धृष्टद्युम्नो विराटश्च सात्यकिश्चापराजितः  ॥ १.१७ ॥  मूल श्लोक
\end{quotation}

\begin{quotation}

काश्यश् च परमेष्वासः, शिखण्डी च महारथः  ।  

धृष्ट-द्युम्-नो विराटश् च, सात्यकिश् च पराजितः  ॥ १.१७ ॥  उच्चारण

\noindent\rule{16cm}{0.4pt} 
\end{quotation}


\begin{quotation} 

द्रुपदो द्रौपदेयाश्च सर्वशः पृथिवीपते  ।  

सौभद्रश्च महाबाहुः शंखान्दध्मुः पृथक्पृथक्‌  ॥ १.१८ ॥  मूल श्लोक
\end{quotation}

\begin{quotation}

द्रुपदो द्रौपदेयाश् च, सर्वशः पृथिवीपते  ।  

सौभद्रश् च महाबाहुः, शङ्खान् दध्मुः पृथक्-पृथक्  ॥ १.१८ ॥  उच्चारण

\noindent\rule{16cm}{0.4pt} 
\end{quotation}


\begin{quotation} 

स घोषो धार्तराष्ट्राणां हृदयानि व्यदारयत्‌  ।  

नभश्च पृथिवीं चैव तुमुलो व्यनुनादयन्‌  ॥ १.१९ ॥  मूल श्लोक
\end{quotation}

\begin{quotation}

स घोषो धार्त-राष्ट्राणां, हृदयानि व्यदार-यत्  ।  

नभश् च पृथिवीं च एैव, तुमुलो भयनु नादयन्  ॥ १.१९ ॥  उच्चारण

\noindent\rule{16cm}{0.4pt} 
\end{quotation}


\begin{quotation} 

अथ व्यवस्थितान्दृष्ट्वा धार्तराष्ट्रान्‌ कपिध्वजः  ।  

प्रवृत्ते शस्त्रसम्पाते धनुरुद्यम्य पाण्डवः  ।  
 
हृषीकेशं तदा वाक्यमिदमाह महीपते   ॥ १.२० ॥  मूल श्लोक
\end{quotation}

\begin{quotation}

अथ व्यवस्थितान् दृष्ट्वा, धार्त-राष्ट्रान् कपिध्वजः ।  

प्रवृत्ते शस्त्र सम्पाते, धनुर उद्यम्य पाण्डवः  ।  

हृषीकेशं तदा वाक्यम,  इदम अहा महीपते  ॥ १.२० ॥  उच्चारण

\noindent\rule{16cm}{0.4pt} 
\end{quotation}

\paragraph{\sanskrit अर्जुन उवाच}

\begin{quotation} 



सेनयोरुभयोर्मध्ये रथं स्थापय मेऽच्युत  ।  

यावदेतान्निरीक्षेऽहं योद्धुकामानवस्थितान्‌  ॥ १.२१ ॥ 

कैर्मया सह योद्धव्यमस्मिन् रणसमुद्यमे  ॥ १.२२ ॥  मूल श्लोक
\end{quotation}

\begin{quotation}

सेनयोर उभयोर मध्ये, रथं स्थापय मेऽ अच्युत  ।  

यावद इतान् निरीक्षेऽ अहं, योद्धु कामान् अवस्थितान्  ॥ १.२१ ॥ 

कैर्मया सह योद्ध-व्यम, अस्मिन् रण समुद्यमे  ॥ १.२२ ॥  उच्चारण

\noindent\rule{16cm}{0.4pt} 
\end{quotation}


\begin{quotation} 

योत्स्यमानानवेक्षेऽहं य एतेऽत्र समागताः  ।  

धार्तराष्ट्रस्य दुर्बुद्धेर्युद्धे प्रियचिकीर्षवः  ॥ १.२३ ॥  मूल श्लोक
\end{quotation}

\begin{quotation}

योत्स्य मानान् अवेक्षेऽ अहं, य एतेऽ अत्र समागताः  ।  

धार्त-राष्ट्रस्य दुर्बुद्धेर, युद्ध प्रिये चिकीर्षवः  ॥ १.२३ ॥  उच्चारण

\noindent\rule{16cm}{0.4pt} 
\end{quotation}


\paragraph{\sanskrit सञ्जय उवाच}
\begin{quotation} 


एवमुक्तो हृषीकेशो गुडाकेशेन भारत  ।  

सेनयोरुभयोर्मध्ये स्थापयित्वा रथोत्तमम्‌  ॥ १.२४ ॥  मूल श्लोक
\end{quotation}

\begin{quotation}

एवम उक्तो हृषीकेशो, गुडाकेशेन भारत  ।  

सेनयोर उभयोर मध्ये, स्थापयित्वा रथ-उत्तमम्  ॥ १.२४ ॥  उच्चारण

\noindent\rule{16cm}{0.4pt} 
\end{quotation}


\begin{quotation} 

भीष्मद्रोणप्रमुखतः सर्वेषां च महीक्षिताम्‌  ।  

उवाच पार्थ पश्यैतान्‌ समवेतान्‌ कुरूनिति  ॥ १.२५ ॥  मूल श्लोक
\end{quotation}

\begin{quotation}

भीष्म द्रोण प्रमुखतः, सर्वेषां च महीक्षिताम्  ।  

उवाच पार्थ पश्यैतान्, समवेतान् कुरून् इति  ॥ १.२५ ॥  उच्चारण

\noindent\rule{16cm}{0.4pt} 
\end{quotation}


\begin{quotation} 

तत्रापश्यत्स्थितान्‌ पार्थः पितृनथ पितामहान्‌  ।  

आचार्यान्मातुलान्भ्रातृन्पुत्रान्पौत्रान्सखींस्तथा  ।  

श्वशुरान्‌ सुहृदश्चैव सेनयोरुभयोरपि  ॥ १.२६ - १.२७ ॥  मूल श्लोक
\end{quotation}

\begin{quotation}

तत्रा पश्यत स्थितान् पार्थः, पितृ़न् अथ पितामहान्  ।  

आचार्यान् मातुलान् भ्रातेन्, पुत्रान् पौत्रान् सखींस तथा  । 

श्वशुरान् सुहृदश च एैव, सेनयोर उभयोर अपि  ॥ १.२६ - १.२७ ॥  उच्चारण

\noindent\rule{16cm}{0.4pt} 
\end{quotation}


\begin{quotation} 


तान्समीक्ष्य स कौन्तेयः सर्वान्‌ बन्धूनवस्थितान्‌  ।  
 
कृपया परयाविष्टो विषीदत्रिदमब्रवीत्‌  ॥ १.२७ - १.२८ ॥  मूल श्लोक
\end{quotation}

\begin{quotation}

तान समीक्ष्य स कौन्तेयः, सर्वान् बन्धून अवस्थितान्  ।  

कृपया परयाविष्टो, विषीदन्न इदं अब्रवीत्  ॥ १.२७ - १.२८ ॥  उच्चारण

\noindent\rule{16cm}{0.4pt} 
\end{quotation}

\paragraph{\sanskrit अर्जुन उवाच}

\begin{quotation} 


दृष्टेवमं स्वजनं कृष्ण युयुत्सुं समुपस्थितम्‌  ।  

सीदन्ति मम गात्राणि मुखं च परिशुष्यति  । 
 
वेपथुश्च शरीरे में रोमहर्षश्च जायते  ॥ १.२८ - १.२९ ॥  मूल श्लोक
\end{quotation}

\begin{quotation}

दृष्ट्वेमं स्वजनं कृष्ण, युयुत्सुं सम-उपस्थितम्  ।  

सीदन्ति मम गात्राणि, मुखं च परिश उष्यति  । 

वेपथुश् च शरीरे में, रोमहर्षश् च जायते  ॥ १.२८ - १.२९ ॥  उच्चारण

\noindent\rule{16cm}{0.4pt} 
\end{quotation}


\begin{quotation} 

गाण्डीवं स्रंसते हस्तात्त्वक्चैव परिदह्यते  ।  
 
न च शक्नोम्यवस्थातुं भ्रमतीव च मे मनः  ॥ १.३० ॥  मूल श्लोक
\end{quotation}

\begin{quotation} 

गाण्डीवं स्रंसते हस्तात्, त्वक च एैव परिदह्यते  ।  

न च शक्नोम्य अवस्थातुं, भ्रमतीव च मे मनः  ॥ १.३० ॥  उच्चारण

\noindent\rule{16cm}{0.4pt} 
\end{quotation}


\begin{quotation} 

निमित्तानि च पश्यामि विपरीतानि केशव  ।  
 
न च श्रेयोऽनुपश्यामि हत्वा स्वजनमाहवे  ॥ १.३१ ॥  मूल श्लोक
\end{quotation}

\begin{quotation}

निमित्तानि च पश्यामि, विपरीतानि केशव  ।  
 
न च श्रेयोऽ अनु-पश्यामि, हत्वा स्वजनम आहवे  ॥ १.३१ ॥  उच्चारण

\noindent\rule{16cm}{0.4pt} 
\end{quotation}


\begin{quotation} 

न काङ्‍क्षे विजयं कृष्ण न च राज्यं सुखानि च  ।  
 
किं नो राज्येन गोविंद किं भोगैर्जीवितेन वा  ॥ १.३२ ॥  मूल श्लोक
\end{quotation}

\begin{quotation}

न काङ्क्षे विजयं कृष्ण, न च राज्यं सुखानि च  ।  
 
किं नो राज्येन गोविन्द, किं भोगैर् जीवितेन वा  ॥ १.३२ ॥  उच्चारण

\noindent\rule{16cm}{0.4pt} 
\end{quotation}


\begin{quotation} 

येषामर्थे काङक्षितं नो राज्यं भोगाः सुखानि च  ।  

त इमेऽवस्थिता युद्धे प्राणांस्त्यक्त्वा धनानि च  ॥ १.३३ ॥  मूल श्लोक
\end{quotation}

\begin{quotation}

येषाम अर्थे काङ्क्षितं नो, राज्यं भोगाः सुखानि च  ।  

त इमेऽ अवस्थिता युद्धे, प्राणांस्य त्यक्त्वा धनानि च  ॥ १.३३ ॥  उच्चारण

\noindent\rule{16cm}{0.4pt} 
\end{quotation}


\begin{quotation} 

आचार्याः पितरः पुत्रास्तथैव च पितामहाः  ।  
 
मातुलाः श्वशुराः पौत्राः श्यालाः संबंधिनस्तथा  ॥ १.३४ ॥  मूल श्लोक
\end{quotation}

\begin{quotation}

आचार्याः पितरः पुत्रास, तथैव च पितामहाः ।  

मातुलाः च श्रशुराः पौत्राः श्यालाः सम्बन्धिनस तथा  ॥ १.३४ ॥  उच्चारण

\noindent\rule{16cm}{0.4pt} 
\end{quotation}


\begin{quotation} 

एतान्न हन्तुमिच्छामि घ्नतोऽपि मधुसूदन  ।  
 
अपि त्रैलोक्यराज्यस्य हेतोः किं नु महीकृते  ॥ १.३५ ॥  मूल श्लोक
\end{quotation}

\begin{quotation}

एतान् न हन्तुम इच्छामि, घ्नतोऽ अपि मधुसूदन ।  

अपि त्रैलोक्य राज्यस्य हेतोः, किं नु महीकृते  ॥ १.३५ ॥   उच्चारण

\noindent\rule{16cm}{0.4pt} 
\end{quotation}


\begin{quotation} 

निहत्य धार्तराष्ट्रान्न का प्रीतिः स्याज्जनार्दन  ।  


पापमेवाश्रयेदस्मान्‌ हत्वैतानाततायिनः  ॥ १.३६ ॥  मूल श्लोक
\end{quotation}

\begin{quotation}

निहत्य धार्तराष्ट्रान् नः, का प्रीतिः स्याज् जनार्दन ।  


पापम एव आश्रयेद अस्मां, हत्वैतान आततायिनः  ॥ १.३६ ॥  उच्चारण

\noindent\rule{16cm}{0.4pt} 
\end{quotation}


\begin{quotation} 

तस्मान्नार्हा वयं हन्तुं धार्तराष्ट्रान्स्वबान्धवान्‌  ।  
 

स्वजनं हि कथं हत्वा सुखिनः स्याम माधव  ॥ १.३७ ॥  मूल श्लोक
\end{quotation}

\begin{quotation}

तस्मान् नर्हा वयं हन्तुं, धार्त-राष्ट्रान् स बान्धवान्  ।  


स्वजनं हि कथं हत्वा, सुखिनः स्याम माधव  ॥ १.३७ ॥  उच्चारण

\noindent\rule{16cm}{0.4pt}
\end{quotation}


\begin{quotation}
यद्यप्येते न पश्यन्ति लोभोपहतचेतसः  ।  

कुलक्षयकृतं दोषं मित्रद्रोहे च पातकम्‌  ॥ १.३८ ॥  मूल श्लोक
\end{quotation}

\begin{quotation}

यद्य अप्य एते न पश्यन्ति, लोभो-पहत चेतसः  ।  


कुलक्षय कृतं दोषं, मित्र-द्रोहे च पातकम्  ॥ १.३८ ॥  उच्चारण

\noindent\rule{16cm}{0.4pt} 
\end{quotation}


\begin{quotation} 
कथं न ज्ञेयमस्माभिः पापादस्मान्निवर्तितुम्‌  ।  


कुलक्षयकृतं दोषं प्रपश्यद्भिर्जनार्दन  ॥ १.३९ ॥  मूल श्लोक
\end{quotation}

\begin{quotation}

कथं न ज्ञेयम अस्माभिः, पापाद अस्मान नि-वर्तितुम्‌  ।  


कुलक्षय कृतं दोषं, प्रपश्य-यद्भिर जनार्दन  ॥ १.३९ ॥  उच्चारण

\noindent\rule{16cm}{0.4pt} 
\end{quotation}


\begin{quotation} 

कुलक्षये प्रणश्यन्ति कुलधर्माः सनातनाः  ।  


धर्मे नष्टे कुलं कृत्स्नमधर्मोऽभिभवत्युत  ॥ १.४० ॥  मूल श्लोक
\end{quotation}

\begin{quotation}

कुलक्षये प्रणश्यन्ति, कुलधर्माः सनातनाः  ।  


धर्मे नष्टे कुलं कृत्सनं, अधर्मोऽ अभि-भवत्य उत  ॥ १.४० ॥  उच्चारण

\noindent\rule{16cm}{0.4pt} 
\end{quotation}


\begin{quotation} 

अधर्माभिभवात्कृष्ण प्रदुष्यन्ति कुलस्त्रियः  ।  


स्त्रीषु दुष्टासु वार्ष्णेय जायते वर्णसंकरः  ॥ १.४१ ॥  मूल श्लोक
\end{quotation}

\begin{quotation}

अधर्मा अभि-भवात कृष्ण, प्रदुष्यन्ति कुल-स्त्रियः  ।  


स्त्रीषु दुष्टासु वार्ष्णेय, जायते वर्णसङ्करः  ॥ १.४१ ॥  उच्चारण

\noindent\rule{16cm}{0.4pt} 
\end{quotation}


\begin{quotation} 

संकरो नरकायैव कुलघ्नानां कुलस्य च  ।  


पतन्ति पितरो ह्येषां लुप्तपिण्डोदकक्रियाः  ॥ १.४२ ॥  मूल श्लोक
\end{quotation}

\begin{quotation}

संकरो नरकाय एैव, कुलघ्नानां कुलस्य च  ।  


पतन्ति पितरो ह्य एषां, लुप्त पिण्डोदक क्रियाः  ॥ १.४२ ॥  उच्चारण

\noindent\rule{16cm}{0.4pt} 
\end{quotation}


\begin{quotation} 

दोषैरेतैः कुलघ्नानां वर्णसंकरकारकैः  ।  


उत्साद्यन्ते जातिधर्माः कुलधर्माश्च शाश्वताः  ॥ १.४३ ॥  मूल श्लोक
\end{quotation}

\begin{quotation}

दोषैर एतैः कुलघ्नानां, वर्णसङ्कर कारकैः  ।  


उत्साद्-यन्ते जाति-धर्माः, कुल धर्माश् च शाश्वताः  ॥ १.४३ ॥  उच्चारण

\noindent\rule{16cm}{0.4pt} 
\end{quotation}


\begin{quotation} 

उत्सन्नकुलधर्माणां मनुष्याणां जनार्दन  ।  


नरकेऽनियतं वासो भवतीत्यनुशुश्रुम  ॥ १.४४ ॥  मूल श्लोक
\end{quotation}

\begin{quotation}

उत्सन्न कुल धर्माणां, मनुष्याणां जनार्दन  ।  


नरकेऽ नियतं वासो, भवतीत अनु-शु-श्रुम  ॥ १.४४ ॥  उच्चारण

\noindent\rule{16cm}{0.4pt} 
\end{quotation}


\begin{quotation} 
अहो बत महत्पापं कर्तुं व्यवसिता वयम्‌  ।  


यद्राज्यसुखलोभेन हन्तुं स्वजनमुद्यताः  ॥ १.४५ ॥  मूल श्लोक
\end{quotation}

\begin{quotation}

अहो बत महत्पापं, कर्तुं व्यवसिता वयम्  ।  


यद राज्य सुख लोभेन, हन्तुं स्वजनं उद्यताः  ॥ १.४५ ॥  उच्चारण

\noindent\rule{16cm}{0.4pt} 
\end{quotation}


\begin{quotation} 

यदि मामप्रतीकारमशस्त्रं शस्त्रपाणयः  ।  


धार्तराष्ट्रा रणे हन्युस्तन्मे क्षेमतरं भवेत्‌  ॥ १.४६ ॥  मूल श्लोक
\end{quotation}

\begin{quotation}

यदि माम अ-प्रतीकारम, अशस्त्रं शस्त्र पाणयः  ।  


धार्त-राष्ट्रा रणे हन्युस्, तन्मे क्षेमतरं भवेत्  ॥ १.४६ ॥  उच्चारण

\noindent\rule{16cm}{0.4pt} 
\end{quotation}

\paragraph{\sanskrit सञ्जय उवाच}

\begin{quotation} 


एवमुक्त्वार्जुनः सङ्‍ख्ये रथोपस्थ उपाविशत्‌  ।  


विसृज्य सशरं चापं शोकसंविग्नमानसः  ॥ १.४७ ॥  मूल श्लोक
\end{quotation}

\begin{quotation}

एव मुक्त्वाऽ अर्जुनः संख्ये, रथोपस्थ उपाविशत्  ।  


वि-सृज्य स शरं चापं, शोक संविग्न मानसः  ॥ १.४७ ॥  उच्चारण

\noindent\rule{16cm}{0.4pt} 
\end{quotation}


\begin{center} ***** \end{center}
\begin{quotation}
ॐ तत् सद इति श्री मद्-भगवद्-गीतास उपनिषत्सु ब्रह्म विद्यायां योगशास्त्रे श्री कृष्णार्जुन संवादे अर्जुनविषादयोगो नाम प्रथमोऽ अध्यायः  ॥  १ ॥ 
\end{quotation}