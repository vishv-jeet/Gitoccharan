\chapter{\sanskrit कर्मसंन्यासयोग}
\paragraph{\sanskrit अर्जुन उवाच}

\begin{quotation}

सन्न्यासं कर्मणां कृष्ण पुनर्योगं च शंससि  ।  

यच्छ्रेय एतयोरेकं तन्मे ब्रूहि सुनिश्चितम्‌  ॥ ५.१ ॥  मूल श्लोक
\end{quotation}

\begin{quotation}

सन्न्यासं कर्मणां कृष्ण, पुनर्योगं च शंससि  ।  

यच्छ्रेय एतयोर एकं, तन्मे ब्रूहि सुनिश्चितम्‌  ॥ ५.१ ॥  उच्चारण

\noindent\rule{16cm}{0.4pt} 
\end{quotation}

\paragraph{\sanskrit श्रीभगवानुवाच}
\begin{quotation}  


सन्न्यासः कर्मयोगश्च निःश्रेयसकरावुभौ  ।  

तयोस्तु कर्मसन्न्यासात्कर्मयोगो विशिष्यते  ॥ ५.२ ॥  मूल श्लोक
\end{quotation}

\begin{quotation}

सन्न्यासः कर्मयोगश् च, निःश्रेयस-कराव उभौ  ।  

तयोस्तु कर्म सन्न्यासात्,  कर्म-योगो विशिष्यते  ॥ ५.२ ॥  उच्चारण

\noindent\rule{16cm}{0.4pt} 
\end{quotation}


\begin{quotation}  

ज्ञेयः स नित्यसन्न्यासी यो न द्वेष्टि न काङ्‍क्षति  ।  

निर्द्वन्द्वो हि महाबाहो सुखं बन्धात्प्रमुच्यते  ॥ ५.३ ॥  मूल श्लोक
\end{quotation}

\begin{quotation}

ज्ञेयः स नित्य-सन्न्यासी, यो न द्वेष्टि न काङ्‍क्षति  ।  

निर्द्वन्द्वो हि महाबाहो, सुखं बन्धात प्रमुच्यते  ॥ ५.३ ॥  उच्चारण

\noindent\rule{16cm}{0.4pt} 
\end{quotation}


\begin{quotation}  

साङ्‍ख्ययोगौ पृथग्बालाः प्रवदन्ति न पण्डिताः  ।  

एकमप्यास्थितः सम्यगुभयोर्विन्दते फलम्‌  ॥ ५.४ ॥  मूल श्लोक
\end{quotation}

\begin{quotation}

साङ्‍ख्य योगौ पृथग्बालाः, प्रवदन्ति न पण्डिताः  ।  

एकम अपि अस्थितः सम्यग, उभयोर विन्दते फलम्‌  ॥ ५.४ ॥  उच्चारण

\noindent\rule{16cm}{0.4pt} 
\end{quotation}


\begin{quotation}  

यत्साङ्‍ख्यैः प्राप्यते स्थानं तद्यौगैरपि गम्यते  ।  

एकं साङ्‍ख्यं च योगं च यः पश्यति स पश्यति  ॥ ५.५ ॥  मूल श्लोक
\end{quotation}

\begin{quotation}

यत साङ्‍ख्यैः प्राप्यते स्थानं, तद योगेर अपि गम्यते  ।  

एकं साङ्‍ख्यं च योगं च, यः पश्यति स पश्यति  ॥ ५.५ ॥  उच्चारण

\noindent\rule{16cm}{0.4pt} 
\end{quotation}


\begin{quotation}  

सन्न्यासस्तु महाबाहो दुःखमाप्तुमयोगतः  ।  

योगयुक्तो मुनिर्ब्रह्म नचिरेणाधिगच्छति  ॥ ५.६ ॥  मूल श्लोक
\end{quotation}

\begin{quotation}

सन्न्यास अस्तु महाबाहो, दुःखं अप्तुम अयोगतः  ।  

योग-युक्तो मुनिर्ब्रह्म, न चिरेणा अधिगच्छति  ॥ ५.६ ॥  उच्चारण

\noindent\rule{16cm}{0.4pt} 
\end{quotation}


\begin{quotation}  

योगयुक्तो विशुद्धात्मा विजितात्मा जितेन्द्रियः  ।  

सर्वभूतात्मभूतात्मा कुर्वन्नपि न लिप्यते  ॥ ५.७ ॥  मूल श्लोक
\end{quotation}

\begin{quotation}

योग-युक्तो विशुद्धात्मा, विजित आत्मा जितेन्द्रियः  ।  

सर्व-भूतात्म-भूतात्मा, कुर्वन्न अपि न लिप्यते  ॥ ५.७ ॥  उच्चारण

\noindent\rule{16cm}{0.4pt} 
\end{quotation}


\begin{quotation}  

नैव किंचित्करोमीति युक्तो मन्येत तत्ववित्‌  ।  

पश्यञ्शृण्वन्स्पृशञ्जिघ्रन्नश्नन्गच्छन्स्वपन्श्वसन्‌  ॥ ५.८ ॥  मूल श्लोक
\end{quotation}

\begin{quotation}

नैव किंचित करोमीति, युक्तो मन्येत तत्व वित्‌  ।  

पश्यन् शृण्वन् स्पर्शन् जिघ्रन्न, अस्नन् गच्छन् स्वपन् श्वसन्‌  ॥ ५.८ ॥  उच्चारण

\noindent\rule{16cm}{0.4pt} 
\end{quotation}


\begin{quotation}  

प्रलपन्विसृजन्गृह्णन्नुन्मिषन्निमिषन्नपि  । 

इन्द्रियाणीन्द्रियार्थेषु वर्तन्त इति धारयन्‌  ॥ ५.९ ॥  मूल श्लोक
\end{quotation}

\begin{quotation}

प्रलपन् विसृजन् गृहणन्, उन्मिषन् निमिषन्न अपि  । 

इन्द्रियाणी इन्द्रिय-आर्थेषु, वर्तन्त इति धारयन्‌  ॥ ५.९ ॥  उच्चारण

\noindent\rule{16cm}{0.4pt} 
\end{quotation}


\begin{quotation}  

ब्रह्मण्याधाय कर्माणि सङ्‍गं त्यक्त्वा करोति यः  ।  

लिप्यते न स पापेन पद्मपत्रमिवाम्भसा  ॥ ५.१० ॥  मूल श्लोक
\end{quotation}

\begin{quotation}

ब्रह्मण्य अधाय कर्माणि, सङ्‍गं त्यक्त्वा करोति यः  ।  

लिप्यते न स पापेन, पदम्-पत्रम इवाम्-भसा  ॥ ५.१० ॥  उच्चारण

\noindent\rule{16cm}{0.4pt} 
\end{quotation}


\begin{quotation}  

कायेन मनसा बुद्धया केवलैरिन्द्रियैरपि  ।  

योगिनः कर्म कुर्वन्ति संग त्यक्त्वात्मशुद्धये  ॥ ५.११ ॥  मूल श्लोक
\end{quotation}

\begin{quotation}
कायेन मनसा बुद्धया, केवलैर इन्द्रियैर अपि  ।  

योगिनः कर्म कुर्वन्ति, संग त्यक्त्व-आत्मा शुद्धये  ॥ ५.११ ॥  उच्चारण

\noindent\rule{16cm}{0.4pt} 
\end{quotation}


\begin{quotation}  

युक्तः कर्मफलं त्यक्त्वा शान्तिमाप्नोति नैष्ठिकीम्‌  ।  

अयुक्तः कामकारेण फले सक्तो निबध्यते  ॥ ५.१२ ॥  मूल श्लोक
\end{quotation}

\begin{quotation}

युक्तः कर्मफलं त्यक्त्वा, शान्तिम् अपनोति नैष्ठिकीम्‌  ।  

अयुक्तः काम-कारेण, फले सक्तो निबध्यते  ॥ ५.१२ ॥  उच्चारण

\noindent\rule{16cm}{0.4pt} 
\end{quotation}


\begin{quotation}  

सर्वकर्माणि मनसा संन्यस्यास्ते सुखं वशी  ।  

नवद्वारे पुरे देही नैव कुर्वन्न कारयन्‌  ॥ ५.१३ ॥  मूल श्लोक
\end{quotation}

\begin{quotation}

सर्व-कर्माणि मनसा, संन्यस्या-अस्ते सुखं वशी  ।  

नवद्वारे पुरे देही, नैव कुर्वन्न कारयन्‌  ॥ ५.१३ ॥  उच्चारण

\noindent\rule{16cm}{0.4pt} 
\end{quotation}


\begin{quotation}  

न कर्तृत्वं न कर्माणि लोकस्य सृजति प्रभुः  ।  

न कर्मफलसंयोगं स्वभावस्तु प्रवर्तते  ॥ ५.१४ ॥  मूल श्लोक
\end{quotation}

\begin{quotation}

न कर-तृत्वं न कर्माणि, लोकस्य सृजति प्रभुः  ।  

न कर्मफल संयोगं, स्वभावस्तु प्रवर्तते  ॥ ५.१४ ॥  उच्चारण

\noindent\rule{16cm}{0.4pt} 
\end{quotation}


\begin{quotation}  

नादत्ते कस्यचित्पापं न चैव सुकृतं विभुः  ।  

अज्ञानेनावृतं ज्ञानं तेन मुह्यन्ति जन्तवः  ॥ ५.१५ ॥  मूल श्लोक
\end{quotation}

\begin{quotation}

नादत्ते कस्यचित पापं, न चैव सुकृतं विभुः  ।  

अज्ञाने न अवृतं ज्ञानं, तेन मुह्यन्ति जन्तवः  ॥ ५.१५ ॥  उच्चारण

\noindent\rule{16cm}{0.4pt} 
\end{quotation}


\begin{quotation}  

ज्ञानेन तु तदज्ञानं येषां नाशितमात्मनः  ।  

तेषामादित्यवज्ज्ञानं प्रकाशयति तत्परम्‌  ॥ ५.१६ ॥  मूल श्लोक
\end{quotation}

\begin{quotation}

ज्ञानेन तु तद अज्ञानं, येषां नाशितम् आत्मनः  ।  

तेषां आदित्य वज ज्ञानं, प्रकाश-यति तत्परम्‌  ॥ ५.१६ ॥  उच्चारण

\noindent\rule{16cm}{0.4pt} 
\end{quotation}


\begin{quotation}  

तद्‍बुद्धयस्तदात्मानस्तन्निष्ठास्तत्परायणाः  ।  

गच्छन्त्यपुनरावृत्तिं ज्ञाननिर्धूतकल्मषाः  ॥ ५.१७ ॥  मूल श्लोक
\end{quotation}

\begin{quotation}
तद्‍-बुद्धयास तद-आत्मानः, तन्निष्ठाः तत परायणाः  ।  

गच्छन्त्य पुनरावृत्तिं, ज्ञान निर्धूत कल्मषाः  ॥ ५.१७ ॥  उच्चारण

\noindent\rule{16cm}{0.4pt} 
\end{quotation}


\begin{quotation}  

विद्याविनयसम्पन्ने ब्राह्मणे गवि हस्तिनि  ।  

शुनि चैव श्वपाके च पण्डिताः समदर्शिनः  ॥ ५.१८ ॥  मूल श्लोक
\end{quotation}

\begin{quotation}

विद्या विनय सम्पन्ने, ब्राह्मणे गवि हस्तिनि  ।  

शुनि चैव श्वपाके च, पण्डिताः समदर्शिनः  ॥ ५.१८ ॥  उच्चारण

\noindent\rule{16cm}{0.4pt} 
\end{quotation}


\begin{quotation}  

इहैव तैर्जितः सर्गो येषां साम्ये स्थितं मनः  ।  

निर्दोषं हि समं ब्रह्म तस्माद् ब्रह्मणि ते स्थिताः  ॥ ५.१९ ॥  मूल श्लोक
\end{quotation}

\begin{quotation}

इहैव तैर जितः सर्गो, येषां साम्ये स्थितं मनः  ।  

निर्दोषं हि समं ब्रह्म, तस्माद् ब्रह्मणि ते स्थिताः  ॥ ५.१९ ॥  उच्चारण

\noindent\rule{16cm}{0.4pt} 
\end{quotation}


\begin{quotation}  

न प्रहृष्येत्प्रियं प्राप्य नोद्विजेत्प्राप्य चाप्रियम्‌  ।  

स्थिरबुद्धिरसम्मूढो ब्रह्मविद् ब्रह्मणि स्थितः  ॥ ५.२० ॥  मूल श्लोक
\end{quotation}

\begin{quotation}

न प्रहृष्य एत प्रियं प्राप्य, न उद-विजेत प्राप्य च प्रियम्‌  ।  

स्थिर बुद्धिर सम्मूढो, ब्रह्म-विद् ब्रह्मणि स्थितः  ॥ ५.२० ॥  उच्चारण

\noindent\rule{16cm}{0.4pt} 
\end{quotation}


\begin{quotation}  

बाह्यस्पर्शेष्वसक्तात्मा विन्दत्यात्मनि यत्सुखम्‌  ।  

स ब्रह्मयोगयुक्तात्मा सुखमक्षयमश्नुते  ॥ ५.२१ ॥  मूल श्लोक
\end{quotation}

\begin{quotation}

बाह्य स्पर्श-एष्व असक्त-आत्मा, विन्दत्य-आत्मनि यत्-सुखम्‌  ।  

स ब्रह्म-योग-युक्तात्मा, सुखम-अक्षयम-अश्-नुते  ॥ ५.२१ ॥  उच्चारण

\noindent\rule{16cm}{0.4pt} 
\end{quotation}


\begin{quotation}  

ये हि संस्पर्शजा भोगा दुःखयोनय एव ते  ।  

आद्यन्तवन्तः कौन्तेय न तेषु रमते बुधः  ॥ ५.२२ ॥  मूल श्लोक
\end{quotation}

\begin{quotation}

ये हि संस्पर्श-जा भोगा, दुःख-योनय एव ते  ।  

आद्य-अन्त-वन्तः कौन्तेय, न तेषु रमते बुधः  ॥ ५.२२ ॥  उच्चारण

\noindent\rule{16cm}{0.4pt} 
\end{quotation}


\begin{quotation}  

शक्नोतीहैव यः सोढुं प्राक्शरीरविमोक्षणात्‌  ।  

कामक्रोधोद्भवं वेगं स युक्तः स सुखी नरः  ॥ ५.२३ ॥  मूल श्लोक
\end{quotation}

\begin{quotation}
शक्नोती-हैव यः सोढुं, प्राक-शरीर विमोक्षणात्‌  ।  

काम-क्रोध-उद्भवं वेगं, स युक्तः स सुखी नरः  ॥ ५.२३ ॥  उच्चारण

\noindent\rule{16cm}{0.4pt} 
\end{quotation}


\begin{quotation}  

योऽन्तःसुखोऽन्तरारामस्तथान्तर्ज्योतिरेव यः  ।  

स योगी ब्रह्मनिर्वाणं ब्रह्मभूतोऽधिगच्छति  ॥ ५.२४ ॥  मूल श्लोक
\end{quotation}

\begin{quotation}

योऽ अन्तः सुखोऽ अन्तर-अरामस्, तथान्तर ज्योतिर एव यः  ।  

स योगी ब्रह्म-निर्वाणं, ब्रह्म-भूतोऽ अधिगच्छति  ॥ ५.२४ ॥  उच्चारण

\noindent\rule{16cm}{0.4pt} 
\end{quotation}


\begin{quotation}  

लभन्ते ब्रह्मनिर्वाणमृषयः क्षीणकल्मषाः  ।  

छिन्नद्वैधा यतात्मानः सर्वभूतहिते रताः  ॥ ५.२५ ॥  मूल श्लोक
\end{quotation}

\begin{quotation}

लभन्ते ब्रह्म-निर्वाणम्, ऋषयः क्षीण कल्मषाः  ।  

छिन्न-द्वैधा यतात्मानः, सर्वभूत हिते रताः  ॥ ५.२५ ॥  उच्चारण

\noindent\rule{16cm}{0.4pt} 
\end{quotation}


\begin{quotation}  

कामक्रोधवियुक्तानां यतीनां यतचेतसाम्‌  ।  

अभितो ब्रह्मनिर्वाणं वर्तते विदितात्मनाम्‌  ॥ ५.२६ ॥  मूल श्लोक
\end{quotation}

\begin{quotation}

काम क्रोध वियुक्तानां, यतीनां यत-चेतसाम्‌  ।  

अभितो ब्रह्म-निर्वाणं, वर्तते विदित आत्मनाम्‌  ॥ ५.२६ ॥  उच्चारण

\noindent\rule{16cm}{0.4pt} 
\end{quotation}


\begin{quotation}  

स्पर्शान्कृत्वा बहिर्बाह्यांश्चक्षुश्चैवान्तरे भ्रुवोः  ।  

प्राणापानौ समौ कृत्वा नासाभ्यन्तरचारिणौ  ॥ ५.२७ ॥  मूल श्लोक
\end{quotation}

\begin{quotation}

स्पर्शान्-कृत्वा बहिर बाह्यांश्, चक्षुश चैवा-अन्तरे भ्रुवोः  ।  

प्राणा-पानौ समौ कृत्वा, नासाभ्यं तर-चारिणौ  ॥ ५.२७ ॥  उच्चारण

\noindent\rule{16cm}{0.4pt} 
\end{quotation}


\begin{quotation}  

यतेन्द्रियमनोबुद्धिर्मुनिर्मोक्षपरायणः  ।  

विगतेच्छाभयक्रोधो यः सदा मुक्त एव सः  ॥ ५.२८ ॥  मूल श्लोक
\end{quotation}

\begin{quotation}

यतेन्द्रिय मनो बुद्धिर, मुनिर मोक्ष परायणः  ।  

विगत इच्छा भय क्रोधो, यः सदा मुक्त एव सः  ॥ ५.२८ ॥  उच्चारण

\noindent\rule{16cm}{0.4pt} 
\end{quotation}


\begin{quotation}  

भोक्तारं यज्ञतपसां सर्वलोकमहेश्वरम्‌  ।  

सुहृदं सर्वभूतानां ज्ञात्वा मां शान्तिमृच्छति  ॥ ५.२९ ॥  मूल श्लोक
\end{quotation}

\begin{quotation}
भोक्तारं यज्ञ तपसां, सर्व लोक महेश्वरम्‌  ।  

सुहृदं सर्व-भूतानां, ज्ञात्वा मां शान्तिम्-रच्छति  ॥ ५.२९ ॥  उच्चारण

\noindent\rule{16cm}{0.4pt} 
\end{quotation}

\begin{center} ***** \end{center}

\begin{quotation}  


ॐ तत् सद इति श्री मद्-भगवद्-गीतास उपनिषत्सु ब्रह्म विद्यायां योगशास्त्रे श्री कृष्णार्जुन संवादे कर्मसंन्यासयोगो नाम पंचमोऽ अध्यायः  ॥  ५  ॥ 


\end{quotation}

