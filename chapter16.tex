\chapter{\sanskrit दैवासुरसम्पद् विभागयोग} 
\paragraph{\sanskrit श्रीभगवानुवाच}
\begin{quotation} 
अभयं सत्त्वसंशुद्धिर्ज्ञानयोगव्यवस्थितिः ।  

दानं दमश्च यज्ञश्च स्वाध्यायस्तप आर्जवम्‌  ॥ १६.१ ॥  मूल श्लोक
\end{quotation}

\begin{quotation}

अभयं सत्त्व संशुद्धिर्, ज्ञानयोग व्यवस्थितिः ।  

दानं दमश् च यज्ञश् च, स्वाध्यायस् तप आर्जवम्‌  ॥ १६.१ ॥  उच्चारण

\noindent\rule{16cm}{0.4pt} 
\end{quotation}


\begin{quotation}

अहिंसा सत्यमक्रोधस्त्यागः शान्तिरपैशुनम्‌ ।  

दया भूतेष्वलोलुप्त्वं मार्दवं ह्रीरचापलम्‌  ॥ १६.२ ॥  मूल श्लोक
\end{quotation}

\begin{quotation}

अहिंसा सत्यम अक्रोधस्, त्यागः शान्तिर अपैशुनम्‌ ।  

दया भूतेष्व अलोलुप्त्वं,  मार्दवं ह्रीर अचापलम्‌  ॥ १६.२ ॥  उच्चारण

\noindent\rule{16cm}{0.4pt} 
\end{quotation}


\begin{quotation}

तेजः क्षमा धृतिः शौचमद्रोहोनातिमानिता ।  

भवन्ति सम्पदं दैवीमभिजातस्य भारत  ॥ १६.३ ॥  मूल श्लोक
\end{quotation}

\begin{quotation}

तेजः क्षमा धृतिः शौचम, अद्रोहो नाति-मानिता ।  

भवन्ति सम्पदं दैवीम, अभि-जातस्य भारत  ॥ १६.३ ॥  उच्चारण

\noindent\rule{16cm}{0.4pt} 
\end{quotation}


\begin{quotation}

दम्भो दर्पोऽभिमानश्च क्रोधः पारुष्यमेव च ।  

अज्ञानं चाभिजातस्य पार्थ सम्पदमासुरीम्‌  ॥ १६.४ ॥  मूल श्लोक
\end{quotation}

\begin{quotation}

दम्भो दर्पोऽ अभिमानश् च, क्रोधः पारुष्यम एव च ।  

अज्ञानं चा अभि-जातस्य, पार्थ सम्पदम आसुरीम्‌  ॥ १६.४ ॥  उच्चारण

\noindent\rule{16cm}{0.4pt} 
\end{quotation}


\begin{quotation}

दैवी सम्पद्विमोक्षाय निबन्धायासुरी मता ।  

मा शुचः सम्पदं दैवीमभिजातोऽसि पाण्डव  ॥ १६.५ ॥  मूल श्लोक
\end{quotation}

\begin{quotation}

दैवी सम्पद् विमोक्षाय, निबन्धाय आसुरी मता ।  

मा शुचः सम्पदं दैवीम, अभिजातोऽ असि पाण्डव  ॥ १६.५ ॥  उच्चारण

\noindent\rule{16cm}{0.4pt} 
\end{quotation}


\begin{quotation}
द्वौ भूतसर्गौ लोकेऽस्मिन्दैव आसुर एव च ।  

दैवो विस्तरशः प्रोक्त आसुरं पार्थ में शृणु  ॥ १६.६ ॥  मूल श्लोक
\end{quotation}

\begin{quotation}

द्वौ भूत-सर्गौ लोकेऽ अस्मिन्, दैव आसुर एव च ।  

दैवो विस्तरशः प्रोक्त, आसुरं पार्थ में शृणु  ॥ १६.६ ॥  उच्चारण

\noindent\rule{16cm}{0.4pt} 
\end{quotation}


\begin{quotation}

प्रवृत्तिं च निवृत्तिं च जना न विदुरासुराः ।  

न शौचं नापि चाचारो न सत्यं तेषु विद्यते  ॥ १६.७ ॥  मूल श्लोक
\end{quotation}

\begin{quotation}

प्रवृत्तिं च निवृत्तिं च, जना न विदुर आसुराः ।  

न शौचं नापि चाचारो, न सत्यं तेषु विद्यते  ॥ १६.७ ॥  उच्चारण

\noindent\rule{16cm}{0.4pt} 
\end{quotation}


\begin{quotation}

असत्यमप्रतिष्ठं ते जगदाहुरनीश्वरम्‌ ।  

अपरस्परसम्भूतं किमन्यत्कामहैतुकम्‌  ॥ १६.८ ॥  मूल श्लोक
\end{quotation}

\begin{quotation}

असत्यम अप्रतिष्ठं ते, जगद आहुर अनीश्वरम्‌ ।  

अपरस्पर सम्भूतं, किम अन्यत् काम हैतुकम्‌  ॥ १६.८ ॥  उच्चारण

\noindent\rule{16cm}{0.4pt} 
\end{quotation}


\begin{quotation}

एतां दृष्टिमवष्टभ्य नष्टात्मानोऽल्पबुद्धयः ।  

प्रभवन्त्युग्रकर्माणः क्षयाय जगतोऽहिताः  ॥ १६.९ ॥  मूल श्लोक
\end{quotation}

\begin{quotation}

एतां दृष्टिम अवष्-टभ्य, नष्ट आत्मानोऽ अल्प बुद्धयः ।  

प्रभवन्त्य उग्र-कर्माणः, क्षयाय जगतोऽ अहिताः  ॥ १६.९ ॥  उच्चारण

\noindent\rule{16cm}{0.4pt} 
\end{quotation}


\begin{quotation}

काममाश्रित्य दुष्पूरं दम्भमानमदान्विताः ।  

मोहाद्‍गृहीत्वासद्ग्राहान्प्रवर्तन्तेऽशुचिव्रताः  ॥ १६.१० ॥  मूल श्लोक
\end{quotation}

\begin{quotation}

कामम आश्रित्य दुष्पूरं, दम्भ मान मदान्विताः ।  

मोहाद्‍ गृहीत्वा-असद्-ग्राहान्, प्र-वर्तन्तेऽ अशुचि व्रताः  ॥ १६.१० ॥  उच्चारण

\noindent\rule{16cm}{0.4pt} 
\end{quotation}


\begin{quotation}

चिन्तामपरिमेयां च प्रलयान्तामुपाश्रिताः ।  

कामोपभोगपरमा एतावदिति निश्चिताः  ॥ १६.११ ॥  मूल श्लोक
\end{quotation}

\begin{quotation}

चिन्ताम अपरिमेयां च, प्रलया अन्ताम उपाश्रिताः ।  

काम-उपभोग परमा, एतावद इति निश्चिताः  ॥ १६.११ ॥  उच्चारण

\noindent\rule{16cm}{0.4pt} 
\end{quotation}


\begin{quotation}
आशापाशशतैर्बद्धाः कामक्रोधपरायणाः ।  

ईहन्ते कामभोगार्थमन्यायेनार्थसञ्चयान्‌  ॥ १६.१२ ॥  मूल श्लोक
\end{quotation}

\begin{quotation}

आशा-पाश-शतैर् बद्धाः, काम क्रोध परायणाः ।  

ईहन्ते काम भोगार्थम, अन्यायेना-अर्थ सञ्चयान्‌  ॥ १६.१२ ॥  उच्चारण

\noindent\rule{16cm}{0.4pt} 
\end{quotation}


\begin{quotation}

इदमद्य मया लब्धमिमं प्राप्स्ये मनोरथम्‌ ।  

इदमस्तीदमपि मे भविष्यति पुनर्धनम्‌  ॥ १६.१३ ॥  मूल श्लोक
\end{quotation}

\begin{quotation}

इद मद्य मया लब्धम, इमं प्राप्स्ये मनोरथम्‌ ।  

इदम अस्तीदम अपि मे, भविष्यति पुनर्-धनम्‌  ॥ १६.१३ ॥  उच्चारण

\noindent\rule{16cm}{0.4pt} 
\end{quotation}


\begin{quotation}

असौ मया हतः शत्रुर्हनिष्ये चापरानपि ।  

ईश्वरोऽहमहं भोगी सिद्धोऽहं बलवान्सुखी  ॥ १६.१४ ॥  मूल श्लोक
\end{quotation}

\begin{quotation}

असौ मया हतः शत्रुर् हनिष्ये चापरान अपि ।  

ईश्वरोऽ अहम अहं भोगी, सिद्धोऽ अहं बलवान् सुखी  ॥ १६.१४ ॥  उच्चारण

\noindent\rule{16cm}{0.4pt} 
\end{quotation}


\begin{quotation}

आढयोऽभिजनवानस्मि कोऽन्योऽस्ति सदृशो मया ।  

यक्ष्ये दास्यामि मोदिष्य इत्यज्ञानविमोहिताः  ॥ १६.१५ ॥  मूल श्लोक
\end{quotation}

\begin{quotation}

आढयोऽ अभि-जनवान अस्मि, कोऽ अन्योऽ अस्ति सदृशो मया ।  

यक्ष्ये दास्यामि मोदिष्य, इत्य अज्ञान विमोहिताः  ॥ १६.१५ ॥  उच्चारण

\noindent\rule{16cm}{0.4pt} 
\end{quotation}


\begin{quotation}

अनेकचित्तविभ्रान्ता मोहजालसमावृताः ।  

प्रसक्ताः कामभोगेषु पतन्ति नरकेऽशुचौ  ॥ १६.१६ ॥  मूल श्लोक
\end{quotation}

\begin{quotation}

अनेक चित्त वि-भ्रान्ता, मोहजाल समावृताः ।  

प्रसक्ताः काम भोगेषु, पतन्ति नरकेऽ अशुचौ  ॥ १६.१६ ॥  उच्चारण

\noindent\rule{16cm}{0.4pt} 
\end{quotation}


\begin{quotation}

आत्मसम्भाविताः स्तब्धा धनमानमदान्विताः ।  

यजन्ते नामयज्ञैस्ते दम्भेनाविधिपूर्वकम्‌  ॥ १६.१७ ॥  मूल श्लोक
\end{quotation}

\begin{quotation}

आत्म सम्भाविताः स्तब्धा, धन मान मदान्विताः ।  

यजन्ते नाम यज्ञैस्ते, दम्भेना विधि पूर्वकम्‌  ॥ १६.१७ ॥  उच्चारण

\noindent\rule{16cm}{0.4pt} 
\end{quotation}


\begin{quotation}
अहङ्‍कारं बलं दर्पं कामं क्रोधं च संश्रिताः ।  

मामात्मपरदेहेषु प्रद्विषन्तोऽभ्यसूयकाः  ॥ १६.१८ ॥  मूल श्लोक
\end{quotation}

\begin{quotation}

अहङ्‍कारं बलं दर्पं, कामं क्रोधं च संश्रिताः ।  

मामात्म पर-देहेषु, प्रद्-विषन्तोऽ अभ्य-सूयकाः  ॥ १६.१८ ॥  उच्चारण

\noindent\rule{16cm}{0.4pt} 
\end{quotation}


\begin{quotation}

तानहं द्विषतः क्रूरान्संसारेषु नराधमान्‌ ।  

क्षिपाम्यजस्रमशुभानासुरीष्वेव योनिषु  ॥ १६.१९ ॥  मूल श्लोक
\end{quotation}

\begin{quotation}

तान अहं द्विषतः क्रूरान्, संसारेषु नराधमान्‌ ।  

क्षिपाम्य अज-स्रम अशुभान, आसुरीष्व एव योनिषु  ॥ १६.१९ ॥  उच्चारण

\noindent\rule{16cm}{0.4pt} 
\end{quotation}


\begin{quotation}

आसुरीं योनिमापन्ना मूढा जन्मनि जन्मनि ।  

मामप्राप्यैव कौन्तेय ततो यान्त्यधमां गतिम्‌  ॥ १६.२० ॥  मूल श्लोक
\end{quotation}

\begin{quotation}

आसुरीं योनिम आपन्ना, मूढा जन्मनि जन्मनि ।  

माम अप्राप्य एैव कौन्तेय, ततो यान्त्य अधमां गतिम्‌  ॥ १६.२० ॥  उच्चारण

\noindent\rule{16cm}{0.4pt} 
\end{quotation}


\begin{quotation}

त्रिविधं नरकस्येदं द्वारं नाशनमात्मनः ।  

कामः क्रोधस्तथा लोभस्तस्मादेतत्त्रयं त्यजेत्‌  ॥ १६.२१ ॥  मूल श्लोक
\end{quotation}

\begin{quotation}

त्रिविधं नरकस्य एदं, द्वारं नाशनम् आत्मनः ।  

कामः क्रोधस् तथा लोभस्, तस्माद एतत् त्रयं त्यजेत्‌  ॥ १६.२१ ॥  उच्चारण

\noindent\rule{16cm}{0.4pt} 
\end{quotation}


\begin{quotation}

एतैर्विमुक्तः कौन्तेय तमोद्वारैस्त्रिभिर्नरः ।  

आचरत्यात्मनः श्रेयस्ततो याति परां गतिम्‌  ॥ १६.२२ ॥  मूल श्लोक
\end{quotation}

\begin{quotation}

एतैर् विमुक्तः कौन्तेय, तमो द्वारैस् त्रिभिर् नरः ।  

आचरत्य आत्मनः श्रेयस्, ततो याति परां गतिम्‌  ॥ १६.२२ ॥  उच्चारण

\noindent\rule{16cm}{0.4pt} 
\end{quotation}


\begin{quotation}

यः शास्त्रविधिमुत्सृज्य वर्तते कामकारतः ।  

न स सिद्धिमवाप्नोति न सुखं न परां गतिम्‌  ॥ १६.२३ ॥  मूल श्लोक
\end{quotation}

\begin{quotation}

यः शास्त्र विधिम उत्सृज्य, वर्तते काम कारतः ।  

न स सिद्धिम अवाप्-नोति, न सुखं न परां गतिम्‌  ॥ १६.२३ ॥  उच्चारण

\noindent\rule{16cm}{0.4pt} 
\end{quotation}


\begin{quotation}
तस्माच्छास्त्रं प्रमाणं ते कार्याकार्यव्यवस्थितौ ।  

ज्ञात्वा शास्त्रविधानोक्तं कर्म कर्तुमिहार्हसि  ॥ १६.२४ ॥  मूल श्लोक
\end{quotation}

\begin{quotation}

तस्माच् छास्त्रं प्रमाणं ते, कार्या-कार्य-व्यवस्थितौ ।  

ज्ञात्वा शास्त्र विधानोक्तं, कर्म कर्तुम इहा-अर्हसि  ॥ १६.२४ ॥  उच्चारण

\noindent\rule{16cm}{0.4pt} 
\end{quotation}

\begin{center} ***** \end{center}

\begin{quotation}


ॐ तत् सद इति श्री मद्-भगवद्-गीतास उपनिषत्सु ब्रह्म विद्यायां योगशास्त्रे श्री कृष्णार्जुन संवादे  दैवासुरसम्पद् विभागयोगो नाम षोडशोऽ अध्यायः  ॥  १६  ॥ 
\end{quotation} 