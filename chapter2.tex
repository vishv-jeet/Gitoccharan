\chapter{\sanskrit {साङ्ख्ययोग}}

\sanskrit
\paragraph{\sanskrit सञ्जय उवाच}

\begin{quotation}
तं तथा कृपयाविष्टमश्रुपूर्णाकुलेक्षणम्‌  ।  

विषीदन्तमिदं वाक्यमुवाच मधुसूदनः  ॥ २.१ ॥  मूल श्लोक
\end{quotation}

\begin{quotation}

तं तथा कृपया विष्टम, अश्रुपूर्णा कुलेक्षणम्‌  ।  

विषीदन्तम इदं वाक्यं, उवाच मधुसूदनः  ॥ २.१ ॥  उच्चारण

\noindent\rule{16cm}{0.4pt} 
\end{quotation}

\paragraph{\sanskrit श्रीभगवानुवाच}

\begin{quotation}

कुतस्त्वा कश्मलमिदं विषमे समुपस्थितम्‌  ।  

अनार्यजुष्टमस्वर्ग्यमकीर्तिकरमर्जुन  ॥ २.२ ॥  मूल श्लोक
\end{quotation}

\begin{quotation}

कुतस् त्वा कश्मल इदं, विषमे सम-उपस्थितम्‌  ।  

अनार्य जुष्टम अस्वर्ग्यम, अकीर्ति करं अर्जुन  ॥ २.२ ॥  उच्चारण

\noindent\rule{16cm}{0.4pt} 
\end{quotation}


\begin{quotation}

क्लैब्यं मा स्म गमः पार्थ नैतत्त्वय्युपपद्यते  ।  

क्षुद्रं हृदयदौर्बल्यं त्यक्त्वोत्तिष्ठ परन्तप  ॥ २.३ ॥  मूल श्लोक
\end{quotation}

\begin{quotation}

क्लैब्यं मा स्म गमः पार्थ, नैतत तवय्य उपपद्यते  ।  

क्षुद्रं हृदय दौर्बल्यं, त्यक्त्वो-त्तिष्ठ परन्तप  ॥ २.३ ॥  उच्चारण

\noindent\rule{16cm}{0.4pt} 
\end{quotation}

\paragraph{\sanskrit अर्जुन उवाच}

\begin{quotation}

कथं भीष्ममहं सङ्‍ख्ये द्रोणं च मधुसूदन  ।  

इषुभिः प्रतियोत्स्यामि पूजार्हावरिसूदन  ॥ २.४ ॥  मूल श्लोक
\end{quotation}

\begin{quotation}

कथं भीष्मम अहं सङ्‍ख्ये, द्रोणं च मधुसूदन  ।  

इषुभिः प्रति-योत्स्यामि, पूजार्हाव अरि सूदन  ॥ २.४ ॥  उच्चारण

\noindent\rule{16cm}{0.4pt} 
\end{quotation}


\begin{quotation}
गुरूनहत्वा हि महानुभावाञ्छ्रेयो भोक्तुं भैक्ष्यमपीह लोके  ।  

हत्वार्थकामांस्तु गुरूनिहैवभुंजीय भोगान्‌ रुधिरप्रदिग्धान्‌  ॥ २.५ ॥  मूल श्लोक
\end{quotation}

\begin{quotation}

गुरून् अहत्वा हि महानुभावान्‌, 
श्रेयो भोक्तुं भैक्ष्यम् अपीह लोके  ।  

हत्वार्थ कामांस् तु गुरून् इहैव, 
भुंजीय भोगान्‌ रुधिर प्रदिग्धान्‌  ॥ २.५ ॥   उच्चारण

\noindent\rule{16cm}{0.4pt} 
\end{quotation}


\begin{quotation}

न चैतद्विद्मः कतरन्नो गरीयो
यद्वा जयेम यदि वा नो जयेयुः  ।  

यानेव हत्वा न जिजीविषाम
स्तेऽवस्थिताः प्रमुखे धार्तराष्ट्राः  ॥ २.६ ॥  मूल श्लोक
\end{quotation}

\begin{quotation}

न च ऐतद विद्मः कतरन् नो गरीयो, 
यद् वा जयेम यदि वा नो जयेयुः  ।  

यान एव हत्वा न जिजीविषाम, 
स्तेऽ अवस्थिताः प्रमुखे धार्त-राष्ट्राः  ॥ २.६ ॥  उच्चारण

\noindent\rule{16cm}{0.4pt} 
\end{quotation}


\begin{quotation}

कार्पण्यदोषोपहतस्वभावः
पृच्छामि त्वां धर्मसम्मूढचेताः  ।  

यच्छ्रेयः स्यान्निश्चितं ब्रूहि तन्मे
शिष्यस्तेऽहं शाधि मां त्वां प्रपन्नम्‌  ॥ २.७ ॥  मूल श्लोक
\end{quotation}

\begin{quotation}

कार्पण्य दोषोपहत स्वभावः, 
पृच्छामि त्वां धर्म सम्मूढ चेताः  ।  

यच छ्रेयः स्यां निश्चितं ब्रूहि तन्मे, 
शिष्यस् तेऽहं शाधि मां त्वां प्रपन्नम्‌  ॥ २.७ ॥  उच्चारण

\noindent\rule{16cm}{0.4pt} 
\end{quotation}


\begin{quotation}

न हि प्रपश्यामि ममापनुद्या-
द्यच्छोकमुच्छोषणमिन्द्रियाणाम्‌  ।  

अवाप्य भूमावसपत्रमृद्धं-
राज्यं सुराणामपि चाधिपत्यम्‌  ॥ २.८ ॥  मूल श्लोक
\end{quotation}

\begin{quotation}



न हि प्रपश्यामि मम अपनुदयाद, 
यच छोकं उच्छोषणं इन्द्रियाणाम्‌  ।  

अवाप्य भूमाव असपतनं रद्धं, 
राज्यं सुराणां अपि च आधिपत्यम्‌  ॥ २.८ ॥  उच्चारण

\noindent\rule{16cm}{0.4pt} 
\end{quotation}

\paragraph{\sanskrit सञ्जय उवाच}

\begin{quotation}

एवमुक्त्वा हृषीकेशं गुडाकेशः परन्तप  ।  

न योत्स्य इतिगोविन्दमुक्त्वा तूष्णीं बभूव ह  ॥ २.९ ॥  मूल श्लोक
\end{quotation}

\begin{quotation}

एवं उक्त्वा हृषीकेशं, गुडाकेशः परन्तप  ।  

न योत्स्य इति गोविन्दं, उक्त्वा तूष्णीं बभूव ह  ॥ २.९ ॥  उच्चारण

\noindent\rule{16cm}{0.4pt} 
\end{quotation}


\begin{quotation}

तमुवाच हृषीकेशः प्रहसन्निव भारत  ।  

सेनयोरुभयोर्मध्ये विषीदंतमिदं वचः  ॥ २.१० ॥  मूल श्लोक
\end{quotation}

\begin{quotation}

तमुवाच हृषीकेशः, प्रहसन्न इव भारत  ।  

सेनयोर उभयोर मध्ये, विषीदंतम् इदम वचः  ॥ २.१० ॥  उच्चारण

\noindent\rule{16cm}{0.4pt} 
\end{quotation}


\paragraph{\sanskrit श्रीभगवानुवाच}

\begin{quotation}
अशोच्यानन्वशोचस्त्वं प्रज्ञावादांश्च भाषसे  ।  

गतासूनगतासूंश्च नानुशोचन्ति पण्डिताः  ॥ २.११ ॥  मूल श्लोक
\end{quotation}

\begin{quotation}

अशोच्यान अन्व-शोचस त्वं, प्रज्ञा वादांश् च भाषसे  ।  

गता सून गतासूंश् च, नानु शोचन्ति पण्डिताः  ॥ २.११ ॥  उच्चारण

\noindent\rule{16cm}{0.4pt} 
\end{quotation}


\begin{quotation}

न त्वेवाहं जातु नासं न त्वं नेमे जनाधिपाः  ।  

न चैव न भविष्यामः सर्वे वयमतः परम्‌  ॥ २.१२ ॥  मूल श्लोक
\end{quotation}

\begin{quotation}

न त्व एवाहं जातु नासं, न त्वं नेमे जनाधिपाः  ।  

न चैव न भविष्यामः, सर्वे वयमतः परम्‌  ॥ २.१२ ॥  उच्चारण

\noindent\rule{16cm}{0.4pt} 
\end{quotation}


\begin{quotation}
देहिनोऽस्मिन्यथा देहे कौमारं यौवनं जरा  ।  

तथा देहान्तरप्राप्तिर्धीरस्तत्र न मुह्यति  ॥ २.१३ ॥  मूल श्लोक
\end{quotation}

\begin{quotation}

देहिनोऽ अस्मिन् यथा देहे, कौमारं यौवनं जरा  ।  

तथा देहान्तर प्राप्तिर, धीरस तत्र न मुह्यति  ॥ २.१३ ॥  उच्चारण

\noindent\rule{16cm}{0.4pt} 
\end{quotation}


\begin{quotation}

मात्रास्पर्शास्तु कौन्तेय शीतोष्णसुखदुःखदाः  ।  

आगमापायिनोऽनित्यास्तांस्तितिक्षस्व भारत  ॥ २.१४ ॥  मूल श्लोक
\end{quotation}

\begin{quotation}

मात्रा स्पर्शास् तु कौन्तेय, शीतोष्ण सुख दुःख दाः  ।  

आगमा पायिनोऽ अनित्यास, ताम्स तितिक्षस्व भारत  ॥ २.१४ ॥  उच्चारण

\noindent\rule{16cm}{0.4pt} 
\end{quotation}


\begin{quotation}

यं हि न व्यथयन्त्येते पुरुषं पुरुषर्षभ  ।  

समदुःखसुखं धीरं सोऽमृतत्वाय कल्पते  ॥ २.१५ ॥  मूल श्लोक
\end{quotation}

\begin{quotation}

यं हि न व्यथ-यंतय इते, पुरुषं पुरुषर्षभ  ।  

सम दुःख सुखं धीरं, सोऽ अमृत-त्वाय कल्पते  ॥ २.१५ ॥  उच्चारण

\noindent\rule{16cm}{0.4pt} 
\end{quotation}


\begin{quotation}

नासतो विद्यते भावो नाभावो विद्यते सतः  ।  

उभयोरपि दृष्टोऽन्तस्त्वनयोस्तत्वदर्शिभिः  ॥ २.१६ ॥  मूल श्लोक
\end{quotation}

\begin{quotation}

नासतो विद्यते भावो, नाभावो विद्यते सतः  ।  

उभयोर अपि दृष्टोऽ अन्तस:, त्व अनयोस तत्व दर्शिभिः  ॥ २.१६ ॥  उच्चारण

\noindent\rule{16cm}{0.4pt} 
\end{quotation}


\begin{quotation}

अविनाशि तु तद्विद्धि येन सर्वमिदं ततम्‌  ।  

विनाशमव्ययस्यास्य न कश्चित्कर्तुमर्हति  ॥ २.१७ ॥  मूल श्लोक
\end{quotation}

\begin{quotation}

अविनाशि तु तद विद्धि, येन सर्वम् इदं ततम्‌  ।  

विनाशम अ-व्ययस्य-अस्य, न कश्चित कर्तुम अर्हति  ॥ २.१७ ॥  उच्चारण

\noindent\rule{16cm}{0.4pt} 
\end{quotation}


\begin{quotation}

अन्तवन्त इमे देहा नित्यस्योक्ताः शरीरिणः  ।  

अनाशिनोऽप्रमेयस्य तस्माद्युध्यस्व भारत  ॥ २.१८ ॥  मूल श्लोक
\end{quotation}

\begin{quotation}

अन्त वन्त इमे देहा, नित्यस्य उक्ताः शरीरिणः  ।  

अनाशिनोऽ अ-प्रमेयस्य, तस्माद युध्यस्व भारत  ॥ २.१८ ॥  उच्चारण

\noindent\rule{16cm}{0.4pt} 
\end{quotation}


\begin{quotation}
य एनं वेत्ति हन्तारं यश्चैनं मन्यते हतम्‌  ।  

उभौ तौ न विजानीतो नायं हन्ति न हन्यते  ॥ २.१९ ॥  मूल श्लोक
\end{quotation}

\begin{quotation}

य एनं वेत्ति हन्तारं, यश्चैनं मन्यते हतम्‌  ।  

उभौ तौ न विजानीतो, नायं हन्ति न हन्यते  ॥ २.१९ ॥  उच्चारण

\noindent\rule{16cm}{0.4pt} 
\end{quotation}


\begin{quotation}

न जायते म्रियते वा कदाचिन्नायं भूत्वा भविता वा न भूयः  ।  

अजो नित्यः शाश्वतोऽयं पुराणो, न हन्यते हन्यमाने शरीरे  ॥ २.२० ॥  मूल श्लोक
\end{quotation}

\begin{quotation}

न जायते म्रियते वा कदाचिन्, 
नायं भूत्वा भविता वा न भूयः  ।  

अजो नित्यः शाश्वतोऽ अयं पुराणो, 
न हन्यते हन्यमाने शरीरे  ॥ २.२० ॥  उच्चारण

\noindent\rule{16cm}{0.4pt} 
\end{quotation}


\begin{quotation}

वेदाविनाशिनं नित्यं य एनमजमव्ययम्‌  ।  

कथं स पुरुषः पार्थ कं घातयति हन्ति कम्‌  ॥ २.२१ ॥  मूल श्लोक
\end{quotation}

\begin{quotation}

वेदा विनाशिनं नित्यं, य एनं अजं अव्ययम्‌  ।  

कथं स पुरुषः पार्थ, कं घातयति हन्ति कम्‌  ॥ २.२१ ॥  उच्चारण

\noindent\rule{16cm}{0.4pt} 
\end{quotation}


\begin{quotation}

वासांसि जीर्णानि यथा विहायनवानि गृह्णाति नरोऽपराणि  ।  

तथा शरीराणि विहाय जीर्णान्यन्यानि संयाति नवानि देही  ॥ २.२२ ॥  मूल श्लोक
\end{quotation}

\begin{quotation}

वासांसि जीर्णानि यथा विहाय, 
नवानि गृहणाति नरोऽ अपराणि  ।  

तथा शरीराणि विहाय जीर्णान्य, 
अन्यानि संयाति नवानि देही  ॥ २.२२ ॥  उच्चारण

\noindent\rule{16cm}{0.4pt} 
\end{quotation}


\begin{quotation}

नैनं छिन्दन्ति शस्त्राणि नैनं दहति पावकः  ।  

न चैनं क्लेदयन्त्यापो न शोषयति मारुतः  ॥ २.२३ ॥  मूल श्लोक
\end{quotation}

\begin{quotation}

नैनं छिन्दन्ति शस्त्राणि, नैनं दहति पावकः  ।  

न चैनं क्लेदयन्त्-यापो, न शोषयति मारुतः  ॥ २.२३ ॥  उच्चारण

\noindent\rule{16cm}{0.4pt} 
\end{quotation}


\begin{quotation}

अच्छेद्योऽयमदाह्योऽयमक्लेद्योऽशोष्य एव च  ।  

नित्यः सर्वगतः स्थाणुरचलोऽयं सनातनः  ॥ २.२४ ॥  मूल श्लोक
\end{quotation}

\begin{quotation}

अच्छेद्योऽ अयम अदाह्योऽ अयम, अक्लेद्योऽ अशोष्य एव च  ।  

नित्यः सर्व-गतः स्थाणुर, अचलोऽ अयं सनातनः  ॥ २.२४ ॥  उच्चारण

\noindent\rule{16cm}{0.4pt} 
\end{quotation}


\begin{quotation}

अव्यक्तोऽयमचिन्त्योऽयमविकार्योऽयमुच्यते  ।  

तस्मादेवं विदित्वैनं नानुशोचितुमर्हसि  ॥ २.२५ ॥  मूल श्लोक
\end{quotation}

\begin{quotation}

अव्यक्तोऽ अयम अचिन्त्योऽ अयम, अविकार्योऽ अयम उच्यते  ।  

तस्माद एवं विदित्व-एैनं, नानु शोचितुम अर्हसि  ॥ २.२५ ॥  उच्चारण

\noindent\rule{16cm}{0.4pt} 
\end{quotation}


\begin{quotation}

अथ चैनं नित्यजातं नित्यं वा मन्यसे मृतम्‌  ।  

तथापि त्वं महाबाहो नैवं शोचितुमर्हसि  ॥ २.२६ ॥  मूल श्लोक
\end{quotation}

\begin{quotation}

अथ चैनं नित्य-जातं, नित्यं वा मन्यसे मृतम्‌  ।  

तथापि त्वं महाबाहो, नैवं शोचितुम अर्हसि  ॥ २.२६ ॥  उच्चारण

\noindent\rule{16cm}{0.4pt} 
\end{quotation}


\begin{quotation}

जातस्त हि ध्रुवो मृत्युर्ध्रुवं जन्म मृतस्य च  ।  

तस्मादपरिहार्येऽर्थे न त्वं शोचितुमर्हसि  ॥ २.२७ ॥  मूल श्लोक
\end{quotation}

\begin{quotation}

जातस्त हि ध्रुवो मृत्युर, ध्रुवं जन्म मृतस्य च  ।  

तस्माद अपरिहार्येऽ अर्थे, न त्वं शोचितुम अर्हसि  ॥ २.२७ ॥  उच्चारण

\noindent\rule{16cm}{0.4pt} 
\end{quotation}


\begin{quotation}

अव्यक्तादीनि भूतानि व्यक्तमध्यानि भारत  ।  

अव्यक्तनिधनान्येव तत्र का परिदेवना  ॥ २.२८ ॥  मूल श्लोक
\end{quotation}

\begin{quotation}

अव्यक्ता-दीनि भूतानि, व्यक्त-मध्यानि भारत  ।  

अव्यक्त-निधनान्य एव, तत्र का परिदेवना  ॥ २.२८ ॥  उच्चारण

\noindent\rule{16cm}{0.4pt} 
\end{quotation}


\begin{quotation}

आश्चर्यवत्पश्यति कश्चिदेनमाश्चर्यवद्वदति तथैव चान्यः  ।  

आश्चर्यवच्चैनमन्यः श्रृणोति श्रुत्वाप्येनं वेद न चैव कश्चित्‌  ॥ २.२९ ॥  मूल श्लोक
\end{quotation}

\begin{quotation}
आश्चर्य-वत् पश्यति कश्चिद एनम्, 
आश्चर्य-वद् वदति तथैव चान्यः  ।  

आश्चर्य-वच् चैनम अन्यः श्रृणोति, 
श्रुत्वा अप्येनं वेद न चैव कश्चित्‌  ॥ २.२९ ॥  उच्चारण

\noindent\rule{16cm}{0.4pt} 
\end{quotation}


\begin{quotation}

देही नित्यमवध्योऽयं देहे सर्वस्य भारत  ।  

तस्मात्सर्वाणि भूतानि न त्वं शोचितुमर्हसि  ॥ २.३० ॥  मूल श्लोक
\end{quotation}

\begin{quotation}

देही नित्यम अवध्योऽ अयं, देहे सर्वस्य भारत  ।  

तस्मात सर्वाणि भूतानि, न त्वं शोचितुम अर्हसि  ॥ २.३० ॥  उच्चारण

\noindent\rule{16cm}{0.4pt} 
\end{quotation}


\begin{quotation}

स्वधर्ममपि चावेक्ष्य न विकम्पितुमर्हसि  ।  

धर्म्याद्धि युद्धाच्छ्रेयोऽन्यत्क्षत्रियस्य न विद्यते  ॥ २.३१ ॥  मूल श्लोक
\end{quotation}

\begin{quotation}

स्व-धर्मम् अपि चावेक्ष्य, न विकम्पि-तुम अर्हसि  ।  

धर्म्याद् धी युद्धाच् छ्रेयोऽ अन्यते, क्षत्रियस्य न विद्यते  ॥ २.३१ ॥  उच्चारण

\noindent\rule{16cm}{0.4pt} 
\end{quotation}


\begin{quotation}

यदृच्छया चोपपन्नां स्वर्गद्वारमपावृतम्‌  ।  

सुखिनः क्षत्रियाः पार्थ लभन्ते युद्धमीदृशम्‌  ॥ २.३२ ॥  मूल श्लोक
\end{quotation}

\begin{quotation}

यदृच्छया च उपपन्नाम्, स्वर्ग-द्वारम पावृतम्‌  ।  

सुखिनः क्षत्रियाः पार्थ, लभन्ते युद्ध मी-दृशम्‌  ॥ २.३२ ॥  उच्चारण

\noindent\rule{16cm}{0.4pt} 
\end{quotation}


\begin{quotation}

अथ चेत्त्वमिमं धर्म्यं सङ्‍ग्रामं न करिष्यसि  ।  

ततः स्वधर्मं कीर्तिं च हित्वा पापमवाप्स्यसि  ॥ २.३३ ॥  मूल श्लोक
\end{quotation}

\begin{quotation}

अथ चेत् त्वम इमं धर्म्यं, सङ्‍ग्रामं न करिष्यसि  ।  

ततः स्वधर्मं कीर्तिं च, हित्वा पापम अवाप्स्यसि  ॥ २.३३ ॥  उच्चारण

\noindent\rule{16cm}{0.4pt} 
\end{quotation}


\begin{quotation}

अकीर्तिं चापि भूतानि कथयिष्यन्ति तेऽव्ययाम्‌  ।  

सम्भावितस्य चाकीर्तिर्मरणादतिरिच्यते  ॥ २.३४ ॥  मूल श्लोक
\end{quotation}

\begin{quotation}

अकीर्तिं चापि भूतानि, कथय-इष्यन्ति तेऽ-अव्ययाम्‌  ।  

सम्भावित-अस्य चाकीर्तिर्, मरणाद अतिरिच्यते  ॥ २.३४ ॥  उच्चारण

\noindent\rule{16cm}{0.4pt} 
\end{quotation}


\begin{quotation}

भयाद्रणादुपरतं मंस्यन्ते त्वां महारथाः  ।  

येषां च त्वं बहुमतो भूत्वा यास्यसि लाघवम्‌  ॥ २.३५ ॥  मूल श्लोक
\end{quotation}

\begin{quotation}

भयाद् रणाद उपरतं, मम्-स्यन्ते त्वां महारथाः  ।  

येषां च त्वं बहुमतो, भूत्वा यास्यसि लाघवम्‌  ॥ २.३५ ॥  उच्चारण

\noindent\rule{16cm}{0.4pt} 
\end{quotation}


\begin{quotation}

अवाच्यवादांश्च बहून्‌ वदिष्यन्ति तवाहिताः  ।  

निन्दन्तस्तव सामर्थ्यं ततो दुःखतरं नु किम्‌  ॥ २.३६ ॥  मूल श्लोक
\end{quotation}

\begin{quotation}

अवाच्य वादांश् च बहून्,‌ वदिष्यन्ति तवाहिताः  ।  

निन्दन्तस तव सामर्थ्यं, ततो दुःखतरं नु किम्‌  ॥ २.३६ ॥  उच्चारण

\noindent\rule{16cm}{0.4pt} 
\end{quotation}


\begin{quotation}

हतो वा प्राप्स्यसि स्वर्गं जित्वा वा भोक्ष्यसे महीम्‌  ।  

तस्मादुत्तिष्ठ कौन्तेय युद्धाय कृतनिश्चयः  ॥ २.३७ ॥  मूल श्लोक
\end{quotation}

\begin{quotation}

हतो वा प्राप्स्यसि स्वर्गं, जित्वा वा भोक्ष्यसे महीम्‌  ।  

तस्माद उत्तिष्ठ कौन्तेय, युद्धाय कृत निश्चयः  ॥ २.३७ ॥  उच्चारण

\noindent\rule{16cm}{0.4pt} 
\end{quotation}


\begin{quotation}

सुखदुःखे समे कृत्वा लाभालाभौ जयाजयौ  ।  

ततो युद्धाय युज्यस्व नैवं पापमवाप्स्यसि  ॥ २.३८ ॥  मूल श्लोक
\end{quotation}

\begin{quotation}

सुख दुःखे समे कृत्वा, लाभा लाभौ जयाजयौ  ।  

ततो युद्धाय युज्यस्व, नैवं पापम अवाप्स्यसि  ॥ २.३८ ॥  उच्चारण

\noindent\rule{16cm}{0.4pt} 
\end{quotation}


\begin{quotation}

एषा तेऽभिहिता साङ्‍ख्ये बुद्धिर्योगे त्विमां श्रृणु  ।  

बुद्ध्‌या युक्तो यया पार्थ कर्मबन्धं प्रहास्यसि  ॥ २.३९ ॥  मूल श्लोक
\end{quotation}

\begin{quotation}

एषा तेऽ-अभिहिता साङ्‍ख्ये, बुद्धिर् योगे त्व इमाम् श्रृणु  ।  

बुद्ध्‌या युक्तो यया पार्थ, कर्म-बन्धं प्रहास्यसि  ॥ २.३९ ॥  उच्चारण

\noindent\rule{16cm}{0.4pt} 
\end{quotation}


\begin{quotation}
नेहाभिक्रमनाशोऽस्ति प्रत्यवातो न विद्यते  ।  

स्वल्पमप्यस्य धर्मस्य त्रायते महतो भयात्‌  ॥ २.४० ॥  मूल श्लोक
\end{quotation}

\begin{quotation}

न एह अभिक्रम नाशोऽ अस्ति, प्रत्यवातो न विद्यते  ।  

स्वल्पम अप्यस्य धर्मस्य, त्रायते महतो भयात्‌  ॥ २.४० ॥  उच्चारण

\noindent\rule{16cm}{0.4pt} 
\end{quotation}


\begin{quotation}

व्यवसायात्मिका बुद्धिरेकेह कुरुनन्दन  ।  

बहुशाका ह्यनन्ताश्च बुद्धयोऽव्यवसायिनाम्‌  ॥ २.४१ ॥  मूल श्लोक
\end{quotation}

\begin{quotation}

व्यवसाय-आत्मिका बुद्धिर, एकेह कुरुनन्दन  ।  

बहुशाका ह्य अनन्ताश् च, बुद्धयोऽ अ-व्यव-सायिनाम्‌  ॥ २.४१ ॥  उच्चारण

\noindent\rule{16cm}{0.4pt} 
\end{quotation}


\begin{quotation}

यामिमां पुष्पितां वाचं प्रवदन्त्यविपश्चितः  ।  

वेदवादरताः पार्थ नान्यदस्तीति वादिनः  ॥ २.४२ ॥  मूल श्लोक
\end{quotation}

\begin{quotation}

यामिमां पुष्पितां वाचं, प्रवदन्त्य विपश्चितः  ।  

वेद वादरताः पार्थ, नान्य दस्तीति वादिनः  ॥ २.४२ ॥  उच्चारण

\noindent\rule{16cm}{0.4pt} 
\end{quotation}


\begin{quotation}

कामात्मानः स्वर्गपरा जन्मकर्मफलप्रदाम्‌  ।  

क्रियाविश्लेषबहुलां भोगैश्वर्यगतिं प्रति  ॥ २.४३ ॥  मूल श्लोक
\end{quotation}

\begin{quotation}

काम-आत्मानः स्वर्गपरा, जन्म कर्मफल प्रदाम्‌  ।  

क्रिया विशेष बहुलां, भोग एैश्वर्य गतिं प्रति  ॥ २.४३ ॥  उच्चारण

\noindent\rule{16cm}{0.4pt} 
\end{quotation}


\begin{quotation}

भोगैश्वर्यप्रसक्तानां तयापहृतचेतसाम्‌  ।  

व्यवसायात्मिका बुद्धिः समाधौ न विधीयते  ॥ २.४४ ॥  मूल श्लोक
\end{quotation}

\begin{quotation}

भोग एैश्वर्य प्रसक्तानां, तया अपहृत चेतसाम्‌  ।  

व्यवसाय-आत्मिका बुद्धिः, समाधौ न विधीयते  ॥ २.४४ ॥  उच्चारण

\noindent\rule{16cm}{0.4pt} 
\end{quotation}


\begin{quotation}

त्रैगुण्यविषया वेदा निस्त्रैगुण्यो भवार्जुन  ।  

निर्द्वन्द्वो नित्यसत्वस्थो निर्योगक्षेम आत्मवान्‌  ॥ २.४५ ॥  मूल श्लोक
\end{quotation}

\begin{quotation}
त्रैगुण्य विषया वेदा, निस्त्रै-गुण्यो भवार्जुन  ।  

निर्द्वन्द्वो नित्य-सत्व-स्थो, निर्योग-क्षेम आत्मवान्‌  ॥ २.४५ ॥  उच्चारण

\noindent\rule{16cm}{0.4pt} 
\end{quotation}


\begin{quotation}

यावानर्थ उदपाने सर्वतः सम्प्लुतोदके  ।  

तावान्सर्वेषु वेदेषु ब्राह्मणस्य विजानतः  ॥ २.४६ ॥  मूल श्लोक
\end{quotation}

\begin{quotation}

यावान अर्थ उदपाने, सर्वतः सम्प्लुत-उदके  ।  

तावान्-सर्वेषु वेदेषु, ब्राह्मणस्य विजानतः  ॥ २.४६ ॥  उच्चारण

\noindent\rule{16cm}{0.4pt} 
\end{quotation}


\begin{quotation}

कर्मण्येवाधिकारस्ते मा फलेषु कदाचन  ।  

मा कर्मफलहेतुर्भुर्मा ते संगोऽस्त्वकर्मणि  ॥ २.४७ ॥  मूल श्लोक
\end{quotation}

\begin{quotation}

कर्मण्य एवा-अधिकारस् ते , मा फलेषु कदाचन  ।  

मा कर्म फल-हेतुर भुर्, मा ते संगोऽ अस्त्व कर्मणि  ॥ २.४७ ॥  उच्चारण

\noindent\rule{16cm}{0.4pt} 
\end{quotation}


\begin{quotation}

योगस्थः कुरु कर्माणि संग त्यक्त्वा धनंजय  ।  

सिद्धयसिद्धयोः समो भूत्वा समत्वं योग उच्यते  ॥ २.४८ ॥  मूल श्लोक
\end{quotation}

\begin{quotation}

योगस्थः कुरु कर्माणि, संग त्यक्त्वा धनंजय  ।  

सिद्धय-असिद्धयोः समो भूत्वा, समत्वं योग उच्यते  ॥ २.४८ ॥  उच्चारण

\noindent\rule{16cm}{0.4pt} 
\end{quotation}


\begin{quotation}

दूरेण ह्यवरं कर्म बुद्धियोगाद्धनंजय  ।  

बुद्धौ शरणमन्विच्छ कृपणाः फलहेतवः  ॥ २.४९ ॥  मूल श्लोक
\end{quotation}

\begin{quotation}

दूरेण ह्य अवरं कर्म, बुद्धि-योगाद धनंजय  ।  

बुद्धौ शरणम अन्विच्छ, कृपणाः फल हेतवः  ॥ २.४९ ॥  उच्चारण

\noindent\rule{16cm}{0.4pt} 
\end{quotation}


\begin{quotation}

बुद्धियुक्तो जहातीह उभे सुकृतदुष्कृते  ।  

तस्माद्योगाय युज्यस्व योगः कर्मसु कौशलम्‌  ॥ २.५० ॥  मूल श्लोक
\end{quotation}

\begin{quotation}

बुद्धि-युक्तो जहा-तीह, उभे सुकृत-दुष्कृते  ।  

तस्माद्-योगाय युज्य-स्व, योगः कर्मसु कौशलम्‌  ॥ २.५० ॥  उच्चारण

\noindent\rule{16cm}{0.4pt} 
\end{quotation}


\begin{quotation}

कर्मजं बुद्धियुक्ता हि फलं त्यक्त्वा मनीषिणः  ।  

जन्मबन्धविनिर्मुक्ताः पदं गच्छन्त्यनामयम्‌  ॥ २.५१ ॥  मूल श्लोक
\end{quotation}

\begin{quotation}

कर्मजं बुद्धियुक्ता हि, फलं त्यक्त्वा मनीषिणः  ।  

जन्म-बन्ध-विनिर्-मुक्ताः, पदं गच्छन्त्य अनामयम्‌  ॥ २.५१ ॥  उच्चारण

\noindent\rule{16cm}{0.4pt} 
\end{quotation}


\begin{quotation}

यदा ते मोहकलिलं बुद्धिर्व्यतितरिष्यति  ।  

तदा गन्तासि निर्वेदं श्रोतव्यस्य श्रुतस्य च  ॥ २.५२ ॥  मूल श्लोक
\end{quotation}

\begin{quotation}

यदा ते मोह-कलिलं, बुद्धिर व्यतितर-इष्यति  ।  

तदा गन्तासि निर्वेदं, श्रोतव्य-अस्य श्रुतस्य च  ॥ २.५२ ॥  उच्चारण

\noindent\rule{16cm}{0.4pt} 
\end{quotation}


\begin{quotation}
श्रुतिविप्रतिपन्ना ते यदा स्थास्यति निश्चला  ।  

समाधावचला बुद्धिस्तदा योगमवाप्स्यसि  ॥ २.५३ ॥  मूल श्लोक
\end{quotation}

\begin{quotation}

श्रुति विप्रति-पन्ना ते, यदा स्थास्यति निश्चला  ।  

समाधा वचला बुद्धिस , तदा योगम अवाप्स्यसि  ॥ २.५३ ॥  उच्चारण

\noindent\rule{16cm}{0.4pt} 
\end{quotation}

\paragraph{\sanskrit अर्जुन उवाच}


\begin{quotation}

स्थितप्रज्ञस्य का भाषा समाधिस्थस्य केशव  ।  

स्थितधीः किं प्रभाषेत किमासीत व्रजेत किम्‌  ॥ २.५४ ॥  मूल श्लोक
\end{quotation}

\begin{quotation}

स्थित प्रज्ञस्य का भाषा, समाधिस्थ-अस्य केशव  ।  

स्थित-धीः किं प्रभाषेत, किमासीत व्रजेत किम्‌  ॥ २.५४ ॥  उच्चारण

\noindent\rule{16cm}{0.4pt} 
\end{quotation}

\paragraph{\sanskrit श्रीभगवानुवाच}
\begin{quotation}



प्रजहाति यदा कामान्‌ सर्वान्पार्थ मनोगतान्‌  ।  

आत्मयेवात्मना तुष्टः स्थितप्रज्ञस्तदोच्यते  ॥ २.५५ ॥  मूल श्लोक
\end{quotation}

\begin{quotation}

प्रजहाति यदा कामान्‌, सर्वान्पार्थ मनोगतान्‌  ।  

आत्मय एव आत्मना तुष्टः, स्थित-प्रज्ञस् तोदोच्यते  ॥ २.५५ ॥  उच्चारण

\noindent\rule{16cm}{0.4pt} 
\end{quotation}


\begin{quotation}

दुःखेष्वनुद्विग्नमनाः सुखेषु विगतस्पृहः  ।  

वीतरागभयक्रोधः स्थितधीर्मुनिरुच्यते  ॥ २.५६ ॥  मूल श्लोक
\end{quotation}

\begin{quotation}

दुःखेष्व अनुद्विग्न मनाः, सुखेषु विगत-स्पृहः  ।  

वीत राग भय क्रोधः, स्थित-धीर मुनिर उच्यते  ॥ २.५६ ॥  उच्चारण

\noindent\rule{16cm}{0.4pt} 
\end{quotation}


\begin{quotation}

यः सर्वत्रानभिस्नेहस्तत्तत्प्राप्य शुभाशुभम्‌  ।  

नाभिनंदति न द्वेष्टि तस्य प्रज्ञा प्रतिष्ठिता  ॥ २.५७ ॥  मूल श्लोक
\end{quotation}

\begin{quotation}
यः सर्वत्रा आनभि-स्नेहः, तत तत प्राप्य शुभा-शुभम्‌  ।  

नाभि नन्दति न द्वेष्टि, तस्य प्रज्ञा प्रतिष्ठिता  ॥ २.५७ ॥  उच्चारण

\noindent\rule{16cm}{0.4pt} 
\end{quotation}


\begin{quotation}
यदा संहरते चायं कूर्मोऽङ्गनीव सर्वशः  ।  

इन्द्रियाणीन्द्रियार्थेभ्यस्तस्य प्रज्ञा प्रतिष्ठिता  ॥ २.५८ ॥  मूल श्लोक
\end{quotation}

\begin{quotation}

यदा संहरते चायं, कूर्मोऽ अङ्गनीव सर्वशः  ।  

इन्द्रियाणी इन्द्रियार्थे-अभ्यस, तस्य प्रज्ञा प्रतिष्ठिता  ॥ २.५८ ॥  उच्चारण

\noindent\rule{16cm}{0.4pt} 
\end{quotation}


\begin{quotation}

विषया विनिवर्तन्ते निराहारस्य देहिनः  ।  

रसवर्जं रसोऽप्यस्य परं दृष्टवा निवर्तते  ॥ २.५९ ॥  मूल श्लोक
\end{quotation}

\begin{quotation}

विषया विनि-वर्तन्ते, निराहारस्य देहिनः  ।  

रसवर्जं रसोऽ अप्यस्य, परं दृष्टवा नि-वर्तते  ॥ २.५९ ॥  उच्चारण

\noindent\rule{16cm}{0.4pt} 
\end{quotation}


\begin{quotation}

यततो ह्यपि कौन्तेय पुरुषस्य विपश्चितः  ।  

इन्द्रियाणि प्रमाथीनि हरन्ति प्रसभं मनः  ॥ २.६० ॥  मूल श्लोक
\end{quotation}

\begin{quotation}

यततो ह्य अपि कौन्तेय, पुरुषस्य विपश्चितः  ।  

इन्द्रियाणि प्रमाथीनि, हरन्ति प्रसभं मनः  ॥ २.६० ॥  उच्चारण

\noindent\rule{16cm}{0.4pt} 
\end{quotation}


\begin{quotation}

तानि सर्वाणि संयम्य युक्त आसीत मत्परः  ।  

वशे हि यस्येन्द्रियाणि तस्य प्रज्ञा प्रतिष्ठिता  ॥ २.६१ ॥  मूल श्लोक
\end{quotation}

\begin{quotation}

तानि सर्वाणि संयम्य, युक्त आसीत मत्परः  ।  

वशे हि यस्ये इन्द्रियाणि, तस्य प्रज्ञा प्रतिष्ठिता  ॥ २.६१ ॥  उच्चारण

\noindent\rule{16cm}{0.4pt} 
\end{quotation}


\begin{quotation}

ध्यायतो विषयान्पुंसः संगस्तेषूपजायते  ।  

संगात्संजायते कामः कामात्क्रोधोऽभिजायते  ॥ २.६२ ॥  मूल श्लोक
\end{quotation}

\begin{quotation}

ध्यायतो विषयान् पुंसः, संगस् तेषू-पजायते  ।  

संगात् संजायते कामः, कामात् क्रोधोऽ अभिजायते  ॥ २.६२ ॥  उच्चारण

\noindent\rule{16cm}{0.4pt} 
\end{quotation}


\begin{quotation}

क्रोधाद्‍भवति सम्मोहः सम्मोहात्स्मृतिविभ्रमः  ।  

स्मृतिभ्रंशाद् बुद्धिनाशो बुद्धिनाशात्प्रणश्यति  ॥ २.६३ ॥  मूल श्लोक
\end{quotation}

\begin{quotation}

क्रोधाद्‍ भवति सम्मोहः, सम्मोहात स्मृति विभ्रमः  ।  

स्मृति-भ्रंशाद् बुद्धिनाशो, बुद्धिनाशात प्रणश्यति  ॥ २.६३ ॥  उच्चारण

\noindent\rule{16cm}{0.4pt} 
\end{quotation}


\begin{quotation}
रागद्वेषवियुक्तैस्तु विषयानिन्द्रियैश्चरन्‌  ।  

आत्मवश्यैर्विधेयात्मा प्रसादमधिगच्छति  ॥ २.६४ ॥  मूल श्लोक
\end{quotation}

\begin{quotation}

रागद्वेष वियुक्तै अस्तु, विषयानि इन्द्रियैश् चरन्‌  ।  

आत्म-वश्यैर विधेय-आत्मा, प्रसादम अधि-गच्छति  ॥ २.६४ ॥  उच्चारण

\noindent\rule{16cm}{0.4pt} 
\end{quotation}


\begin{quotation}

प्रसादे सर्वदुःखानां हानिरस्योपजायते  ।  

प्रसन्नचेतसो ह्याशु बुद्धिः पर्यवतिष्ठते  ॥ २.६५ ॥  मूल श्लोक
\end{quotation}

\begin{quotation}

प्रसादे सर्व-दुःखानां, हानिर अस्यो-पजायते  ।  

प्रसन्न-चेतसो ह्य आशु, बुद्धिः पर्य अव-तिष्ठते  ॥ २.६५ ॥  उच्चारण

\noindent\rule{16cm}{0.4pt} 
\end{quotation}


\begin{quotation}

नास्ति बुद्धिरयुक्तस्य न चायुक्तस्य भावना  ।  

न चाभावयतः शान्तिरशान्तस्य कुतः सुखम्‌  ॥ २.६६ ॥  मूल श्लोक
\end{quotation}

\begin{quotation}

नास्ति बुद्धिर युक्तस्य, न चायुक्तस्य भावना  ।  

न च आभावयतः शान्तिर, अशान्तस्य कुतः सुखम्‌  ॥ २.६६ ॥  उच्चारण

\noindent\rule{16cm}{0.4pt} 
\end{quotation}


\begin{quotation}

इन्द्रियाणां हि चरतां यन्मनोऽनुविधीयते  ।  

तदस्य हरति प्रज्ञां वायुर्नावमिवाम्भसि  ॥ २.६७ ॥  मूल श्लोक
\end{quotation}

\begin{quotation}

इन्द्रियाणां हि चरतां, यन्मनोऽ अनु-विधीयते  ।  

तदस्य हरति प्रज्ञां, वायुर नावम्  इव अम्भसि  ॥ २.६७ ॥  उच्चारण

\noindent\rule{16cm}{0.4pt} 
\end{quotation}


\begin{quotation}

तस्माद्यस्य महाबाहो निगृहीतानि सर्वशः  ।  

इन्द्रियाणीन्द्रियार्थेभ्यस्तस्य प्रज्ञा प्रतिष्ठिता  ॥ २.६८ ॥  मूल श्लोक
\end{quotation}

\begin{quotation}

तस्माद् यस्य महाबाहो, निगृही तानि सर्वशः  ।  

इन्द्रियाणी इन्द्रियार्थे-अभ्यस, तस्य प्रज्ञा प्रतिष्ठिता  ॥ २.६८ ॥  उच्चारण

\noindent\rule{16cm}{0.4pt} 
\end{quotation}


\begin{quotation}

या निशा सर्वभूतानां तस्यां जागर्ति संयमी  ।  

यस्यां जाग्रति भूतानि सा निशा पश्यतो मुनेः  ॥ २.६९ ॥  मूल श्लोक
\end{quotation}

\begin{quotation}

या निशा सर्व-भूतानां, तस्यां जागर्ति संयमी  ।  

यस्यां जाग्रति भूतानि, सा निशा पश्यतो मुनेः  ॥ २.६९ ॥  उच्चारण

\noindent\rule{16cm}{0.4pt} 
\end{quotation}


\begin{quotation}
आपूर्यमाणमचलप्रतिष्ठं-
समुद्रमापः प्रविशन्ति यद्वत्‌  ।  

तद्वत्कामा यं प्रविशन्ति सर्वे
स शान्तिमाप्नोति न कामकामी  ॥ २.७० ॥  मूल श्लोक
\end{quotation}

\begin{quotation}

आपूर्य-माणम अचल प्रतिष्ठं, 
समुद्रम आपः प्रविशन्ति यद्वत्‌  ।  

तद्-वत् कामा यं प्रविशन्ति सर्वे, 
स शान्तिम अप्-नोति न काम-कामी  ॥ २.७० ॥  उच्चारण

\noindent\rule{16cm}{0.4pt} 
\end{quotation}


\begin{quotation}

विहाय कामान्यः सर्वान्पुमांश्चरति निःस्पृहः  ।  

निर्ममो निरहंकारः स शान्तिमधिगच्छति  ॥ २.७१ ॥  मूल श्लोक
\end{quotation}

\begin{quotation}

विहाय कामान्यः सर्वान्, पुमांश् चरति निःस्पृहः  ।  

निर्ममो निर-अहंकारः, स शान्तिम अधि-गच्छति  ॥ २.७१ ॥  उच्चारण

\noindent\rule{16cm}{0.4pt} 
\end{quotation}


\begin{quotation}

एषा ब्राह्मी स्थितिः पार्थ नैनां प्राप्य विमुह्यति  ।  

स्थित्वास्यामन्तकालेऽपि ब्रह्मनिर्वाणमृच्छति  ॥ २.७२ ॥  मूल श्लोक
\end{quotation}

\begin{quotation}

एषा ब्राह्मी-स्थितिः पार्थ, नैनां प्राप्य विमुह्यति  ।  

स्थित्वा-अस्याम अन्त कालेऽ अपि, ब्रह्म-निर्वाणम रच्छति  ॥ २.७२ ॥  उच्चारण

\noindent\rule{16cm}{0.4pt} 
\end{quotation}

\begin{center}  ***** \end{center}

\begin{quotation}
ॐ तत् सद इति श्री मद्-भगवद्-गीतास उपनिषत्सु ब्रह्म विद्यायां योगशास्त्रे श्री कृष्णार्जुन संवादे साङ्ख्ययोगो नाम द्वितीयोऽ अध्यायः  ॥  २  ॥ 
\end{quotation}

