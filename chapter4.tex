\chapter{\sanskrit ज्ञानकर्मसंन्यासयोग}

\sanskrit

\paragraph{\sanskrit श्रीभगवानुवाच}

\begin{quotation}
इमं विवस्वते योगं प्रोक्तवानहमव्ययम्‌  ।  

विवस्वान्मनवे प्राह मनुरिक्ष्वाकवेऽब्रवीत्‌  ॥ ४.१ ॥  मूल श्लोक
\end{quotation}

\begin{quotation}

इमं विव-स्वते योगं, प्रोक्तवान अहम अव्ययम्‌  ।  

विवस्वां मनवे प्राह, मनुर इक्ष्वा-कवेऽ अब्रवीत्‌  ॥ ४.१ ॥  उच्चारण

\noindent\rule{16cm}{0.4pt} 
\end{quotation}


\begin{quotation}

एवं परम्पराप्राप्तमिमं राजर्षयो विदुः  ।  

स कालेनेह महता योगो नष्टः परन्तप   ॥ ४.२ ॥  मूल श्लोक
\end{quotation}

\begin{quotation}

एवं परम्परा प्राप्तं,  इमं राजर्षयो विदुः  ।  

स काले-नेह महता, योगो नष्टः परन्तप  ॥ ४.२ ॥  उच्चारण

\noindent\rule{16cm}{0.4pt} 
\end{quotation}


\begin{quotation}

स एवायं मया तेऽद्य योगः प्रोक्तः पुरातनः  ।  

भक्तोऽसि मे सखा चेति रहस्यं ह्येतदुत्तमम्‌  ॥ ४.३ ॥  मूल श्लोक
\end{quotation}

\begin{quotation}

स एवायं मया तेऽद्य, योगः प्रोक्तः पुरातनः  ।  

भक्तोऽसि मे सखा चेति, रहस्यं ह्ये एतद उत्तमम्‌  ॥ ४.३ ॥  उच्चारण

\noindent\rule{16cm}{0.4pt} 
\end{quotation}

\paragraph{\sanskrit अर्जुन उवाच}
\begin{quotation}



अपरं भवतो जन्म परं जन्म विवस्वतः  ।  

कथमेतद्विजानीयां त्वमादौ प्रोक्तवानिति  ॥ ४.४ ॥  मूल श्लोक
\end{quotation}

\begin{quotation}

अपरं भवतो जन्म, परं जन्म विव-स्वतः  ।  

कथं एतद विजानीयां, त्वमादौ प्रोक्तवान इति  ॥ ४.४ ॥  उच्चारण

\noindent\rule{16cm}{0.4pt} 
\end{quotation}


\paragraph{\sanskrit श्रीभगवानुवाच} 
\begin{quotation}




बहूनि मे व्यतीतानि जन्मानि तव चार्जुन  ।  

तान्यहं वेद सर्वाणि न त्वं वेत्थ परन्तप  ॥ ४.५ ॥  मूल श्लोक
\end{quotation}

\begin{quotation}
बहूनि मे व्यती तानि, जन्मानि तव चार्जुन  ।  

तान्यहं वेद सर्वाणि, न त्वं वेत्थ परन्तप  ॥ ४.५ ॥  उच्चारण

\noindent\rule{16cm}{0.4pt} 
\end{quotation}


\begin{quotation}

अजोऽपि सन्नव्ययात्मा भूतानामीश्वरोऽपि सन्‌  ।  

प्रकृतिं स्वामधिष्ठाय सम्भवाम्यात्ममायया  ॥ ४.६ ॥  मूल श्लोक
\end{quotation}

\begin{quotation}

अजोऽपि सन्न अव्यय-आत्मा, भूतानामी श्वरोऽपि सन्‌  ।  

प्रकृतिं स्वामधिष्ठाय, सम्-भवाम्य आत्म-मायया  ॥ ४.६ ॥  उच्चारण

\noindent\rule{16cm}{0.4pt} 
\end{quotation}


\begin{quotation}

यदा यदा हि धर्मस्य ग्लानिर्भवति भारत  ।  

अभ्युत्थानमधर्मस्य तदात्मानं सृजाम्यहम्‌  ॥ ४.७ ॥  मूल श्लोक
\end{quotation}

\begin{quotation}

यदा यदा हि धर्मस्य, ग्लानिर भवति भारत  ।  

अभ्युत्-थानम अधर्मस्य, तदात्मानं सृजाम्य अहम्‌  ॥ ४.७ ॥  उच्चारण

\noindent\rule{16cm}{0.4pt} 
\end{quotation}


\begin{quotation}

परित्राणाय साधूनां विनाशाय च दुष्कृताम्‌  ।  

धर्मसंस्थापनार्थाय सम्भवामि युगे युगे  ॥ ४.८ ॥  मूल श्लोक
\end{quotation}

\begin{quotation}

परित्राणाय साधूनां, विनाशाय च दुष्कृताम्‌  ।  

धर्म-संस्थापना अर्थाय, सम्-भवामि युगे युगे  ॥ ४.८ ॥  उच्चारण

\noindent\rule{16cm}{0.4pt} 
\end{quotation}


\begin{quotation}

जन्म कर्म च मे दिव्यमेवं यो वेत्ति तत्वतः  ।  

त्यक्तवा देहं पुनर्जन्म नैति मामेति सोऽर्जुन  ॥ ४.९ ॥  मूल श्लोक
\end{quotation}

\begin{quotation}

जन्म कर्म च मे दिव्यम, एवं यो वेत्ति तत्वतः  ।  

त्यक्तवा देहं पुनर्जन्म, नैति मामेति सोऽर्जुन  ॥ ४.९ ॥  उच्चारण

\noindent\rule{16cm}{0.4pt} 
\end{quotation}


\begin{quotation}

वीतरागभय क्रोधा मन्मया मामुपाश्रिताः  ।  

बहवो ज्ञानतपसा पूता मद्भावमागताः  ॥ ४.१० ॥  मूल श्लोक
\end{quotation}

\begin{quotation}

वीत राग भय क्रोधा, मन्मया माम उपाश्रिताः  ।  

बहवो ज्ञान-तपसा, पूता मद्भावम आगताः  ॥ ४.१० ॥  उच्चारण

\noindent\rule{16cm}{0.4pt} 
\end{quotation}


\begin{quotation}

ये यथा माँ प्रपद्यन्ते तांस्तथैव भजाम्यहम्‌  ।  

मम वर्त्मानुवर्तन्ते मनुष्याः पार्थ सर्वशः  ॥ ४.११ ॥  मूल श्लोक
\end{quotation}

\begin{quotation}

ये यथा माम प्रपद्-यन्ते, तांस तथैव भजाम्य अहम्‌  ।  

मम वर्त्मा-अनुवर्तन्ते, मनुष्याः पार्थ सर्वशः  ॥ ४.११ ॥  उच्चारण

\noindent\rule{16cm}{0.4pt} 
\end{quotation}


\begin{quotation}

काङ्‍क्षन्तः कर्मणां सिद्धिं यजन्त इह देवताः  ।  

क्षिप्रं हि मानुषे लोके सिद्धिर्भवति कर्मजा  ॥ ४.१२ ॥  मूल श्लोक
\end{quotation}

\begin{quotation}

काङ्‍क्षन्तः कर्मणां सिद्धिं, यजन्त इह देवताः  ।  

क्षिप्रं हि मानुषे लोके, सिद्धिर भवति कर्मजा  ॥ ४.१२ ॥  उच्चारण

\noindent\rule{16cm}{0.4pt} 
\end{quotation}


\begin{quotation}

चातुर्वर्ण्यं मया सृष्टं गुणकर्मविभागशः  ।  

तस्य कर्तारमपि मां विद्धयकर्तारमव्ययम्‌  ॥ ४.१३ ॥  मूल श्लोक
\end{quotation}

\begin{quotation}

चातुर्-वर्ण्यं मया सृष्टं ,गुण कर्म विभागशः  ।  

तस्य कर्तारं अपि माम, विद्धय अकर्तारं अव्ययम्‌  ॥ ४.१३ ॥  उच्चारण

\noindent\rule{16cm}{0.4pt} 
\end{quotation}


\begin{quotation}

न मां कर्माणि लिम्पन्ति न मे कर्मफले स्पृहा  ।  

इति मां योऽभिजानाति कर्मभिर्न स बध्यते  ॥ ४.१४ ॥  मूल श्लोक
\end{quotation}

\begin{quotation}

न मां कर्माणि लिम्पन्ति, न मे कर्मफले स्पृहा  ।  

इति मां योऽ अभिजानाति, कर्मभिर न स बध्यते  ॥ ४.१४ ॥  उच्चारण

\noindent\rule{16cm}{0.4pt} 
\end{quotation}


\begin{quotation}

एवं ज्ञात्वा कृतं कर्म पूर्वैरपि मुमुक्षुभिः  ।  

कुरु कर्मैव तस्मात्वं पूर्वैः पूर्वतरं कृतम्‌  ॥ ४.१५ ॥  मूल श्लोक
\end{quotation}

\begin{quotation}

एवं ज्ञात्वा कृतं कर्म, पूर्वैर अपि मुमुक्ष-उभिः  ।  

कुरु कर्मैव तस्मात्वं, पूर्वैः पूर्वतरं कृतम्‌  ॥ ४.१५ ॥  उच्चारण

\noindent\rule{16cm}{0.4pt} 
\end{quotation}


\begin{quotation}

किं कर्म किमकर्मेति कवयोऽप्यत्र मोहिताः  ।  

तत्ते कर्म प्रवक्ष्यामि यज्ज्ञात्वा मोक्ष्यसेऽशुभात्‌  ॥ ४.१६ ॥  मूल श्लोक
\end{quotation}

\begin{quotation}
किं कर्म किम अकर्मेति, कवयोऽ अपि अत्र मोहिताः  ।  

तत ते कर्म प्रवक्ष्यामि, यज ज्ञात्वा मोक्ष्यसेऽ अशुभात्‌  ॥ ४.१६ ॥  उच्चारण

\noindent\rule{16cm}{0.4pt} 
\end{quotation}


\begin{quotation}

कर्मणो ह्यपि बोद्धव्यं बोद्धव्यं च विकर्मणः  ।  

अकर्मणश्च बोद्धव्यं गहना कर्मणो गतिः  ॥ ४.१७ ॥  मूल श्लोक
\end{quotation}

\begin{quotation}

कर्मणो ह्य अपि बोद्ध-अव्यं, बोद्ध-अव्यं च विकर्मणः  ।  

अकर्मणश् च बोद्ध-अव्यं, गहना कर्मणो गतिः  ॥ ४.१७ ॥  उच्चारण

\noindent\rule{16cm}{0.4pt} 
\end{quotation}


\begin{quotation}

कर्मण्यकर्म यः पश्येदकर्मणि च कर्म यः  ।  

स बुद्धिमान्मनुष्येषु स युक्तः कृत्स्नकर्मकृत्‌  ॥ ४.१८ ॥  मूल श्लोक
\end{quotation}

\begin{quotation}

कर्मण्य अकर्म यः पश्येद, अकर्मणि च कर्म यः  ।  

स बुद्धिमान मनुष्येषु, स युक्तः कृत्सं-कर्म-कृत्‌  ॥ ४.१८ ॥  उच्चारण

\noindent\rule{16cm}{0.4pt} 
\end{quotation}


\begin{quotation}

यस्य सर्वे समारम्भाः कामसंकल्पवर्जिताः  ।  

ज्ञानाग्निदग्धकर्माणं तमाहुः पंडितं बुधाः  ॥ ४.१९ ॥  मूल श्लोक
\end{quotation}

\begin{quotation}

यस्य सर्वे समारम्भाः, काम संकल्प वर्जिताः  ।  

ज्ञानाग्नि दग्ध कर्माणं, तमाहुः पंडितं बुधाः  ॥ ४.१९ ॥  उच्चारण

\noindent\rule{16cm}{0.4pt} 
\end{quotation}


\begin{quotation}

त्यक्त्वा कर्मफलासङ्गं नित्यतृप्तो निराश्रयः  ।  

कर्मण्यभिप्रवृत्तोऽपि नैव किंचित्करोति सः  ॥ ४.२० ॥  मूल श्लोक
\end{quotation}

\begin{quotation}

त्यक्त्वा कर्म फला-ससङ्गम्, नित्य तृप्तो निराश्रयः  ।  

कर्मण्य अभि-प्रवृत्तोऽ अपि, नैव किंचित करोति सः  ॥ ४.२० ॥  उच्चारण

\noindent\rule{16cm}{0.4pt} 
\end{quotation}


\begin{quotation}

निराशीर्यतचित्तात्मा त्यक्तसर्वपरिग्रहः  ।  

शारीरं केवलं कर्म कुर्वन्नाप्नोति किल्बिषम्‌  ॥ ४.२१ ॥  मूल श्लोक
\end{quotation}

\begin{quotation}

निराशीर यत-चित्तात्मा, त्यक्त सर्व परिग्रहः  ।  

शारीरं केवलं कर्म, कुर्वन् न अपनोति किल्बिषम्‌  ॥ ४.२१ ॥  उच्चारण

\noindent\rule{16cm}{0.4pt} 
\end{quotation}


\begin{quotation}

यदृच्छालाभसंतुष्टो द्वंद्वातीतो विमत्सरः  ।  

समः सिद्धावसिद्धौ च कृत्वापि न निबध्यते  ॥ ४.२२ ॥  मूल श्लोक
\end{quotation}

\begin{quotation}
यदृच्छा लाभ संतुष्टो, द्वंद्वा तीतो विमत्सरः  ।  

समः सिद्धाव असिद्धौ च, कृत्वापि न निबध्यते  ॥ ४.२२ ॥  उच्चारण

\noindent\rule{16cm}{0.4pt} 
\end{quotation}


\begin{quotation}

गतसङ्‍गस्य मुक्तस्य ज्ञानावस्थितचेतसः  ।  

यज्ञायाचरतः कर्म समग्रं प्रविलीयते  ॥ ४.२३ ॥  मूल श्लोक
\end{quotation}

\begin{quotation}

गत सङ्‍गस्य मुक्तस्य, ज्ञाना अवस्थित चेतसः  ।  

यज्ञाय-आचरतः कर्म, समग्रं प्रवि-लीयते  ॥ ४.२३ ॥  उच्चारण

\noindent\rule{16cm}{0.4pt} 
\end{quotation}


\begin{quotation}

ब्रह्मार्पणं ब्रह्म हविर्ब्रह्माग्रौ ब्रह्मणा हुतम्‌  ।  

ब्रह्मैव तेन गन्तव्यं ब्रह्मकर्मसमाधिना  ॥ ४.२४ ॥  मूल श्लोक
\end{quotation}

\begin{quotation}

ब्रह्मा अर्पणं ब्रह्म हविर, ब्रह्म अगनौ ब्रह्मणा हुतम्‌  ।  

ब्रह्म-ऐव तेन गन्तव्यं, ब्रह्म-कर्म-समाधिना  ॥ ४.२४ ॥  उच्चारण

\noindent\rule{16cm}{0.4pt} 
\end{quotation}


\begin{quotation}

दैवमेवापरे यज्ञं योगिनः पर्युपासते  ।  

ब्रह्माग्नावपरे यज्ञं यज्ञेनैवोपजुह्वति  ॥ ४.२५ ॥  मूल श्लोक
\end{quotation}

\begin{quotation}

दैवम् एवापरे यज्ञं, योगिनः पर्य-उपासते  ।  
 
ब्रह्मा-अगनाव अपरे यज्ञं, यज्ञेन एैव उप-जुह्वति  ॥ ४.२५ ॥  उच्चारण

\noindent\rule{16cm}{0.4pt} 
\end{quotation}


\begin{quotation}

श्रोत्रादीनीन्द्रियाण्यन्ये संयमाग्निषु जुह्वति ।  

शब्दादीन्विषयानन्य इन्द्रियाग्निषु जुह्वति  ॥ ४.२६ ॥  मूल श्लोक
\end{quotation}

\begin{quotation}

श्रोत्रा दीनी इन्द्रियाण्य अन्ये, संयम-अग्निषु जुह्वति ।  

शब्दादीन् विषयान अन्य, इन्द्रिय-अग्निषु जुह्वति  ॥ ४.२६ ॥  उच्चारण

\noindent\rule{16cm}{0.4pt} 
\end{quotation}


\begin{quotation}

सर्वाणीन्द्रियकर्माणि प्राणकर्माणि चापरे  ।  

आत्मसंयमयोगाग्नौ जुह्वति ज्ञानदीपिते  ॥ ४.२७ ॥  मूल श्लोक
\end{quotation}

\begin{quotation}

सर्वाण इन्द्रिय कर्माणि, प्राण कर्माणि चापरे  ।  

आत्म संयम योगा अग्नौ, जुह्वति ज्ञान दीपिते  ॥ ४.२७ ॥  उच्चारण

\noindent\rule{16cm}{0.4pt} 
\end{quotation}


\begin{quotation}

द्रव्ययज्ञास्तपोयज्ञा योगयज्ञास्तथापरे  ।  

स्वाध्यायज्ञानयज्ञाश्च यतयः संशितव्रताः  ॥ ४.२८ ॥  मूल श्लोक
\end{quotation}

\begin{quotation}
द्रव्य यज्ञा-स्तपो यज्ञा, योग यज्ञा स्तथा-परे  ।  

स्वाध्याय ज्ञान यज्ञाश् च, यतयः संशित-व्रताः  ॥ ४.२८ ॥  उच्चारण

\noindent\rule{16cm}{0.4pt} 
\end{quotation}


\begin{quotation}

अपाने जुह्वति प्राणं प्राणेऽपानं तथापरे  ।  

प्राणापानगती रुद्ध्वा प्राणायामपरायणाः  ॥ ४.२९ ॥  मूल श्लोक
\end{quotation}

\begin{quotation}

अपाने जुह्वति प्राणं, प्राणेऽ अपानं तथा-परे  ।  

प्राणा-पान गती रुद्ध्वा, प्राणायाम परायणाः  ॥ ४.२९ ॥  उच्चारण

\noindent\rule{16cm}{0.4pt} 
\end{quotation}


\begin{quotation}

अपरे नियताहाराः प्राणान्प्राणेषु जुह्वति  ।  

सर्वेऽप्येते यज्ञविदो यज्ञक्षपितकल्मषाः  ॥ ४.३० ॥  मूल श्लोक
\end{quotation}

\begin{quotation}

अपरे नियत-आहाराः, प्राणान्-प्राणेषु जुह्वति  ।  

सर्वेऽ-अप्य एते यज्ञविदो, यज्ञ-क्षपित-कल्मषाः  ॥ ४.३० ॥  उच्चारण

\noindent\rule{16cm}{0.4pt} 
\end{quotation}


\begin{quotation}

यज्ञशिष्टामृतभुजो यान्ति ब्रह्म सनातनम्‌  ।  

नायं लोकोऽस्त्ययज्ञस्य कुतोऽन्यः कुरुसत्तम  ॥ ४.३१ ॥  मूल श्लोक
\end{quotation}

\begin{quotation}

यज्ञ शिष्टामृत भुजो, यान्ति ब्रह्म सनातनम्‌  ।  

नायं लोकोऽ अस्तय अयज्ञस्य, कुतोऽ अन्यः कुरु-सत्-तम  ॥ ४.३१ ॥  उच्चारण

\noindent\rule{16cm}{0.4pt} 
\end{quotation}


\begin{quotation}

एवं बहुविधा यज्ञा वितता ब्रह्मणो मुखे  ।  

कर्मजान्विद्धि तान्सर्वानेवं ज्ञात्वा विमोक्ष्यसे  ॥ ४.३२ ॥  मूल श्लोक
\end{quotation}

\begin{quotation}

एवं बहुविधा यज्ञा, वितता ब्रह्मणो मुखे  ।  

कर्मजान् विद्धि तान सर्वां, एवं ज्ञात्वा वि-मोक्ष्यसे  ॥ ४.३२ ॥  उच्चारण

\noindent\rule{16cm}{0.4pt} 
\end{quotation}


\begin{quotation}

श्रेयान्द्रव्यमयाद्यज्ञाज्ज्ञानयज्ञः परन्तप  ।  

सर्वं कर्माखिलं पार्थ ज्ञाने परिसमाप्यते  ॥ ४.३३ ॥  मूल श्लोक
\end{quotation}

\begin{quotation}

श्रेयान् द्रव्य-मयाद्य यज्ञनाज्, ज्ञान यज्ञः परन्तप  ।  

सर्वं कर्मा-खिलं पार्थ, ज्ञाने परि-समाप्यते  ॥ ४.३३ ॥  उच्चारण

\noindent\rule{16cm}{0.4pt} 
\end{quotation}


\begin{quotation}

तद्विद्धि प्रणिपातेन परिप्रश्नेन सेवया  ।  

उपदेक्ष्यन्ति ते ज्ञानं ज्ञानिनस्तत्वदर्शिनः  ॥ ४.३४ ॥  मूल श्लोक
\end{quotation}

\begin{quotation}
तद विद्धि प्रणि-पातेन, परि-प्रश्नेन सेवया  ।  

उप-देक्ष्य-अन्ति ते ज्ञानं, ज्ञानिनस तत्व-दर्शिनः  ॥ ४.३४ ॥  उच्चारण

\noindent\rule{16cm}{0.4pt} 
\end{quotation}


\begin{quotation}

यज्ज्ञात्वा न पुनर्मोहमेवं यास्यसि पाण्डव  ।  

येन भुतान्यशेषेण द्रक्ष्यस्यात्मन्यथो मयि  ॥ ४.३५ ॥  मूल श्लोक
\end{quotation}

\begin{quotation}

यज्ज्ञात्वा न पुनर्-मोहम्, एवं यास्यसि पाण्डव  ।  

येन भुतान्य अशेषेण, द्रक्ष्यस्य आत्मन्य अथो मयि  ॥ ४.३५ ॥  उच्चारण

\noindent\rule{16cm}{0.4pt} 
\end{quotation}


\begin{quotation}

अपि चेदसि पापेभ्यः सर्वेभ्यः पापकृत्तमः  ।  

सर्वं ज्ञानप्लवेनैव वृजिनं सन्तरिष्यसि  ॥ ४.३६ ॥  मूल श्लोक
\end{quotation}

\begin{quotation}

अपि चेद असि पापेभ्यः, सर्वेभ्यः पापकृत् तमः  ।  

सर्वं ज्ञान-प्लवेन एैव, वृजिनं सन्त-रिष्यसि  ॥ ४.३६ ॥  उच्चारण

\noindent\rule{16cm}{0.4pt} 
\end{quotation}


\begin{quotation}

यथैधांसि समिद्धोऽग्निर्भस्मसात्कुरुतेऽर्जुन  ।  

ज्ञानाग्निः सर्वकर्माणि भस्मसात्कुरुते तथा  ॥ ४.३७ ॥  मूल श्लोक
\end{quotation}

\begin{quotation}

यथै-धांसि समिद्धोऽ अग्नि:र, भस्मसात कुरुतेऽ अर्जुन  ।  

ज्ञानाग्निः सर्व-कर्माणि, भस्मसात कुरुते तथा  ॥ ४.३७ ॥  उच्चारण

\noindent\rule{16cm}{0.4pt} 
\end{quotation}


\begin{quotation}

न हि ज्ञानेन सदृशं पवित्रमिह विद्यते  ।  

तत्स्वयं योगसंसिद्धः कालेनात्मनि विन्दति  ॥ ४.३८ ॥  मूल श्लोक
\end{quotation}

\begin{quotation}

न हि ज्ञानेन सदृशं, पवित्रम इह विद्यते  ।  

तत स्वयं योग-संसिद्धः, कालेन आत्मनि विन्दति  ॥ ४.३८ ॥  उच्चारण

\noindent\rule{16cm}{0.4pt} 
\end{quotation}


\begin{quotation}

श्रद्धावाँल्लभते ज्ञानं तत्परः संयतेन्द्रियः  ।  

ज्ञानं लब्धवा परां शान्तिमचिरेणाधिगच्छति  ॥ ४.३९ ॥  मूल श्लोक
\end{quotation}

\begin{quotation}

श्रद्धा-वाँल लभते ज्ञानं, तत्परः संयत इन्द्रियः  ।  

ज्ञानं लब्धवा परां शान्तिम, अचिरेणा अधिगच्छति  ॥ ४.३९ ॥  उच्चारण

\noindent\rule{16cm}{0.4pt} 
\end{quotation}


\begin{quotation}

अज्ञश्चश्रद्दधानश्च संशयात्मा विनश्यति  ।  

नायं लोकोऽस्ति न परो न सुखं संशयात्मनः  ॥ ४.४० ॥  मूल श्लोक
\end{quotation}

\begin{quotation}
अज्ञश् च अश्रद्धांश च, संशयात्मा विनश्यति  ।  

नायं लोकोऽ अस्ति न परो, न सुखं संशयात्मनः  ॥ ४.४० ॥  उच्चारण

\noindent\rule{16cm}{0.4pt} 
\end{quotation}


\begin{quotation}

योगसन्नयस्तकर्माणं ज्ञानसञ्न्निसंशयम्‌  ।  

आत्मवन्तं न कर्माणि निबध्नन्ति धनञ्जय  ॥ ४.४१ ॥  मूल श्लोक
\end{quotation}

\begin{quotation}

योग संन्यस्त कर्माणं, ज्ञान संचिन्न-संशयम्‌  ।  

आत्म वन्तं न कर्माणि, निबध्-नन्ति धनञ्जय  ॥ ४.४१ ॥  उच्चारण

\noindent\rule{16cm}{0.4pt} 
\end{quotation}


\begin{quotation}

तस्मादज्ञानसम्भूतं हृत्स्थं ज्ञानासिनात्मनः  ।  

छित्वैनं संशयं योगमातिष्ठोत्तिष्ठ भारत  ॥ ४.४२ ॥  मूल श्लोक
\end{quotation}

\begin{quotation}

तस्माद अज्ञान-सम्भूतं, हृत्-स्थं  ज्ञान आसिन आत्मनः  ।  

छित्वैनं संशयं योगं, अतिष्ठ-उत्तिष्ठ भारत  ॥ ४.४२ ॥   उच्चारण

\noindent\rule{16cm}{0.4pt} 
\end{quotation}

\begin{center} ***** \end{center}

\begin{quotation}

ॐ तत् सद इति श्री मद्-भगवद्-गीतास उपनिषत्सु ब्रह्म विद्यायां योगशास्त्रे श्री कृष्णार्जुन संवादे ज्ञानकर्मसंन्यास योगो नाम चतुर्थोऽ अध्यायः  ॥  ४  ॥ 

\end{quotation}
