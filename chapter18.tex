\chapter{\sanskrit मोक्षसन्न्यासयोग} 
\paragraph{\sanskrit अर्जुन उवाच}
\begin{quotation} 
सन्न्यासस्य महाबाहो तत्त्वमिच्छामि वेदितुम्‌  ।  

त्यागस्य च हृषीकेश, पृथक केशि-निषूदन  ॥ १८.१ ॥  मूल श्लोक
\end{quotation}

\begin{quotation}

सन्न्यास-अस्य महाबाहो, तत्त्वम इच्छामि वेदितुम्‌  ।  

त्यागस्य च हृषीकेश, पृथक केशि-निषूदन  ॥ १८.१ ॥  उच्चारण

\noindent\rule{16cm}{0.4pt} 
\end{quotation}




\paragraph{\sanskrit श्रीभगवानुवाच}
\begin{quotation} 
काम्यानां कर्मणा न्यासं सन्न्यासं कवयो विदुः  ।  

सर्वकर्मफलत्यागं प्राहुस्त्यागं विचक्षणाः  ॥ १८.२ ॥  मूल श्लोक
\end{quotation}

\begin{quotation}

काम्यानाम् कर्मणा न्यासं, सन्न्यासं कवयो विदुः  ।  

सर्व कर्म फल त्यागं, प्राहुस् त्यागं वि-चक्षणाः  ॥ १८.२ ॥  उच्चारण

\noindent\rule{16cm}{0.4pt} 
\end{quotation}


\begin{quotation}

त्याज्यं दोषवदित्येके कर्म प्राहुर्मनीषिणः  ।  

यज्ञदानतपःकर्म न त्याज्यमिति चापरे  ॥ १८.३ ॥  मूल श्लोक
\end{quotation}

\begin{quotation}

त्याज्यं दोष वद इत्य एके, कर्म प्राहुर् मनीषिणः  ।  

यज्ञ दान तपः कर्म, न त्याज्यम इति चापरे  ॥ १८.३ ॥  उच्चारण

\noindent\rule{16cm}{0.4pt} 
\end{quotation}


\begin{quotation}

निश्चयं श्रृणु में तत्र त्यागे भरतसत्तम  ।  

त्यागो हि पुरुषव्याघ्र त्रिविधः सम्प्रकीर्तितः  ॥ १८.४ ॥  मूल श्लोक
\end{quotation}

\begin{quotation}

निश्चयं श्रृणु में तत्र, त्यागे भरत-सत्तम  ।  

त्यागो हि पुरुष-व्याघ्र, त्रि-विधः सम्-प्रकीर्-तितः  ॥ १८.४ ॥  उच्चारण

\noindent\rule{16cm}{0.4pt} 
\end{quotation}


\begin{quotation}

यज्ञदानतपःकर्म न त्याज्यं कार्यमेव तत्‌  ।  

यज्ञो दानं तपश्चैव पावनानि मनीषिणाम्‌  ॥ १८.५ ॥  मूल श्लोक
\end{quotation}

\begin{quotation}
यज्ञ दान तपः कर्म, न त्याज्यं कार्यम एव तत्‌  ।  

यज्ञो दानं तपश् चैव, पावनानि मनीषिणाम्‌  ॥ १८.५ ॥  उच्चारण

\noindent\rule{16cm}{0.4pt} 
\end{quotation}


\begin{quotation}

एतान्यपि तु कर्माणि सङ्‍गं त्यक्त्वा फलानि च  ।  

कर्तव्यानीति में पार्थ निश्चितं मतमुत्तमम्‌  ॥ १८.६ ॥  मूल श्लोक
\end{quotation}

\begin{quotation}

एतान्य अपि तु कर्माणि, सङ्‍गं त्यक्त्वा फलानि च  ।  

कर्तव्यान इति में पार्थ, निश्चितं मतम उत्तमम्‌  ॥ १८.६ ॥  उच्चारण

\noindent\rule{16cm}{0.4pt} 
\end{quotation}


\begin{quotation}

नियतस्य तु सन्न्यासः कर्मणो नोपपद्यते  ।  

मोहात्तस्य परित्यागस्तामसः परिकीर्तितः  ॥ १८.७ ॥  मूल श्लोक
\end{quotation}

\begin{quotation}

नियतस्य तु सन्न्यासः, कर्मणो न उप-पद्यते  ।  

मोहात् तस्य परि-त्यागस्, तामसः परि-कीर्तितः  ॥ १८.७ ॥  उच्चारण

\noindent\rule{16cm}{0.4pt} 
\end{quotation}


\begin{quotation}

दुःखमित्येव यत्कर्म कायक्लेशभयात्त्यजेत्‌  ।  

स कृत्वा राजसं त्यागं नैव त्यागफलं लभेत्‌  ॥ १८.८ ॥  मूल श्लोक
\end{quotation}

\begin{quotation}

दुःखम इत्य एव यत् कर्म, काय-क्लेश भयात् त्यजेत्‌  ।  

स कृत्वा राजसं त्यागं, नैव त्याग-फलं लभेत्‌  ॥ १८.८ ॥  उच्चारण

\noindent\rule{16cm}{0.4pt} 
\end{quotation}


\begin{quotation}

कार्यमित्येव यत्कर्म नियतं क्रियतेअर्जुन  ।  

सङ्‍गं त्यक्त्वा फलं चैव स त्यागः सात्त्विको मतः  ॥ १८.९ ॥  मूल श्लोक
\end{quotation}

\begin{quotation}

कार्यम इत्य एव यत् कर्म, नियतं क्रियतेऽ अर्जुन  ।  

सङ्‍गं त्यक्त्वा फलं चैव, स त्यागः सात्त्विको मतः  ॥ १८.९ ॥  उच्चारण

\noindent\rule{16cm}{0.4pt} 
\end{quotation}


\begin{quotation}

न द्वेष्ट्यकुशलं कर्म कुशले नानुषज्जते  ।  

त्यागी सत्त्वसमाविष्टो मेधावी छिन्नसंशयः  ॥ १८.१० ॥  मूल श्लोक
\end{quotation}

\begin{quotation}

न द्वेष्ट्य अ-कुशलं कर्म, कुशले ना अनु-षज्जते  ।  

त्यागी सत्त्व-समा-विष्टो, मेधावी छिन्न-संशयः  ॥ १८.१० ॥  उच्चारण

\noindent\rule{16cm}{0.4pt} 
\end{quotation}


\begin{quotation}

न हि देहभृता शक्यं त्यक्तुं कर्माण्यशेषतः  ।  

यस्तु कर्मफलत्यागी स त्यागीत्यभिधीयते  ॥ १८.११ ॥  मूल श्लोक
\end{quotation}

\begin{quotation}
न हि देह भृता शक्यं, त्यक्तुं कर्माण्य अ-शेषतः  ।  

यस्तु कर्म फल त्यागी, स त्यागीत्य अभि-धीयते  ॥ १८.११ ॥  उच्चारण

\noindent\rule{16cm}{0.4pt} 
\end{quotation}


\begin{quotation}

अनिष्टमिष्टं मिश्रं च त्रिविधं कर्मणः फलम्‌  ।  

भवत्यत्यागिनां प्रेत्य न तु सन्न्यासिनां क्वचित्‌  ॥ १८.१२ ॥  मूल श्लोक
\end{quotation}

\begin{quotation}

अनिष्टम इष्टं मिश्रं च, त्रि-विधं कर्मणः फलम्‌  ।  

भवत्य अत्यागिनां प्रेत्य, न तु सन्न्यासि-नाम् क्वचित ‌ ॥ १८.१२ ॥  उच्चारण

\noindent\rule{16cm}{0.4pt} 
\end{quotation}


\begin{quotation}

पञ्चैतानि महाबाहो कारणानि निबोध मे  ।  

साङ्ख्ये कृतान्ते प्रोक्तानि सिद्धये सर्वकर्मणाम्‌  ॥ १८.१३ ॥  मूल श्लोक
\end{quotation}

\begin{quotation}

पञ्च एैतानि महाबाहो, कारणानि निबोध मे  ।  

साङ्ख्ये कृतान्ते प्रोक्तानि, सिद्धये सर्व कर्मणाम्‌  ॥ १८.१३ ॥  उच्चारण

\noindent\rule{16cm}{0.4pt} 
\end{quotation}


\begin{quotation}

अधिष्ठानं तथा कर्ता करणं च पृथग्विधम्‌  ।  

विविधाश्च पृथक्चेष्टा दैवं चैवात्र पञ्चमम्‌  ॥ १८.१४ ॥  मूल श्लोक
\end{quotation}

\begin{quotation}

अधिष्ठानं तथा कर्ता, करणं च पृथग्-विधम्‌  ।  

विविधाश् च पृथक् चेष्टा, दैवं चैवात्र पञ्चमम्‌  ॥ १८.१४ ॥  उच्चारण

\noindent\rule{16cm}{0.4pt} 
\end{quotation}


\begin{quotation}

शरीरवाङ्‍मनोभिर्यत्कर्म प्रारभते नरः  ।  

न्याय्यं वा विपरीतं वा पञ्चैते तस्य हेतवः  ॥ १८.१५ ॥  मूल श्लोक
\end{quotation}

\begin{quotation}

शरीर-वाङ्‍ मनोभिर् यत्, कर्म प्रारभते नरः  ।  

न्याय्-यम् वा विपरीतं वा, पञ्च एैते तस्य हेतवः  ॥ १८.१५ ॥  उच्चारण

\noindent\rule{16cm}{0.4pt} 
\end{quotation}


\begin{quotation}

तत्रैवं सति कर्तारमात्मानं केवलं तु यः  ।  

पश्यत्यकृतबुद्धित्वान्न स पश्यति दुर्मतिः  ॥ १८.१६ ॥  मूल श्लोक
\end{quotation}

\begin{quotation}

तत्र एैवं सति कर्तारम, आत्मानं केवलं तु यः  ।  

पश्यत्य अकृत-बुद्धित्-वान्, न स पश्यति दुर्मतिः  ॥ १८.१६ ॥  उच्चारण

\noindent\rule{16cm}{0.4pt} 
\end{quotation}


\begin{quotation}

यस्य नाहङ्‍कृतो भावो बुद्धिर्यस्य न लिप्यते  ।  

हत्वापि स इमाँल्लोकान्न हन्ति न निबध्यते  ॥ १८.१७ ॥  मूल श्लोक
\end{quotation}

\begin{quotation}
यस्य ना अहङ्‍कृतो भावो, बुद्धिर् यस्य न लिप्यते  ।  

हत्वा अपि स इमाँल्-लोकान्, न हन्ति न निबध्यते  ॥ १८.१७ ॥  उच्चारण

\noindent\rule{16cm}{0.4pt} 
\end{quotation}


\begin{quotation}

ज्ञानं ज्ञेयं परिज्ञाता त्रिविधा कर्मचोदना  ।  

करणं कर्म कर्तेति त्रिविधः कर्मसङ्ग्रहः  ॥ १८.१८ ॥  मूल श्लोक
\end{quotation}

\begin{quotation}

ज्ञानं ज्ञेयं परिज्ञाता, त्रि-विधा कर्मचो-दना  ।  

करणं कर्म कर्तेति, त्रि-विधः कर्म सङ्ग्रहः  ॥ १८.१८ ॥  उच्चारण

\noindent\rule{16cm}{0.4pt} 
\end{quotation}


\begin{quotation}

ज्ञानं कर्म च कर्ता च त्रिधैव गुणभेदतः  ।  

प्रोच्यते गुणसङ्ख्याने यथावच्छृणु तान्यपि  ॥ १८.१९ ॥  मूल श्लोक
\end{quotation}

\begin{quotation}

ज्ञानं कर्म च कर्ता च, त्रि-धैव गुण-भेदतः  ।  

प्रोच्यते गुण-सङ्ख्याने, यथावच्-छृणु तान्य अपि  ॥ १८.१९ ॥  उच्चारण

\noindent\rule{16cm}{0.4pt} 
\end{quotation}


\begin{quotation}

सर्वभूतेषु येनैकं भावमव्ययमीक्षते  ।  

अविभक्तं विभक्तेषु तज्ज्ञानं विद्धि सात्त्विकम्  ॥ १८.२० ॥  मूल श्लोक
\end{quotation}

\begin{quotation}

सर्व-भूतेषु येन एैकं, भावम अ-व्ययम इक्षते  ।  

अ-विभक्तं विभक्तेषु, तज् ज्ञानं विद्धि सात्त्विकम्  ॥ १८.२० ॥  उच्चारण

\noindent\rule{16cm}{0.4pt} 
\end{quotation}


\begin{quotation}

पृथक्त्वेन तु यज्ज्ञानं नानाभावान्पृथग्विधान्‌  ।  

वेत्ति सर्वेषु भूतेषु तज्ज्ञानं विद्धि राजसम्‌  ॥ १८.२१ ॥  मूल श्लोक
\end{quotation}

\begin{quotation}

पृथक्त्वेन तु यज् ज्ञानं, नाना भावान् पृथग् विधान्‌  ।  

वेत्ति सर्वेषु भूतेषु, तज् ज्ञानं विद्धि राजसम्‌  ॥ १८.२१ ॥  उच्चारण

\noindent\rule{16cm}{0.4pt} 
\end{quotation}


\begin{quotation}

यत्तु कृत्स्नवदेकस्मिन्कार्ये सक्तमहैतुकम्‌ ।  

अतत्त्वार्थवदल्पंच तत्तामसमुदाहृतम्‌  ॥ १८.२२ ॥  मूल श्लोक
\end{quotation}

\begin{quotation}

यत् तु कृत्-सन्-वद एकस्मिन्, कार्ये सक्तम अहैतु-कम्‌ ।  

अतत्त्वा-अर्थ-वद अल्पं च, तत् तामसम उदा-हृतम्‌  ॥ १८.२२ ॥  उच्चारण

\noindent\rule{16cm}{0.4pt} 
\end{quotation}


\begin{quotation}

नियतं सङ्‍गरहितमरागद्वेषतः कृतम ।  

अफलप्रेप्सुना कर्म यत्तत्सात्त्विकमुच्यते  ॥ १८.२३ ॥  मूल श्लोक
\end{quotation}

\begin{quotation}
नियतं सङ्‍ग-रहितम, अराग द्वेषतः कृतम ।  

अफल-प्रेप्सुना कर्म, यत् तत् सात्त्विकम उच्यते  ॥ १८.२३ ॥  उच्चारण

\noindent\rule{16cm}{0.4pt} 
\end{quotation}


\begin{quotation}

यत्तु कामेप्सुना कर्म साहङ्‍कारेण वा पुनः ।  

क्रियते बहुलायासं तद्राजसमुदाहृतम्‌  ॥ १८.२४ ॥  मूल श्लोक
\end{quotation}

\begin{quotation}

यत् तु कामेप्सुना कर्म, सा अहङ्‍कारेण वा पुनः ।  

क्रियते बहुला-यासं, तद् राजसम उदा-हृतम्‌  ॥ १८.२४ ॥  उच्चारण

\noindent\rule{16cm}{0.4pt} 
\end{quotation}


\begin{quotation}

अनुबन्धं क्षयं हिंसामनवेक्ष्य च पौरुषम्‌  ।  

मोहादारभ्यते कर्म यत्तत्तामसमुच्यते  ॥ १८.२५ ॥  मूल श्लोक
\end{quotation}

\begin{quotation}

अनुबन्धं क्षयं हिंसाम, अनवेक्ष्य च पौरुषम्‌  ।  

मोहाद आरभ्यते कर्म, यत् तत् तामसम उच्यते  ॥ १८.२५ ॥  उच्चारण

\noindent\rule{16cm}{0.4pt} 
\end{quotation}


\begin{quotation}

मुक्तसङ्‍गोऽनहंवादी धृत्युत्साहसमन्वितः  ।  

सिद्धयसिद्धयोर्निर्विकारः कर्ता सात्त्विक उच्यते  ॥ १८.२६ ॥  मूल श्लोक
\end{quotation}

\begin{quotation}

मुक्त सङ्‍गोऽ अन-अहं-वादी, धृत्य-उत्साह-समन्वितः  ।  

सिद्धय-अ-सिद्धयोर् निर्विकारः, कर्ता सात्त्विक उच्यते  ॥ १८.२६ ॥  उच्चारण

\noindent\rule{16cm}{0.4pt} 
\end{quotation}


\begin{quotation}

रागी कर्मफलप्रेप्सुर्लुब्धो हिंसात्मकोऽशुचिः ।  

हर्षशोकान्वितः कर्ता राजसः परिकीर्तितः  ॥ १८.२७ ॥  मूल श्लोक
\end{quotation}

\begin{quotation}

रागी कर्म फल प्रेप्सुर्, लुब्धो हिंसात्मकोऽ अशुचिः ।  

हर्ष शोकान्वितः कर्ता, राजसः परि-कीर्तितः  ॥ १८.२७ ॥  उच्चारण

\noindent\rule{16cm}{0.4pt} 
\end{quotation}


\begin{quotation}

आयुक्तः प्राकृतः स्तब्धः शठोनैष्कृतिकोऽलसः  ।  

विषादी दीर्घसूत्री च कर्ता तामस उच्यते  ॥ १८.२८ ॥  मूल श्लोक
\end{quotation}

\begin{quotation}

आयुक्तः प्राकृतः स्तब्धः, शठो नैष्कृतिकोऽ अलसः  ।  

विषादी दीर्घ-सूत्री च, कर्ता तामस उच्यते  ॥ १८.२८ ॥  उच्चारण

\noindent\rule{16cm}{0.4pt} 
\end{quotation}


\begin{quotation}

बुद्धेर्भेदं धृतेश्चैव गुणतस्त्रिविधं श्रृणु  ।  

प्रोच्यमानमशेषेण पृथक्त्वेन धनंजय  ॥ १८.२९ ॥  मूल श्लोक
\end{quotation}

\begin{quotation}
बुद्धेर् भेदं धृतेश् चैव, गुणतस् त्रि-विधं श्रृणु  ।  

प्रोच्य-मानम अ-शेषेण, पृथक्-त्वेन धनंजय  ॥ १८.२९ ॥  उच्चारण

\noindent\rule{16cm}{0.4pt} 
\end{quotation}


\begin{quotation}

प्रवत्तिं च निवृत्तिं च कार्याकार्ये भयाभये ।  

बन्धं मोक्षं च या वेति बुद्धिः सा पार्थ सात्त्विकी  ॥ १८.३० ॥  मूल श्लोक
\end{quotation}

\begin{quotation}

प्रवत्तिं च निवृत्तिं च, कार्या कार्ये भया भये ।  

बन्धं मोक्षं च या वेति, बुद्धिः सा पार्थ सात्त्विकी  ॥ १८.३० ॥  उच्चारण

\noindent\rule{16cm}{0.4pt} 
\end{quotation}


\begin{quotation}

यया धर्ममधर्मं च कार्यं चाकार्यमेव च ।  

अयथावत्प्रजानाति बुद्धिः सा पार्थ राजसी  ॥ १८.३१ ॥  मूल श्लोक
\end{quotation}

\begin{quotation}

यया धर्मम अधर्मं च, कार्यं चा अकार्यम एव च ।  

अ-यथावत् प्रजानाति, बुद्धिः सा पार्थ राजसी  ॥ १८.३१ ॥  उच्चारण

\noindent\rule{16cm}{0.4pt} 
\end{quotation}


\begin{quotation}

अधर्मं धर्ममिति या मन्यते तमसावृता ।  

सर्वार्थान्विपरीतांश्च बुद्धिः सा पार्थ तामसी  ॥ १८.३२ ॥  मूल श्लोक
\end{quotation}

\begin{quotation}

अधर्मं धर्म-मिति या, मन्यते तमस आवृता ।  

सर्वार्थान् विपरीतांश् च, बुद्धिः सा पार्थ तामसी  ॥ १८.३२ ॥  उच्चारण

\noindent\rule{16cm}{0.4pt} 
\end{quotation}


\begin{quotation}

धृत्या यया धारयते मनःप्राणेन्द्रियक्रियाः ।  

योगेनाव्यभिचारिण्या धृतिः सा पार्थ सात्त्विकी  ॥ १८.३३ ॥  मूल श्लोक
\end{quotation}

\begin{quotation}

धृत्या यया धारयते, मनः प्राणेन्द्रिय-क्रियाः ।  

योगेना अव्यभि-चारिण्या, धृतिः सा पार्थ सात्त्विकी  ॥ १८.३३ ॥  उच्चारण

\noindent\rule{16cm}{0.4pt} 
\end{quotation}


\begin{quotation}

यया तु धर्मकामार्थान्धृत्या धारयतेऽर्जुन ।  

प्रसङ्‍गेन फलाकाङ्क्षी धृतिः सा पार्थ राजसी  ॥ १८.३४ ॥  मूल श्लोक
\end{quotation}

\begin{quotation}

यया तु धर्म कामार्थान्, धृत्या धारयतेऽ अर्जुन ।  

प्रसङ्‍गेन फलाकाङ्क्षी, धृतिः सा पार्थ राजसी  ॥ १८.३४ ॥  उच्चारण

\noindent\rule{16cm}{0.4pt} 
\end{quotation}


\begin{quotation}

यया स्वप्नं भयं शोकं विषादं मदमेव च ।  

न विमुञ्चति दुर्मेधा धृतिः सा पार्थ तामसी  ॥ १८.३५ ॥  मूल श्लोक
\end{quotation}

\begin{quotation}
यया स्वप्नं भयं शोकं, विषादं मदम एव च ।  

न विमुञ्चति दुर्मेधा, धृतिः सा पार्थ तामसी  ॥ १८.३५ ॥  उच्चारण

\noindent\rule{16cm}{0.4pt} 
\end{quotation}


\begin{quotation}

सुखं त्विदानीं त्रिविधं श्रृणु मे भरतर्षभ ।  

अभ्यासाद्रमते यत्र दुःखान्तं च निगच्छति  ॥ १८.३६ ॥  मूल श्लोक
\end{quotation}

\begin{quotation}

सुखं त्व इदानीं त्रि-विधं, श्रृणु मे भरतर्षभ ।  

अभ्यासाद् रमते यत्र, दुःखान्तं च निगच्छति  ॥ १८.३६ ॥  उच्चारण

\noindent\rule{16cm}{0.4pt} 
\end{quotation}


\begin{quotation}

यत्तदग्रे विषमिव परिणामेऽमृतोपमम्‌ ।  

तत्सुखं सात्त्विकं प्रोक्तमात्मबुद्धिप्रसादजम्‌  ॥ १८.३७ ॥  मूल श्लोक
\end{quotation}

\begin{quotation}

यत् तद अग्रे विषम इव, परिणामेऽ अमृत-उपमम्‌ ।  

तत् सुखं सात्त्विकं प्रोक्तम, आत्म बुद्धि प्रसादजम्‌  ॥ १८.३७ ॥  उच्चारण

\noindent\rule{16cm}{0.4pt} 
\end{quotation}


\begin{quotation}

विषयेन्द्रियसंयोगाद्यत्तदग्रेऽमृतोपमम्‌ ।  

परिणामे विषमिव तत्सुखं राजसं स्मृतम्‌  ॥ १८.३८ ॥  मूल श्लोक
\end{quotation}

\begin{quotation}

विषय एन्द्रिय संयोगाद्, यत् तद अग्रेऽ अमृत-उपमम्‌ ।  

परिणामे विषम इव, तत् सुखं राजसं स्मृतम्‌  ॥ १८.३८ ॥  उच्चारण

\noindent\rule{16cm}{0.4pt} 
\end{quotation}


\begin{quotation}

यदग्रे चानुबन्धे च सुखं मोहनमात्मनः ।  

निद्रालस्यप्रमादोत्थं तत्तामसमुदाहृतम्‌  ॥ १८.३९ ॥  मूल श्लोक
\end{quotation}

\begin{quotation}

यद अग्रे चा अनुबन्धे च, सुखं मोहनम आत्मनः ।  

निद्रा आलस्य प्रमाद उत्थं, तत् तामसम उदा-हृतम्‌  ॥ १८.३९ ॥  उच्चारण

\noindent\rule{16cm}{0.4pt} 
\end{quotation}


\begin{quotation}

न तदस्ति पृथिव्यां वा दिवि देवेषु वा पुनः ।  

सत्त्वं प्रकृतिजैर्मुक्तं यदेभिःस्यात्त्रिभिर्गुणैः  ॥ १८.४० ॥  मूल श्लोक
\end{quotation}

\begin{quotation}

न तद अस्ति पृथिव्यां वा, दिवि देवेषु वा पुनः ।  

सत्त्वं प्रकृतिजैर्  मुक्तं, यद एभिः स्यात् त्रिभिर् गुणैः  ॥ १८.४० ॥  उच्चारण

\noindent\rule{16cm}{0.4pt} 
\end{quotation}


\begin{quotation}

ब्राह्मणक्षत्रियविशां शूद्राणां च परन्तप ।  

कर्माणि प्रविभक्तानि स्वभावप्रभवैर्गुणैः  ॥ १८.४१ ॥  मूल श्लोक
\end{quotation}

\begin{quotation}
ब्राह्मण-क्षत्रिय-विशां, शूद्राणां च परन्तप ।  

कर्माणि प्रवि-भक्तानि, स्वभाव प्रभवैर् गुणैः  ॥ १८.४१ ॥  उच्चारण

\noindent\rule{16cm}{0.4pt} 
\end{quotation}


\begin{quotation}

शमो दमस्तपः शौचं क्षान्तिरार्जवमेव च ।  

ज्ञानं विज्ञानमास्तिक्यं ब्रह्मकर्म स्वभावजम्‌  ॥ १८.४२ ॥  मूल श्लोक
\end{quotation}

\begin{quotation}

शमो दमस् तपः शौचं, क्षान्तिर आर्जवम एव च ।  

ज्ञानं विज्ञानम आस्तिक्-यम्, ब्रह्म कर्म स्वभाव-जम्‌  ॥ १८.४२ ॥  उच्चारण

\noindent\rule{16cm}{0.4pt} 
\end{quotation}


\begin{quotation}

शौर्यं तेजो धृतिर्दाक्ष्यं युद्धे चाप्यपलायनम्‌ ।  

दानमीश्वरभावश्च क्षात्रं कर्म स्वभावजम्‌  ॥ १८.४३ ॥  मूल श्लोक
\end{quotation}

\begin{quotation}

शौर्यं तेजो धृतिर् दाक्ष्यं, युद्धे चाप्य अ-पलायनम्‌ ।  

दानम ईश्वर भावश् च, क्षात्रं कर्म स्वभाव-जम्‌  ॥ १८.४३ ॥  उच्चारण

\noindent\rule{16cm}{0.4pt} 
\end{quotation}


\begin{quotation}

कृषिगौरक्ष्यवाणिज्यं वैश्यकर्म स्वभावजम्‌ ।  

परिचर्यात्मकं कर्म शूद्रस्यापि स्वभावजम्‌  ॥ १८.४४ ॥  मूल श्लोक
\end{quotation}

\begin{quotation}

कृषि गौ-रक्ष्य वाणिज्यं, वैश्य कर्म स्वभाव-जम्‌ ।  

परिचर्य आत्मकं कर्म, शूद्रस्य अपि स्वभाव-जम्‌  ॥ १८.४४ ॥  उच्चारण

\noindent\rule{16cm}{0.4pt} 
\end{quotation}


\begin{quotation}

स्वे स्वे कर्मण्यभिरतः संसिद्धिं लभते नरः ।  

स्वकर्मनिरतः सिद्धिं यथा विन्दति तच्छृणु  ॥ १८.४५ ॥  मूल श्लोक
\end{quotation}

\begin{quotation}

स्वे स्वे कर्मण्य अ-भिरतः, संसिद्धिं लभते नरः ।  

स्व-कर्म-निरतः सिद्धिं, यथा विन्दति तच् छृणु  ॥ १८.४५ ॥  उच्चारण

\noindent\rule{16cm}{0.4pt} 
\end{quotation}


\begin{quotation}

यतः प्रवृत्तिर्भूतानां येन सर्वमिदं ततम्‌ ।  

स्वकर्मणा तमभ्यर्च्य सिद्धिं विन्दति मानवः  ॥ १८.४६ ॥  मूल श्लोक
\end{quotation}

\begin{quotation}

यतः प्रवृत्तिर् भूतानां, येन सर्वम इदं ततम्‌ ।  

स्व-कर्मणा तम अभ्यर्च्य, सिद्धिं विन्दति मानवः  ॥ १८.४६ ॥  उच्चारण

\noindent\rule{16cm}{0.4pt} 
\end{quotation}


\begin{quotation}

श्रेयान्स्वधर्मो विगुणः परधर्मात्स्वनुष्ठितात्‌ ।  

स्वभावनियतं कर्म कुर्वन्नाप्नोति किल्बिषम्‌  ॥ १८.४७ ॥  मूल श्लोक
\end{quotation}

\begin{quotation}
श्रेयान् स्वधर्मो विगुणः, परधर्मात् स्व अनुष्ठि-तात्‌ ।  

स्वभाव नियतं कर्म, कुर्वन् ना-अप्-नोति किल्बिषम्‌  ॥ १८.४७ ॥  उच्चारण

\noindent\rule{16cm}{0.4pt} 
\end{quotation}


\begin{quotation}

सहजं कर्म कौन्तेय सदोषमपि न त्यजेत्‌ ।  

सर्वारम्भा हि दोषेण धूमेनाग्निरिवावृताः  ॥ १८.४८ ॥  मूल श्लोक
\end{quotation}

\begin{quotation}

सहजं कर्म कौन्तेय, सदोषम अपि न त्यजेत्‌ ।  

सर्वारम्भा हि दोषेण, धूमेना अग्निर इव-आवृताः  ॥ १८.४८ ॥  उच्चारण

\noindent\rule{16cm}{0.4pt} 
\end{quotation}


\begin{quotation}

असक्तबुद्धिः सर्वत्र जितात्मा विगतस्पृहः ।  

नैष्कर्म्यसिद्धिं परमां सन्न्यासेनाधिगच्छति  ॥ १८.४९ ॥  मूल श्लोक
\end{quotation}

\begin{quotation}

असक्त बुद्धिः सर्वत्र, जितात्मा विगत-स्पृहः ।  

नैष्कर्म्य सिद्धिं परमां, सन्न्यासेना अधि-गच्छति  ॥ १८.४९ ॥  उच्चारण

\noindent\rule{16cm}{0.4pt} 
\end{quotation}


\begin{quotation}

सिद्धिं प्राप्तो यथा ब्रह्म तथाप्नोति निबोध मे ।  

समासेनैव कौन्तेय निष्ठा ज्ञानस्य या परा  ॥ १८.५० ॥  मूल श्लोक
\end{quotation}

\begin{quotation}

सिद्धिं प्राप्तो यथा ब्रह्म, तथा अप्-नोति निबोध मे ।  

समासेन एैव कौन्तेय, निष्ठा ज्ञानस्य या परा  ॥ १८.५० ॥  उच्चारण

\noindent\rule{16cm}{0.4pt} 
\end{quotation}


\begin{quotation}

बुद्ध्‌या विशुद्धया युक्तो धृत्यात्मानं नियम्य च ।  

शब्दादीन्विषयांस्त्यक्त्वा रागद्वेषौ व्युदस्य च  ॥ १८.५१ ॥  मूल श्लोक
\end{quotation}

\begin{quotation}

बुद्ध्‌या विशुद्धया युक्तो, धृत्य आत्मानं नियम्य च ।  

शब्दादीन् विषयांस् त्यक्त्वा, राग द्वेषौ व्युदस्य च  ॥ १८.५१ ॥  उच्चारण

\noindent\rule{16cm}{0.4pt} 
\end{quotation}


\begin{quotation}

विविक्तसेवी लघ्वाशी यतवाक्कायमानस ।  

ध्यानयोगपरो नित्यं वैराग्यं समुपाश्रितः  ॥ १८.५२ ॥  मूल श्लोक
\end{quotation}

\begin{quotation}

विविक्त सेवी लघ्वाशी, यत वाक्काय मानस ।  

ध्यान योग परो नित्यं, वैराग्यं सम उपाश्रितः  ॥ १८.५२ ॥  उच्चारण

\noindent\rule{16cm}{0.4pt} 
\end{quotation}


\begin{quotation}

अहङकारं बलं दर्पं कामं क्रोधं परिग्रहम्‌ ।  

विमुच्य निर्ममः शान्तो ब्रह्मभूयाय कल्पते  ॥ १८.५३ ॥  मूल श्लोक
\end{quotation}

\begin{quotation}
अहङकारं बलं दर्पं, कामं क्रोधं परिग्रहम्‌ ।  

विमुच्य निर्ममः शान्तो, ब्रह्म-भूयाय कल्पते  ॥ १८.५३ ॥  उच्चारण

\noindent\rule{16cm}{0.4pt} 
\end{quotation}


\begin{quotation}

ब्रह्मभूतः प्रसन्नात्मा न शोचति न काङ्क्षति ।  

समः सर्वेषु भूतेषु मद्भक्तिं लभते पराम्‌  ॥ १८.५४ ॥  मूल श्लोक
\end{quotation}

\begin{quotation}

ब्रह्म भूतः प्रसन्न आत्मा, न शोचति न काङ्क्षति ।  

समः सर्वेषु भूतेषु, मद् भक्तिं लभते पराम्‌  ॥ १८.५४ ॥  उच्चारण

\noindent\rule{16cm}{0.4pt} 
\end{quotation}


\begin{quotation}

भक्त्या मामभिजानाति यावान्यश्चास्मि तत्त्वतः ।  

ततो मां तत्त्वतो ज्ञात्वा विशते तदनन्तरम्‌  ॥ १८.५५ ॥  मूल श्लोक
\end{quotation}

\begin{quotation}

भक्त्या माम अभि-जानाति, यावान्-यश् चास्मि तत्त्व-तः ।  

ततो मां तत्त्वतो ज्ञात्वा, विशते तद अनन्त-रम्‌  ॥ १८.५५ ॥  उच्चारण

\noindent\rule{16cm}{0.4pt} 
\end{quotation}


\begin{quotation}

सर्वकर्माण्यपि सदा कुर्वाणो मद्व्यपाश्रयः ।  

मत्प्रसादादवाप्नोति शाश्वतं पदमव्ययम्‌  ॥ १८.५६ ॥  मूल श्लोक
\end{quotation}

\begin{quotation}

सर्व कर्माण्य अपि सदा, कुर्वाणो मद्-व्यपाश्रयः ।  

मत्-प्रसादाद अवाप्-नोति, शाश्वतं पदम अ-व्ययम्‌  ॥ १८.५६ ॥  उच्चारण

\noindent\rule{16cm}{0.4pt} 
\end{quotation}


\begin{quotation}

चेतसा सर्वकर्माणि मयि सन्न्यस्य मत्परः ।  

बुद्धियोगमुपाश्रित्य मच्चित्तः सततं भव  ॥ १८.५७ ॥  मूल श्लोक
\end{quotation}

\begin{quotation}

चेतसा सर्व कर्माणि, मयि सन्न्यस्य मत्परः ।  

बुद्धि योगम उपाश्रित्य, मच्चित्तः सततं भव  ॥ १८.५७ ॥  उच्चारण

\noindent\rule{16cm}{0.4pt} 
\end{quotation}


\begin{quotation}

मच्चित्तः सर्वदुर्गाणि मत्प्रसादात्तरिष्यसि ।  

अथ चेत्त्वमहङ्कारान्न श्रोष्यसि विनङ्क्ष्यसि  ॥ १८.५८ ॥  मूल श्लोक
\end{quotation}

\begin{quotation}

मच्चित्तः सर्व-दुर्गाणि, मत् प्रसादात् तरिष्यसि ।  

अथ चेत् त्वं अहन्कारान्, न श्रोष्यसि विनङ्क्ष्य-सि  ॥ १८.५८ ॥  उच्चारण

\noindent\rule{16cm}{0.4pt} 
\end{quotation}


\begin{quotation}

यदहङ्‍कारमाश्रित्य न योत्स्य इति मन्यसे  ।  

मिथ्यैष व्यवसायस्ते प्रकृतिस्त्वां नियोक्ष्यति  ॥ १८.५९ ॥  मूल श्लोक
\end{quotation}

\begin{quotation}
यद अहङ्‍कारम आश्रित्य, न योत्स्य इति मन्यसे  ।  

मिथ्यैष व्यवसाय-स्ते, प्रकृतिस् त्वां नियो-क्ष्यति  ॥ १८.५९ ॥  उच्चारण

\noindent\rule{16cm}{0.4pt} 
\end{quotation}


\begin{quotation}

स्वभावजेन कौन्तेय निबद्धः स्वेन कर्मणा  ।  

कर्तुं नेच्छसि यन्मोहात्करिष्यस्यवशोऽपि तत्‌  ॥ १८.६० ॥  मूल श्लोक
\end{quotation}

\begin{quotation}

स्वभाव-जेन कौन्तेय, निबद्धः स्वेन कर्मणा  ।  

कर्तुं नेच्छसि यन् मोहात्, करिष्यस्य वशोऽ अपि तत्‌  ॥ १८.६० ॥  उच्चारण

\noindent\rule{16cm}{0.4pt} 
\end{quotation}


\begin{quotation}

ईश्वरः सर्वभूतानां हृद्देशेऽर्जुन तिष्ठति ।  

भ्रामयन्सर्वभूतानि यन्त्रारुढानि मायया  ॥ १८.६१ ॥  मूल श्लोक
\end{quotation}

\begin{quotation}

ईश्वरः सर्व भूतानां, हृद्-देशेऽ अर्जुन तिष्ठति ।  

भ्रामयन् सर्व भूतानि, यन्त्रा-रुढानि मायया  ॥ १८.६१ ॥  उच्चारण

\noindent\rule{16cm}{0.4pt} 
\end{quotation}


\begin{quotation}

तमेव शरणं गच्छ सर्वभावेन भारत ।  

तत्प्रसादात्परां शान्तिं स्थानं प्राप्स्यसि शाश्वतम्‌  ॥ १८.६२ ॥  मूल श्लोक
\end{quotation}

\begin{quotation}

तमेव शरणं गच्छ, सर्वभावेन भारत ।  

तत्-प्रसादात् पराम् शान्तिं, स्थानं प्राप्स्-यसि शाश्वतम्‌  ॥ १८.६२ ॥  उच्चारण

\noindent\rule{16cm}{0.4pt} 
\end{quotation}


\begin{quotation}

इति ते ज्ञानमाख्यातं गुह्याद्‍गुह्यतरं मया  ।  

विमृश्यैतदशेषेण यथेच्छसि तथा कुरु  ॥ १८.६३ ॥  मूल श्लोक
\end{quotation}

\begin{quotation}

इति ते ज्ञानम आख्यातं, गुह्याद्‍ गुह्यतरं मया  ।  

विमृश्य एैतद अ-शेषेण, यथ एच्छसि तथा कुरु  ॥ १८.६३ ॥  उच्चारण

\noindent\rule{16cm}{0.4pt} 
\end{quotation}


\begin{quotation}

सर्वगुह्यतमं भूतः श्रृणु मे परमं वचः  ।  

इष्टोऽसि मे दृढमिति ततो वक्ष्यामि ते हितम्‌  ॥ १८.६४ ॥  मूल श्लोक
\end{quotation}

\begin{quotation}

सर्व गुह्य तमं भूतः, श्रृणु मे परमं वचः  ।  

इष्टोऽ असि मे दृढम इति, ततो वक्ष्यामि ते हितम्‌  ॥ १८.६४ ॥  उच्चारण

\noindent\rule{16cm}{0.4pt} 
\end{quotation}


\begin{quotation}

मन्मना भव मद्भक्तो मद्याजी मां नमस्कुरु  ।  

मामेवैष्यसि सत्यं ते प्रतिजाने प्रियोऽसि मे  ॥ १८.६५ ॥  मूल श्लोक
\end{quotation}

\begin{quotation}
मन्मना भव मद्-भक्तो, मद्-याजी मां नमस्कुरु  ।  

माम एवैष्यसि सत्यं ते, प्रति जाने प्रियोऽसि मे  ॥ १८.६५ ॥  उच्चारण

\noindent\rule{16cm}{0.4pt} 
\end{quotation}


\begin{quotation}

सर्वधर्मान्परित्यज्य मामेकं शरणं व्रज  ।  

अहं त्वा सर्वपापेभ्यो मोक्षयिष्यामि मा शुचः  ॥ १८.६६ ॥  मूल श्लोक
\end{quotation}

\begin{quotation}

सर्व धर्मान् परि-त्यज्य, माम एकं शरणं व्रज  ।  

अहं त्वा सर्व पापेभ्यो, मोक्षय इष्यामि मा शुचः  ॥ १८.६६ ॥  उच्चारण

\noindent\rule{16cm}{0.4pt} 
\end{quotation}


\begin{quotation}

इदं ते नातपस्काय नाभक्ताय कदाचन  ।  

न चाशुश्रूषवे वाच्यं न च मां योऽभ्यसूयति  ॥ १८.६७ ॥  मूल श्लोक
\end{quotation}

\begin{quotation}

इदं ते न अ-तपस्काय, न अ-भक्ताय कदाचन  ।  

न च अशु-श्रूष-वे वाच्यं, न च मां योऽभ्य सूयति  ॥ १८.६७ ॥  उच्चारण

\noindent\rule{16cm}{0.4pt} 
\end{quotation}


\begin{quotation}

य इमं परमं गुह्यं मद्भक्तेष्वभिधास्यति  ।  

भक्तिं मयि परां कृत्वा मामेवैष्यत्यसंशयः  ॥ १८.६८ ॥  मूल श्लोक
\end{quotation}

\begin{quotation}

य इमं परमं गुह्यं, मद् भक्तेष्व अभि-धास्यति  ।  

भक्तिं मयि परां कृत्वा, माम एवैष्-यत्य असंशयः  ॥ १८.६८ ॥  उच्चारण

\noindent\rule{16cm}{0.4pt} 
\end{quotation}


\begin{quotation}

न च तस्मान्मनुष्येषु कश्चिन्मे प्रियकृत्तमः  ।  

भविता न च मे तस्मादन्यः प्रियतरो भुवि  ॥ १८.६९ ॥  मूल श्लोक
\end{quotation}

\begin{quotation}

न च तस्मान् मनुष्येषु, कश्चिन् मे प्रिय-कृत्-तमः  ।  

भविता न च मे तस्माद, अन्यः प्रिय तरो भुवि  ॥ १८.६९ ॥  उच्चारण

\noindent\rule{16cm}{0.4pt} 
\end{quotation}


\begin{quotation}

अध्येष्यते च य इमं धर्म्यं संवादमावयोः  ।  

ज्ञानयज्ञेन तेनाहमिष्टः स्यामिति मे मतिः  ॥ १८.७० ॥  मूल श्लोक
\end{quotation}

\begin{quotation}

अध्येष्-यते च य इमं, धर्म्यं संवादम आवयोः  ।  

ज्ञान यज्ञेन तेनाहम, इष्टः स्याम इति मे मतिः  ॥ १८.७० ॥  उच्चारण

\noindent\rule{16cm}{0.4pt} 
\end{quotation}


\begin{quotation}

श्रद्धावाननसूयश्च श्रृणुयादपि यो नरः  ।  

सोऽपि मुक्तः शुभाँल्लोकान्प्राप्नुयात्पुण्यकर्मणाम्‌  ॥ १८.७१ ॥  मूल श्लोक
\end{quotation}

\begin{quotation}
श्रद्धावान अनसूयश् च, 
श्रृणु-याद अपि यो नरः  ।  

सोऽ अपि मुक्तः शुभाँल् लोकान्, 
प्राप्-नुयात् पुण्य कर्मणाम्‌  ॥ १८.७१ ॥  उच्चारण

\noindent\rule{16cm}{0.4pt} 
\end{quotation}


\begin{quotation}

कच्चिदेतच्छ्रुतं पार्थ त्वयैकाग्रेण चेतसा  ।  

कच्चिदज्ञानसम्मोहः प्रनष्टस्ते धनञ्जय  ॥ १८.७२ ॥  मूल श्लोक
\end{quotation}

\begin{quotation}

कच्-चिद एतच् छ्रुतं पार्थ, त्वय एकाग्रेण चेतसा  ।  

कच्-चिद अज्ञान सम्मोहः, प्रनष्टस् ते धनञ्जय  ॥ १८.७२ ॥  उच्चारण

\noindent\rule{16cm}{0.4pt} 
\end{quotation}

\paragraph{\sanskrit अर्जुन उवाच}

\begin{quotation}

नष्टो मोहः स्मृतिर्लब्धा त्वप्रसादान्मयाच्युत  ।  

स्थितोऽस्मि गतसंदेहः करिष्ये वचनं तव  ॥ १८.७३ ॥  मूल श्लोक
\end{quotation}

\begin{quotation}

नष्टो मोहः स्मृतिर् लब्धा, त्व प्रसादान् मया अच्युत  ।  

स्थितोऽ अस्मि गत संदेहः, करिष्ये वचनं तव  ॥ १८.७३ ॥  उच्चारण

\noindent\rule{16cm}{0.4pt} 
\end{quotation}




\paragraph{\sanskrit सञ्जय उवाच}
\begin{quotation} 
इत्यहं वासुदेवस्य पार्थस्य च महात्मनः  ।  

संवादमिममश्रौषमद्भुतं रोमहर्षणम्‌  ॥ १८.७४ ॥  मूल श्लोक
\end{quotation}

\begin{quotation}

इत्य अहं वासु-देवस्य, पार्थस्य च महात्मनः  ।  

संवादम इमम अ-श्रौषम, अद्भुतं रोम हर्षणम्‌  ॥ १८.७४ ॥  उच्चारण

\noindent\rule{16cm}{0.4pt} 
\end{quotation}


\begin{quotation}

व्यासप्रसादाच्छ्रुतवानेतद्‍गुह्यमहं परम्‌  ।  

योगं योगेश्वरात्कृष्णात्साक्षात्कथयतः स्वयम्‌  ॥ १८.७५ ॥  मूल श्लोक
\end{quotation}

\begin{quotation}

व्यास प्रसादाच् छ्रुतवान्, एतद्‍ गुह्यम अहं परम्‌  ।  

योगं योगेश्वरात् कृष्णात्, साक्षात् कथयतः स्वयम्‌  ॥ १८.७५ ॥  उच्चारण

\noindent\rule{16cm}{0.4pt} 
\end{quotation}


\begin{quotation}


राजन्संस्मृत्य संस्मृत्य संवादमिममद्भुतम्‌  ।  

केशवार्जुनयोः पुण्यं हृष्यामि च मुहुर्मुहुः  ॥ १८.७६ ॥  मूल श्लोक
\end{quotation}

\begin{quotation}

राजन् संस्मृत्य संस्मृत्य, संवादम इमम अद्भुतम्‌  ।  

केशवा अर्जुनयोः पुण्यं, हृष्यामि च मुहुर् मुहुः  ॥ १८.७६ ॥  उच्चारण

\noindent\rule{16cm}{0.4pt} 
\end{quotation}


\begin{quotation}

तच्च संस्मृत्य संस्मृत्य रूपमत्यद्भुतं हरेः  ।  

विस्मयो मे महान्‌ राजन्हृष्यामि च पुनः पुनः  ॥ १८.७७ ॥  मूल श्लोक
\end{quotation}

\begin{quotation}

तच्च संस्मृत्य संस्मृत्य, रूपम अत्य अद्भुतं हरेः  ।  

विस्मयो मे महान्‌ राजन, ह्रष्यामि च पुनः पुनः  ॥ १८.७७ ॥  उच्चारण

\noindent\rule{16cm}{0.4pt} 
\end{quotation}


\begin{quotation}

यत्र योगेश्वरः कृष्णो यत्र पार्थो धनुर्धरः  ।  

तत्र श्रीर्विजयो भूतिर्ध्रुवा नीतिर्मतिर्मम  ॥ १८.७८ ॥  मूल श्लोक
\end{quotation}

\begin{quotation}

यत्र योगेश्वरः कृष्णो, यत्र पार्थो धनुर्धरः  ।  

तत्र श्रीर्-विजयो भूतिर्  ध्रुवा नीतिर् मतिर् मम्  ॥ १८.७८ ॥  उच्चारण

\noindent\rule{16cm}{0.4pt} 
\end{quotation}
\begin{center} ***** \end{center}

\begin{quotation}



ॐ तत् सद इति श्री मद्-भगवद्-गीतास उपनिषत्सु ब्रह्म विद्यायां योगशास्त्रे श्री कृष्णार्जुन संवादे  मोक्षसन्न्यासयोगो नामाष्टादशोऽ अध्यायः  ॥  १८  ॥ 
\end{quotation}

